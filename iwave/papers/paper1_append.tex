\append{Annotated Parameter Files}

All IWAVE applications are parameter-driven: that is, they accept as
input a {\em map} or associative array, defined by 
a list of {\tt key = value} pairs. These parameter specifications can
be included on the command line. However, because the number of such
parameter specifications is rather large, it's convenient to store
them in a parameter file (``par file''). The use of a par file has the
added advantage that the file may include annotations and white space to
improve readability. 

The examples displayed in this paper are created in the directory {\tt
  \$TOP/demo/data}. The par file {\tt parfile} is a by-product of data
creation - the SConstruct script text-processes it from prototype
files including macros, which are resolved when the scripts are
run. Four such prototype par files are present in {\tt data}, each one defining
a modeling task corresponding to a given level of grid refinement. 
The actual input to the modeling command is {\tt parfile}. 

The meaning of each parameter in the par file is described in the IWAVE
web documentation:
\begin{verbatim}
http://www.trip.caam.rice.edu/software/iwave/doc/html/index.html.
\end{verbatim}
This appendix gives a brief description of the parameter assignments
appearing in the {\tt parfile} generated for the 20 m grid example. To
run this example, and coincidentally generate its parameter file,
\begin{itemize}
\item {\tt cd \$TOP/demo/data}
\item {\tt scons demo20m}
\end{itemize}

The file {\tt parfile} groups job parameters into blocks. The first
block looks like this:
\begin{verbatim}
INPUT DATA FOR iwave

------------------------------------------------------------------------
FD:

         order = 2          spatial half-order
           cfl = 0.4        proportion of max dt/dx
          cmin = 1.0
          cmax = 4.5
          dmin = 0.5
          dmax = 5.0
         fpeak = 0.010      central frequency

\end{verbatim}
Note that comments, block labels, and typographical separators are all
accommodated. The IWAVE parameter parser identifies parameter
specifications by strings of the form 
\begin{verbatim}
          key = value
\end{verbatim}
consisting of a string with no embedded whitespace, followed by an
{\tt =} sign surrounded by any amount of whitespace on either side,
followed by another string with no embedded whitespace. Strings with embedded
whitespace are also allowed, provided that they are double-quoted -
thus {\tt ``this is a value''} is a legitimate value string. Other
capabilities of the parser are described in its html
documentation. All values are first read as strings, then
converted to other types as required.

The parameters appearing in {\tt parfile} are as follows:

\begin{itemize}
\item {\tt order = 2}: half-order of the spatial difference scheme -
  {\tt asg} implements schemes of order 2 in time, and 2k in space,
  for certain values of k, the spatial half-order. A this paper is
  written, permissible values of k are 1, 2, and 4.
\item {\tt cfl = 0.4}: max time step is computed using one osf several
  criteria - see html docs for details. This number is the fraction of
  the max step used. Must lie between 0.0 and 1.0.
\item{\tt cmin, cmax, dmin, dmax}: sanity checks on density and
  velocity values. The max permitted velocity {\tt cmax} also figures
  in two of the max time step criteria. Violation of these bounds
  causes an exception to be raised, an informative error message to be
  written to the output file {\tt cout[n].txt}, where {\tt n} is the
  process number ( = 0 for serial execution), and the program to
  exit. All trappable fatal errors are handled in this way.
\item {\tt fpeak = 0.010}: nominal central frequency, in kHz. Used in
  two ways: (1) to set the width of absorbing boundary layers by
  defining a wavelength at max velocity {\tt cmax}, and (2) as the
  center frequency of a Ricker wavelet in case the point source Ricker
  option is chosen for source generation. Plays no other role.
\item {\tt nl1, nr1, nl2, nr2}: specify 2D PML layer thicknesses: {\tt
    nl1} describes the layer thickness, in wavelengths determined as
  above, of the {\em left} boundary (with lesser coordinate) in the
  axis 1 direction, etc. Set = 0.0 for no layer, in which case the
  free (pressure-release) boundary condition is applied.
\item {\tt srctype = point}: this application implements two source
  representations, a point source with amplitude options and a very flexible array
  source option.
\item{\tt ``source  = ....''}: a quoted parameter spec is just a
  string, from the IWAVE parser's point of view, so does not define
  anything: this parameter is commented out.
\item{\tt sampord = 0}: order of spatial interpolation. Legal values
  are 0 and 1. 0 signifies rounding down the source coordinates to the
  nearest gridpoint with smaller coordinates. 1 signifies piecewise
  multilinear (adjoint) interpolation, so that the source is
  represented by a convex linear combination of point sources at
  the corners of the grid cube in which it lies.
\item {\tt refdist = 1000.0, refamp = 1.0}: point source calibration
  rule developed for the SEAM project - the wavelet amplitude is
  adjusted to produce the prescribed amplitude (in GPa) at the
  prescribed distance (in m).
\item {\tt hdrfile = ...}: IWAVE specifies acquisition parameters
  such as source and receiver locations, time sample rates and delays,
  and so on, by supplying trace headers in a file: the traces produced
  in simulation have the same headers. At present, the only
  implemented option for specifying headers is via a path to an SU
  file, that is, a SEGY-formatted file with reel header stripped
  off. Other options are planned for future releases.
\item {\tt datafile = ...}: output data file; contains traces with
  same headers as in {\tt hdrfile}, computed trace samples. Note that
  sample rate of output traces is whatever is specified in {\tt
    hdrfile}, and generally is not the same as the time step used in
  the simulation, the trace samples being resampled on output. Note
  also that pathnames may be either fully qualified (as in the {\tt hdrfile}
  entry) or relative.


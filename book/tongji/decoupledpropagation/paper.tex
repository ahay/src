\published{Geophysics, 81, no. 2, T63-T77, (2016)}

\title{Simulating propagation of decoupled elastic waves using low-rank approximate mixed-domain integral operators for anisotropic media}

\author{
Jiubing Cheng\footnotemark[1], Tariq Alkhalifah\footnotemark[2], Zedong Wu\footnotemark[3], Peng Zou\footnotemark[4] and Chenlong Wang\footnotemark[5]
}

\address{
\footnotemark[1] Tongji University, State Key Laboratory of Marine Geology, Shanghai, China. E-mail: cjb1206@tongji.edu.cn \\
\footnotemark[2] King Abdullah University of Science and Technology, Thuwal, Saudi Arabia. E-mail: tariq.alkhalifah@kaust.edu.sa \\
\footnotemark[3] King Abdullah University of Science and Technology, Thuwal, Saudi Arabia. E-mail: zedong.wu@kaust.edu.sa \\
\footnotemark[4] Tongji University, Shanghai, China. E-mail: 1533006@tongji.edu.cn \\
\footnotemark[5] Tongji University, Shanghai, China. E-mail: skoltyz.10000@163.com 
}

\lefthead{Cheng et al.}
\righthead{Propagate decoupled anisotropic elastic waves}

\maketitle

\begin{abstract}
	In elastic imaging, the extrapolated vector fields are decoupled
	into pure wave modes, such that the imaging condition produces interpretable
	images.
	Conventionally, mode decoupling in anisotropic media is costly as
	the operators involved are dependent on the velocity, and thus are not stationary.
	We develop an efficient pseudo-spectral approach to directly extrapolate
	the decoupled elastic waves using low-rank approximate
	mixed-domain integral operators on the basis of \new{the} elastic displacement wave equation.
	We apply $k$-space adjustment to the pseudo-spectral solution to allow for
	a relatively large extrapolation time-step. 
	The low-rank approximation is, thus, applied to the spectral operators that
	simultaneously extrapolate and decompose the elastic wavefields.
    Synthetic examples \new{on transversely isotropic and orthorhombic models}
	show that, our \old{approaches have}\new{approach has} the potential to efficiently and
	accurately simulate the propagations of the decoupled quasi-P and quasi-S
	modes as well as the total wavefields, for elastic wave modeling, imaging
	and inversion.
\end{abstract}

\section{Introduction}

Multicomponent seismic data are increasingly acquired on land and at the ocean bottom
in an attempt to better understand the geological structure and characterize oil and
gas reservoirs.
Seismic modeling, reverse-time migration (RTM), and full-waveform inversion (FWI) in
areas with complex geology all require high-accuracy numerical algorithms for time
extrapolation of waves.
Because seismic waves propagate through the earth as a superposition of P- and S-wave
modes, an elastic wave equation is usually more accurate for wavefield
extrapolation than an acoustic wave equation.
Wave mode decoupling can not only help elastic imaging to produce physically interpretable images,
which characterize reflectivities of various reflection types \cite[]{wapenaar:1987,dellinger.etgen:1990,yan.sava:2008},
but also provide more opportunity to mitigate the parameter trade-offs in elastic waveform inversion \cite[]{wang.cheng:2015}.

For isotropic media, far-field P and S waves can be separated by
taking the divergence and curl in the extrapolated elastic wavefields
\cite[]{aki.richards:1980,sun.mcmechan:2001}. 
Alternatively, \cite{ma.zhu:2003} and \cite{zhang:2007} extrapolated vector P and S modes separately in an
elastic wavefield by decomposing the wave equation into P- and S-wave components.
In the meantime, decoupling of wave modes yields familiar scalar wave equations
for P and S modes \cite[]{aki.richards:1980}.
In anisotropic media, one cannot, so simply, derive explicit single-mode
time-space-domain differential wave equations.
Generally, P and S modes do not respectively polarize parallel and perpendicular to
the wave vectors, and thus are called quasi-P (qP) and quasi-S (qS) waves. They cannot
be fully decoupled with divergence and curl operations \cite[]{dellinger.etgen:1990}.

Anisotropic wave propagation can be formally decoupled in the wavenumber-domain to
yield single-mode pseudo-differential equations \cite[]{liu:2009}. 
Unfortunately, these equations in time-space domain cannot be solved with traditional
numerical schemes unless further approximation to the dispersion relation or phase
velocity is applied
\cite[]{etgen:2009,chu:2011,fowler.king:2011,zhan:2012,song:2013,wu:2014,du:2014}.
To avoid solving the pseudo-differential equation, \cite{xu.zhou:2014}
proposed a nonlinear wave equation for pseudoacoustic qP-wave with an auxiliary
scalar operator depending on the material parameters and the phase direction of the
propagation at each spatial location.
All these efforts are restricted to pure-mode scalar waves and do not honor
the elastic effects such as mode conversion.
\cite{cheng.kang:2014} and \cite{kang.cheng:2012} have proposed approaches to propagate the
partially decoupled wave modes using the so-called pseudo-pure-mode wave equations,
 and then obtain completely decoupled qP or qS waves by
correcting the polarization deviation of the
pseudo-pure-mode wavefields.
Their approaches honor the kinematics of all wave modes
but may distort the reflection/transmission coefficients if high contrasts exist
in the velocity fields.

Alternatively, many have developed approaches to decouple qP- and qS-wave modes
from the extrapolated elastic wavefields.
\cite {dellinger.etgen:1990} generalized the divergence and curl operations to anisotropic media
by constructing separators as polarization projection in the wavenumber-domain.
To tackle heterogeneity, these mode separators were rewritten
by \cite{yan.sava:2009} as nonstationary spatial filters determined by the local
polarization directions.
\cite{zhang.mcmechan:2010} proposed a wavefield decomposition method to separate
elastic wavefields into vector qP- and qS-wave fields
for vertically transverse isotropic (VTI) media. 
Accordingly, \cite{cheng.fomel:2014} proposed fast mixed-domain algorithms
for mode separation and vector decomposition in heterogeneous anisotropic media by applying low-rank
approximation to the involved Fourier integral operators (FIOs) of the general form.

The motivation of this study is to develop an efficient approach to propagate and
decouple the elastic waves for general anisotropic media. The primary strategy is to
merge the numerical solutions for time extrapolation and vector decomposition into a unified
Fourier integral framework and speed up the solutions using the low-rank approximation.
This paper is organized as follows.
We first demonstrate a pseudo-spectral \old{(pseudo-spectral)} solution to extrapolate the elastic displacement wavefields in
time-domain.
Then we propose to merge the spectral operations for time extrapolation into the
integral framework for vector decomposition of the wave modes. Applying
low-rank approximations to the involved mixed-domain matrices, we obtain
an efficient algorithm for simultaneous propagating and decoupling the elastic
wavefields.
We demonstrate the validity of the proposed method using 2-D and 3-D synthetic
examples on the transversely isotropic (TI) \new{and orthorhombic} models with \new{increasing complexity}\old{vertical or tilted symmetry axis}.

\section{Propagating coupled elastic wavefields}

Following \cite{carcione:2007},
we denote the spatial variables $x$, $y$ and $z$ of a right-hand Cartesian system by
the indices $i, j,$ \ldots $=1$, $2$ and $3$, respectively, the position vector by
$\mathbf{x}$, a partial derivative with respect to a variable $x_i$ with $\partial_i$,
and the first and second time derivatives with $\partial_t$ and $\partial_{tt}$.
Matrix transposition is denoted by the superscript $``T"$. We also denote $\sqrt{-1}$ by $i$, the
scalar and matrix products by the symbol $``\cdot"$,
the dyadic product by the symbol $``\otimes"$.
The Einstein convention of repeated indices is assumed unless otherwise specified.

\subsection{Pseudo-spectral solution of the elastic wave equation}

Wave propagation in general anisotropic elastic media is governed by the linearized
momentum balance law and a linear constitutive relation between the stress and strain
tensors. These governing equations can be combined to write the displacement
formalism as,
\begin{equation}
\label{eq:3de}
\rho\partial^2_{tt}\mathbf{u} = \bigtriangledown{\cdot{[\mathbf{C}\cdot(\bigtriangledown^{T}\cdot\mathbf{u})]}} +
\mathbf{f},
\end{equation}
with $\mathbf{u}=(u_x,u_y,u_z)^{T}$ represents the vector wavefields, $\mathbf{f}$
is the body force vector per unit volume, $\mathbf{C}$ is the $6\times6$ elasticity
matrix representing the stiffness tensor with the Voigt's menu, and the spatial
differential operator $\bigtriangledown$ has the following matrix representation:
\begin{equation}
\label{eq:grad}
\bigtriangledown =
\begin{pmatrix}{\partial}_x &0 &0 &0& {\partial}_z & {\partial}_y \cr
0& {\partial}_y &0 & {\partial}_z  &0 & {\partial}_x \cr
0& 0& {\partial}_z & {\partial}_y & {\partial}_x &0 \end{pmatrix}.
\end{equation}

The pseudo-spectral method calculates the spatial derivatives using the fast Fourier transform
(FFT), while
approximating the temporal derivative with a finite-difference.
Neglecting the source term, equation \ref{eq:3de} is rewritten in the spatial
Fourier-domain for a homogeneous medium as,
\begin{equation}
\label{eq:3deh}
\partial^2_{tt}\hat{\mathbf{u}} + \mathbf{\Gamma}\hat{\mathbf{u}} =
\mathbf{0},
\end{equation}
in which $\hat{\mathbf{u}}$ is the wavefields in the wavenumber-domain,
$\mathbf{k}=(k_x,k_y,k_z)$ is the wavenumber vector, and
$\mathbf{\Gamma}=1/\rho\mathbf{L}\cdot\bar{\mathbf{C}}\cdot\mathbf{L}^{T}$ represents the
$3\times3$ \new{density normalized} Christoffel matrix with the wavenumber-domain counterpart of the space
differential operator (removing the imaginary unit $i$) satisfies,
\begin{equation}
\label{eq:gradientL}
\mathbf{L}=
\begin{pmatrix}k_x & 0 &0 &0 & k_z & k_y \cr 
               0 & k_y & 0 & k_z &0 & k_x \cr
               0 & 0 & k_z & k_y & k_x &0\end{pmatrix}.
\end{equation}

To calculate the 2nd-order temporal derivatives, we use the standard leapfrog scheme,
i.e.,
\begin{equation}
\label{eq:leapfrog}
\partial^2_{tt}{u^{(n)}_{i}} = \frac{u^{(n+1)}_i - 2u^{(n)}_i +u^{(n-1)}_i}{\Delta{t^2}},
\end{equation}
in which $\Delta{t}=t^{n+1}-t^n$ is the time-step.
For constant density homogeneous media, applying the two-step time-marching scheme
leads to the pseudo-spectral formula:
\begin{equation}
\label{eq:ps}
\partial^2_{tt}{\mathbf{u}^{(n)}} = \mathbf{\Psi} \mathbf{u}^{(n)},
\end{equation}
with the spectral operator defined with the following kernel:
\begin{equation}
\label{eq:spect}
\mathbf{\Psi}:=
(2\pi)^{-3}\int{\int{\mathbf{\Gamma}(\mathbf{k})e^{i\mathbf{k}\cdot(\mathbf{x}-\mathbf{y})}d\mathbf{y}d\mathbf{k}}}.
\end{equation}
Phase terms in the integral operator can be absorbed into forward and inverse Fourier transforms.
This implies that the wavefields are first transformed into
wavenumber-domain using forward FFTs,
then multiplied with the corresponding components of the Christoffel matrix, and finally
transformed back into space-domain using inverse FFTs.
For locally smooth media, we use a spatially varying Christoffel matrix to
tackle the heterogeneity, i.e.,
\begin{equation}
\label{eq:spectx}
\mathbf{\Psi}:=
(2\pi)^{-3}\int{\int{\mathbf{\Gamma}(\mathbf{x},\mathbf{k})e^{i\mathbf{k}\cdot(\mathbf{x}-\mathbf{y})}d\mathbf{y}d\mathbf{k}}}.
\end{equation}
The extended formation of this pseudo-spectral elastic wave propagator is shown in Appendix B.
Spectral methods are charaterized by the use of Fourier basis functions to describe the field variables and 
have the advantages over finite-difference schemes that the mesh requirements are more relaxed \cite[]{kosloff:1989,liu.li:2000}.

\subsection{Adjustment to the pseudo-spectral solution}

Generally, the two-step time-marching pseudo-spectral solution is limited to \new{a} small time-step,
as larger time-steps lead to numerical dispersion and stability issues.
At more computational costs, high-order finite-difference \cite[]{dablain:1986}
can be applied to address this difficulty.
As an alternative to second-order temporal differencing, a time integration technique based on 
rapid expansion method (REM) can provide higher accuracy with less computational efforts
\cite[]{kosloff:1989}.
As \cite{du:2014} demonstrated, one-step time marching schemes \cite[]{zhang.zhang:2009,sun.fomel:2013},
especially using optimized polynomial expansion, usually give more accurate
approximations to heterogeneous extrapolators for larger time-steps.
In this section, we discuss a strategy to extend the time-step for the previous two-step time-marching
pseudo-spectral scheme
according to the eigenvalue decomposition of the Christoffel matrix.

Since the Christoffel matrix is symmetric positive definite, it has a unique
eigen-decomposition of the form:
\begin{equation}
\label{eq:decomp1}
\mathbf{\Gamma} = \sum_{i=1}^{3}{\lambda^2_i\mathbf{a}_i\otimes{\mathbf{a}_i}},
\end{equation}
where $\lambda^2_i$'s are the eigenvalues and $\mathbf{a}_i$'s are the eigenvectors
of $\mathbf{\Gamma}$, with $\mathbf{a}_i\cdot{\mathbf{a}_j}=\delta_{ij}$.
The three eigenvalues correspond to phase velocities of the three wave modes with
$\lambda_i = v_{i}k$ (in which $k=|\mathbf{k}|$, and $v_i$ represents the phase velocity) representing the circular frequency,
and the corresponding eigenvector $\mathbf{a}_i = ({a_i}_x, {a_i}_y,
{a_i}_z)$ represents the normalized polarization vector for the given mode.
An alternative form of the above decomposition yields:
\begin{equation}
\label{eq:decomp2}
\mathbf{\Gamma} = \mathbf{Q}\mathbf{\Lambda}{\mathbf{Q}^{T}},
\end{equation}
with
\begin{equation}
\label{eq:lambda}
\mathbf{\Lambda}=
\begin{pmatrix}{\lambda^2_1} & 0 &0 \cr 
	0 & {\lambda^2_2} & 0\cr
	0 & 0 & {\lambda^2_3}\end{pmatrix},
\end{equation}
\begin{equation}
\label{eq:qq}
\mathbf{Q}=
\begin{pmatrix}{a_1}_x & {a_2}_x &{a_3}_x \cr 
	{a_1}_y & {a_2}_y &{a_3}_y \cr
	{a_1}_z & {a_2}_z &{a_3}_z \end{pmatrix}.
\end{equation}
Note $\mathbf{Q}$ is an \old{orthorgonal}\new{orthogonal} matrix, i.e.,
$\mathbf{Q}^{-1}=\mathbf{Q}^{T}$.

The eigenvalues represent the frequencies and satisfy the condition given by,
\begin{equation}
\label{eq:freqband}
\lambda_i = v_{i}k \le 2\pi{f_{max}},
\end{equation}
in which $f_{max}$ is the maximum frequency of the source.
Therefore, we suggest to filter out the high-wavenumber components in the wavefields
beyond $2\pi{f_{max}}/v_{min}$ ($v_{min}$ is the minimum phase velocity in the computational model)
caused by the numerical errors to enhance numerical stability.

According to above eigen-decomposition, we apply the $k$-space adjustment to our
pseudo-spectral scheme by modifying the eigenvalues
of Christoffel matrix for the anisotropic elastic wave equation (see Appendix C), i.e.,
\begin{equation}
\label{eq:lambda1}
\overline{\mathbf{\Lambda}}=
\begin{pmatrix}{\lambda^2_1}{sinc}^2(\lambda_1{\Delta{t}/2}) & 0 &0 \cr 
	0 & {\lambda^2_2}{sinc}^2(\lambda_2{\Delta{t}/2}) & 0\cr
	0 & 0 & {\lambda^2_3}{sinc}^2(\lambda_3{\Delta{t}/2})\end{pmatrix}.
\end{equation}
Thus this adjustment inserts a modified Christoffel matrix, i.e.,
$\overline{\mathbf{\Gamma}} = \mathbf{Q}\overline{\mathbf{\Lambda}}{\mathbf{Q}^{T}}$,
into the original pseudo-spectral formula on the basis of equations \ref{eq:ps} and \ref{eq:spectx}.
Note that the $k$-space adjustment to the pseudo-spectral solution has been widely used in acoustic and ultrasound \cite[]{bojarski:1982,tabei:2002}
and elastic isotropic wavefield simulation \cite[]{liu:1995,firouzi:2012}.

\section{Propagating decoupled elastic wavefields}

Above pseudo-spectral operators propagate the elastic wavefields as a superposition of qP- and qS-wave modes.
To obtain physically interpretable results for seismic imaging and waveform inversion, wave
mode decoupling is required during wavefield extrapolation.
The key concept of mode decoupling is based on polarization.
In a general anisotropic medium, the qP and qS
modes do not polarize parallel and perpendicular to the wave vectors.
Moreover, unlike the well-behaved qP mode, the two qS modes do not consistently polarize as a
function of the propagation direction (or wavenumber) and
thus cannot be designated as SV and SH waves, except in isotropic and TI media
\cite[]{winterstein,crampin:1991}.
Even for a TI medium, it is a challenge to find the right solution of the shear singularity problem and
obtain two completely separated S-modes with correct amplitudes
\cite[]{yan.sava:2011,cheng.fomel:2014}.
Therefore, we restrict to extrapolate the decoupled qP- and qS- wave modes in this paper.

\subsection{Vector decomposition of the elastic wave modes}

According to \cite{zhang.mcmechan:2010}, one can decompose qP and qS modes in
the elastic wavefields for a homogeneous anisotropic medium using:
\begin{equation}
\label{eq:decomPS}
u_i^{(m)}(\mathbf{k})=d_{ij}^{(m)}(\mathbf{k})\tilde{u}_j(\mathbf{k}),
\end{equation} 
where $m=\{qP, qS\}$, $i, j=\{x, y, z\}$, and the decomposition operators satisfy:
\begin{equation}
\begin{array}{lcl}
\label{eq:decP}
    d_{xx}^{(qP)}(\mathbf{k}) = a_{x}^2(\mathbf{k}), \\ 
    d_{yy}^{(qP)}(\mathbf{k}) = a_{y}^2(\mathbf{k}), \\ 
    d_{zz}^{(qP)}(\mathbf{k}) = a_{z}^2(\mathbf{k}), \\
    d_{xy}^{(qP)}(\mathbf{k}) = a_{x}(\mathbf{k})a_{y}(\mathbf{k}),
    \\
    d_{xz}^{(qP)}(\mathbf{k}) = a_{x}(\mathbf{k})a_{z}(\mathbf{k}),
    \\
    d_{yz}^{(qP)}(\mathbf{k}) = a_{y}(\mathbf{k})a_{z}(\mathbf{k}),
\end{array}
\end{equation}
and 
\begin{equation}
\begin{array}{lcl}
\label{eq:decS}
    d_{xx}^{(qS)}(\mathbf{k}) = a_{y}^2(\mathbf{k})+a_{z}^2(\mathbf{k}), \\ 
    d_{yy}^{(qS)}(\mathbf{k}) = a_{x}^2(\mathbf{k})+a_{z}^2(\mathbf{k}), \\ 
    d_{zz}^{(qS)}(\mathbf{k}) = a_{x}^2(\mathbf{k})+a_{y}^2\mathbf{k}), \\ 
    d_{xy}^{(qS)}(\mathbf{k}) = -a_{x}(\mathbf{k})a_{y}(\mathbf{k}),
    \\
    d_{xz}^{(qS)}(\mathbf{k}) = -a_{x}(\mathbf{k})a_{z}(\mathbf{k}),
    \\
    d_{yz}^{(qS)}(\mathbf{k}) = -a_{y}(\mathbf{k})a_{z}(\mathbf{k}),
\end{array}
\end{equation}
in which $a_{x}(\mathbf{k})$, $a_{y}(\mathbf{k})$ and
$a_{z}(\mathbf{k})$ represent the $x$-, $y$- and $z$-components
of the normalized polarization vector of qP-wave.

As demonstrated by \cite{cheng.fomel:2014}, one can decompose the wave modes
in a heterogeneous anisotropic medium using the following mixed-domain integral operations:
\begin{equation}
\begin{array}{lcl}
\label{eq:decomXK3}
u_{x}^{(m)}(\mathbf{x})
&=&\int{e^{i\mathbf{k}\mathbf{x}}d_{xx}^{(m)}(\mathbf{x},\mathbf{k})\tilde{u}_{x}(\mathbf{k})}\,\mathrm{d}\mathbf{k}
+\int{e^{i\mathbf{k}\mathbf{x}}d_{xy}^{(m)}(\mathbf{x},\mathbf{k})\tilde{u}_{y}(\mathbf{k})}\,\mathrm{d}\mathbf{k}
\\
&+&\int{ e^{i\mathbf{k}\mathbf{x}}d_{xz}^{(m)}(\mathbf{x},\mathbf{k})
			  \tilde{u}_{z}(\mathbf{k})}\,\mathrm{d}\mathbf{k},\\
u_{y}^{(m)}(\mathbf{x})
&=&\int{e^{i\mathbf{k}\mathbf{x}}d_{xy}^{(m)}(\mathbf{x},\mathbf{k})\tilde{u}_{x}(\mathbf{k})}\,\mathrm{d}\mathbf{k}
+\int{e^{i\mathbf{k}\mathbf{x}}d_{yy}^{(m)}(\mathbf{x},\mathbf{k})\tilde{u}_{y}(\mathbf{k})}\,\mathrm{d}\mathbf{k}
\\
&+&\int{ e^{i\mathbf{k}\mathbf{x}}d_{yz}^{(m)}(\mathbf{x},\mathbf{k})
			  \tilde{u}_{z}(\mathbf{k})}\,\mathrm{d}\mathbf{k},\\
u_{z}^{(m)}(\mathbf{x})
&=&\int{e^{i\mathbf{k}\mathbf{x}}d_{xz}^{(m)}(\mathbf{x},\mathbf{k})\tilde{u}_{x}(\mathbf{k})}\,\mathrm{d}\mathbf{k}
+\int{e^{i\mathbf{k}\mathbf{x}}d_{yz}^{(m)}(\mathbf{x},\mathbf{k})\tilde{u}_{y}(\mathbf{k})}\,\mathrm{d}\mathbf{k}
\\
&+&\int{ e^{i\mathbf{k}\mathbf{x}}d_{zz}^{(m)}(\mathbf{x},\mathbf{k}) \tilde{u}_{z}(\mathbf{k})}\,\mathrm{d}\mathbf{k}.
\end{array}
\end{equation}

\subsection{Extrapolating the decoupled elastic waves}

For heterogeneous anisotropic media, the wavefield propagator (equation~\ref{eq:ps}) and the
vector decomposition operators (equation~\ref{eq:decomXK3}) are both in the general form of
FIOs. Naturally, we merge them to derive a new mixed-domain integral solution
for extrapolating the decoupled elastic wavefields:
\begin{equation}
\begin{array}{lcl}
\label{eq:extrapdecom3}
u_{x}^{(m)}(\mathbf{x},t+\Delta{t})&=&-u_{x}^{(m)}(\mathbf{x},t-\Delta{t})
+\int{e^{i\mathbf{k}\mathbf{x}}\overline{w}_{xx}^{(m)}(\mathbf{x},\mathbf{k})\tilde{u}_x(\mathbf{k},t)}\,\mathrm{d}\mathbf{k}
\\
&+&\int{e^{i\mathbf{k}\mathbf{x}}\overline{w}_{xy}^{(m)}(\mathbf{x},\mathbf{k})\tilde{u}_y(\mathbf{k},t)}\,\mathrm{d}\mathbf{k}
+\int{e^{i\mathbf{k}\mathbf{x}}\overline{w}_{xz}^{(m)}(\mathbf{x},\mathbf{k})\tilde{u}_z(\mathbf{k},t)}\,\mathrm{d}\mathbf{k}, \\
u_{y}^{(m)}(\mathbf{x},t+\Delta{t})&=&-u_{y}^{(m)}(\mathbf{x},t-\Delta{t})
+\int{e^{i\mathbf{k}\mathbf{x}}\overline{w}_{yx}^{(m)}(\mathbf{x},\mathbf{k})\tilde{u}_x(\mathbf{k},t)}\,\mathrm{d}\mathbf{k}
\\
&+&\int{e^{i\mathbf{k}\mathbf{x}}\overline{w}_{yy}^{(m)}(\mathbf{x},\mathbf{k})\tilde{u}_y(\mathbf{k},t)}\,\mathrm{d}\mathbf{k}
+\int{e^{i\mathbf{k}\mathbf{x}}\overline{w}_{yz}^{(m)}(\mathbf{x},\mathbf{k})\tilde{u}_z(\mathbf{k},t)}\,\mathrm{d}\mathbf{k}, \\
u_{z}^{(m)}(\mathbf{x},t+\Delta{t})&=&-u_{z}^{(m)}(\mathbf{x},t-\Delta{t})
+\int{e^{i\mathbf{k}\mathbf{x}}\overline{w}_{zx}^{(m)}(\mathbf{x},\mathbf{k})\tilde{u}_x(\mathbf{k},t)}\,\mathrm{d}\mathbf{k}
\\
&+&\int{e^{i\mathbf{k}\mathbf{x}}\overline{w}_{zy}^{(m)}(\mathbf{x},\mathbf{k})\tilde{u}_y(\mathbf{k},t)}\,\mathrm{d}\mathbf{k}
+\int{e^{i\mathbf{k}\mathbf{x}}\overline{w}_{zz}^{(m)}(\mathbf{x},\mathbf{k})\tilde{u}_z(\mathbf{k},t)}\,\mathrm{d}\mathbf{k},
\end{array}
\end{equation}
with the propagation matrices for the decoupled wave modes given as
\begin{equation}
\label{eq:op}
\overline{w}_{ij}^{(m)}(\mathbf{x},\mathbf{k}) =
d_{ki}^{(m)}(\mathbf{x},\mathbf{k})w_{kj}{(\mathbf{x},\mathbf{k})},
\end{equation}
in which $w_{kj}(\mathbf{x},\mathbf{k})$ is defined by the spatially varying Christoffel
matrix and the length of time-step, namely
\begin{equation}
	\begin{array}{lcl} 
		\label{eq:extrap}
		w_{kk}(\mathbf{x},\mathbf{k})=2-\Delta{t}^2\Gamma_{kk}{(\mathbf{x},\mathbf{k})}, \\
		w_{kj}(\mathbf{x},\mathbf{k})=-\Delta{t}^2\Gamma_{kj}{(\mathbf{x},\mathbf{k})}. \\
	\end{array}
\end{equation}
The extended formulation of equation~\ref{eq:op} is given in Appendix B.
Note the symmetry properties exist: $d_{ki}^{(m)}=d_{ik}^{(m)}$ and $w_{kj}=w_{jk}$,
and the modified Christoffel matrix will be used if the $k$-space adjustment is applied for the 
pseudo-spectral solutions.

To drive the time extrapolation of the decomposed wavefields using
equation~\ref{eq:extrapdecom3}, we must update the total elastic
wavefields by superposing qP- and qS-waves at each time-step using
\begin{equation}
\begin{array}{lcl}
\label{eq:couple3}
	u_{x}(\mathbf{x}) =
	u_{x}^{(qP)}(\mathbf{x})+u_{x}^{(qS)}(\mathbf{x}),\\
	u_{y}(\mathbf{x}) =
	u_{y}^{(qP)}(\mathbf{x})+u_{y}^{(qS)}(\mathbf{x}),\\
	u_{z}(\mathbf{x}) =
	u_{z}^{(qP)}(\mathbf{x})+u_{z}^{(qS)}(\mathbf{x}).\\
\end{array}
\end{equation}
Thus equations~\ref{eq:extrapdecom3} to \ref{eq:couple3} compose the
spectral-like operators to simultaneously extrapolate and decouple
the elastic wavefields for 3D anisotropic media.
The computation complexity of the straightforward implementation of the integral
operators in equations~\ref{eq:spectx} and \ref{eq:extrapdecom3} is $O(N^2_x)$,
which is prohibitively expensive when the size of model $N_x$ is large.

To tackle strong heterogeneity due to fast varying stiffness coefficients, we suggest to split the displacement equation into the
displacement-stress equation and then solve it using the staggered-grid pseudo-spectral scheme
\cite[]{ozdenvar.mcmechan:1996,carcione:1999,bale:2003}.
Note when using staggered grids, the operators to extrapolate the decoupled wave modes must be modified in order to
account for the shifts in medium properties and fields variables. We will investigate this issue in the future work.

\section{Fast algorithm using low-rank decomposition}

As proposed by \cite{cheng.fomel:2014},
low-rank decomposition of the mixed-domain matrix
$d(\mathbf{x},\mathbf{k})$ in equation \ref{eq:decomXK3}
yields very efficient algorithm for mode decoupling
in heterogeneous anisotropic media. We find that the same strategy works for numerical
implementations of above pseudo-spectral operators for elastic wave propagation.

For example, the mixed-domain matrix, i.e., $w(\mathbf{x},\mathbf{k})$ or
$\overline{w}(\mathbf{x},\mathbf{k})$ in
the FIOs, can be approximated by the following separated representation
\cite[]{fomel:2013}: 
\begin{equation}
	\begin{array}{lcl}
    \label{eq:lowrank}
    W(\mathbf{x},\mathbf{k})\approx
        \sum_{m=1}^M \sum_{n=1}^N B(\mathbf{x},\mathbf{k}_{m})A_{mn}C(\mathbf{x}_{n},\mathbf{k}),
\end{array}
\end{equation}
in which $B(\mathbf{x},\mathbf{k}_{m})$
is a mixed-domain matrix with reduced wavenumber dimension $M$, $C(\mathbf{x}_{n},\mathbf{k})$ is a mixed-domain
matrix with reduced spatial dimension $N$, $A_{mn}$ is a ${M}\times{N}$ matrix
with $N$ and $M$ representing the rank of this decomposition.
Physically, a separable low-rank approximation amounts to selecting
a set of $N$ ($N\ll{N_{x}}$) representative spatial locations and $M$ ($M\ll{N_{x}}$) representative wavenumbers.
Construction of the separated representation follows the method of \cite{engquist.ying:2009}.
The ranks ${M}$ and ${N}$ are dependent on the complexities (heterogeneity and anisotropy) of the medium and 
the estimate of the approximation accuracy
to the mixed-domain matrices (In the numerical examples, we aim for the relative single-precision accuracy of $10^{-6}$).
More explainations on low-rank decomposition is available in \cite{fomel:2013} and \cite{cheng.fomel:2014}.
As we observe, the ranks are generally very small for our applications.
For homogeneous media, the ranks naturally reduce to $1$. If there is heterogeneity, the ranks increase to $2$
for isotropic media but exceed $2$ for anisotropic media. The $k$-space adjustment may slightly increase the ranks for the heterogeneous media.

Thus the above low-rank approximation speeds up computation of the FIOs since
\begin{equation}
	\begin{array}{lcl}
    \label{eq:lowrankFIO}
\int{e^{i\mathbf{k}\mathbf{x}}W(\mathbf{x},\mathbf{k})\tilde{u}_{j}(\mathbf{k})}\,\mathrm{d}\mathbf{k}\\
\approx
\sum_{m=1}^M B(\mathbf{x},\mathbf{k}_{m})\left(\sum_{n=1}^NA_{mn}\left(\int{e^{i\mathbf{k}\mathbf{x}}
C(\mathbf{x}_{n},\mathbf{k})\tilde{u}_{j}(\mathbf{k})\,\mathrm{d}\mathbf{k}}\right)\right).
\end{array}
\end{equation}
Evaluation of the last formula is effectively equivalent to applying $N$ inverse FFTs
each time-step. Accordingly, the computation complexity reduces to
$O(NN_x\log{N_x})$.
In multiple-core implementations, the matrix operations in equation~\ref{eq:lowrankFIO} are easy to parallelize.

\section{examples}
We will \new{first} demonstrate the proposed approach on two-layer TI and {orthorhombic} models, and \new{then on the complex} SEG Hess VTI and BP 2007 TTI models, repectively.

\subsection{2D two-layer VTI/TTI model}
The first example is on a 2D two-layer model, in which the first layer is a
VTI medium with $v_{p0}=2500 m/s$, $v_{s0}=1200 m/s$, $\epsilon=0.2$, and $\delta=-0.2$,
and the second layer is a tilted TI (TTI) medium with $v_{p0}=3600 m/s$, $v_{s0}=1800 m/s$,
$\epsilon=0.2$, $\delta=0.1$ and $\theta=30^{\circ}$. 
A point source is placed at the center of this model.
Firstly, we compare the synthetic elastic wavefields by solving the elastic
displacement wave equation using the 10th-order explicit finite-difference (FD) and low-rank
pseudo-spectral schemes (with or without the $k$-space adjustment), respectively.
Figure 1 shows the wavefield snapshots at the time of 0.3 s using the spatial sampling
$\Delta{x}=\Delta{z}=5 m$ and time-step $\Delta{t}=0.5 ms$.
Only the low-rank pseudo-spectral solutions with the $k$-space adjustment are displayed because 
the three schemes produce very similar results. The vertical slices through the z-components of 
the elastic wavefields show little differences among them (Figure 2).
For the low-rank pseudo-spectral scheme, the ranks are all $2$ for the decomposition of the mixed-domain matrices
$w_{xx}$, $w_{zz}$ and $w_{xz}$ in equation \ref{eq:extrap},
and the $k$-space adjustment doesn't change the ranks.
It takes CPU time of 0.20, 0.23 and 0.23 seconds for them to finish the wavefield extrapolation of one time-step.
Additional 4.3 and 8.2 seconds have been used to finish the low-rank decomposition of the involved mixed-domain matrices
before wavefield extrapolation.
We observe the FD scheme unstable if the time-step is increased to 1.0 ms and 
the low-rank pseudo-spectral scheme unstable if the time-step is increased to 2.0
ms (with unchanged spatial sampling). However, the low-rank pseudo-spectral solution using the $k$-space adjustment 
produces acceptable results even the time-step is increased to 3.0 ms
and the maximum time exceeds 3 s.
Figure 3 and Figure 4 compare the wavefield snapshots and the vertical slices at the time of 0.6 s
using the three schemes with the increased spatial sampling (namely $\Delta{x}=\Delta{z}=10$ m).
The FD scheme tends to exhibit dispersion artifacts with the chosen model size and extrapolation step,
while low-rank pseudo-spectral scheme exhibit acceptable accuracy. 
The $k$-space adjustment permits larger time-steps without reducing accuracy or introducing instability.
For this example, it has produced the best results with less numerical dispersion.
Thanks to the larger spatial and temporal sampling, the same CPU time is used for each scheme as in Figure 1.
In addition, only the ranks for the low-rank decomposition of the matrix $w_{12}$ reduce to $1$ when we change the tilt angle of the second layer to $0$.

\inputdir{twolayer2dti.2nd.coupled.0.5ms}
\multiplot{2}{ElasticxKSSInterf,ElasticzKSSInterf}{width=0.23\textwidth}
{
Horizontal and vertical components of the elastic wavefields at the time of 0.3 s synthesized by solving the
	2nd-order elastic wave equation with $\Delta{x}=\Delta{z}=5$ m and $\Delta{t}=0.5$ ms.
}
\multiplot{3}{ElasticzFDwave,ElasticzPSLRwave,ElasticzKSwave}{width=0.4\textwidth}
{
Vertical slices through the vertical components of the synthetic elastic wavefields at $x=0.75$ km: (a) 10th-order FD, (b) low-rank pseudo-spectral and (c) low-rank pseudo-spectral using the $k$-space adjustment.
}

\inputdir{twolayer2dti.2nd.coupled.compare}
\multiplot{3}{ElasticzFDInterfC,ElasticzPSLRInterfC,ElasticzKSSInterfC}{width=0.23\textwidth}
{
Vertical components of the elastic wavefields at the time of 0.6 s synthesized using three schemes with the same spatial sampling
$\Delta{x}=\Delta{z}=10$ m: (a) 10th-order FD and (b) low-rank pseudo-spectral with $\Delta{t}=1.5$ ms,
and (c) low-rank pseudo-spectral solution using the $k$-space adjustment with $\Delta{t}=3.0$ ms.
}
\multiplot{3}{ElasticzFDwave,ElasticzPSLRwave,ElasticzKSSwave}{width=0.4\textwidth}
{
Vertical slices through the vertical components at $x=1.5$ km in Figure 3: (a) 10th-order FD,
(b) low-rank pseudo-spectral and (c) low-rank pseudo-spectral using the $k$-space adjustment.
}

Secondly, we compare two approaches to get the decoupled elastic wavefields
during time extrapolation.
The first approach uses the low-rank pseudo-spectral algorithm
to synthesize the elastic wavefields and then apply the low-rank vector
decomposition algorithm \cite[]{cheng.fomel:2014} to get the vector qP- and
qSV-wave fields (Figure 5).
The second extrapolates the decoupled qP- and
qSV-wave fields using the proposed low-rank mixed-domain integral operations (Figure 6).
Extrapolation steps of $\Delta{x}=\Delta{z}=10$ m and $\Delta{t}=1.0$ ms are used in this example.
The ranks are still $2$
for the involved low-rank decomposition of the propagation matrices defined in equation~\ref{eq:op}. 
The Two approaches produce comparable elastic wavefields,
in which we can observe all transmitted and reflected waves including mode
conversions.
For one step of time extrapolation, it takes the CPU time of 0.6 ms for the first approach
and 0.5 ms for the second.
This means that merging time extrapolation and vector decomposition into a unified
Fourier integral framework 
provides more efficient solution than operating them in sequence
for anisotropic media thanks to the reduced number of forward and inverse FFTs.

\inputdir{twolayer2dti.2nd.decoupled.1ms}
\multiplot{6}{ElasticxPSLR1Interf,ElasticzPSLR1Interf,ElasticPxPSLR1Interf,ElasticPzPSLR1Interf,ElasticSxPSLR1Interf,ElasticSzPSLR1Interf}{width=0.21\textwidth}
{
Elastic wavefields at the time of 0.6 s synthesized by using low-rank pseudo-spectral solution of the displacement
wave equation followed with low-rank vector decomposition:
(a) x- and (b) z-components of the displacement wavefields;
(c) x- and (d) z-components of the qP-wave fields;
(e) x- and (f) z-components of the qSV-wave fields.
}
\multiplot{6}{ElasticPxPSLRInterf,ElasticPzPSLRInterf,ElasticSxPSLRInterf,ElasticSzPSLRInterf,ElasticxPSLRInterf,ElasticzPSLRInterf}{width=0.21\textwidth}
{
Elastic wavefields at the time of 0.6 s synthesized by using low-rank pseudo-spectral operators
for extrapolating and decomposing the elastic waves simultaneously:
(a) x- and (b) z-components of the qP-wave displacement wavefields;
(c) x- and (d) z-components of the qSV-wave displacement wavefields;
(e) x- and (f) z-components of the total elastic wavefields.
}

\subsection{3D two-layer VTI/orthorhombic model}

\inputdir{twolayer3delr.ort}
Figure 7 shows synthetic vector displacement fields using the proposed approach for a 3-D two-layer model, with a horizontal reflector at 1.167 km.
The first layer is a VTI medium with $v_{p0}=2500 m/s$, $v_{s0}=1400 m/s$, $\epsilon=0.25$, $\delta=0.05$, and $\gamma=0.15$,
and the second is an orthorhombic medium representing a vertically fratcured TI formation \cite[]{schoenberg:1997,tsvankin:2001}, which has
the parameters 
$v_{p0}=3000 m/s$, $v_{s0}=1600 m/s$, $\epsilon_1=0.30$, $\epsilon_2=0.15$, $\delta_1=0.08$, $\delta_2=-0.05$, $\delta_3=-0.10$, $\gamma_1=0.20$ and $\gamma_2=0.05$.
A exploration source is located at the center of the model.
We achieve efficient simulation of dispersion-free 3D elastic wave propagation for the decoupled and total displacement fields.
Shear wave splitting can be observed in the qS-wave fields.

\multiplot{9}{ElasticPx,ElasticSx,Elasticx,ElasticPy,ElasticSy,Elasticy,ElasticPz,ElasticSz,Elasticz}{width=0.28\textwidth}
{
Synthesized decomposed and total elastic wavefields for a orthorhombic model with a VTI overburden:
qP (left), qS (mid) and total (right) elastic displacement fields (top: x-component, mid: y-component, bottom: z-component).
}

\subsection{SEG Hess VTI model}

Then we demonstrate the approach in the 2D Hess VTI model (Figure 8).
Vertical qS-wave velocity is set to equal half the vertical qP-wave velocity everywhere.
A point-source is placed at location of (13.264, 4.023) km.
For comparison, spatial step length $\Delta{x}=\Delta{z}=40.0$ ft and time-step $\Delta{t}=1.0$
ms are used in this example.
Figure 9 displays the decoupled and total displacement fields synthesized by using the low-rank pseudo-spectral algorithm
that simultaneously extrapolate the decoupled qP- and qSV-wave modes. The ranks $N, M$ are in [8,10] for the low-rank
	decomposition of the involved matrices (The ranks reduce to [1,3] if we only propagate the coupled elastic wavefields).
The wavefield snapshots show that the proposed wave propagator honors the elastic effects such as mode conversion.
It takes the CPU time of 111.6 s to decompose the mixed-domain matrices in advance, and about 9397.7 s to extrapolate the
	decoupled wavefields to the maximum time of 1100.0 s.
Figure 10 displays the total displacement fields synthesized by the 10th-order FD solution of the elastic
wave equation and the decoupled qP- and qSV-wave fields using the low-rank vector decomposition
for heterogeneous TI media \cite[]{cheng.fomel:2014}.
The ranks are in $[6,7]$ for the decomposition of the involved mode decoupling matrices $d_{ij}$.
The FD solution shows strong numerical dispersion of the qSV-waves due to inadequate
sampling because the modeling of qSV-wave using FD scheme demands finer grid cell size.
Except the CPU time of 36.7 s to decompose the mixed-domain matrices for mode decoupling,
 it takes 568.0 s to extrapolate and 4690.3 s to decouple the elastic wavefields
to get qP- and qS-wave fields for all the time-steps.
To achieve the same good quality as the low-rank pseudo-spectral solution in Figure 8, we decrease the spatial
sampling to $\Delta{x}=\Delta{z}=20.0$ ft and the temporal sampling to 0.5 ms.
Except the CPU time of 133.8 s to decompose the mixed-domain matrices for mode decoupling,
 it takes 3922.8 s to extrapolate and 14212.0 s to decouple the elastic wavefields to the maximum time.
This means the low-rank pseudo-spectral scheme more efficient to obtain decoupled elastic wavefields
for TI media.

\inputdir{hessvti.2nd.decoupled}
\multiplot{3}{vp-hess,epsilon-hess,delta-hess}{width=0.4\textwidth}
  {
SEG/Hess VTI model with parameters of (a) vertical P-wave velocity, Thomsen coefficients
(b) $\epsilon$ and (c) $\delta$.
}

\multiplot{6}{ElasticPxKS,ElasticPzKS,ElasticSxKS,ElasticSzKS,ElasticxKS,ElasticzKS}{width=0.3\textwidth}
{
  Synthesized decoupled and total displacement fields using the low-rank pseudo-spectral
  solution with the $k$-space adjustment that simultaneously extrapolate and decouple qP- and qSV-wave fields in SEG/Hess VTI model:
  (a) x- and (b) z-components of qP-wave fields;
  (c) x- and (d) z-components of qSV-wave fields;
  (e) x- and (f) z-components of the total displacement fields.
}

\multiplot{6}{ElasticxFD,ElasticzFD,ElasticPxFD,ElasticPzFD,ElasticSxFD,ElasticSzFD}{width=0.3\textwidth}
  {
 Elastic wavefield extrapolation using 10th-order FD scheme and low-rank vector decomposition
 in SEG/Hess VTI model:
 (a) x- and (b) z-components of the synthetic elastic displacement wavefields at 1.1 s;
 (c) x- and (d) z-components of vector qP-wave fields;
 (e) x- and (f) z-components of vector qSV-wave fields.
  } 

\subsection{BP 2007 TTI model}
{
The last example displays the application to the BP 2007 TTI  model (Figure 11).
Vertical qS-wave velocity is set to equal sixty percent of the vertical qP-wave velocity everywhere.
Extrapolation steps of $\Delta{x}=\Delta{z}=12.5 $ m and $\Delta{t}=1.0$ ms are used here.
Because the principal axes of the medium are not aligned with the Cartesian axes, we have apply the Bond transformation
to get the stiffness matrix under the Cartesian system.
Before wavefield extrapolation, separated representations of the mixed operator matrixes are constructed using the
low-rank decomposition approach within the computational zone.
For this complex model, the ranks are about $30$ for the decomposition of the involved matrixes.
As shown in Figure 12, the approach describes very well the propagations of the decoupled qP- and qS-waves as well as the total
 displacement fields even for this complex TTI model. 
We can clearly observe the converted waves from the dipping salt flanks and other strong-contrast interfaces.
And the qP- and qS-waves are free of numerical dispersion in the decoupled and total wavefields.
}
\inputdir{bptti2007.smooth}
\multiplot{4}{vp0,epsi,del,the}{width=0.25\textwidth}
{
Partial of BP 2007 TTI model with parameters of (a) vertical P-wave velocity, Thomsen coefficients
(b) $\epsilon$ and (c) $\delta$, and (d) tilt angle $\theta$.
}

\multiplot{6}{ElasticPxKS,ElasticPzKS,ElasticSxKS,ElasticSzKS,ElasticxKS,ElasticzKS}{width=0.25\textwidth}
{
 Synthesized decoupled and total displacement fields at the time of 1.2 s using the low-rank pseudo-spectral
 solution with the $k$-space adjustment that simultaneously extrapolate and decouple qP- and qSV-wave fields in the BP 2007 TI model:
  (a) x- and (b) z-components of qP-wave fields;
  (c) x- and (d) z-components of qSV-wave fields;
  (e) x- and (f) z-components of the total displacement fields.
}


\section{conclusions}
We have proposed a recursive integral method to simultaneously extrapolate and 
decompose the elastic wavefields on the base of second-order displacement equation for heterogeneous anisotropic media.
The computational efficiency is guaranteed by merging the operations of 
time extrapolation and vector decomposition into a unified Fourier integral
framework and speeding up the solutions using the low-rank approximation.
The use of the $k$-space adjustment permits larger time-steps without reducing accuracy
or introducing instability in the low-rank pseudo-spectral scheme.
The synthetic example shows that our method could produce dispersion-free decoupled
and total elastic wavefields efficiently.
We expect that the proposed approaches to extrapolate the decoupled elastic waves have great potential for applications such
as elastic RTM and FWI of multicomponent seismic data acquired on land and at the
ocean bottom.
The focus for future work will be on the staggered-grid pseudo-spectral solution of the displacement-stress or velocity-stress equation
	for anisotropic media with strong heterogeneity and lower order of symmetry.

\section{ACKNOWLEDGMENTS}
We would like to thank Sergey Fomel for sharing his experience in
designing low-rank approximate algorithms for wave propagation.
The first author appreciates Tengfei Wang and Junzhe Sun for useful discussion in this study.
We acknowledge supports from the National
Natural Science Foundation of China (No.41474099) and Shanghai Natural Science Foundation
(No.14ZR1442900).
This publication is also based upon work supported by
the King Abdullah University of Science and Technology (KAUST) Office of Sponsored Research (OSR) under Award No. 2230.
We thank SEG, BP and HESS Corporation for making the 2D VTI and TTI models available.

\append{Components of the Christoffel matrix}
For a general anisotropic medium, the components of the \new{density normalized} Christoffel matrix
$\mathbf{\Gamma}$ are given as follows,
\begin{equation}
\label{eq:gama}
\begin{split}
\Gamma_{11} &= [C_{11}k^2_x + C_{66}k^2_y + C_{55}k^2_z +2C_{56}k_yk_z +2C_{15}k_xk_z +2C_{16}k_xk_y]/{\rho}, \\
\Gamma_{22} &= [C_{66}k^2_x + C_{22}k^2_y + C_{44}k^2_z +2C_{24}k_yk_z +2C_{46}k_xk_z +2C_{26}k_xk_y]/{\rho}, \\
\Gamma_{33} &= [C_{55}k^2_x + C_{44}k^2_y + C_{33}k^2_z +2C_{34}k_yk_z +2C_{35}k_xk_z +2C_{45}k_xk_y]/{\rho}, \\
\Gamma_{12} &= [C_{16}k^2_x + C_{26}k^2_y + C_{45}k^2_z +(C_{46}+C_{25})k_yk_z +(C_{14}+C_{56})k_xk_z +(C_{12}+C_{66})k_xk_y]/{\rho}, \\
\Gamma_{13} &= [C_{15}k^2_x + C_{46}k^2_y + C_{35}k^2_z +(C_{45}+C_{36})k_yk_z +(C_{13}+C_{55})k_xk_z +(C_{14}+C_{56})k_xk_y]/{\rho}, \\
\Gamma_{23} &= [C_{56}k^2_x + C_{24}k^2_y + C_{34}k^2_z +(C_{44}+C_{23})k_yk_z +(C_{36}+C_{45})k_xk_z +(C_{25}+C_{46})k_xk_y]/{\rho}. 
\end{split}
\end{equation}

\append{Extended formulations of the pseudo-spectral operators}
According to equations \ref{eq:ps} and \ref{eq:spectx}, we express the pseudo-spectral operator that can be used to extrapolate
the coupled elastic wavefields in its extended formation:
\begin{equation}
\begin{array}{lcl}
\label{eq:extrap3}
u_{x}(\mathbf{x},t+\Delta{t})&=&-u_{x}(\mathbf{x},t-\Delta{t})
+\int{e^{i\mathbf{k}\mathbf{x}}w_{xx}(\mathbf{x},\mathbf{k})\tilde{u}_x(\mathbf{k},t)}\,\mathrm{d}\mathbf{k}
\\
&+&\int{e^{i\mathbf{k}\mathbf{x}}w_{xy}(\mathbf{x},\mathbf{k})\tilde{u}_y(\mathbf{k},t)}\,\mathrm{d}\mathbf{k}
+\int{e^{i\mathbf{k}\mathbf{x}}w_{xz}(\mathbf{x},\mathbf{k})\tilde{u}_z(\mathbf{k},t)}\,\mathrm{d}\mathbf{k}, \\
u_{y}(\mathbf{x},t+\Delta{t})&=&-u_{y}(\mathbf{x},t-\Delta{t})
+\int{e^{i\mathbf{k}\mathbf{x}}w_{xy}(\mathbf{x},\mathbf{k})\tilde{u}_x(\mathbf{k},t)}\,\mathrm{d}\mathbf{k}
\\
&+&\int{e^{i\mathbf{k}\mathbf{x}}w_{yy}(\mathbf{x},\mathbf{k})\tilde{u}_y(\mathbf{k},t)}\,\mathrm{d}\mathbf{k}
+\int{e^{i\mathbf{k}\mathbf{x}}w_{yz}(\mathbf{x},\mathbf{k})\tilde{u}_z(\mathbf{k},t)}\,\mathrm{d}\mathbf{k}, \\
u_{z}(\mathbf{x},t+\Delta{t})&=&-u_{z}(\mathbf{x},t-\Delta{t})
+\int{e^{i\mathbf{k}\mathbf{x}}w_{xz}(\mathbf{x},\mathbf{k})\tilde{u}_x(\mathbf{k},t)}\,\mathrm{d}\mathbf{k}
\\
&+&\int{e^{i\mathbf{k}\mathbf{x}}w_{yz}(\mathbf{x},\mathbf{k})\tilde{u}_y(\mathbf{k},t)}\,\mathrm{d}\mathbf{k}
+\int{e^{i\mathbf{k}\mathbf{x}}w_{zz}(\mathbf{x},\mathbf{k})\tilde{u}_z(\mathbf{k},t)}\,\mathrm{d}\mathbf{k},
\end{array}
\end{equation}
in which $\tilde{u}_x(\mathbf{k},t)$, $\tilde{u}_y(\mathbf{k},t)$ and $\tilde{u}_z(\mathbf{k},t)$ represent the three
	components of the elastic wavefields in wavenumber-domain at the time of $t$.

For a VTI or orthorhombic medium, we express the stiffness tensor as a Voigt matrix: 
\begin{equation}
\mathbf{C} = 
\begin{pmatrix}C_{11} &C_{12} &C_{13} &0 &0 &0 \cr
         C_{12} &C_{22} &C_{23} &0 &0 &0 \cr
         C_{13} &C_{23} &C_{33} &0 &0 &0 \cr 
         0& 0&  0 & C_{44} &0 &0 \cr
         0& 0&  0 &0 & C_{55} &0 \cr
         0& 0&  0 &0 &0 &C_{66}\end{pmatrix},
\end{equation}
in which there are only five independent coefficient with
$C_{12}=C_{11}-2C_{66}$, $C_{22}=C_{11}$, $C_{23}=C_{13}$ and $C_{55}=C_{44}$, for a VTI medium.
Therefore, the propagation matrix has the following extended formulation,
\begin{equation}
\label{eq:ww}
\begin{array}{lcl} 
w_{xx}(\mathbf{k})=2-\Delta{t}^2[C_{11}k^2_x+C_{66}k^2_y+C_{55}k^2_z], \\
w_{yy}(\mathbf{k})=2-\Delta{t}^2[C_{66}k^2_x+C_{22}k^2_y+C_{44}k^2_z], \\
w_{zz}(\mathbf{k})=2-\Delta{t}^2[C_{55}k^2_x+C_{44}k^2_y+C_{33}k^2_z], \\
w_{xy}(\mathbf{k})=-\Delta{t}^2[C_{12}+C_{66}]{k_x}{k_y}, \\
w_{xz}(\mathbf{k})=-\Delta{t}^2[C_{13}+C_{55}]{k_x}{k_z}, \\
w_{yz}(\mathbf{k})=-\Delta{t}^2[C_{23}+C_{44}]{k_y}{k_z}.
\end{array}
\end{equation}

If the principal axes of the medium are not aligned with the Cartesian axes, e.g., for the tilted TI and orthorhombic media,
we should apply the Bond transformation \cite[]{winterstein,carcione:2007} to get the stiffness matrix
under the Cartesian system.
This will introduce more mixed partial derivative terms in the wave equation, which demands lots of computational effort
if a finite-difference algorithm is used to extrapolate the wavefields.
Fortunately, for the pseudo-spectral solution, it only introduces negligible computation to prepare the propagation matrix
and no extra computation for the wavefield extrapolation.

Similarly, we can write the propagation matrix $\overline{w}_{ij}^{(m)}$ (in equation~\ref{eq:op}) for the decoupled elastic waves
in its extended formulation:
\begin{equation}
\begin{array}{lcl}
\label{eq:com}
\overline{w}_{xx}^{(m)}(\mathbf{x},\mathbf{k}) =
d_{xx}^{(m)}(\mathbf{x},\mathbf{k})w_{xx}(\mathbf{x},\mathbf{k})+d_{xy}^{(m)}(\mathbf{x},\mathbf{k})w_{xy}(\mathbf{x},\mathbf{k})+d_{xz}^{(m)}(\mathbf{x},\mathbf{k})w_{xz}(\mathbf{x},\mathbf{k}), \\
\overline{w}_{xy}^{(m)}(\mathbf{x},\mathbf{k}) =
d_{xx}^{(m)}(\mathbf{x},\mathbf{k})w_{xy}(\mathbf{x},\mathbf{k})+d_{xy}^{(m)}(\mathbf{x},\mathbf{k})w_{yy}(\mathbf{x},\mathbf{k})+d_{xz}^{(m)}(\mathbf{x},\mathbf{k})w_{yz}(\mathbf{x},\mathbf{k}), \\
\overline{w}_{xz}^{(m)}(\mathbf{x},\mathbf{k}) =
d_{xx}^{(m)}(\mathbf{x},\mathbf{k})w_{xz}(\mathbf{x},\mathbf{k})+d_{xy}^{(m)}(\mathbf{x},\mathbf{k})w_{yz}(\mathbf{x},\mathbf{k})+d_{xz}^{(m)}(\mathbf{x},\mathbf{k})w_{zz}(\mathbf{x},\mathbf{k}), \\
\overline{w}_{yx}^{(m)}(\mathbf{x},\mathbf{k}) =
d_{xy}^{(m)}(\mathbf{x},\mathbf{k})w_{xx}(\mathbf{x},\mathbf{k})+d_{yy}^{(m)}(\mathbf{x},\mathbf{k})w_{xy}(\mathbf{x},\mathbf{k})+d_{yz}^{(m)}(\mathbf{x},\mathbf{k})w_{xz}(\mathbf{x},\mathbf{k}), \\
\overline{w}_{yy}^{(m)}(\mathbf{x},\mathbf{k}) =
d_{xy}^{(m)}(\mathbf{x},\mathbf{k})w_{xy}(\mathbf{x},\mathbf{k})+d_{yy}^{(m)}(\mathbf{x},\mathbf{k})w_{yy}(\mathbf{x},\mathbf{k})+d_{yz}^{(m)}(\mathbf{x},\mathbf{k})w_{yz}(\mathbf{x},\mathbf{k}), \\
\overline{w}_{yz}^{(m)}(\mathbf{x},\mathbf{k}) =
d_{xy}^{(m)}(\mathbf{x},\mathbf{k})w_{xz}(\mathbf{x},\mathbf{k})+d_{yy}^{(m)}(\mathbf{x},\mathbf{k})w_{yz}(\mathbf{x},\mathbf{k})+d_{yz}^{(m)}(\mathbf{x},\mathbf{k})w_{zz}(\mathbf{x},\mathbf{k}), \\
\overline{w}_{zx}^{(m)}(\mathbf{x},\mathbf{k}) =
d_{xz}^{(m)}(\mathbf{x},\mathbf{k})w_{xx}(\mathbf{x},\mathbf{k})+d_{yz}^{(m)}(\mathbf{x},\mathbf{k})w_{xy}(\mathbf{x},\mathbf{k})+d_{zz}^{(m)}(\mathbf{x},\mathbf{k})w_{xz}(\mathbf{x},\mathbf{k}),
\\
\overline{w}_{zy}^{(m)}(\mathbf{x},\mathbf{k}) =
d_{xz}^{(m)}(\mathbf{x},\mathbf{k})w_{xy}(\mathbf{x},\mathbf{k})+d_{yz}^{(m)}(\mathbf{x},\mathbf{k})w_{yy}(\mathbf{x},\mathbf{k})+d_{zz}^{(m)}(\mathbf{x},\mathbf{k})w_{yz}(\mathbf{x},\mathbf{k}),
\\
\overline{w}_{zz}^{(m)}(\mathbf{x},\mathbf{k}) =
d_{xz}^{(m)}(\mathbf{x},\mathbf{k})w_{xz}(\mathbf{x},\mathbf{k})+d_{yz}^{(m)}(\mathbf{x},\mathbf{k})w_{yz}(\mathbf{x},\mathbf{k})+d_{zz}^{(m)}(\mathbf{x},\mathbf{k})w_{zz}(\mathbf{x},\mathbf{k}).
\end{array}
\end{equation}
%%%%%%%%%%%%%%%%%%%%%%%%%%%%%%%%%%%%%%%%%%%%%%%%%%%%%%%%%%%%%%%%%%%%%%%%%%%%%%%%%%%%%%%%%%%%%%%%%%%%%%%%%%%%%%%%%%%%%%%%%%%

\append{K-space adjustment to the pseudo-spectral solution}
According to the eigen-decomposition of the Christoffel matrix (see Equations 9 to 12), we can obtain the scalar
wavefields for homogeneous anisotropic media using the theory of mode separation \cite []{dellinger.etgen:1990},
\begin{equation}
\label{eq:timestep}
\hat{\overline{u}}_i = Q_{ij}\hat{u}_j,
\end{equation}
in which $\hat{\overline{u}}_i$ with $i=1, 2,3$ represents the scalar qP-, qS$_1$- and
qS$_2$-wave fields.
So these wavefields satisfy the same scalar wave equation
\begin{equation}
\label{eq:scalarh}
\partial^2_{tt}\hat{\overline{u}}_i + (v_i{k})^2\hat{\overline{u}}_i = 0.
\end{equation}
The standard leapfrog scheme for this equation can be expressed as
\begin{equation}
\label{eq:timestep}
\frac{\hat{\overline{u}}^{(n+1)}_i - 2\hat{\overline{u}}^{(n)}_i +
\hat{\overline{u}}^{(n-1)}_i}{\Delta{t}^2} = -\lambda^2_i\hat{\overline{u}}^{(n)}_i.
\end{equation}
It is well known that this solution is limited to small time-steps for stable wave
propagation.

Fortunately, there is an exact time-steping solution  
for the second-order time derivatives allowing for any size of time-steps 
for homogeneous medium \cite[]{cox:2007,etgen:2009}, namely:
\begin{equation}
\label{eq:kspace}
\frac{\hat{\overline{u}}^{(n+1)}_i - 2\hat{\overline{u}}^{(n)}_i +
\hat{\overline{u}}^{(n-1)}_i}{\Delta{t}^2} =
\frac{-\sin^2(\lambda_i\Delta{t}/2)}{(\Delta{t}/2)^2}\hat{\overline{u}}^{(n)}_i.
\end{equation}
Comparing equations \ref{eq:timestep} and \ref{eq:kspace} shows that, it is possible
to extend the length of time-step without reducing the
accuracy by replacing $(\lambda_i\Delta{t}/2)^2$ with $\sin^2{(\lambda_i\Delta{t}/2})$.
This opens up a possibility by replacing $k^2$ with
$k^2{sinc^2(\lambda_i\Delta{t}/2)}$ as
a $k$-space adjustment to the spatial derivatives, which may convert the time-stepping
pseudo-spectral solution into an exact one for homogeneous media, and stable for larger
time-steps (for a given level of accuracy) in heterogeneous media
\cite[]{bojarski:1982}.

Nowadays, the $k$-space scheme is widely used to improve the approximation of the temporal
derivative in acoustic and ultrasound \cite[]{tabei:2002,cox:2007,fang.fomel:2014}.
As far as we know, \cite{liu:1995} was the first to apply $k$-space ideas to elastic wave problems. He
derived a $k$-space form of the dyadic Green's function for the
second-order wave equation and used it to calculate the scattered field
iteratively in a Born series. \cite{firouzi:2012} proposed a 
$k$-space scheme on the base of the first-order elastic wave equation for isotropic media.

Accordingly, we apply the $k$-space adjustment to improve the performance of our two-step time-marching
pseudo-spectral solution of the anisotropic elastic wave equation.
To propagate the elastic waves on the base of equations \ref{eq:ps} and
\ref{eq:spectx},
we need modify the eigenvalues of Christoffel matrix as in Equation~\ref{eq:lambda1}.


\newpage
\bibliographystyle{seg} 
\bibliography{reference}

\section{PSEUDO-PURE-MODE COUPLED SYSTEM FOR qP-WAVES}

\subsection{Plane-wave analysis of the elastic wave equation}
Vector and component notations are used alternatively throughout the paper. The wave equation in general
 heterogeneous anisotropic media can be expressed as \cite[]{carcione:2001},
\begin{equation}
\label{eq:elastic}
\rho\frac{\partial^2\mathbf{u}}{\partial t^2} = [{\bigtriangledown}{\mathbf{C}{\bigtriangledown}^{T}}]\mathbf{u} + \mathbf{f}
\end{equation}
where $\mathbf{u}=(u_x,u_y,u_z)^{T}$ is the particle displacement vector, $\mathbf{f}=(f_x,f_y,f_z)^{T}$ represents
 the force term, $\rho$ is the density, $\mathbf{C}$ is the matrix representing the stiffness tensor in a
 two-index notation called the “Voigt recipe”, and the symmetric gradient operator has
 the following matrix representation:
\begin{equation}
\label{eq:grad}
\tensor{\bigtriangledown} =
\begin{pmatrix}\frac{\partial}{\partial x} &0 &0 &0& \frac{\partial}{\partial z} & \frac{\partial}{\partial y} \cr
0& \frac{\partial}{\partial y} &0 & \frac{\partial}{\partial z}  &0 & \frac{\partial}{\partial x} \cr
0& 0& \frac{\partial}{\partial z} & \frac{\partial}{\partial y} & \frac{\partial}{\partial x} &0 \end{pmatrix}.
\end{equation}
\new{Assuming that the material properties vary sufficiently slowly so that spatial derivatives of the stiffnesses 
can be ignored, e}\old{E}quation 1 can be \old{rewritten} \new{simplified} as
\begin{equation}
\label{eq:elastic1}
\rho\frac{\partial^2\mathbf{u}}{\partial t^2} = \Gamma\mathbf{u} + \mathbf{f},
\end{equation}
where $\Gamma$ is the $3\times3$ symmetric Christoffel differential-operator matrix, of which the elements
 are given for locally smooth media as follows \cite[]{auld:1973},
\begin{equation}
\label{eq:gama}
\begin{split}
\Gamma_{11} &= 
              C_{11}\frac{\partial^2}{\partial x^2}
            + C_{66}\frac{\partial^2}{\partial y^2}
            + C_{55}\frac{\partial^2}{\partial z^2} 
            + 2C_{56}\frac{\partial^2}{{\partial y}{\partial z}} 
            + 2C_{15}\frac{\partial^2}{{\partial x}{\partial z}}
            + 2C_{16}\frac{\partial^2}{{\partial x}{\partial y}}, \\
\Gamma_{22} &= 
              C_{66}\frac{\partial^2}{\partial x^2}
            + C_{22}\frac{\partial^2}{\partial y^2}
            + C_{44}\frac{\partial^2}{\partial z^2}
            + 2C_{24}\frac{\partial^2}{{\partial y}{\partial z}}
            + 2C_{46}\frac{\partial^2}{{\partial x}{\partial z}}
            + 2C_{26}\frac{\partial^2}{{\partial x}{\partial y}}, \\
\Gamma_{33} &= 
              C_{55}\frac{\partial^2}{\partial x^2}
            + C_{44}\frac{\partial^2}{\partial y^2}
            + C_{33}\frac{\partial^2}{\partial z^2}
            + 2C_{34}\frac{\partial^2}{{\partial y}{\partial z}}
            + 2C_{35}\frac{\partial^2}{{\partial x}{\partial z}}
            + 2C_{45}\frac{\partial^2}{{\partial x}{\partial y}}, \\
\Gamma_{23} &= 
              C_{56}\frac{\partial^2}{\partial x^2}
            + C_{24}\frac{\partial^2}{\partial y^2}
            + C_{34}\frac{\partial^2}{\partial z^2}
            +(C_{44}+C_{23})\frac{\partial^2}{{\partial y}{\partial z}}
            +(C_{36}+C_{45})\frac{\partial^2}{{\partial x}{\partial z}} \\
            &+(C_{25}+C_{46})\frac{\partial^2}{{\partial x}{\partial y}}, \\
\Gamma_{13} &= 
              C_{15}\frac{\partial^2}{\partial x^2}
            + C_{46}\frac{\partial^2}{\partial y^2}
            + C_{35}\frac{\partial^2}{\partial z^2}
            +(C_{45}+C_{36})\frac{\partial^2}{{\partial y}{\partial z}}
            +(C_{13}+C_{55})\frac{\partial^2}{{\partial x}{\partial z}} \\
            &+(C_{14}+C_{56})\frac{\partial^2}{{\partial x}{\partial y}}, \\
\Gamma_{12} &= 
              C_{16}\frac{\partial^2}{\partial x^2}
            + C_{26}\frac{\partial^2}{\partial y^2}
            + C_{45}\frac{\partial^2}{\partial z^2} 
            +(C_{46}+C_{25})\frac{\partial^2}{{\partial y}{\partial z}}
            +(C_{14}+C_{56})\frac{\partial^2}{{\partial x}{\partial z}} \\
            &+(C_{12}+C_{66})\frac{\partial^2}{{\partial x}{\partial y}}.
\end{split}
\end{equation}
For the most important types of seismic anisotropy such as transverse isotropy and \old{orhtorhombic} \new{orthorhombic} anisotropy,
 some terms in equation~\ref{eq:gama} vanish
because the corresponding stiffness coefficients become zeros.

Neglecting the source term, a plane-wave analysis of the elastic anisotropic wave equation yields the 
Christoffel equation,
\begin{equation}
\label{eq:chris1}
\widetilde{\mathbf{\Gamma}}\widetilde{\mathbf{u}} = \rho{\omega}^2\widetilde{\mathbf{u}},
\end{equation}
or
\begin{equation}
\label{eq:chris2}
(\widetilde{\mathbf{\Gamma}} - \rho{\omega}^2\mathbf{I})\widetilde{\mathbf{u}} = \mathbf{0},
\end{equation}
where $\omega$ is the frequency, $\widetilde{\mathbf{u}}=(\widetilde{u}_x,\widetilde{u}_y,\widetilde{u}_z)^{T}$
 is the wavefield in Fourier domain, $\widetilde{\Gamma} = \widetilde{\mathbf{L}}\mathbf{C}\widetilde{\mathbf{L}}^{T}$ 
is the symmetric Christoffel matrix, $\mathbf{I}$ is a $3\times3$ identity matrix.
To support the sign notation in equations~\ref{eq:chris1} and \ref{eq:chris2}, 
we remove the imaginary unit $i$ of the wavenumber-domain counterpart of the gradient operator
 $\tensor{\bigtriangledown}$ and thus express matrix $\widetilde{\mathbf{L}}$ as: 
\begin{equation}
\label{eq:wavenumber}
\widetilde{\mathbf{L}}=
\begin{pmatrix}k_x & 0 &0 &0 & k_z & k_y \cr
         0 & k_y & 0 & k_z &0 & k_x \cr
         0 & 0 & k_z & k_y & k_x &0\end{pmatrix}.
\end{equation}
Setting the determinant of 
$\widetilde{\mathbf{\Gamma}} - \rho{\omega}^2\mathbf{I}$ in equation 6 to zero gives the
 characteristic equation, and expanding that determinant gives the (angular) dispersion relation. For a given
 spatial direction specified by a wave vector $\mathbf{k} = (k_x, k_y, k_z)^{T}$, the characteristic equation
 poses a standard $3\times3$ eigenvalue problem. The three eigenvalues correspond to the phase velocities of
 the qP-wave and two qS waves. Inserting one of the eigenvalues back into the Christoffel equation gives
 ratios of the components of $\mathbf{\widetilde{u}}$, from which the polarization or displacement direction
 can be determined for the given wave mode. In general, these directions are neither parallel nor
 perpendicular to the wave vector, and depend on the local material parameters for the anisotropic
 medium. For a given wave vector or slowness direction, the polarization vectors of the three wave modes are
 always mutually orthogonal.

Applying an inverse Fourier transform to the dispersion relation yields a high-order PDE in time and space
 and contains mixed space and time derivatives. Setting the shear velocity along the axis of symmetry to zero
while using Thomsen's parameter notation yields the pseudo-acoustic dispersion relation and wave equation
 in VTI media \cite[]{alkhalifah:2000}. Most published methods instead
 have used coupled PDEs (derived from the pseudo-acoustic dispersion relation) that are only second-order in time and
 eliminate the mixed space-time derivatives, e.g., \cite{zhou:2006eage}. Many kinematically equivalent
 coupled second-order systems can be generated from the dispersion relation
 by similarity transformations \cite[]{fowler:2010}. In the next section, we present a particular similarity
 transformation to the Christoffel equation in order to derive a minimal second-order coupled system,
 which is helpful for simulating propagation of separated qP-waves in anisotropic media.

\subsection{Pseudo-pure-mode qP-wave equation}
To describe propagation of separated qP-waves in anisotropic media, we first revisit the classical wave mode
 separation theory. In isotropic media, scalar P-wave can be separated from the extrapolated
 vector wavefield $\mathbf{u}$ by applying a divergence operation: $P = \bigtriangledown\cdot{\mathbf{u}}$.
 In the wavenumber domain,
this can be equivalently expressed as a dot product that essentially projects the wavefield
	$\widetilde{\mathbf{u}}$ onto the wave vector $\mathbf{k}$, i.e.,
\begin{equation}
\label{eq:PSep}
\widetilde{P} = i\mathbf{k}\cdot{\widetilde{\mathbf{u}}},
\end{equation}
Similarly, for an anisotropic medium, scalar qP-waves can be separated
by projecting the vector wavefields onto the true polarization
 directions for qP-waves by
\cite[]{dellinger.etgen:1990, yan.sava:2009},
\begin{equation}
\label{eq:qPSep}
q\widetilde{P} = i\mathbf{a_{p}}\cdot{\widetilde{\mathbf{u}}},
\end{equation}
where $\mathbf{a}_{p}=(a_{px},a_{py},a_{pz})^{T}$ represents the polarization vector for qP-waves.
For heterogeneous models, this scalar projection can be performed using
nonstationary spatial filtering depending on local material parameters \cite[]{yan.sava:2009}.

To provide more flexibility for describing wave propagation in anisotropic media, we suggest to split
the one-step projection into two steps, of which the first step
implicitly implements partail wave mode separation (like in equation~\ref{eq:PSep}) during wavefield
extrapolation with a transformed wave equation, while the second step is designed to correct the
projection deviation implied by equations~\ref{eq:PSep} and \ref{eq:qPSep}.
We achieve this on the base of the following observations:
 the difference of the polarization between an ordinary anisotropic medium and its isotropic reference 
at a given wave vector direction is usually
 small, though exceptions are possible \cite[]{thomsen:1986, tsvankin.chesnokov:1990}; The wave vector can be
taken as the isotropic reference of the polarization vector for qP-waves; It is a material-independent
operation to project the elastic wavefield onto the wave vector.

Therefore, we introduce a similarity transformation to the Christoffel matrix, i.e.,
\begin{equation}
\label{eq:tansChrisM}
\widetilde{\overline{\mathbf{\Gamma}}} = \mathbf{M_p}\widetilde{\mathbf{\Gamma}}\mathbf{M_p}^{-1},
\end{equation}
with a invertible $3\times3$ matrix $\mathbf{M}_{p}$ related to the wave vector:
\begin{equation}
\label{eq:tansM}
\mathbf{M_p}=
\begin{pmatrix}i{k_x} & 0 &0 \cr
         0 & i{k_y} &0 \cr
         0 & 0 & i{k_z}\end{pmatrix}.
\end{equation}
Accordingly, we derive an equivalent Christoffel equation,
\begin{equation}
\label{eq:tansChris}
\widetilde{\overline{\mathbf{\Gamma}}}\widetilde{\overline{\mathbf{u}}} = \rho{\omega}^2\widetilde{\overline{\mathbf{u}}},
\end{equation}
for a transformed wavefield:
\begin{equation}
\label{eq:similarT}
\widetilde{\overline{\mathbf{u}}} = \mathbf{M_p}\widetilde{\mathbf{u}}.
\end{equation}
The above similarity transformation does not change the eigeinvalues of the Christoffel matrix and thus 
introduces no kinematic errors for the wavefields. By the way, we can obtain 
the same transformed Christoffel equation if matrix $\mathbf{M}_{p}$ is constructed using the normalized wavenumbers
to ensure all spatial frequencies are uniformly scaled.
For a locally smooth medium, applying an inverse Fourier transform to
equation~\ref{eq:tansChris}, we obtain a coupled
 linear second-order system kinematically equivalent to the original elastic wave equation:
\begin{equation}
\label{eq:tansElastic}
\rho\frac{\partial^2\overline{\mathbf{u}}}{\partial t^2} = \overline{\mathbf{\Gamma}}\overline{\mathbf{u}},
\end{equation}
where $\overline{\mathbf{u}}$ represents the time-space domain wavefields, and $\overline{\mathbf{\Gamma}}$
 represents the Christoffel differential-operator matrix after the similarity transformation.

For the transformed elastic wavefield in the wavenumber-domain, we have
\begin{equation}
\label{eq:sumKdomain}
\widetilde{\overline{u}} = \widetilde{\overline{u}}_x + \widetilde{\overline{u}}_y + \widetilde{\overline{u}}_z
             = i\mathbf{k}\cdot{\widetilde{\mathbf{u}}}.
\end{equation}
And in space-domain, we also have
\begin{equation}
\label{eq:sum}
\overline{u} = \overline{u}_x + \overline{u}_y + \overline{u}_z
             = \bigtriangledown\cdot{\mathbf{u}},
\end{equation}
with
\begin{equation}
\label{eq:Deriv}
\overline{u}_x = \frac{\partial u_x}{\partial x},\qquad \overline{u}_y = \frac{\partial u_y}{\partial y},\qquad \mbox{and} \qquad \overline{u}_z=\frac{\partial u_z}{\partial z}.
\end{equation}
These imply that the new wavefield components essentially represent the spatial derivatives of the original
components of the displacement wavefield, and the transformation (equation~\ref{eq:similarT})
plus the summation of the transformed wavefield components (like in equation~\ref{eq:sumKdomain} or~\ref{eq:sum})
essentially finishes a scalar projection of the displacement wavefield onto the wave vector. 
For isotropic media, such a projection directly produces scalar P-wave data. In an anisotropic medium,
 however, only a partial wave-mode separation is achieved becuase there is usually a direction deviation 
between the wave vector and the polarization vector of qP-wave.
Generally, this deviation turns out to be small and
its maximum value rarely exceeds $20^\circ$ for typical anisotropic earth media\mbox{\cite[]{psencik:1998}}.
Due to the orthognality of qP- and qS-wave polarizations, the projection deviations of qP-waves are generally
far less than those of the qSV-waves when the elastic wavefields are projected onto the isotropic references
of the qP-wave's polarization vectors.
As demonstrated in the synthetic examples of various symmetry and strength of anisotropy, 
	the scalar wavefield $\overline{u}$ represents dominantly
the energy of qP-waves but contains some weak residual qS-waves.
This is why we call the coupled system (equation~\ref{eq:tansElastic}) a pseudo-pure-mode wave equation
for qP-wave in anisotropic media.

Substituting the corresponding stiffness matrix into the above derivations, we get the extended expression of
pseudo-pure-mode qP-wave equation for any anisotropic media.
As demonstrated in Appendix A, pseudo-pure-mode qP-wave equation in vertical TI and orthorhombic media can be expressed as
\begin{equation}
\label{eq:pseudo}
\begin{split}
\rho\frac{\partial^2\overline{u}_x}{\partial t^2} &= C_{11}\frac{\partial^2{\overline{u}_x}}{\partial x^2}
                                                  + C_{66}\frac{\partial^2{\overline{u}_x}}{\partial y^2}
                                                  + C_{55}\frac{\partial^2{\overline{u}_x}}{\partial z^2}
                                                  +(C_{12}+C_{66})\frac{\partial^2{\overline{u}_y}}{\partial x^2}
                                                  +(C_{13}+C_{55})\frac{\partial^2{\overline{u}_z}}{\partial x^2}, \\
\rho\frac{\partial^2\overline{u}_y}{\partial t^2} &= C_{66}\frac{\partial^2{\overline{u}_y}}{\partial x^2}
                                                  + C_{22}\frac{\partial^2{\overline{u}_y}}{\partial y^2}
                                                  + C_{44}\frac{\partial^2{\overline{u}_y}}{\partial z^2}
                                                  +(C_{12}+C_{66})\frac{\partial^2{\overline{u}_x}}{\partial y^2}
                                                  +(C_{23}+C_{44})\frac{\partial^2{\overline{u}_z}}{\partial y^2}, \\
\rho\frac{\partial^2\overline{u}_z}{\partial t^2} &= C_{55}\frac{\partial^2{\overline{u}_z}}{\partial x^2}
                                                  + C_{44}\frac{\partial^2{\overline{u}_z}}{\partial y^2}
                                                  + C_{33}\frac{\partial^2{\overline{u}_z}}{\partial z^2}
                                                  +(C_{13}+C_{55})\frac{\partial^2{\overline{u}_x}}{\partial z^2}
                                                  +(C_{23}+C_{44})\frac{\partial^2{\overline{u}_y}}{\partial z^2}.
\end{split}
\end{equation}
Note that, unlike the original elastic wave equation, pseudo-pure-mode wave equation dose not contain mixed partial
derivatives.
This is a good news because it takes more computational cost to compute the mixed partial derivatives
using \new{a} finite-difference algorithm with required accuracy.
In the forthcoming text, we focus on demonstration of pseudo-pure-mode qP-wave equations for TI media while
briefly supplement similar derivation for orthorhombic media in Appendix B.

\subsubsection{Pseudo-pure-mode qP-wave equation in VTI media}
For a VTI medium, there are only five independent parameters: $C_{11}$, $C_{33}$, $C_{44}$, $C_{66}$ and $C_{13}$, 
with $C_{12}=C_{11}-2C_{66}$, $C_{22}=C_{11}$, $C_{23}=C_{13}$ and $C_{55}=C_{44}$.
 So we rewrite equation~\ref{eq:pseudo} as,
\begin{equation}
\label{eq:pseudoVTI0}
\begin{split}
\rho\frac{\partial^2{\overline{u}_x}}{\partial t^2} &= C_{11}\frac{\partial^2{\overline{u}_x}}{\partial x^2}
                                         + C_{66}\frac{\partial^2{\overline{u}_x}}{\partial y^2}
                                         + C_{44}\frac{\partial^2{\overline{u}_x}}{\partial z^2}
                                         +(C_{11}-C_{66})\frac{\partial^2{\overline{u}_y}}{\partial x^2}
                                         +(C_{13}+C_{44})\frac{\partial^2{\overline{u}_z}}{\partial x^2}, \\
\rho\frac{\partial^2{\overline{u}_y}}{\partial t^2} &= C_{66}\frac{\partial^2{\overline{u}_y}}{\partial x^2}
                                         + C_{11}\frac{\partial^2{\overline{u}_y}}{\partial y^2}
                                         + C_{44}\frac{\partial^2{\overline{u}_y}}{\partial z^2} 
                                         +(C_{11}-C_{66})\frac{\partial^2{\overline{u}_x}}{\partial y^2}
                                         +(C_{13}+C_{44})\frac{\partial^2{\overline{u}_z}}{\partial y^2}, \\
\rho\frac{\partial^2{\overline{u}_z}}{\partial t^2} &= C_{44}\frac{\partial^2{\overline{u}_z}}{\partial x^2}
                                         + C_{44}\frac{\partial^2{\overline{u}_z}}{\partial y^2}
                                         + C_{33}\frac{\partial^2{\overline{u}_z}}{\partial z^2} 
                                         +(C_{13}+C_{44})\frac{\partial^2{\overline{u}_x}}{\partial z^2}
                                         +(C_{13}+C_{44})\frac{\partial^2{\overline{u}_y}}{\partial z^2}.
\end{split}
\end{equation}
Since a TI material has cylindrical symmetry in its elastic properties, it is safe to sum the first two equations
 in equation~\ref{eq:pseudoVTI0} to yield a simplified form for wavefield modeling and RTM, namely
\begin{equation}
\label{eq:pseudoVTI1}
\begin{split}
\rho\frac{\partial^2{\overline{u}_{xy}}}{\partial t^2} &=
                   C_{11}(\frac{\partial^2}{\partial x^2}+\frac{\partial^2}{\partial y^2}){\overline{u}_{xy}}
                  + C_{44}\frac{\partial^2{\overline{u}_{xy}}}{\partial z^2}
                  +(C_{13}+C_{44})(\frac{\partial^2}{\partial x^2}+\frac{\partial^2}{\partial y^2}){\overline{u}_z}, \\
\rho\frac{\partial^2{\overline{u}_z}}{\partial t^2} &= 
                   C_{44}(\frac{\partial^2}{\partial x^2}+\frac{\partial^2}{\partial y^2}){\overline{u}_z}
                  + C_{33}\frac{\partial^2{\overline{u}_z}}{\partial z^2} 
                  +(C_{13}+C_{44})\frac{\partial^2{\overline{u}_{xy}}}{\partial z^2},
\end{split}
\end{equation}
where $\overline{u}_{xy}=\overline{u}_{x}+\overline{u}_{y}$ represents the sum of the two horizontal components.
Pure SH-waves horizontally polarize in the isotropic planes of VTI media
with the polarization given by $(-k_{y}, k_{x}, 0)$, which implies $ik_{x}\widetilde{u}_{x}+ik_{y}\widetilde{u}_{y}=0$,
i.e., $\overline{u}_{xy}=0$, for the SH-wave.
Therefore, the above partial summation (after the first-step projection) completes divergence operation and removes the SH-waves from
the three-component pseudo-pure-mode qP-wave fields.
As a result, there are no terms related to $C_{66}$ any more in equation~\ref{eq:pseudoVTI1}.
Compared with original elastic wave equation, equation~\ref{eq:pseudoVTI1} further reduces the compuational
costs for 3D wavefield modeling and RTM for VTI media.

Applying the Thomsen notation \cite[]{thomsen:1986},
\begin{equation}
\label{eq:ThomsenVTI}
\begin{split}
C_{11} &= (1+2\epsilon)\rho{v_{p0}^2}, \\
C_{33} &= \rho{v_{p0}^2}, \\
C_{44} &= \rho{v_{s0}^2}, \\
C_{66} &= (1+2\gamma)\rho{v_{s0}^2}, \\
(C_{13}+C_{44})^2 &= \rho^2({v_{p0}^2}-{v_{s0}^2})({v_{pn}^2}-{v_{s0}^2}),
\end{split}
\end{equation}
the pseudo-pure-mode qP-wave equation can be expressed as,
\begin{equation}
\label{eq:pseudoVTIxy}
\begin{split}
\frac{\partial^2\overline{u}_{xy}}{\partial t^2} & =
 {v_{px}^2}(\frac{\partial^2}{\partial x^2}+\frac{\partial^2}{\partial y^2}){\overline{u}_{xy}}
+{v_{s0}^2}\frac{\partial^2{\overline{u}_{xy}}}{\partial z^2}
+ \sqrt{({v_{p0}^2}-{v_{s0}^2})({v_{pn}^2}-{v_{s0}^2})}
(\frac{\partial^2}{\partial x^2}+\frac{\partial^2}{\partial y^2}){\overline{u}_z}, \\
\frac{\partial^2\overline{u}_{z}}{\partial t^2} & =
{v_{s0}^2}(\frac{\partial^2}{\partial x^2}+\frac{\partial^2}{\partial y^2}){\overline{u}_{z}}
+ {v_{p0}^2}\frac{\partial^2{\overline{u}_z}}{\partial z^2}
+ \sqrt{({v_{p0}^2}-{v_{s0}^2})({v_{pn}^2}-{v_{s0}^2}) }
  \frac{\partial^2\overline{u}_{xy}}{\partial z^2},
\end{split}
\end{equation}
where $v_{p0}$ and $v_{s0}$ represent the vertical velocities of qP- and qSV-waves,
 $v_{pn} = v_{p0}\sqrt{1+2\delta}$ represents the interval
NMO velocity, $v_{px} = v_{p0}\sqrt{1+2\epsilon}$ 
represents the horizontal velocity of qP-waves,
$\delta$, $\epsilon$ and $\gamma$ are the other three Thomsen coefficients.
 Unlike other coupled second-order systems derived from the dispersion relation 
of VTI media \cite[]{zhou:2006eage}, the wavefield components in
 equations~\ref{eq:pseudoVTI1} and \ref{eq:pseudoVTIxy}
have clear physical meaning and their summation automatically produces scalar wavefields dominant of qP-wave energy.
 Equation~\ref{eq:pseudoVTIxy} is also similar to a minimal coupled system (equation 30 in their paper) 
demonstrated by Fowler
 et al. (2010), except that it is now derived from a significant similarity transformation that helps to
enhance qP-waves and suppress qS-waves after summing the transformed wavefield components. 
 This is undoubtedly useful for migration of conventional seismic data representing mainly qP-wave data.

We can also obtain a pseudo-acoustic coupled system by setting $v_{s0}=0$ in equation~\ref{eq:pseudoVTIxy}, namely:
\begin{equation}
\label{eq:acoustic}
\begin{split}
\frac{\partial^2\overline{u}_{xy}}{\partial t^2} & =
 (1+2\epsilon){v_{p0}^2}(\frac{\partial^2}{\partial x^2}+\frac{\partial^2}{\partial y^2}){\overline{u}_{xy}}
+ \sqrt{1+2\delta}{v_{p0}^2}(\frac{\partial^2}{\partial x^2}+\frac{\partial^2}{\partial y^2}){\overline{u}_z}, \\
\frac{\partial^2\overline{u}_{z}}{\partial t^2} & =
{v_{p0}^2}\frac{\partial^2{\overline{u}_z}}{\partial z^2}
+ \sqrt{1+2\delta}{v_{p0}^2}\frac{\partial^2\overline{u}_{xy}}{\partial z^2}.
\end{split}
\end{equation}
The pseudo-acoustic approximation does not significantly
 affect the kinematic signatures but may distort the reflection,
 transmission and conversion coefficients (thus the amplitudes) of waves in elastic media.

If we further apply the isotropic assumption (seting $\delta=0$ and $\epsilon=0$) and sum the two equations in
 equation~\ref{eq:acoustic}, we get the familar constant-density acoustic wave equation:
\begin{equation}
\frac{\partial^2\overline{u}}{\partial t^2} = 
{v_{p}^2}(\frac{\partial^2}{\partial x^2}+\frac{\partial^2}{\partial y^2}+\frac{\partial^2}{\partial z^2}){\overline{u}},
\end{equation}
where $\overline{u}=\overline{u}_{xy}+\overline{u}_{z}$ represents the acoustic pressure wavefield, and $v_{p}$ is
 the propagation velocity of isotropic P-wave.

\subsubsection{Pseudo-pure-mode qP-wave equation in TTI media}

In the case of transversely isotropic media with a tilted symmetry axis, the elastic tensor loses its simple
 form. Written in Voigt notation, it contains nonzero entries in all four quadrants if expressed in global
 Cartesian coordinates $\mathbf{x}=(x,y,z)$. The generalization of pseudo-pure-mode wave equation to a tilted symmetry
 axis involves no additional physics but greatly complicates the algebra. One strategy to derive the wave
 equations in TTI media is to locally rotate the coordinate system so that its third axis coincides with
 the symmetry axis, and make use of the simple form in VTI media.

We introduce a transformation to a rotated coordinate system $\widehat{\mathbf{x}}=(\widehat{x},\widehat{y},\widehat{z})$,
\begin{equation}
\widehat{\mathbf{x}}=\mathbf{R}^{T}\mathbf{x},
\end{equation}
where the rotation matrix $\mathbf{R}$ is dependent on the tilt angle $\theta$ and the azimuth $\varphi$ of the
 symmetry axis, namely,
\begin{equation}
\mathbf{R}=
\begin{pmatrix}r_{11} & r_{12} &r_{13} \cr
         r_{21} & r_{22} &r_{23} \cr
         r_{31} & r_{32} &r_{33}\end{pmatrix}
=\begin{pmatrix}\cos{\varphi} & -\sin{\varphi} &0 \cr
          \sin{\varphi} & \cos{\varphi} &0 \cr
          0 & 0 & 1\end{pmatrix}
\begin{pmatrix}\cos{\theta} & 0 & -\sin{\theta} \cr
          0 & 1 & 0 \cr
          \sin{\theta} & 0 & \cos{\theta}\end{pmatrix}.
\end{equation}
So,
\begin{equation}
\begin{split}
r_{11}&=\cos{\theta}\cos{\varphi}, \\
r_{12}&=-\sin{\varphi}, \\
r_{13}&=-\sin{\theta}\cos{\varphi}, \\
r_{21}&=\cos{\theta}\sin{\varphi}, \\
r_{22}&=\cos{\varphi}, \\
r_{23}&=-\sin{\theta}\sin{\varphi}, \\
r_{31}&=\sin{\theta}, \\
r_{32}&=0, \\
r_{33}&=\cos{\theta}.
\end{split}
\end{equation}
Assuming that the rotation operator $\mathbf{R}$ varies slowly so that its spatial derivatives can be ignored,
the second-order differential operators in the rotated coordinate system aligned with the symmetry axis are given as:
\begin{equation}
\label{eq:difoper}
\begin{split}
\frac{\partial^2}{\partial{\widehat{x}^2}} &= {r_{11}^2}\frac{\partial^2}{\partial x^2}
                                           + {r_{21}^2}\frac{\partial^2}{\partial y^2}
                                           + {r_{31}^2}\frac{\partial^2}{\partial z^2}
                                           + 2r_{11}r_{21}\frac{\partial^2}{{\partial x}{\partial y}}
                                           + 2r_{11}r_{31}\frac{\partial^2}{{\partial x}{\partial z}}
                                           + 2r_{21}r_{31}\frac{\partial^2}{{\partial y}{\partial z}}, \\
\frac{\partial^2}{\partial{\widehat{y}^2}} &= {r_{12}^2}\frac{\partial^2}{\partial x^2}
                                           + {r_{22}^2}\frac{\partial^2}{\partial y^2}
                                           + {r_{32}^2}\frac{\partial^2}{\partial z^2}
                                           + 2r_{12}r_{22}\frac{\partial^2}{{\partial x}{\partial y}}
                                           + 2r_{12}r_{32}\frac{\partial^2}{{\partial x}{\partial z}}
                                           + 2r_{22}r_{32}\frac{\partial^2}{{\partial y}{\partial z}}, \\
\frac{\partial^2}{\partial{\widehat{z}^2}} &= {r_{13}^2}\frac{\partial^2}{\partial x^2}
                                           + {r_{23}^2}\frac{\partial^2}{\partial y^2}
                                           + {r_{33}^2}\frac{\partial^2}{\partial z^2}
                                           + 2r_{13}r_{23}\frac{\partial^2}{{\partial x}{\partial y}}
                                           + 2r_{13}r_{33}\frac{\partial^2}{{\partial x}{\partial z}}
                                           + 2r_{23}r_{33}\frac{\partial^2}{{\partial y}{\partial z}}.
\end{split}
\end{equation}
Substituting these differential operators into the pseudo-pure-mode qP-wave equation of VTI media 
 yields the pseudo-pure-mode qP-wave equation for TTI media in the global Cartesian coordinates.
Likewise, the pseudo-pure-mode qP-wave equation in TTI media can be further simplified by applying the pseudo-acoustic approximation.
We must mention that, the above
coordinate rotation in deriving the wave equations for TTI and tilted orthorhombic media (see Appendix B)
 should be improved to enhance numerical stability according to some significant insights provided in recent literatures  
 \cite[]{duveneck:2011,macesanu,zhang:2011,bube:2012}.

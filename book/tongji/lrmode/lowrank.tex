\section{Low-rank approximation solutions}
We observe that both elastic wave mode separation and vector decomposition are based on polarization of wave modes,
and any wave mode shares the \old{program}\new{algorithm structure} of separation or decomposition. We will use qP-wave as an example.

We first write the equivalent version of the first equation of equation~\ref{eq:div.curl.like} in the space-domain as
\begin{equation}
\label{eq:sepP}
qP(\mathbf{x})=\int{e^{i\mathbf{k}\mathbf{x}}
\left[ia_{px}(\mathbf{k})\tilde{U}_{x}(\mathbf{k})+ia_{py}(\mathbf{k})\tilde{U}_{y}(\mathbf{k})+ia_{pz}(\mathbf{k})\tilde{U}_{z}(\mathbf{k})\right]
}\,\mathrm{d}\mathbf{k}.
\end{equation}
To tackle spatial variations of the polarization in heterogeneous media, we extend the integral operators using
\begin{eqnarray}
\label{eq:sepPXK}
qP(\mathbf{x})&=& \int{ e^{i\mathbf{k}\mathbf{x}}\left[ia_{px}(\mathbf{x},\mathbf{k})\right] \tilde{U}_{x}(\mathbf{k}) }\,\mathrm{d}\mathbf{k} 
                + \int{ e^{i\mathbf{k}\mathbf{x}}\left[ia_{py}(\mathbf{x},\mathbf{k})\right] \tilde{U}_{y}(\mathbf{k}) }\,\mathrm{d}\mathbf{k} \nonumber \\
              &+& \int{ e^{i\mathbf{k}\mathbf{x}}\left[ia_{pz}(\mathbf{x},\mathbf{k})\right] \tilde{U}_{z}(\mathbf{k}) }\,\mathrm{d}\mathbf{k},
\end{eqnarray}
where $a_{px}(\mathbf{x},\mathbf{k})$, $a_{py}(\mathbf{x},\mathbf{k})$ and $a_{pz}(\mathbf{x},\mathbf{k})$ represent the $x$-, $y$- and $z$-components 
of the normalized polarization vector of qP waves at location $\mathbf{x}$.
Compared with nonstationary filtering \cite[]{yan.sava:2009}, these pseudospectral-like operations are more accurate but less efficient.
The computation complexity of the straight-forward implementation is $O(N_{\mathbf{x}}^2)$, which is prohibitively expensive when the size of model $N_{\mathbf{x}}$ is large.

Similarly, from equation~\ref{eq:decomPKTI}, we can derive the space-wavenumber-domain operators for decomposing qP-waves. For example, the $x$-component of qP-wave satisfies,
\begin{eqnarray}
\label{eq:decomPXK}
U_{x}^{qP}(\mathbf{x})&=&\int{e^{i\mathbf{k}\mathbf{x}}\left[a_{px}(\mathbf{x},\mathbf{k})a_{px}(\mathbf{x},\mathbf{k})\right]\tilde{U}_{x}(\mathbf{k})}\,\mathrm{d}\mathbf{k}
                        +\int{e^{i\mathbf{k}\mathbf{x}}\left[a_{px}(\mathbf{x},\mathbf{k})a_{py}(\mathbf{x},\mathbf{k})\right]\tilde{U}_{y}(\mathbf{k})}\,\mathrm{d}\mathbf{k} \nonumber \\
                      &+&\int{ e^{i\mathbf{k}\mathbf{x}}\left[a_{px}(\mathbf{x},\mathbf{k})a_{pz}(\mathbf{x},\mathbf{k})\right] \tilde{U}_{z}(\mathbf{k})}\,\mathrm{d}\mathbf{k}.
\end{eqnarray}
Note that more multiplication operations are needed for vector decomposition.

The discrete implementation of each integral operation in equation~\ref{eq:sepPXK} or \ref{eq:decomPXK} naturally arises as a numerical approximation of a continuous Fourier
integral operator (FIO) of the general form. 
Underlying fast solutions to FIOs is a mathematical insight concerning restriction of the integral kernel to subsets of space and wavenumber domains.
Whenever these subsets obey a simple geometric condition, the restricted kernel is approximately low rank \cite[]{candes:2007,ying:2012}.
Recently, several two-way wave extrapolation operators have been developed with the help of a low-rank approximation of the space-wavenumber-domain wave-propagator
matrix in variable and possibly anisotropic media \cite[]{fomel:2010,fomel:2013,song:2011,song:2013b,alkhalifah:2013}.

For both mode separation and vector decomposition, phase terms in the FIOs are relatively simple and can be absorbed into inverse Fourier transforms. 
Therefore our main task is to respectively construct low-rank decomposition for the amplitude terms in the kernel. \new{Previously, \cite{fomel:2013} applied 
low-rank approximation to the phase-only terms in wave extrapolation operators.}
\old{For example,} Any amplitude term bracketed in equations~\ref{eq:sepPXK} and \ref{eq:decomPXK} can be approximated by the following separated representation:
\begin{eqnarray}
\label{eq:lowrank}
W_{j}(\mathbf{x},\mathbf{k})\approx
        \sum_{m=1}^M \sum_{n=1}^N B(\mathbf{x},\mathbf{k}_{m})A_{mn}C(\mathbf{x}_{n},\mathbf{k}),
\end{eqnarray}
in which $j=x, y, z$\new{, $N$ and $M$ represent the rank of this decomposition, $W_{j}(\mathbf{x},\mathbf{k})$ 
represents the mixed-domain mode separation or vector decomposition operator,
$B(\mathbf{x},\mathbf{k}_{m})$
is a mixed-domain matrix with reduced wavenumber dimension, $C(\mathbf{x}_{n},\mathbf{k})$ is a mixed-domain 
matrix with reduced spatial dimension, and $A_{mn}$ is a ${M}\times{N}$ matrix}.
\old{
This representation speeds up the computation of the FIOs since 
\parbox{\textwidth}{
\begin{eqnarray}
\label{eq:lowrankFIO}
\begin{split}
&\int{e^{i\mathbf{k}\mathbf{x}}W_{j}(\mathbf{x},\mathbf{k})\tilde{U}_{j}(\mathbf{k})}\,\mathrm{d}\mathbf{k} \approx \\
&\sum_{m=1}^M B(\mathbf{x},\mathbf{k}_{m})\left(\sum_{n=1}^Na_{mn}\left(\int{e^{i\mathbf{k}\mathbf{x}}
              C(\mathbf{x}_{n},\mathbf{k})\tilde{U}_{j}(\mathbf{k})\,\mathrm{d}\mathbf{k}}\right)\right). 
\end{split}
\end{eqnarray}
}
The evaluation of the last formula is effectively equivalent to applying $N$ inverse fast Fourier transforms.
	Physically, a separable low-rank
approximation amounts to selecting a set of $N$ representative spatial locations and $M$ representative wavenumbers.
}
\old{C}\new{The c}onstruction of \old{approximation}{\new{the separated form} \ref{eq:lowrank} \new{follows the method
of \cite{engquist.ying:2009}.
It} can be viewed as a matrix decomposition problem, i.e.,
\begin{equation}
\label{eq:lowrankMatrix}
\mathbf{W} \approx \mathbf{B}\mathbf{A}\mathbf{C},
\end{equation}
where $\mathbf{W}$ is the $N_{\mathbf{x}}\times{N_{\mathbf{x}}}$ matrix with entries $W_{j}(\mathbf{x},\mathbf{k})$, $\mathbf{B}$ is the submatrix of $\mathbf{W}$ that
 consists \old{in}\new{of} columns associated with $\{\mathbf{k}_{m}\}$, $\mathbf{C}$ is the submatrix that consists \old{in}\new{of} rows associated with
 $\{\mathbf{x}_{n}\}$, and $\mathbf{A}=\{A_{mn}\}$. 
\new{
Physically, a separable low-rank approximation amounts to selecting
a set of $N$ ($N\ll{N_{x}}$) representative spatial locations and $M$ ($M\ll{N_{x}}$) representative wavenumbers.
As explained by \cite{fomel:2013} in detail, we first need to restrict the mixed-domain $\mathbf{W}$ to $n$
randomly selected rows. In practice, $n$ can be scaled as 
$O(r\log{N_{x}})$ and $r$ represents the numerical rank of $\mathbf{W}$. Then we apply pivoted QR algorithm \cite[]{golub:1996}
to find the corresponding columns for $B(\mathbf{x},\mathbf{k}_{m})$. To find the rows for $C(\mathbf{x}_{n},\mathbf{k})$, we
apply the pivoted QR algorithm to $\mathbf{\mathbf{W}^{\ast}}$.
The algorithm does not require, at any step, access to the full-matrix $\mathbf{W}$, only to its selected rows and columns.
}

\new{
Representation \ref{eq:lowrank} speeds up the computation of the FIOs in equations \ref{eq:sepPXK} and \ref{eq:decomPXK} since 
\begin{eqnarray}
\label{eq:lowrankFIO}
\begin{split}
&\int{e^{i\mathbf{k}\mathbf{x}}W_{j}(\mathbf{x},\mathbf{k})\tilde{U}_{j}(\mathbf{k})}\,\mathrm{d}\mathbf{k} \approx \\
&\sum_{m=1}^M B(\mathbf{x},\mathbf{k}_{m})\left(\sum_{n=1}^Na_{mn}\left(\int{e^{i\mathbf{k}\mathbf{x}}
              C(\mathbf{x}_{n},\mathbf{k})\tilde{U}_{j}(\mathbf{k})\,\mathrm{d}\mathbf{k}}\right)\right). 
\end{split}
\end{eqnarray}
The evaluation of the last formula is effectively equivalent to applying $N$ inverse fast Fourier transforms (FFTs).
Accordingly, with low-rank approximation, the computation complexity reduces to $O(NN_{\mathbf{x}}\log{N_{\mathbf{x}}})$.
In other word, the costs are mainly controled by the model size $N_{x}$ and the rank $N$,
   which depends on the complexity of the anisotropic velocity model.
 For isotropic models with arbitrary heterogeneity, the rank automatically reduces to $1$ because the
 polarization directions are material-independent. 
Similarly to the observation by \cite{fomel:2013}, there is a natural tradeoff in the selection of N: larger values lead to
a more accurate separated representation but require a longer computational time. In the examples of the next section, the 
ranks are automatically calculated based on the estimate of the approximation accuracy and generally aiming for the relative single-precision accuracy 
(namely the maximum allowable error in low-rank decomposition) of $10^{-6}.$
}
\old{In practice, we find that matrix $\mathbf{W}$ for mode separation or vector decomposition dose has a low-rank
separated representation, at least in TI media.
Construction of the separated representation follows the method of \mbox{\cite{engquist.ying:2009}}.
The algorithm does not require, at any step, access to the full-matrix $\mathbf{W}$, only to its selected rows and columns.
 Once the related low-rank decomposition is complete, it is used in equation~\ref{eq:lowrankFIO} at every time step.
For mode separation, computation complexity reduces to $O(3NN_{\mathbf{x}}\log_{2}{N_{\mathbf{x}}})$,
 where $N_{\mathbf{x}}\log_{2}{N_{\mathbf{x}}}$ refers to the cost of the Fourier transform. 
}
In multiple-core implementations, the matrix operations in equation~\ref{eq:lowrankFIO} are easy to parallelize.

\new{For some applications such as ERTM in TI media, one should construct the separated representations of the operator matrixes in advance,
and then implement mode separation or/and decomposition of the extrapolated elastic wavefields before applying the imaging condition.
To further save computational cost, appropriate relaxing of the accuracy requirement for low-rank approximation and applying
mode separation only every two or three time steps are both good choices in practice.}

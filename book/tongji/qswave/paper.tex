\title{Simulating propagation of separated wave modes in general anisotropic media,
Part II: qS-wave propagators}

\author{Jiubing Cheng\footnotemark[1] and Wei Kang\footnotemark[2]}

\address{
\footnotemark[1] State Key Laboratory of Marine Geology,\\
Tongji University, Shanghai, China. E-mail: cjb1206@tongji.edu.cn\\
\footnotemark[2] Formerly Tongji University, Shanghai, China;\\
presently Schlumberger, Houston, Texas, USA. E-mail: wkang@slb.com\\
}

\lefthead{Cheng \& Kang}
\righthead{Propagate separated qS-wave modes}

\maketitle

\newpage
%\begin{abstract}
Wave propagation in an anisotropic medium is inherently described by elastic wave equations, with P- and S-wave modes
intrinsically coupled. We present an approach to simulate propagation of separated wave modes for
 forward modeling,
 migration, waveform inversion and other applications in general anisotropic media.
The proposed approach consists of two cascaded computational steps. First, we simulate equivalent
 elastic anisotropic wavefields with a minimal second-order coupled system 
(that we call here a pseudo-pure-mode wave equation),
which describes propagation of all wave modes with a partial wave mode separation. Such a system for qP-wave
 is derived from the inverse Fourier transform of the Christoffel equation after a
 similarity transformation, which aims to project the original vector displacement wavefields onto isotropic
 references of the polarization directions of qP-waves. It accurately describes the kinematics
 of all wave modes and enhances qP-waves when the pseudo-pure-mode wavefield components are summed.
 The second step is a filtering to further project the pseudo-pure-mode wavefields onto the polarization directions
 of qP-waves so that residual qS-wave energy is removed and scalar qP-wave fields are accurately separated
 at each time step during wavefield extrapolation.
As demonstrated in the numerical examples, pseudo-pure-mode wave equation plus
correction of projection deviation provides a robust and flexible
tool for simulating propagation of separated wave modes in
transversely isotropic and orthorhombic media.
The synthetic example of Hess VTI model shows that
the pseudo-pure-mode qP-wave equation can be used in prestack reverse-time migration (RTM) applications.
\end{abstract}

\begin{abstract}
Shear waves, especially converted modes in multicomponent seismic data, 
provide significant information that allows better delineation of
geological structures and characterization of petroleum reservoirs.
Seismic imaging and inversion based upon the elastic wave equation involve high computational
cost and many challenges in decoupling the wave modes and estimating so many model parameters.
For transversely isotropic media, shear waves can be designated as pure SH and quasi-SV modes.
Through two different similarity transformations to the Christoffel equation aiming to project the
vector displacement wavefields onto the isotropic references of the polarization directions,
we derive simplified second-order systems (i.e., pseudo-pure-mode wave equations)
for SH- and qSV-waves, respectively.
The first system propagates a vector wavefield with two horizontal components, of which the summation
produces pure-mode scalar SH-wave data,
while the second propagates a vector wavefield with a summed horizontal component and a vertical component,
of which the final summation produces a scalar field dominated by qSV-waves in energy.
The simulated SH- or qSV-wave has the same kinematics as its counterpart in the elastic wavefield.
As explained in our previous paper (part I), we can obtain completely separated 
scalar qSV-wave fields after spatial filtering the pseudo-pure-mode qSV-wave fields.
Synthetic examples demonstrate that these wave propagators provide efficient and flexible tools 
for qS-wave extrapolation in general transversely isotropic media.
\end{abstract}

%\section{Introduction}
Riemannian wavefield extrapolation \cite[]{SavaFomel.geo.rwe}
generalizes solutions to the Helmholtz equation in general Riemannian
coordinate systems. Conventionally, the Helmholtz equation is solved
in Cartesian coordinates which represent special cases of Riemannian
coordinates. The main requirements imposed on the Riemannian
coordinate systems are that they maintain orthogonality between the
extrapolation coordinate and the other coordinates (2 in 3D, 1 in
2D). This requirement can be relaxed when using an even more general
form of RWE in non-orthogonal coordinates
\cite[]{Shragge.geo.nonlinear}. In addition, it is desirable that the
coordinate system does not triplicate, although numerical methods can
stabilize extrapolation even in such situations
\cite[]{SavaFomel.geo.rwe}. Thus, wavefield extrapolation in
Riemannian coordinates has the flexibility to be used in many
applications where those basic conditions are fulfilled. Cartesian
coordinate systems, including tilted coordinates, are special cases of
Riemannian coordinate systems.
%
\par Two straightforward applications of wave propagation in
Riemannian coordinates are extrapolation in a coordinate system
created by ray tracing in a smooth background velocity
\cite[]{SavaFomel.geo.rwe}, and extrapolation with a coordinate system
created by conformal mapping of a given geometry to a regular space,
for example migration from topography \cite[]{ShraggeSava.segab.2005}.
%
\par Coordinate systems created by ray tracing in a background medium
often well represent wavefield propagation. In this context, we
effectively split wave propagation effects into two parts: one part
accounting for the general trend of wave propagation, which is
incorporated in the coordinate system, and the other part accounting
for the details of wavefield scattering due to rapid velocity
variations. If the background medium is close to the real one, the
wave-propagation can be properly described with low-order
operators. However, if the background medium is far from the true one,
the wavefield departs from the general direction of the coordinate
system and the low-order extrapolators are not enough for accurate
description of wave propagation.
%
\par For coordinate system describing a geometrical property of the
medium (e.g. migration from topography), there is no guarantee that
waves propagate in the direction of extrapolation. This situation is
similar to that of Cartesian coordinates when waves propagate away
from the vertical direction, except that conformal mapping gives us
the flexibility to define any coordinates, as required by
acquisition. In this case, too, low-order extrapolators are not enough
for accurate description of wave propagation.  
%
\par Therefore, there is need for higher-order Riemannian wavefield
extrapolators in order to handle correctly waves propagating obliquely
relative to the coordinate system. Usually, the high-order
extrapolators are implemented as mixed operators, part in the Fourier
domain using a reference medium, part in the space domain as a
correction from the reference medium. Many methods have been developed
for high-order extrapolation in Cartesian coordinates. In this paper,
we explore some of those extrapolators in Riemannian coordinates, in
particular high-order finite-differences solutions
\cite[]{Claerbout.iei}, and methods from the pseudo-screen family
\cite[]{GEO64-05-15241534} and Fourier finite-differences family
\cite[]{GEO59-12-18821893,GEO67-03-08720882}. In theory, any other
high-order extrapolator developed in Cartesian coordinates can have a
correspondent in Riemannian coordinates.  
%
\par In this paper, we implement the finite-differences portion of the
high-order extrapolators with implicit methods. Such solutions are
accurate and robust, but they face difficulties for 3D implementations
because the finite-differences part cannot be solved by fast
tridiagonal solvers anymore and require more complex and costlier
approaches \cite[]{GEO63-05-15321541,SEG-1998-1124}. The problem of 3D
wavefield extrapolation is addressed in Cartesian coordinates either
by splitting the one-way wave-equation along orthogonal directions
\cite[]{GEO62-02-05540567}, or by explicit numerical solutions
\cite[]{GEO56-11-17701777}. Similar approaches can be employed for 3D
Riemannian extrapolation. The explicit solution seems more
appropriate, since splitting is difficult due to the mixed terms of
the Riemannian equations. In this paper, we concentrate our attention
to higher-order kernels implemented with implicit methods.

\section{Introduction}
Ultrasonic laboratory studies as well as seismic field investigations have shown that many geological materials
and subsurface structures are elastically anisotropic.
It is well known that a shear wave passing through an anisotropic material splits into two mutually
 orthogonal waves, which propagate at different velocities.
Therefore, seismic waves propagate through the earth as a superposition of three body wave modes, one P-wave and two S-waves.
Generally they are polarized neither parallel to nor perpendicular to the direction of wave travel, 
thus are called quasi-P (qP) and quasi-S (qS) waves, with quasi- means similar to but not exactly.
P- and S-waves were originally named for their arrival times,
with P for the first (primary) and S for the second. Today, the indicators P and S are often connected with polarization,
 i.e., P with compressional (or longitudinal) and S with shear (or transverse), with the specification SH and SV for waves
 with transverse displacements in the horizontal and vertical planes, respectively \cite[]{winterstein:1990}.
For vertical transversely isotropic (VTI) media, one often uses this terminology with qP, qSV, and SH, since the first two of
these waves are generally not purely longitudinal and transverse, respectively.

Because seismic anisotropy by nature is an elastic phenomenon, the full elastic wave equation 
is usually more accurate for wavefield extrapolation than the acoustic equation.
However, seismic imaging
using the elastic wave equation involve high computational cost and many
challenges in decoupling wave modes to get
 physically interpretable images of the subsurface \cite[]{dellinger.etgen:1990,zhang.mcmechan:2010,yan.sava:2011,cheng.fomel:2014}.
For real-sized applications of seismic imaging and 
inversion, it is necessary to resort to a simplified description of wave
propagation in anisotropic media.

Pseudoacoustic wave equations are the most common approximations made
to mono-component (mainly pressure) seismic data. They are derived by setting the
qS-wave phase velocity along the symmetry axis to zero  
for VTI or orthorhombic media \cite[]{alkhalifah:2000,
alkhalifah:2003,duveneck:2011}.
Pseudoacoustic wave equations describe the kinematic signatures of qP-waves
   with sufficient accuracy and are simpler than their elastic
   counterparts, which leads to computational savings in practice
   \cite[]{zhou:2006eage,fletcher:2009,zhang.zhang:2011}.
They also have fewer parameters, which is important for inversion.
However, we note several limitations of the acoustic anisotropic wave equation.
First, acoustic approximation does not prevent the propagation of qS-waves in directions other than
   the symmetry axis \cite[]{grechka:2004,zhang:2005}; the residual qS-waves are
   regarded as artifacts in the framework of acoustic modeling, reverse-time migration
   (RTM) and full waveform inversion (FWI)
   \cite[]{alkhalifah:2000,zhang:2009,operto:2009}.
Second, stability analysis based on requiring the stiffness tensor to remain positive definite
\cite[]{helbig:1994} shows that wavefield extrapolation in a pseudo-acoustic TI or orthorhombic medium
can become unstable \cite[]{alkhalifah:2000,grechka:2004,fowler.king:2011}.
Alternatively, qP- and qS-wave propagation can be formally decoupled in the wavenumber domain
to yield pure-mode pseudodifferential equations \cite[]{liu:2009,du:2014}.
Unfortunately, these equations in time-space domain cannot be solved with traditional numerical schemes.
Through factorizing and approximating the qP-qSV dispersion relations or phase velocities,
many authors proposed to simulate propagation of scalar pure-mode waves
using mixed-domain recursive integral operators
\cite[]{etgen:2009,liu:2009,zhang.zhang:2009,crawley:2010,
fowler:2010,chu:2011,pestana:2011,zhan:2012,fomel:2013}
or novel finite-difference operators \cite[]{song:2013}.
To avoid solving the pseudodifferential equation, \cite{xu.zhou:2014}
proposed a nonlinear wave equation for a pseudo-acoustic qP-wave with an auxiliary
scalar operator depending on the material parameters and the phase direction of the
propagation at each spatial location.

 Multicomponent seismic data are increasingly acquired on land and at the ocean
 bottom to better delineate geological structures and to characterize oil and gas
 reservoirs \cite[]{li:1997,thomsen:1999,cary:2001,stewart:2002,hardage:2011}. 
 The development of unconventional reservoirs and the microseismic monitoring of hydraulic fracturing
 has led to more interest in shear waves because microseismic sources emit strong shear energy that is routinely 
 recorded by three-component geophones \cite[]{maxwell} and is widely recognized as
 being useful for locating microseismic events and estimating their focal mechanisms
\cite[]{baig.urbancic:2010,grechka.yaskevich:2014}.
 In fact, anisotropic phenomena are especially noticeable in shear and mode-converted wavefields.
 Therefore, modeling of anisotropic shear waves
may be important both theoretically and practically.
As we know, the pseudoacoustic approximation is not appropriate for qS-waves.
In addition to amplitude errors, the kinematic accuracy of qS-waves is reduced if we use the
existing numerical schemes based on factorizing and approximating the dispersion relations or phase velocities.

In kinematics, there are various forms equivalent to the original elastic wave equations.
In our previous paper (part I), we derived the pseudo-pure-mode wave equation for qP-waves
by applying a particular similarity transformation to the Christoffel equation and demonstrated its features in
describing wave propagation for TI and orthorhombic media.
Except for its application to scalar qP-wave RTM \cite[]{cheng.kang:2014}, the pseudo-pure-mode wave equation
provides new insight into developing approaches for multicomponent qP-wave inversion
\cite[]{djebbi.alkhalifah:2014}.
The same theoretical framework described in part I is applied to qS-waves in this paper.
First we derive the pseudo-pure-mode wave equations for qS-waves in TI media through
new similarity transformations to the original Christoffel equation.
Numerical examples demonstrate the features of the proposed qS-wave propagators in
 2D and 3D TI media.
Then we discuss the dynamic features of the pseudo-pure-mode qS-wave equations and the challenges to
extending them to anisotropic media with lower symmetry.

%\section{Extrapolation kernels}
Extrapolation using \req{rwemain} implies that the coefficients
defining the problem, $a$ and $b$, are not changing spatially. In this
case, we can perform extrapolation using a simple phase-shift
operation
%
\beq
\UU_{\tt+\Delta\tt} = \UU_{\tt} e^{i \kt \Delta\tt} \;,
\eeq
%
where $\UU_{\qt+\Delta\qt}$ and $\UU_{\qt}$ represent the acoustic
wavefield at two successive extrapolation steps, and $\kt$ is the
extrapolation wavenumber defined by \req{rwemain}.  \par For media
with lateral variability of the coefficients $a$ and $b$, due to
either velocity variation or focusing/defocusing of the coordinate
system, we cannot use in extrapolation the wavenumber computed
directly using \req{rwemain}. Like for the case of extrapolation in
Cartesian coordinates, we need to approximate the wavenumber $\kt$
using expansions relative to $a$ and $b$. Such approximations can be
implemented in the space-domain, in the Fourier domain or in mixed
space-Fourier domains.

\subsection{Space-domain extrapolation}
The space-domain finite-differences solution to \req{rwemain} is
derived based on a square-root expansion as suggested by Francis Muir
\cite[]{Claerbout.iei}:
%
\beq \label{eqn:rwexfd}
\kt   \approx  \ww a + \ww \frac{    \nu \yys }
                                {\mu-\ro \yys } \;,
\eeq
%
where the coefficients $\mu$, $\nu$ and $\ro$ take the form derived in Appendix A:
%
\beqa
\nu &=& - \co a \bas \;,
\\
\mu &=&   1          \;,
\\
\ro &=&   \ct   \bas \;. 
\eeqa
%
In the special case of Cartesian coordinates, $a=s$ and $b=1$,
\req{rwexfd} takes the familiar form
%
\beq
\kt   \approx  \ww s - \ww \frac{    \frac{\co}{s}   \yys }
                                {1  -\frac{\ct}{s^2} \yys } \;,
\eeq
%
where the coefficients $\co$ and $\ct$ take different values for
different orders of Muir's expansion: $(\co,\ct)=(0.50,0.00)$ for the
$15^\circ$ equation, and $(\co,\ct)=(0.50,0.25)$ for the $45^\circ$
equation, etc. For extrapolation in Riemannian coordinates, the
meaning of $15^\circ$, $45^\circ$ etc is not defined. We use this
terminology here to indicate orders of accuracy comparable to the ones
defined in Cartesian coordinates.

\subsection{Mixed-domain extrapolation}
Mixed-domain solutions to the one-way wave equation consist of
decompositions of the extrapolation wavenumber defined in
\req{rwemain} in terms computed in the Fourier domain for a reference
of the extrapolation medium, followed by a finite-differences
correction applied in the space-domain. For \req{rwemain}, a generic
mixed-domain solution has the form:
%
\beq \label{eqn:rwemixed}
\kt \approx \kto + \ww \lp a-\ao \rp + 
                   \ww \frac{    \nu \yys }
                            {\mu-\ro \yys } \;,
\eeq
%
where $\ao$ and $\bo$ are reference values for the medium
characterized by the parameters $a$ and $b$, and the coefficients
$\mu$, $\nu$ and $\ro$ take different forms according to the type of
approximation. As for usual Cartesian coordinates, $\kto$ is applied
in the Fourier domain, and the other two terms are applied in the
space domain. If we limit the space-domain correction to the thin lens
term, $\ww \lp a-\ao \rp$, we obtain the equivalent of the split-step
Fourier (SSF) method \cite[]{GEO55-04-04100421} in Riemannian
coordinates.
%
\par
%
Appendix A details the derivations for two types of expansions known
by the names of pseudo-screen \cite[]{GEO64-05-15241534}, and Fourier
finite-differences \cite[]{GEO59-12-18821893,GEO67-03-08720882}. Other
extrapolation approximations are possible, but are not described here,
for simplicity.

\begin{itemize}
\item {\bf Pseudo-screen method:}
\par
The coefficients for the pseudo-screen approximation to \req{rwemixed} are
\beqa \label{eqn:rwepsc}
\nu &=& \ao \lb \co \lp \frac{a}{\ao}-1 \rp - \lp \frac{b}{\bo}-1 \rp\rb \baos \;,
\\
\mu &=&   1 \;,
\\
\ro &=&3\ct\baos \;,
\eeqa
%
where $\ao$ and $\bo$ are reference values for the medium
characterized by parameters $a$ and $b$. In the special case of
Cartesian coordinates, $a=s$ and $b=1$, \req{rwemixed} with
coefficients \req{rwepsc} takes the familiar form
%
\beq
\kt \approx \kto + \ww
\lb 1+ \frac{    \frac{ \co}{s_0^2} \yys }
            {1  -\frac{3\ct}{s_0^2} \yys } 
\rb \lp s-s_0 \rp \;,
\eeq
%
where the coefficients $\co$ and $\ct$ take different values for
different orders of the finite-differences term:
$(\co,\ct)=(0.50,0.00)$, $(\co,\ct)=(0.50,0.25)$, etc. When
$(\co,\ct)=(0.00,0.00)$ we obtain the usual split-step Fourier
equation \cite[]{GEO55-04-04100421}.

\item {\bf Fourier finite-differences method:}
\par
%
The coefficients for the Fourier finite-differences solution to \req{rwemixed} are
%
\beqa \label{eqn:rweffd}
\nu &=& \hf\delta_1^2 \;,
\\
\mu &=&    \delta_1   \;,
\\
\ro &=& \qu\delta_2   \;,
\eeqa
where, by definition,
\beqa
\delta_1 &=& a\bas - \ao \baos \;,
\\
\delta_2 &=& a\baf - \ao \baof \;.
\eeqa
%
$\ao$ and $\bo$ are reference values for the medium characterized by
the parameters $a$ and $b$. In the special case of Cartesian
coordinates, $a=s$ and $b=1$, \req{rwemixed} with coefficients
\req{rweffd} takes the familiar form:
%
\beq
\kt \approx \kto + \ww
\lb 1+ \frac{          \frac{\co}{s s_0}   \yys }
            {1-\ct \lp \frac{1}{s^2}   + 
                       \frac{1}{s s_0} + 
                       \frac{1}{s_0^2} \rp \yys } 
\rb \lp s-s_0 \rp \;,
\eeq
%
where the coefficients $\co$ and $\ct$ take different values for
different orders of the finite-differences term:
$(\co,\ct)=(0.50,0.00)$ for $15^\circ$, $(\co,\ct)=(0.50,0.25)$ for
$45^\circ$, etc. When $\co=\ct=0.0$ we obtain the usual split-step
Fourier equation \cite[]{GEO55-04-04100421}.
\end{itemize}
\section{Phase velocity and polarization characteristics}

Following \cite{carcione:2007},
we denote the spatial variables $x$, $y$ and $z$ of a Cartesian system by
the indices $i, j,$ \ldots $=1$, $2$ and $3$, respectively, the position vector by
$\mathbf{x}$, a partial derivative with respect to a variable $x_i$ with $\partial_i$,
and the first and second time derivatives with $\partial_t$ and $\partial_{tt}$.
Matrix transposition is denoted by the superscript $"\top"$. We also denote
the scalar and matrix products by the symbol $"\cdot"$, and the gradient operator by $\bigtriangledown$.
%The Einstein convention of repeated indices is
%assumed unless otherwise specified.

The wave equation in a general heterogeneous anisotropic medium can be expressed as
\begin{equation}
\label{eq:elastic}
\rho{\partial_{tt}\mathbf{u}} = [{\bigtriangledown}{\mathbf{C}{\bigtriangledown}^{\top}}]\mathbf{u} + \mathbf{f},
\end{equation}
where $\mathbf{u}=(u_x,u_y,u_z)^{\top}$ is the particle displacement vector,
$\mathbf{f}=(f_x,f_y,f_z)^{\top}$ represents
 the force term, $\rho$ the density, $\mathbf{C}$ the matrix representing the stiffness tensor in a
 two-index notation called the “Voigt recipe”. The gradient operator has
 the following matrix representation:
\begin{equation}
\label{eq:grad}
\tensor{\bigtriangledown} =
\begin{pmatrix}{\partial}_x &0 &0 &0& {\partial}_z & {\partial}_y \cr
0& {\partial}_y &0 & {\partial}_z  &0 & {\partial}_x \cr
0& 0& {\partial}_z & {\partial}_y & {\partial}_x &0 \end{pmatrix}.
\end{equation} 
Neglecting the source term, a plane-wave analysis of the elastic wave equation yields the 
Christoffel equation,
\begin{equation}
\label{eq:chris1}
\widetilde{\mathbf{\Gamma}}\widetilde{\mathbf{u}} = \rho{\omega}^2\widetilde{\mathbf{u}},
\end{equation}
where $\omega$ is the angular frequency and
$\widetilde{\mathbf{u}}=(\widetilde{u}_x,\widetilde{u}_y,\widetilde{u}_z)^{\top}$
is the wavefield in Fourier domain; the wavenumber-domain counterpart of the gradient
operator is written as
\begin{equation}
\label{eq:wavenumber}
\widetilde{\mathbf{L}}=
\begin{pmatrix}k_x & 0 &0 &0 & k_z & k_y \cr
         0 & k_y & 0 & k_z &0 & k_x \cr
         0 & 0 & k_z & k_y & k_x &0\end{pmatrix},
\end{equation}
in which the propagation direction is specified by the wave vector
$\mathbf{k}=(k_x,k_y,k_z)^{\top}$, 
and the symmetric Christoffel matrix $\widetilde{\Gamma} =
\widetilde{\mathbf{L}}\mathbf{C}\widetilde{\mathbf{L}}^{\top}$ satisfies:
\begin{equation}
\label{eq:gamma}
\widetilde{\Gamma} = 
 \begin{pmatrix} C_{11}{k_x}^2 + C_{66}{k_y}^2 + C_{55}{k_z}^2  &
 (C_{12}+C_{66}){k_x}{k_y}    &(C_{13}+C_{55}){k_x}{k_z} \cr
 (C_{12}+C_{66}){k_x}{k_y} & C_{66}{k_x}^2+C_{22}{k_y}^2+C_{44}{k_z}^2
 &(C_{23}+       C_{44}){k_y}{k_z} \cr
 (C_{13}+C_{55}){k_x}{k_z} & (C_{23}+C_{44}){k_y}{k_z} &
 C_{55}{k_x}^2+C_{44}{k_y}^2+  C_{33}{k_z}^2\end{pmatrix}.
\end{equation}
The squared phase (or normal) velocities ${V^2}_q (q=1,2,3)$ are eigenvalues of the 
Christoffel matrix. The inequalitities
\begin{equation}
\label{eq:v1v2v3}
{V_1(\mathbf{k})}\geq{V_2(\mathbf{k})}\geq{V_2(\mathbf{k})}
\end{equation}
establish the types of waves.
Except for anomalous cases of elastic anisotropy, which are of little interest in geophysics,
the qP-wave ($q=1$) usually is faster than qS-waves ($q=2,3$)
and the equation
\begin{equation}
\label{eq:v1v2}
{V_1(\mathbf{k})}={V_2(\mathbf{k})}
\end{equation}
is a great rarity \cite[]{yu:1993}. However, the equation
\begin{equation}
\label{eq:v2v3}
{V_2(\mathbf{k})}={V_3(\mathbf{k})}
\end{equation}
is a common event because the phase velocity surfaces (corresponding to qS1 and qS2 waves) can touch or intersect each other \cite[]{musgrave}.
Directions of wave normals along which the two phase velocities are equal to each other are called acoustic axes or singularity directions. 
For the shear singularities, the Christoffel matrix is degenerate.
\cite{crampin:1991} distinguished three kinds of singularity: point, kiss and line.
In point and kiss singularities, the phase velocity surfaces touch at a single point, while in line singularities                      
they intersect.
Generally, inserting a nondegenerate eigenvalue back into the Christoffel equation gives ratios of
the components of $\mathbf{u}$, 
which specify polarization along a given phase direction for a given wave mode.
The polarization consists of the geometrical properties of the particle motion, including trajectory shape and spatial orientation,
but excludes magnitudes of the motion.
Polarization of an isolated body wave in a noise-free perfectly elastic medium is linear \mbox{\cite[]{winterstein:1990}}.
In isotropic media, polarizations of such body waves are either parallel to the direction of wave travel, for P-waves,
or perpendicular to it, for S-waves. The polarization vectors of the S-wave may take an arbitrary orientation in the 
plane orthogonal to the P-wave polarization vector.
In anisotropic media, however, the polarizations are often neither parallel nor
perpendicular to the direction of wave propagation.
Whether the medium is anisotropic or not, the polarizations of the three wave modes are always mutually orthogonal for a given propagation direction.
So we may separate the elastic wavefield into single-mode scalar fields using the polarization-based projection:
\begin{equation}
\label{eq:sep}
\widetilde{w}=i\mathbf{a}_w\cdot\widetilde{\mathbf{u}},
\end{equation}
with $\mathbf{a}_w$ representing the normalized polarization vector of the given mode
$w=\{qP, qS_1, qS_2\}$ \cite[]{dellinger.thesis}.

In practice, horizontally polarized (or SH) and vertically polarized (or SV) are
likely to be the most useful S-wave modes when consideration is restricted to isotropic or TI media,
in which all rays lie in symmetry planes.
For isotropic media, the above polarization-based projection is material-independent,
because the following vectors related to the wave vector $\mathbf{k}$, i.e.,
\begin{equation}
\label{eq:e123}
\mathbf{e}_1=(k_x,k_y,k_z)^{\top},\qquad\mathbf{e}_2=(-k_y,k_x,0)^{\top}\qquad\mbox{and}\qquad\mathbf{e}_3 =(k_xk_z,k_yk_z,-(k^2_x+k^2_y))^{\top},
\end{equation}
indicate the polarization direction of pure P-, SH- and SV-wave, respectively. 
On the contrary, for anisotropic media, the polarization-based projection depends on local material parameters,
because the polarizations are generally specified by the eigenvectors of the original Christoffel equation.
For a VTI medium, the stiffness coefficients satisfy: $C_{12}=C_{11}-2C_{66}$, $C_{22}=C_{11}$, $C_{23}=C_{13}$ and $C_{55}=C_{44}$.
In this case, people still prefer to designate the qS-waves as SH-like and SV-like modes due to the following fact:
The use of qS1 and qS2 distinguished by phase velocities does not always give continuous polarization surfaces,
while the use of SV and SH distinguished by polarization does, except at the kiss singularity at $k_x=k_y=0$
\cite[]{crampin.yedlin:1981,zhang.mcmechan:2010}.
In this case, the SH-waves polarize perpendicular to the symmetry plane and are pure,
and the SV-waves polarize in symmetry planes and are usually quasi-shear.
As shown in Figure~\ref{fig:P_last,SV_last,SH_last}, for both qP and qSV modes, the polarization directions in a VTI material
deviate from those in the corresponding isotropic (reference) medium in most propagation directions,
but no deviation exists for the SH mode in any direction for either medium.
To our interest, the polarization deviations of qP- or qSV-waves between an ordinary VTI
medium and its isotropic reference are usually very small, although exceptions are possible \cite[]{thomsen:1986,tsvankin.chesnokov:1990}.
In addition, taking vectors $\mathbf{e}_1$, $\mathbf{e}_2$ and $\mathbf{e}_3$ as the three mutually perpendicular polarization vectors
in the unperturbed isotropic medium, approximate formulas for the qP- and qS-wave polarizations 
in an arbitrary anisotropic medium can be developed using perturbation theory \cite[]{cerveny.jech:1982,psencik:1998,farra:2001}.

\inputdir{Polar3D}
\multiplot{3}{P_last,SV_last,SH_last}{width=0.72\textwidth}
{
Polarization vectors in a 3D VTI material with $v_{p0}=3.0$ km/s,
$v_{s0}=1.5$ km/s, $\epsilon=0.25$, and $\delta=-0.29$. Its isotropic reference
medium is determined by setting $\epsilon=0$ and $\delta=0$.
One can observe polarization deviations between VTI (red) and its isotropic reference (blue) media for (a) P- and (b) S-waves in
most propagation directions, but no deviation for (c) SH-waves in any direction.}

To provide for more possibilities and flexibility in describing single-mode wave propagation in anisotropic media, \mbox{\cite{cheng.kang:2014}} suggest splitting
the one-step polarization-based projection into two steps, of which the first step implicitly implements partial wave-mode separation
during wavefield extrapolation with a transformed wave equation, while the second step is designed to correct the
projection deviation due to the approximation of polarization directions.
The transformed wave equation (that they called a pseudo-pure-mode qP-wave equation) was derived from the original Christoffel equation through
a similarity transformation aiming to project the displacement wavefield onto the isotropic (reference) polarization direction indicated by $\mathbf{e}_1$.  
In this paper, taking another two orthogonal vectors in equation~\ref{eq:e123}, namely $\mathbf{e}_2$ and $\mathbf{e}_3$,
as the reference polarization directions for SH- and qSV-waves,
we apply the same strategy to derive simplified wave equations for these qS-wave modes in general TI media.

%\input{pseudopure.ti.sh}
\section{Pure-mode SH-wave equation}
Pure SH-waves horizontally polarize in the planes perpendicular to the symmetry axis of VTI media with
$u_z\equiv0$, so we introduce a similarity transformation to the Christoffel matrix 
ignoring the vertical component, i.e.,
\begin{equation}
\label{eq:tansChrisMsh}
\widetilde{\overline{\mathbf{\Gamma}}}_\mathbf{m} =
\mathbf{M}\widetilde{\mathbf{\Gamma}}_2\mathbf{M}^{-1},
\end{equation}
with a generally invertible $2\times2$ matrix $\mathbf{M}$ related to the reference polarization direction $\mathbf{e}_2$:
\begin{equation}
\label{eq:tansMsh}
\mathbf{M}=
\begin{pmatrix}-{k_y} & 0 \cr
	0 & {k_x} \end{pmatrix},
\end{equation}
and
\begin{equation}
\label{eq:gamma2}
\widetilde{\Gamma}_2=\begin{pmatrix}C_{11}{k_x}^2+C_{66}{k_y}^2+C_{44}{k_z}^2&
 (C_{12}+C_{66}){k_x}{k_y} \cr
 (C_{12}+C_{66}){k_x}{k_y} & C_{66}{k_x}^2+C_{22}{k_y}^2+C_{44}{k_z}^2\end{pmatrix}.
\end{equation}
Accordingly, we derive a transformed Christoffel equation,
\begin{equation}
\label{eq:tansChrissh}
\widetilde{\overline{\mathbf{\Gamma}}}_\mathbf{m}\widetilde{\overline{\mathbf{u}}}=
\rho{\omega}^2\widetilde{\overline{\mathbf{u}}},
\end{equation}
for the SH-wave mode:
\begin{equation}
\label{eq:similarTsh}
\widetilde{\overline{\mathbf{u}}}=\mathbf{M}\widetilde{\mathbf{u}}_2,
\end{equation}
in which
$\widetilde{\mathbf{u}}_2=(\widetilde{u}_x,\widetilde{u}_y)^{\top}$ represents the horizontal
components of the original elastic wavefields, and
$\widetilde{\overline{\mathbf{u}}}=(\widetilde{\overline{u}}_x,\widetilde{\overline{u}}_y)^{\top}$
represents the horizontal components of the transformed wavefields.
Note that the matrix $\mathbf{M}$ will be not invertible when $k_x=0$ or/and $k_y=0$.
These special directions don't affect the derivation of the pseudo-pure-mode wave equation for the following reasons:
First, we don't directly project the elastic wavefield into the wavenumber-domain, but instead apply the similarity transformation to the Christoffel equation
and eventually inverse the transformed Christoffell equation back into the time-space-domain.
Second, the original Christofell matrix $\widetilde{\Gamma}_2$ automatically becomes a diagonal matrix in these directions,
so the similarity transformation is not actually needed for the corresponding wavenumber components.

Note the similarity transformation does not change the eigenvalue of the Christoffel
matrix corresponding to the SH-wave and, thus, introduces no kinematic error for this wave mode.
We also can obtain a kinematically equivalent Christoffel equation if $\mathbf{M}$ is constructed using 
the normalized form of $\mathbf{e}_2$ to ensure all spatial frequencies are uniformly scaled.
For a locally smooth medium, applying an inverse Fourier transform to
equation~\ref{eq:tansChrissh}, we obtain a linear second-order system in the time-space domain:
\begin{equation}
\label{eq:tansElasticsh}
\rho\partial_{tt}\overline{\mathbf{u}} =
\overline{\mathbf{\Gamma}}_\mathbf{m}\overline{\mathbf{u}},
\end{equation}
or in its extended form:
\begin{equation}
\label{eq:pseudosh}
\begin{split}
	\rho\partial_{tt}\overline{u}_x &= C_{11}\partial_{xx}{\overline{u}_x}
	+ C_{66}\partial_{yy}{\overline{u}_x}+
	C_{44}\partial_{zz}{\overline{u}_x}
	-(C_{11}-C_{66})\partial_{yy}{\overline{u}_y}, \\
	\rho\partial_{tt}\overline{u}_y &= C_{66}\partial_{xx}{\overline{u}_y}
	+ C_{11}\partial_{yy}{\overline{u}_y} +
	C_{44}\partial_{zz}{\overline{u}_y}
	- (C_{11}-C_{66})\partial_{xx}{\overline{u}_x},
\end{split}
\end{equation}
where $\overline{\mathbf{u}}=(\overline{u}_x, \overline{u}_y)^{\top}$
represents the horizontal components of SH-wave in time-space domain, and
$\overline{\mathbf{\Gamma}}_\mathbf{m}$
 represents the Christoffel differential-operator matrix after the similarity transformation.

Due to the cylindrical symmetry of a TI material, the two equations in equation~\ref{eq:pseudosh} may be summed to 
	produce a scalar wave equation in terms of $\overline{u}$:
\begin{equation}
\label{eq:pseudosh1}
\rho\partial_{tt}\overline{u} =
C_{66}(\partial_{xx}+\partial_{yy}){\overline{u}}+C_{44}\partial_{zz}{\overline{u}},
\end{equation}
with $\overline{u}=\overline{u}_{x}+\overline{u}_{y}$ representing the total horizontal components of the transformed SH-wave fields.
This is consistent with the fact that only $C_{44}$ and $C_{66}$ affect the kinematic signatures of the
SH-wave in VTI media \cite[]{tsvankin:2001}. In addition,  the derived equation naturally reduces to the acoustic wave equation if we apply the isotropic assumption by setting $C_{44}=C_{66}=\rho{V_s}^2$ with $V_s$ representing the velocity of the isotropic shear wave.

%\input{pseudopure.ti.sv}
\section{Pseudo-pure-mode qSV-wave equation}

\subsection{Derivation of pseudo-pure-mode qSV-wave equation}
For the qSV-wave, we should essentially build a projection from the elastic wavefields
$\widetilde{\mathbf{u}}=(\widetilde{u}_x,\widetilde{u}_y, \widetilde{u}_z)^{\top}$ to a pseudo-pure-mode wavefield
$\widetilde{\overline{\mathbf{u}}}=(\widetilde{\overline{u}}_x,\widetilde{\overline{u}}_y,
\widetilde{\overline{u}}_z)^{\top}$.
Naturally, we may introduce the following similarity transformation to the Christoffel
matrix, i.e.,
\begin{equation}
\label{eq:tansChrisMsv}
\widetilde{\overline{\mathbf{\Gamma}}}_\mathbf{n}=\mathbf{N}\widetilde{\mathbf{\Gamma}}\mathbf{N}^{-1},
\end{equation}
with the intuitive projection matrix $\mathbf{N}$ defined by the reference polarization direction $\mathbf{e}_3$:
\begin{equation}
\label{eq:tansMsv0}
\mathbf{N}=
\begin{pmatrix}{k_xk_z} & 0 &0 \cr
         0 & {k_yk_z} &0 \cr
         0 & 0 & {-(k^2_x+k^2_y)}\end{pmatrix},
\end{equation}
or its normalized form.
However, the resulting pseudo-pure-mode wave equation is very complicated and contains mixed
derivatives of time and space. To keep them simple, an intermediate
wavefield
${\widetilde{\overline{\mathbf{u}}}}'=(\widetilde{u}_x,\widetilde{u}_y,{\widetilde{u}}'_z)^{\top}$
is defined by
\begin{equation}
\label{eq:zcomp}
\qquad \widetilde{\overline{u}}_z=(k^2_x+k^2_y){\widetilde{\overline{u}}}'_z.
\end{equation}
So we project the vector displacement wavefields using:
\begin{equation}
\label{eq:similarTsv}
{\widetilde{\overline{\mathbf{u}}}}' =
{\mathbf{N}}'\widetilde{\mathbf{u}},
\end{equation}
with an intermediate projection matrix:
\begin{equation}
\label{eq:tansMsv}
{\mathbf{N}}'=
\begin{pmatrix}{k_xk_z} & 0 &0 \cr
         0 & {k_yk_z} &0 \cr
         0 & 0 & -1\end{pmatrix}.
\end{equation}
Accordingly, we apply the similarity transformation using ${\mathbf{N}}'$ to
equation \ref{eq:chris1} and finally get an equivalent Christoffel equation:
\begin{equation}
\label{eq:tansChrissv}
{\widetilde{\overline{\mathbf{\Gamma}}}}'_\mathbf{n}{\widetilde{\overline{\mathbf{u}}}}' =
\rho{\omega}^2{\widetilde{\overline{\mathbf{u}}}}'.
\end{equation}
with
${\widetilde{\overline{\mathbf{\Gamma}}}}'_\mathbf{n}={\mathbf{N}}'\widetilde{\mathbf{\Gamma}}(\mathbf{N}')^{-1}$.

For a locally smooth medium, applying an inverse Fourier transformation to
equation~\ref{eq:tansChrissv}, we obtain another coupled
forth-order linear system:
\begin{equation}
\label{eq:tansElasticsv}
\rho\partial_{tt}{\overline{\mathbf{u}}}'={\overline{\mathbf{\Gamma}}}'_\mathbf{n}{\overline{\mathbf{u}}}',
\end{equation}
or in its extended form:
\begin{equation}
\label{eq:pseudosv}
\begin{split}
	\rho\partial_{tt}\overline{u}_x&=(C_{11}\partial_{xx}+C_{66}\partial_{yy}+ C_{44}\partial_{zz}){\overline{u}_x}
	+(C_{11}-C_{66})\partial_{xx}{\overline{u}_y} 
	+(C_{13}+C_{44})\partial_{xx}\partial_{zz}{{\overline{u}}'_z}, \\
	\rho\partial_{tt}\overline{u}_y&=(C_{11}-C_{66})\partial_{yy}{\overline{u}_x}
	+ (C_{66}\partial_{xx}+C_{11}\partial_{yy}+C_{44}\partial_{zz}){\overline{u}_y}
	+(C_{13}+C_{44})\partial_{yy}\partial_{zz}{{\overline{u}}'_z}, \\
	\rho\partial_{tt}{\overline{u}}'_z&=(C_{13}+C_{44})\overline{u}_x
	+(C_{13}+C_{44})\overline{u}_y
	+C_{44}(\partial_{xx}+\partial_{yy})\overline{u}'_z+C_{33}\partial_{zz}{\overline{u}}'_z.
\end{split}
\end{equation}
where
${\overline{\mathbf{u}}}'=(\overline{u}_x,\overline{u}_y,{\overline{u}}'_z)^{\top}$
is an intermediate wavefield in the time-space domain, and
${\overline{\mathbf{\Gamma}}}'_\mathbf{n}$
 represents the corresponding Christoffel differential-operator matrix after the similarity
 transformation.
 The intermediate wavefield has the same horizontal components but a different vertical
 component of the pseudo-pure-mode wavefield $\overline{\mathbf{u}}=(u_x, u_y,
 u_z)^{\top}$.
 Equation \ref{eq:zcomp} indicates that the vertical component satisfies:
\begin{equation}
\label{eq:zcomp1}
\overline{u}_z=-(\partial_{xx}+\partial_{yy}){\overline{u}}'_z.
\end{equation}

Due to the symmetry property of a VTI material, we may sum the horizontal components and replace the vertical component
with the relation given in equation~\ref{eq:zcomp1}, and finally obtain a simpler second-order system that honors the kinematics of both qP- and qSV-waves:
\begin{equation}
\label{eq:pseudosvxy}
\begin{split}
\rho\partial_{tt}\overline{u}_{xy}&=C_{11}(\partial_{xx}+\partial_{yy})\overline{u}_{xy}+C_{44}\partial_{zz}{\overline{u}}_{xy}-(C_{13}+C_{44})\partial_{zz}{\overline{u}_z},\\
\rho\partial_{tt}\overline{u}_z&=-(C_{13}+C_{44})(\partial_{xx}+\partial_{yy})\overline{u}_{xy}+C_{44}(\partial_{xx}+\partial_{yy})\overline{u}_z+C_{33}\partial_{zz}\overline{u}_z,
\end{split}
\end{equation}
with $\overline{u}_{xy}=\overline{u}_x+\overline{u}_y$.
Note that pure SH-waves always polarize in the planes perpendicular to the symmetry axis with the polarization direction indicated by $\mathbf{e}_2$,
which implies $(k_xk_z)\widetilde{u}_{x}+(k_yk_z)\widetilde{u}_{y}\equiv0$, i.e., $\overline{u}_{xy}\equiv0$, for the SH-wave.
Therefore, the partial summation (after the similarity transformation) automatically removes the SH component from the transformed wavefields.
As a result, there are no terms related to $C_{66}$ any more in equation~\ref{eq:pseudosvxy}.
In order to produce a pseudo-pure-mode scalar qSV-wave field, we sum all components of the transformed wavefields, namely
\begin{equation}
\label{eq:scalalr}
\overline{u}=\overline{u}_{xy}+\overline{u}_{z}.
\end{equation}

For a 2-D VTI medium, equation ~\ref{eq:pseudosvxy} reduces to the following form:
\begin{equation}
\label{eq:pseudosvx}
\begin{split}
\rho\partial_{tt}\overline{u}_x&=C_{11}\partial_{xx}\overline{u}_x+C_{44}\partial_{zz}{\overline{u}}_x
-(C_{13}+C_{44})\partial_{zz}{\overline{u}_z},\\
\rho\partial_{tt}\overline{u}_z&=-(C_{13}+C_{44})\partial_{xx}\overline{u}_x
+C_{44}\partial_{xx}\overline{u}_z+C_{33}\partial_{zz}\overline{u}_z.
\end{split}
\end{equation}
In fact, we can derive the same pseudo-pure-mode wave equation for a 2-D qSV-wave by
projecting the 2-D Christoffel matrix onto a reference vector
$\mathbf{e}'_3=(k_z, -k_x)^{\top}$.
Similarly, a 2D pseudo-pure-mode scalar qSV-wave field is given by the summation: $\overline{u}=\overline{u}_{x}+\overline{u}_{z}$.

If we apply the isotropic assumption by setting $C_{11}=C_{33}$ and $C_{13}+C_{44}=C_{33}-C_{44}$,
and sum the two equations in equation~\ref{eq:pseudosvxy}, we get the scalar wave equation:
\begin{equation}
\rho{\partial_{tt}\overline{u}} = 
C_{44}(\partial_{xx}+\partial_{yy}+\partial_{zz}){\overline{u}},
\end{equation}
with $\overline{u}=\overline{u}_{xy}+\overline{u}_{z}$ representing a shear wave field, and $C_{44}=\rho{V_s}^2$ with $V_s$
representing the propagation velocity of the isotropic shear wave.

The derived pseudo-pure-mode qSV-wave equations have some interesting and valuable features.
First, the projection using matrix $\mathbf{N}$ yields wave-mode separation to some
extent, because the chosen projection direction, $\mathbf{e}_3$, represents the polarization direction of
the SV-wave in an isotropic medium.
As investigated by \mbox{\cite{tsvankin.chesnokov:1990}} and \mbox{\cite{psencik:1998}}, and also demonstrated in Figure 1b,
the difference in polarization directions between isotropic and VTI media is
generally quite small in most propagation directions for SV-waves.
Therefore, considering the mode separator (namely equation~\ref{eq:sep}) and the small projection deviation,
summing all the pseudo-pure-mode wavefield components in equation \ref{eq:pseudosvxy} or \ref{eq:pseudosvx}
partially achieves wave-mode separation and produces a scalar wavefield dominated by
the energy of qSV-waves. This will be demonstrated in the numerical examples.
Second, the pseudo-pure-mode wave equations are easier to calculate
than the original elastic wave equation because they have no terms of mixed partial
derivatives. More importantly, the summation of the horizontal components further simplifies the
wave equations and reduces the number of parameters needed for scalar qSV-wave extrapolation. 
These features are undoubtedly useful for performing multicomponent seismic imaging and
inversion that mainly use wavefield kinematics when it is necessary to include anisotropy.

\subsection{Removing of the residual qP-waves}

The pseudo-pure-mode qSV-wave equations are derived by using a similarity transformation
that projects the vector displacement wavefield onto the isotropic reference of
the qSV-wave's polarization direction. 
As demonstrated in Figure 1b, even for a very strong VTI medium, the difference bewteen the two directions
is generally quite small in most propagation directions.
However, this difference does result in some qP-wave energy remaining in the pseudo-pure-mode
scalar qSV-wave fields. To remove the residual qP-waves, we have to correct the projection deviations before summing the pseudo-pure-mode wavefield components.
 For heterogeneous VTI media, this can be implemented through nonstationary spatial
 filtering defined by the projection deviations \cite[]{cheng.kang:2014}.

The filters can be constructed once the qSV-wave polarization directions
are determined by solving the Christoffel equation based on local medium properties for every grid point.
However, this operation is computationally expensive, especially in 3D heterogeneous TI media.
We may further reduce the computational cost using a mixed-domain integral algorithm
using a low-rank approximation \cite[]{cheng.fomel:2014}.
We shall observe in the examples that the residual qP-waves in the pseudo-pure-mode qSV-wave fields
 are quite weak, even if the anisotropy becomes strong.
As explained in \cite{cheng.kang:2014}, it is not necessary to apply the filtering
 at every time step for many applications, such as RTM.

In the case of transverse isotropy with a tilted symmetry axis (TTI), the elastic tensor loses its simple form
and the terminology ``in-plane polarization" and ``cross-plane polarization" is to be preferred for qSV- and qSH-waves.
The generalization of pseudo-pure-mode wave equation to a TTI medium
involves no additional physics but greatly complicates the algebra. One strategy for deriving the wave
 equations is to locally rotate the coordinate system so that its third axis coincides with
 the symmetry axis; and to make use of the simple form of the wave equation in VTI media (see
 \mbox{\cite{cheng.kang:2014}}).
 Alternatively, we may use some new strategies to derive more numerically stable pseudo-pure-mode wave
 equations for TTI media with strong variations of parameters
 \mbox{\cite[]{zhang:2011,bube:2012}}. Moreover, the filter to correct the projection deviation can
also be constructed with the coordinate rotation. 

%\section{Examples}

% ------------------------------------------------------------
\inputdir{XFig}
% ------------------------------------------------------------
\plot{zokernel}{width=\textwidth} {Schematic representation of the
forward and adjoint operators for ray-based MVA and wave-based
MVA. The forward operator $F$ applied to a slowness anomaly $\ds$
generates a traveltime perturbation (a) or an image perturbation
(b). The ray-based adjoint MVA operator $A$ applied to the traveltime
perturbation generates a slowness perturbation uniformly distributed
along a ray normal to the reflector (c). The wave-based adjoint MVA
operator $A$ applied to the image perturbation generates a slowness
perturbation with a wider space distribution but with a relative focus
at the location of the original slowness anomaly (d).}

We illustrate the wave-equation migration velocity analysis operators
using impulse responses corresponding to different imaging
configurations. We concentrate on imaging in the zero-offset and
shot-record frameworks, since they also implicitly characterize the
essential elements of the survey-sinking framework. In all cases, we
use wavefield reconstruction based on one-way wavefield extrapolation
with the multi-reference split-step Fourier method
\cite[]{GEO55-04-04100421,GEO61-05-14121416}.

% ------------------------------------------------------------
\inputdir{flatWEMVA}
% ------------------------------------------------------------

\multiplot{2}{vel,img}{width=\textwidth}{ Simple synthetic model with
(a) linear $v\lp z \rp$ velocity and (b) a horizontal reflector. }

A fundamental question concerning the wavefield scattering operator
($\SCAT$) is what is its sensitivity for a given perturbation of the
image or of the slowness model. This sensitivity is usually
characterized using the so-called ``sensitivity kernels'' which are
often discussed in the literature in the context of tomography
problems. For wave-equation MVA, this topic was discussed in the
context of zero-offset imaging by
\cite{SavaBiondi.gp.wemva1,SavaBiondi.gp.wemva2}.  The important topic
of sensitivity and model resolution falls outside the scope of this
paper, so we do not discuss it here in any detail. We merely concern
ourselves with describing the behavior of the wave-equation MVA
operators described earlier.

% ------------------------------------------------------------
\multiplot{2}{ds4,di4}{width=\textwidth}{ (a) Slowness perturbations
used to demonstrate the WEMVA operators in
\rFgs{flatWEMVA-ZFds4}-\rfn{flatWEMVA-SAFds4}, and (b) image
perturbation used to demonstrate the WEMVA operators in
\rFgs{flatWEMVA-ZAdi4}-\rfn{flatWEMVA-SFAdi4}.}

\multiplot{2}{Zdat,Sdat}{width=\textwidth}{ (a) Simulated zero-offset
data and (b) simulated shot-record data for the model depicted in
\rfgs{flatWEMVA-vel}-\rfn{flatWEMVA-img} with a source located at
coordinates $x=6$~km and $z=0$~km.}
% ------------------------------------------------------------

We can analyze the sensitivity of the wavefield scattering operator in
two ways. The first option is to assume a localized slowness
perturbation, compute image perturbations using the forward scattering
operator and then return to the slowness perturbation using the
adjoint scattering operator. The second option is to assume a
localized image perturbation, compute the slowness perturbation using
the adjoint scattering operator and then return to the image
perturbation using the forward scattering operator. 

% ------------------------------------------------------------
\multiplot{2}{ZFds4,ZAFds4}{width=\textwidth}{(a) Zero-offset image
perturbation obtained by the application of the forward scattering
operator to the slowness perturbation from \rFg{flatWEMVA-ds4} and (b)
zero-offset slowness perturbation obtained by the application of the
adjoint scattering operator to the image perturbation from panel (a).
}

\multiplot{2}{SFds4,SAFds4}{width=\textwidth}{(a) Shot-record image
perturbation obtained by the application of the forward scattering
operator to the slowness perturbation from \rFg{flatWEMVA-ds4} and (b)
shot-record slowness perturbation obtained by the application of the
adjoint scattering operator to the image perturbation from panel (a).}
% ------------------------------------------------------------

As discussed in the preceding sections, the main difference between
ray-based and wave-based MVA techniques is that the connection between
measurements on the image and updates to the model is done with rays
and waves, respectively. The impact of this fundamental difference is
best seen if we analyze impulse responses of the wave-equation MVA and
compare them with those of conventional traveltime
tomography. \rFg{zokernel} shows a one-to-one comparison between the
forward and adjoint operators for ray-based MVA (traveltime
tomography) on the left and wave-based MVA on the right in the
context of zero-offset imaging.
%
Assuming a small slowness perturbation $\ds$, we can construct using
the forward MVA operators a traveltime perturbation and an image
perturbation corresponding to ray-based MVA (a) and wave-based MVA
(b), respectively. For this zero-offset configuration, the ray-based
MVA produces a traveltime anomaly strictly located on the reflector
under the slowness anomaly, while the wave-based MVA produces an image
anomaly distributed in space in the vicinity of the reflector.
%
Then, we can construct respective slowness updates if we apply the
ray-based and wave-based adjoint MVA operators to the traveltime
perturbation and image perturbation, respectively. For the ray-based
MVA, the slowness update spreads uniformly along a ray orthogonal to
the reflector (c), while for wave-based MVA, the slowness update is
distributed in space from the image perturbation to the surface, but
with a concentration at the location of the true anomaly (d). Similar
behavior characterizes wave-equation MVA under shot-record or
survey-sinking frameworks.

%% 
 % flat horizontal reflector
 %%

% ------------------------------------------------------------
\multiplot{2}{ZAdi4,ZFAdi4}{width=\textwidth} { (a) Zero-offset
slowness perturbation obtained by the application of the adjoint
scattering operator to the image perturbation from \rFg{flatWEMVA-di4}
and (b) zero-offset image perturbation obtained by the application of
the forward scattering operator to the slowness perturbation from
panel (a).  }

\multiplot{2}{SAdi4,SFAdi4}{width=\textwidth} { (a) Shot-record
slowness perturbation obtained by the application of the adjoint
scattering operator to the image perturbation from \rFg{flatWEMVA-ds4}
and (b) shot-record image perturbation obtained by the application of
the forward scattering operator to the slowness perturbation from
panel (a).  }
% ------------------------------------------------------------

The first set of examples corresponds to a simple model consisting of
a linear $v\lp z \rp$ velocity model and a horizontal reflector,
\rFgs{flatWEMVA-vel}-\rfn{flatWEMVA-img}. The velocity is linearly
increasing from $1.5$~km/s to $2.75$~km/s. We simulate zero-offset
data, \rfg{flatWEMVA-Zdat}, and one shot corresponding to horizontal
position $x=6$~km, \rfg{flatWEMVA-Sdat}.

Assuming a localized slowness perturbation, \rfg{flatWEMVA-ds4}, we
can compute image perturbations using the forward scattering
operators, as defined in the preceding sections. \rFg{flatWEMVA-ZFds4}
shows the image perturbation for the zero-offset case and
\rFg{flatWEMVA-SFds4} shows the similar image perturbation for the
shot-record case. As illustrated in \rFg{zokernel}, the image
perturbations are distributed in the vicinity of the reflector. Two
interfering events are seen for the shot-record case, corresponding to
the source and receiver wavefields, respectively.

Similarly, we can compute slowness perturbations using the adjoint
scattering operators. \rFg{flatWEMVA-ZAFds4} shows the slowness
perturbation for the zero-offset case computed from the image
perturbation in \rFg{flatWEMVA-ZFds4} and \rFg{flatWEMVA-SAFds4} shows
the similar slowness perturbation for the shot-record case computed
from the image perturbation in \rFg{flatWEMVA-SFds4}.  As illustrated
in \rFg{zokernel}, the slowness perturbations are distributed in an
area connecting the reflector to the surface, but with a relative
focus at the location of the original anomaly. For the shot-record
case, the back-projection splits toward the source and receivers,
corresponding to the upward continuation of the source and receiver
wavefields.

We can also analyze the wave-equation MVA operator sensitivity in
another way. Assuming a localized image perturbation,
\rFg{flatWEMVA-di4}, we can compute slowness perturbations using the
adjoint scattering operators, as defined in the preceding
sections. \rFg{flatWEMVA-ZAdi4} shows the slowness perturbation for
the zero-offset case and \rFg{flatWEMVA-SAdi4} shows the similar
slowness perturbation for the shot-record case. Here, too, we see
slowness perturbations distributed in an area connecting the reflector
to the surface, but in this case, there is no relative focus of the
anomaly because the image perturbation is strictly localized on the
reflector. For the shot-record case, the back-projection splits toward
the source and receivers, corresponding to the upward continuation of
the source and receiver wavefields. This case corresponds to the case
of practical MVA where measurements of defocusing features are made on
the image itself.

As we have done in the preceding experiment, we can also compute image
perturbations using the forward scattering operators based on the
back-projections created using the adjoint scattering operators.
\rFg{flatWEMVA-ZFAdi4} shows the image perturbation for the
zero-offset case computed from the slowness perturbation in
\rFg{flatWEMVA-ZAdi4} and \rFg{flatWEMVA-SFAdi4} shows the similar
image perturbation for the shot-record case computed from the slowness
perturbation in \rFg{flatWEMVA-SAdi4}. We can observe that the
resulting image perturbations spread beyond the original location,
indicating wider sensitivity of the wave-based MVA kernels to image
perturbations than that of the corresponding ray-based MVA kernels.

%% 
 % Sigsbee 2A
 %%

Similar sensitivity can be observed for the more complex Sigsbee 2A
model \cite[]{SEG-2002-21222125},
\rFgs{saltWEMVA-vel}-\rfn{saltWEMVA-img}. Similarly to the preceding
example, we simulate zero-offset data, \rfg{saltWEMVA-Zdat}, and one
shot corresponding to horizontal position $x=14.6$~km,
\rfg{saltWEMVA-Sdat}.

\rFgs{saltWEMVA-ZFds2} and \rfn{saltWEMVA-SFds2} correspond to the
image perturbations for the slowness anomaly shown in
\rFg{saltWEMVA-ds2}. We can observe image perturbations that spread in
the vicinity of the reflector, similarly to the simpler example
described earlier. The multi-pathing from the source to the reflector
generates the multiple events characterizing the image
perturbations.
%
\rFgs{saltWEMVA-ZAFds2} and \rfn{saltWEMVA-SAFds2} correspond to the
slowness perturbations constructed by applying the zero-offset and
shot-record adjoint scattering operators to the image perturbations
from \rFgs{saltWEMVA-ZFds2} and \rfn{saltWEMVA-SFds2}. We see similar
back-projection patterns to the ones observed in the preceding
example, except that the propagation pattens are more complicated due
to the presence of the salt body in the background model.

\rFgs{saltWEMVA-ZAdi2} and \rfn{saltWEMVA-SAdi2} correspond to the
slowness perturbations for the image anomaly shown in
\rFg{saltWEMVA-di2}. We can observe slowness perturbations that spread
in the vicinity of the reflector, similarly to the simpler example
described earlier.  
%
Finally, \rFgs{saltWEMVA-ZFAdi2} and \rfn{saltWEMVA-SFAdi2} correspond
to the image perturbations for the slowness perturbations constructed
by the adjoint MVA operators shown in \rFgs{saltWEMVA-ZAdi2} and
\rfn{saltWEMVA-SAdi2} for the zero-offset and shot-record cases,
respectively.

% ------------------------------------------------------------
\inputdir{saltWEMVA}
% ------------------------------------------------------------

\multiplot{2}{vel,img}{width=\textwidth}{ (a) Sigsbee 2A synthetic
model and (b) a sub-salt horizontal reflector.  }

% ------------------------------------------------------------
\multiplot{2}{ds2,di2}{width=\textwidth}{ (a) Slowness perturbations
used to demonstrate the WEMVA operators in
\rFgs{saltWEMVA-ZFds2}-\rfn{saltWEMVA-SAFds2}, and (b) image
perturbation used to demonstrate the WEMVA operators in
\rFgs{saltWEMVA-ZAdi2}-\rfn{saltWEMVA-SFAdi2}.}

\multiplot{2}{Zdat,Sdat}{width=\textwidth}{(a) Simulated zero-offset
data and (b) simulated shot-record data for the model depicted in
\rfgs{saltWEMVA-vel}-\rfn{saltWEMVA-img} with a source located at
coordinates $x=14.6$~km and $z=1.52$~km.}
% ------------------------------------------------------------

% ------------------------------------------------------------
\multiplot{2}{ZFds2,ZAFds2}{width=\textwidth}{(a) Zero-offset
image perturbation obtained by the application of the forward
scattering operator to the slowness perturbation from
\rFg{saltWEMVA-ds2} and (b) zero-offset slowness perturbation obtained
by the application of the adjoint scattering operator to the image
perturbation from panel (a).  }

\multiplot{2}{SFds2,SAFds2}{width=\textwidth}{(a) Shot-record
image perturbation obtained by the application of the forward
scattering operator to the slowness perturbation from
\rFg{saltWEMVA-ds2} and (b) shot-record slowness perturbation obtained
by the application of the adjoint scattering operator to the image
perturbation from panel (a).}
% ------------------------------------------------------------

% ------------------------------------------------------------
\multiplot{2}{ZAdi2,ZFAdi2}{width=\textwidth}{(a) Zero-offset slowness
perturbation obtained by the application of the adjoint scattering
operator to the image perturbation from \rFg{saltWEMVA-di2} and (b)
zero-offset image perturbation obtained by the application of the
forward scattering operator to the slowness perturbation from panel
(a).}

\multiplot{2}{SAdi2,SFAdi2}{width=\textwidth}{(a) Shot-record slowness
perturbation obtained by the application of the adjoint scattering
operator to the image perturbation from \rFg{saltWEMVA-ds2} and (b)
shot-record image perturbation obtained by the application of the
forward scattering operator to the slowness perturbation from panel
(a). }
% ------------------------------------------------------------


\section{EXAMPLES}

The first example compares the synthetic elastic displacement and pseudo-pure-mode qSV-wave field
in 2D homogeneous VTI media with different degrees of anisotropy.
Then we demonstrate the computation of 3D synthetic pseudo-pure-mode qSV-wave and pure-mode SH-wave fields for a two-layer VTI model.
Finally, we investigate the performance of the pseudo-pure-mode qSV-wave propagator on the 2D BP TTI model.
We sue a tenth-order explicit finite-difference scheme on regular grids to solve the involved wave equations.
Point sources are located at the centers of the models.

\subsection{2D homogeneous VTI models}

For comparison, we first apply the original elastic wave equation
to synthesize wavefields in a homogeneous VTI medium with weak anisotropy, in which
$v_{p0}=3000 m/s$, $v_{s0}=1500 m/s$, $\epsilon=0.1$, and $\delta=0.05$. 
Figures 2a and 2b display the horizontal and vertical components of the displacement
wavefields at 0.3 s.
Then we try to simulate the propagation of a single-mode qSV-wave using the pseudo-pure-mode qSV-wave equation
(namely equation \ref{eq:pseudosvx}).
 Figure 2c and 2d display the two components of the pseudo-pure-mode qSV-wave fields,
 and Figure 2e displays their summation, i.e., the pseudo-pure-mode scalar qSV-wave
 fields with weak residual qP-wave energy.
We observe that, in most propagation directions, the polarities are almost reversed for the wavefronts of qP-waves
in the two components of the pseudo-pure-mode qSV-wave fields.
This contributes to suppressing the qP-wave energy through summation of the two components. 
Compared with the theoretical wavefront curves (see Figure 2f) calculated
using group velocities
 and angles, pseudo-pure-mode scalar qSV-wave fields have correct kinematics for
  both qP- and qSV-waves.
 We finally remove residual qP-waves by applying the filtering to correct the projection deviation
 and get completely separated scalar qSV-wave fields (Figure 2g).

\inputdir{homovti.eta0.05}
\multiplot{7}{Elasticx,Elasticz,PseudoPureSVx,PseudoPureSVz,PseudoPureSV,WF,PseudoPureSepSV}{width=0.2\textwidth}
{
Synthesized wavefields in a VTI medium with weak anisotropy: (a) x- and
(b) z-components synthesized by original elastic wave equation; (c) x- and
(d) z-components synthesized by pseudo-pure-mode qSV-wave equation; (e) pseudo-pure-mode scalar qSV-wave fields; 
(f) kinematics of qV- and qSV-waves and (g) separated scalar qSV-wave fields.
}

\inputdir{homovti.eta0.5}

Next, we consider wavefield modeling in a homogeneous VTI medium with strong
anisotropy,
in which $v_{p0}=3000 m/s$, $v_{s0}=1500 m/s$, $\epsilon=0.25$, and $\delta=-0.25$.
Figure 3 displays the wavefield snapshots at 0.3 s synthesized both by using the original
elastic wave equation
and the pseudo-pure-mode qSV-wave equation. Note that the pseudo-pure-mode
  qSV-wave equation still accurately
  represents the qP- and qSV-waves' kinematics. Although the residual qP-wave energy
  becomes stronger when
  the degree of anisotropy increases, the filtering step still removes them effectively.
It takes CPU times of 0.15 and 0.06 seconds to extrapolate the elastic and pseudo-pure-mode qSV-wave fields for one time-step, respectively.
But it takes about 8.2 seconds to remove the residual qP-wave from the pseudo-pure-mode qSV-wave fields using wavenumber-domain filtering.
In fact, separating the pure-mode wavefield from the elastic and pseudo-pure-mode wavefields has almost the same computational cost for TI media \cite[]{cheng.kang:2014}.

  \multiplot{7}{Elasticx,Elasticz,PseudoPureSVx,PseudoPureSVz,PseudoPureSV,WF,PseudoPureSepSV}{width=0.2\textwidth}
  {
  The same plots as Figure 2 but for a VTI medium with stronger anisotropy.
   }

%%%%%%%%%%%%%%%%%%%%%%%%%%%%%%%%%%%%%%%%%%%%%%%%%%%%%%%%%%%%%%%%%%%%%%%%%%%%%%%%%%%%%%%%%%%%%%%%%%%%%%
\subsection{3D two-layer VTI model}

\inputdir{twolayer3dvti}

Figure 4 shows an example of simulating the propagation of pseudo-pure-mode qSV-wave fields in a 3D two-layer
 VTI  model (see Figure 4a), with
$v_{p0}=2500 m/s$, $v_{s0}=1200 m/s$, $\epsilon=0.25$, $\delta=-0.25$ and $\gamma=0.3$ in the first layer, and
$v_{p0}=3600 m/s$, $v_{s0}=1800 m/s$, $\epsilon=0.2$, $\delta=0.1$ and $\gamma=0.05$ in the second layer.
We propagate the 3D pseudo-pure-mode qSV-wave fields using equation~\ref{eq:pseudosvxy}.
Figure 4d displays the pseudo-pure-mode scalar qSV-wave fields resulting from the summation of 
the horizontal (Figure 4b) and vertical (Figure 4c) components, namely $\overline{u}_{xy}$ and $\overline{u}_z$.
We see that the qS-waves dominate the scalar wavefields in energy. 
As shown in Figure 5, we also obtain pure-mode scalar SH-wave fields either using the summation of the horizontal components
synthesized by using the pseudo-pure-mode wave equation~\ref{eq:pseudosh} or directly using the scalar wave equation, i.e., equation~\ref{eq:pseudosh1}. 

\multiplot{4}{vp0Interf,PseudoPureSVxyInterf,PseudoPureSVzInterf,PseudoPureSVInterf}{width=0.45\textwidth}
{
Synthesized wavefield snapshots in a 3D two-layer VTI model using equation~\ref{eq:pseudosvxy} : (a) vertical velocity of qSV-wave,
(b) horizontal component $\overline{u}_{xy}$ and (c) vertical component $\overline{u}_z$ of the pseudo-pure-mode qSV-wave fields,
(d) pseudo-pure-mode scalar qSV-wave fields. The dash line indicates the interface.
}

\multiplot{3}{SHxInterf,SHyInterf,SHInterf}{width=0.45\textwidth}
%\plot{SH}{width=0.45\textwidth}
{
Synthesized wavefield snapshots in a 3D two-layer VTI model using equation~\ref{eq:pseudosh}: (a) x- and (b) y-components
of the pseudo-pure-mode wavefields, (c) pure-mode scalar SH-wave fields calculated as the summation of the two horizontal 
components of the pseudo-pure-mode wavefields. Note that the same scalar wavefields are obtained if we directly
use the scalar wave equation for SH-waves, namely equation~\ref{eq:pseudosh1}.
}

\subsection{BP 2007 TTI model}

Finally, we demonstrate pseudo-pure-mode qSV-wave propagation in the 2D BP TTI model (see Figure 6).
The space grid size is 12.5 m and the time step is 1 ms for
high-order finite-difference operators. Here the vertical velocities for the qSV-wave are set to
 half of the qP-wave velocities.
Figure 7 displays snapshots of wavefield components at the time of 1.4s
synthesized by using the original elastic wave equation and the pseudo-pure-mode qSV-wave equation.
In the elastic wavefields, we observe strong scattered and mode-converted energy in the region with a rapidly varying 
anisotropic symmetry axis direction.
For comparison, in Figures 7c and 7d, we also show the separated qP- and qSV-wave scalar fields
	obtained using the approach proposed by \cite{cheng.fomel:2014}.
Not that in the pseudo-pure-mode qSV-wave fields (see Figure 7g), the incident qP-waves as well as
scattered and converted qP-waves are effectively suppressed.
The spatial filtering appears to remove residual qP-waves and accurately separates qSV-wave data
(including the converted qP-qSV waves) from the pseudo-pure-mode wavefields in this complex model (Figure 7h).
Vertical slices through the scalar fields (Figure 8) provide further proof to evaluate the performance
of the proposed qS-wave propagators.
As we observed, in heterogeneous rough zones with strong variations in tilt angle,
there are differences between the elastic and pseudo-pure-mode qSV-wave fields.
Fortunately, the pseudo-pure-mode qSV-wave equation still captures the shear wave kinematics to a great extent.
For a single time-step, it respectively takes CPU times of 2.71 and 1.22 seconds to extrapolate the elastic
and pseudo-pure-mode qSV-wave fields,
and about 7.50 seconds to separate the qSV-wave fields from both wavefields using low-rank approximate
mixed-domain integral operations based on the qSV-wave's polarization directions \cite[]{cheng.fomel:2014}.

\inputdir{bptti2007}
\multiplot{4}{vp0,epsi,del,the}{width=0.3\textwidth}
{
Partial region of the 2D BP TTI model: (a) vertical qP-wave velocity, Thomsen coefficients
(b) $\epsilon$ and (c) $\delta$, and (d) the tilt angle $\theta$. 
}

\multiplot{8}{Elasticx,Elasticz,ElasticSepP,ElasticSepSV,PseudoPureSVx,PseudoPureSVz,PseudoPureSV,PseudoPureSepSV}{width=0.25\textwidth}
{
Synthesized wavefield snapshots on BP 2007 TTI model using original elastic wave equation and pseudo-pure-mode 
qSV-wave equation respectively: 
(a) x- and (b) z-components synthesized by elastic wave equation; 
(c) scalar qP- and (d) scalar qSV-wave fields separated from the elastic wavefield; 
(e) x- and (f) z-components synthesized by pseudo-pure-mode qSV-wave equation; 
(g) pseudo-pure-mode scalar qSV-wave field and (h) pure-mode scalar qSV-wave field separated from the pseudo-pure-mode qSV-wave field.
}
\multiplot*{3}{ElasticSepSVwave,PseudoPureSVwave,PseudoPureSepSVwave}{width=0.45\textwidth}
{
Vertical slices through the scalar wavefields at $x=50.5$ km in Figure 7: (a) qSV-wave separated from the elastic wavefield;
 (b) pseudo-pure-mode qSV-wave; (c) pure qSV-wave separated from the pseudo-pure-mode wavefield.
}


%\def\EOP{\textbf{extrapolation operator}}
\def\IOP{\textbf{imaging operator}}
\def\SOP{\textbf{scattering operator}}

\def\EOPzo{\textit{ \reqs{PHS-ZO}, \ren{SSR-ZO} }}
\def\EOPsr{\textit{ \reqs{PHSs-SR},\ren{PHSr-SR},\ren{SSR-SR} }}
\def\EOPss{\textit{ \reqs{PHS-SS}, \ren{DSR} }}

\def\IOPzo{\textit{ \req{IMC-ZO} }}
\def\IOPsr{\textit{ \req{IMC-SR} }}
\def\IOPss{\textit{ \req{IMC-SS} }}

\def\SOPzo{\textit{ \reqs{ZOFSOP}, \ren{ZOASOP} }}
\def\SOPsr{\textit{ \reqs{SRFSOPs},\ren{SRFSOPr},\ren{SRASOPs},\ren{SRASOPr} }}
\def\SOPss{\textit{ \reqs{SSFSOP}, \ren{SSASOPs},\ren{SSASOPr} }}

% ------------------------------------------------------------
\section{Summary of operators}

All wave-equation migration velocity analysis operators described in
the preceding sections are similar in that they relate perturbations
of the image with perturbations of the (slowness) model. In all cases,
this velocity estimation procedure takes advantage of features of
migrated images which indicate incorrect imaging. The imaging
inaccuracies can have several causes, i.e. incorrect downward
continuation, irregular illumination, limited acquisition aperture
etc., but the wave-equation MVA operators translate all inaccuracies
in model updates. This feature, however, is a fundamental limitation
of all migration velocity analysis techniques and we do not expand on
this topic further.

Since the migration velocity analysis operators link image
perturbations with slowness perturbations, they are all composed of
several common parts, but with implementations that are specific for
each imaging configuration. Thus, a wave-equation MVA operator is
composed of an extrapolation operator (for wavefield reconstruction
from recorded data), an imaging operator (for image construction from
reconstructed wavefields) and a scattering operator (for relating
wavefield perturbations to slowness perturbations). The following
table summarizes the wave-equation MVA operator components in
different imaging configurations, as described in detail in the
preceding sections.

% ------------------------------------------------------------
\begin{center}
\tiny
%\begin{singlespace}
\begin{table*}
\begin{tabular} { c || ccc | c | c | } 
                        & \multicolumn{3}{c|}{\EOP}                    & \IOP                       & \SOP           \\
\cline{2-6}
                        & equation & velocity & extrapolation     & type                       & equation       \\ \hline \hline
\textbf{zero-offset}    & SSR      & half     & backward               & zero time                  & linearized SSR \\
\cline{2-6}
                        & \multicolumn{3}{c|}{\EOPzo}                  & \IOPzo                     & \SOPzo         \\ \hline \hline
\textbf{survey-sinking} & DSR      & full     & backward               & zero time                  & linearized DSR \\
                        &          &          &                        & implicit cross-corr. &                \\ 
                        & \multicolumn{3}{c|}{\EOPss}                  & \IOPss                     & \SOPss         \\ \hline \hline
\textbf{shot-record}    & SSR      & full     & forward       & zero time                  & linearized SSR \\
                        &          &          & backward     & explicit cross-corr. &                \\
\cline{2-6}
                        & \multicolumn{3}{c|}{\EOPsr}                  & \IOPsr                     & \SOPsr         \\
\end{tabular}
%\end{singlespace}
\end{table*}
\end{center}
% ------------------------------------------------------------


\section{Discussion}

\subsection{Kinematic and dynamic accuracy}
The similarity transformation to the Christoffel equation preserves the kinematics of the qS-waves,
but inevitably change the phases and amplitudes in their wavefields.
Accordingly, the pseudo-pure-mode wave equations may change the radiation from a point source 
(as demonstrated in the examples),
and even distort the amplitude variation with offset (AVO).
In other word, they do not honor the dynamic elasticity of the waves in real media.
In fact, other simplified forms of the elastic wave equation, such as acoustic or pseudo-acoustic wave equations 
and the pure-mode approximate wave equations, have similar limitations
\cite[]{barnes:2009,operto:2009,cheng.kang:2014,shang:2015}.
For heterogeneous rough media, i.e., when scales for variations in the elastic parameters are small compared with the wavelengths
	of the wavefield, the acoustic approximation is no longer reliable \cite[]{cance:2015}.
The pseudo-pure-mode wave equations have similar limitations for shear-wave modeling in high-contrast TI media. 
However, these limitations are not doomed to be catastrophic, because velocity models 
containing high-wavenumber components are rarely involved at most stages of seismic imaging and inversion for real data.

\subsection{Challenge for anisotropy with lower symmetry}
Unlike the well-behaved qP-wave mode, the qS-wave modes do not consistently polarize as a function of propagation direction,
and thus cannot be designated as SV- and SH-waves, except in isotropic and TI media \cite[]{winterstein:1990}.
To demonstrate the difficulties of extending the methodology in this paper to anisotropic media with symmetry lower than TI,
we first compare the polarization features of qS-waves in typical TI and orthorhombic anisotropic rocks.
Figure 9 shows polarizations of qS1- and qS2-waves in a VTI material - Mesaverde shale \cite[]{thomsen:1986},
which has the parameters $v_{p0}=3.749$ km/s, $v_{s0}=2.621$ km/s, $\epsilon=0.225$, $\delta=0.078$, and $\gamma=0.100$.
The polarization directions are either horizontal or vertical (in the symmetry plane), 
so that we can definitely designate qS-waves as qSV- and SH-wave modes, except at the kiss singularity.
Figure 10 shows polarizations of qS1- and qS2-waves in a ``standard" orthorhombic anisotropic material - vertically fractured shale 
\cite[]{Schoenberg:1997}, which has the parameters $v_{p0}=2.437$, $v_{s0}=1.265$ km/s, $\epsilon_1=0.329$,
$\epsilon_2=0.258$, $\delta_1=0.083$, $\delta_2=-0.078$, $\delta_3=-0.106$, $\gamma_1=0.182$ and $\gamma_2=0.0455$.
The qS-waves polarize in a very complicated way and have point singularities in many propagation directions.

Shear-wave modeling is complicated by the presence of the shear-wave singularities.
As investigated by \cite{crampin.yedlin:1981},
a TI material only has line and kiss singularities, while other anisotropic materials excluding those with triclinic symmetry
(e.g., orthorhombic and monoclinic anisotropic materials) have point singularities in many propagation directions.
Line singularities occur only at a fixed angle from the symmetry axis and cause
no distortion of phase velocity surfaces or polarization phenomena.
For kiss singularities (along the direction of symmetry axis), 
qS-wave polarizations vary rapidly in their vicinity but are well-behaved because there is no distortion in phase-velocity surfaces.
These features facilitate the derivations of pseudo-pure-mode qSV-wave and pure-mode SH-wave equations for TI media.
For directions near point singularities, however, the polarization of plane qS-waves changes very rapidly,
and amplitudes and polarizations of qS-waves with curved wavefronts behave quite anomalously.
Therefore, although pseudo-pure-mode qP-wave equations exist for general anisotropic media, 
it may be more confusing than helpful to extend the proposed pseudo-pure-mode qS-wave equations
to symmetry systems lower than TI.

\inputdir{PolarizationMatlab}
\multiplot{2}{polar3dvtiS1,polar3dvtiS2}{width=0.5\textwidth}
{
Polarization vectors of 3D qS-waves in a VTI material (Mesaverde shale): (a) qS1-wave; (b) qS2-wave.
}

\multiplot{2}{polar3dortS1,polar3dortS2}{width=0.5\textwidth}
{
Polarization vectors of 3D qS-waves in an orthorhombic anisotropic material: (a) qS1-wave; (b) qS2-wave.
}

%
\section{Conclusions}
% principles, relationships, generalizations inferred from the results
I present a method of obtaining spatially-varying pseudo derivative
operators with application to wave mode separation in anisotropic
media.  The main idea is to utilize polarization vectors constructed
in the wavenumber domain using the local medium parameters and then
transform these vectors back to the space domain.
% agreements or disagreements with previously published work
The main advantage of applying the pseudo derivative operators in the
space domain constructed in this way is that they are suitable for
heterogeneous media.
% exceptions, problems
%Although the acquired pseudo derivative operators may present minor problems at
%sharp reflection boundaries as can be seen in the separation examples,
% implications and practical applications
The wave mode separators obtained using this method are
spatially-variable filtering operators and can be used to separate
wavefields in VTI media with an arbitrary degree of anisotropy. This
methodology is applicable for elastic RTM in heterogeneous anisotropic
media.
% conclusions 
% ------------------------------------------------------------

\section{Conclusions}
By applying two different similarity transformations to the original Christoffel equation, which aim to project the vector displacement
wavefields onto the isotropic SV- and SH-waves' polarization directions, we have derived the pseudo-pure-mode qSV-wave equation
and the pure-mode SH-wave equation for 2D and 3D heterogeneous TI media, respectively.
These equations are simpler than the original elastic wave equation and involve less material parameters, which reduces computational
cost at least by half if the finite-difference scheme is used in practice.
The theoretical analysis and numerical examples have demonstrated that, the pseudo-pure-mode
qS-wave propagators for TI media have the following features:
First, the qSV-wave equations honor the kinematics for both qP and qSV modes, while the pure-mode SH-wave equation guarantees
the kinematics for the scalar SH-wave.
Second, although qP-waves still remain in the pseudo-pure-mode qSV-wave fields, their horizontal and vertical components have almost
opposite polarities in most propagation directions. As a result, the summation of all components
produces a pseudo-pure-mode scalar qSV-wave field with very weak qP-wave energy. 
Third, the non-SH parts in the pseudo-pure-mode vector SH-wave field have completely opposite polarities,
and thus are thoroughly removed from the scalar SH-wave field once all components are summed.
In addition, 
a filtering step taking into account the polarization deviation can be used to thoroughly remove the residual qP-waves
for pseudo-pure-mode scalar qSV-wave extrapolation.
These features indicate the potential of the proposed qS-wave propagators for developing promising seismic imaging and
inversion algorithms in heterogeneous TI media.
Like the pseudo-acoustic or pseudo-pure-mode qP-wave equations,
the proposed pseudo-pure-mode qS-wave equations take into account ``scalar anisotropy" and may distort the dynamic elasticity
of the real anisotropic media.

%\section{ACKNOWLEDGMENTS}
The research leading to this paper was supported by the National Natural Science Foundation of China (No.41074083)
and the Fundamental Research Funds for the Central Universities (No.1350219123).
We would like to thank Sergey Fomel, Paul Fowler, Yu Zhang, Tengfei Wang and Chenlong Wang
for helpful discussions in the later period of this study.
Constructive comments by Joe Dellinger, Faqi Liu, Mirko van der Baan, Reynam Pestana, and an anonymous reviewer
are much appreciated. We thank SEG, HESS Corporation and BP for making the 2D VTI and TTI synthetic data sets available,
and the authors of \emph{Madagascar} for providing this
software platform for reproducible computational experiments.

\section{ACKNOWLEDGMENTS}
Thanks to the support of the National Natural Science Foundation of China (No.41474099)
and Shanghai Natural Science Foundation (No.14ZR1442900). 
We appreciate Sergey Fomel, Paul Fowler, Chenlong Wang and Tengfei Wang
for helpful discussions in the later period of this study.
Constructive comments by Richard Bale, Alexey Stovas, Joe Dellinger and an anonymous reviewer
are much appreciated.
We thank SEG and BP for making the 2D TTI model available,
and the authors and developers of \emph{Madagascar} for providing resources
for reproducible computational experiments.

%\newpage
%\twocolumn
\bibliographystyle{seg}
\bibliography{reference}

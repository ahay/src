%\def\SEPCLASSLIB{../../../../sepclasslib}
\def\CAKEDIR{.}

\title{Waves in strata}
\author{Jon Claerbout}
\label{paper:wvs}
\maketitle

\def\RMS{{{\sc rms}}}   % Global RMS.  Always type this one, even in equations.
                        % Then we can change it to suit final editor.
                        % This definition dies in equations.

\def\RMS{{\rm RMS}}     % Global RMS.  Always type this one, even in equations.

The waves of practical interest in reflection seismology are usually complicated
because the propagation velocities are generally complex.
In this book, we have chosen to build up the complexity of the waves we
consider, chapter by chapter.  The simplest waves to understand are simple
plane waves and spherical waves propagating through a constant-velocity medium.
In seismology however, the earth's velocity is almost never well approximated
by a constant.
A good first approximation is to assume that the earth's 
velocity increases with depth.
In this situation, the simple planar and circular wavefronts 
are modified by the effects of $v(z)$.
In this chapter we study the basic equations describing plane-like 
and spherical-like waves propagating 
in media where the velocity $v(z)$ is a function only of depth.
This is a reasonable starting point,
even though it neglects the even more complicated distortions that occur
when there are lateral velocity variations.
We will also examine data that shows plane-like waves
and spherical-like waves resulting when waves
from a point source bounce back from a planar reflector.

\section{TRAVEL-TIME DEPTH}
\par
Echo soundings give us a picture of the earth.
A zero-offest section, for example,
is a planar display of traces where the horizontal axis runs along 
the earth's surface
and the vertical axis, running down, seems to measure depth,
but actually measures the two-way echo delay time.
Thus, in practice the vertical axis is almost never depth $z$;
it is the
{\em 
vertical travel time
}
$ \tau $.
In a constant-velocity earth
the time and the depth are related
by a simple scale factor, the speed of sound.  This is analogous to the
way that astronomers measure distances in light-years, always referencing
the speed of light.
The meaning of the scale factor in seismic imaging 
is that the  $(x, \tau$)-plane
has a vertical exaggeration compared to the  $(x,z)$-plane.
In reconnaissance work,
the vertical is often exaggerated by about a factor of five.
By the time prospects have been sufficiently narrowed for a drill
site to be selected,
the vertical exaggeration factor in use is likely to be about unity
(no exaggeration).

\todo{
[JON, THIS IS PROBABLY TRUE BUT IT IS NOT CONVINCING
UNLESS WE SPECIFY THE
DISPLAY SCALES THAT ARE TYPICALLY USED: KM/INCH, SEC/IN.  SEE MY 
RELATED COMMENTS BELOW THE NEXT PARAGRAPH]  [IF WE WISH TO
RETAIN THIS ``EXAGGERATION'' MATERIAL, WHICH I AM INCLINED TO DO, THEN
WE REALLY NEED A FIGURE THAT GRAPHICALLY ILLUSTRATES THE PHENOMENON WITH
ACTUAL DISPLAY SCALES AND AN ACTUAL FUNCTION v(z)].  THE READER SHOULD
BE ABLE TO ``PLAY WITH'' THE DISPLAY PARAMETERS AND WITH v(z) AND SEE
HOW THE EXAGGERATION CHANGES.  WE COULD GENERATE A SET OF SYNTHETIC HORIZONTAL
REFLECTORS AT DEPTHS OF 1.0, 2.0, 3.0 KM, FOR EXAMPLE
}

\par
In seismic reflection imaging, the waves go down and then up,
so the \bx{traveltime depth} $\tau$
is defined as two-way vertical travel time.
\begin{equation}
\label{eqn:twowaytau}
                \tau \eq  {2\,z  \over  v }  \ \ .
\end{equation}
This is the convention that I have chosen to use throughout this book.

\subsection{Vertical exaggeration}
\par
The first task in interpretation of seismic data
is to figure out the approximate numerical value
of the \bx{vertical exaggeration}.
The vertical exaggeration is $2/v$ because it is the ratio of
the apparent slope
$\Delta \tau / \Delta x$
to the actual slope
$\Delta z/ \Delta x$
where $\Delta \tau = 2\ \Delta z / v$.
Since
the velocity generally
{\em  increases} with depth,
the \bx{vertical exaggeration} generally
{\em  decreases} with depth.

\par
For velocity-stratified media,
the time-to-depth conversion formula is
\begin{equation}
\tau (z) \ \ =\ \  \int_0^z \  {2\ dz \over v (z) }
\ \ \ \ \ \ \ \ \  {\rm or} \ \ \ \ \ \ \ \ 
{ d \tau   \over dz }\ \ =\ \  {2 \over v }
\label{eqn:3.2}
\end{equation}

\todo { [JON, IF WE WANT TO
RETAIN THIS ``EXAGGERATION'' MATERIAL, WHICH I REALLY THINK WE SHOULD, THEN
WE MUST HAVE A FIGURE THAT GRAPHICALLY ILLUSTRATES THE PHENOMENON WITH
ACTUAL DISPLAY SCALES AND AN ACTUAL FUNCTION v(z)].  THE READER SHOULD
BE ABLE TO ``PLAY WITH'' THE DISPLAY PARAMETERS AND WITH v(z) AND SEE
HOW THE EXAGGERATION CHANGES.  WE COULD GENERATE A SET OF SYNTHETIC HORIZONTAL
REFLECTORS AT DEPTHS OF 1.0, 2.0, 3.0 KM, FOR EXAMPLE
}

\section{HORIZONTALLY MOVING WAVES}
\inputdir{headray}
In practice, horizontally going waves are easy to recognize
because their travel time is a linear function
of the offset distance between shot and receiver.
There are two kinds of horizontally going waves,
one where the traveltime line goes through the origin,
and the other where it does not.
When the line goes through the origin,
it means the ray path is always near the earth's surface
where the sound source and the receivers are located.
(Such waves are called ``\bx{ground roll}'' on land
or ``\bx{guided wave}s'' at sea;
sometimes they are just called ``\bx{direct arrivals}''.)

\par
When the traveltime line does not pass through the origin
it means parts of the ray path plunge into the earth.
This is usually explained by
the unlikely looking rays shown in Figure~\ref{fig:headray}
which frequently occur in practice.%
\sideplot{headray}{width=3in,height=.8in}{
  Rays associated with \bx{head wave}s.
} %
Later in this chapter we will see that Snell's law
predicts these rays in a model of the earth with two layers,
where the deeper layer is faster and the ray bottom
is along the interface between the slow medium and the fast medium.
Meanwhile, however, notice that these ray paths
imply data with a linear travel time versus distance
corresponding to increasing ray length along the ray bottom.
Where the ray is horizontal in the lower medium,
its wavefronts are vertical.
These waves are called ``\bx{head wave}s,''
perhaps because they are typically fast
and arrive {\em  ahead} of other waves.

\subsection{Amplitudes}
\par
The nearly vertically-propagating waves (reflections)
spread out essentially in three dimensions,
whereas the nearly horizontally-going waves
never get deep into the earth because,
as we will see,
they are deflected back upward by the velocity gradient.
Thus horizontal waves spread out in essentially two dimensions, 
so that energy conservation suggests
that their amplitudes should dominate the amplitudes of reflections
on raw data.
This is often true for \bx{ground roll}.
Head waves, on the other hand,
are often much weaker, often being visible only because
they often arrive before more energetic waves.
The weakness of \bx{head wave}s
is explained by
the small percentage of solid angle occupied by the waves leaving a source
that eventually happen to match up with layer boundaries and propagate as
head waves.
I %\footnote{
  %      In this chapter, and perhaps throughout the book,
  %      the word ``I'' refers to the two authors working
  %      in consultation, and the word ``we'' refers
  %      to readers and authors.
  %      }
selected the examples below because of the strong headwaves.
They are nearly as strong as the guided waves.
To compensate for diminishing energy with distance,
I scaled data displays by multiplying by the offset distance
between the shot and the receiver.
\par
In data display, the slowness (slope of the time-distance curve)
is often called the \bx{stepout}  $p$.  Other commonly-used names for
this slope are \bx{time dip} and \bx{reflection slope}.
The best way to view waves with \bx{linear moveout}
is after time shifting to remove a standard linear moveout
such as that of water.
An equation for the shifted time is
\begin{equation}
\tau \eq t - p x
\label{eqn:lmo}
\end{equation}
where $p$ is often chosen to be the inverse of the velocity of water,
namely, about 1.5 km/s, or $p=.66 {\rm s/km}$
and $x=2h$ is the horizontal separation between
the sound source and receiver, usually referred to as the \bx{offset}.

\inputdir{head}
\par
\bxbx{Ground roll}{ground roll}
and \bx{guided wave}s are typically slow
because materials near the earth's surface typically are slow.
Slow waves are steeply sloped on a time-versus-offset display.
It is not surprising that marine guided waves
typically have speeds comparable to water waves
(near 1.47 km/s approximately 1.5 km/s).
It is perhaps surprising that \bx{ground roll}
also often has the speed of sound in water.
Indeed, the depth to underground water is often determined
by seismology before drilling for water.
Ground roll also often has a speed
comparable to the speed of sound in air,
0.3 km/sec, though, much to my annoyance I could not find
a good example of it today.
%(Record wz.25 seems to have both water speed and air speed head waves,
%but the given parameters imply two speeds exactly twice as fast
%as those two speeds and I am suspicious.)
Figure~\ref{fig:wzl-34} is an example of energetic \bx{ground roll} (land)
that happens to have a speed close to that of water.
\plot{wzl-34}{width=6.00in,height=3.0in}{
  Land shot profile (Yilmaz and Cumro) \#39 from the Middle East
  before (left) and after (right)
  linear moveout at water velocity.
}

\par
The speed of a ray traveling along a layer interface
is the rock speed in the faster layer (nearly always the lower layer).
It is not an average of the layer above and the layer below.

% The discussion below is based on a JLB fantasy figure.
%  I doubt this discussion belongs here.  Meanwhile, I commented it out. -Jon
%Rays are not directly observable; they are theoretical.
%To better understand what is happening, think of 
%the wavefronts for the \bx{head wave} in Figure~\FIG{horzrays}.
%A ray along an interface is perpendicular
%to a vertical wavefront (horizontal ray) in the lower medium
%and a wavefront dipping at angle $\theta_c$ in the upper medium.
%So the horizontal slowness in the upper medium,
%$\sin\theta_c /v_1$, matches the horizontal slowness $1/v_2$
%of the front in the lower medium.  This fact, known as \bx{Snell's law}
%will be proven in Section~{dipping} of this chapter.

\par
Figures~\ref{fig:wzl-20} and~\ref{fig:wzl-32}
are examples of energetic marine guided waves.
In Figure~\ref{fig:wzl-20}
at $\tau=0$ (designated {\tt t-t\_water}) at small offset
is the wave that travels directly from the shot to the receivers.
This wave dies out rapidly with offset
(because it interferes with a wave of opposite polarity
reflected from the water surface).
At near offset slightly later than $\tau=0$ is the water bottom reflection.
At wide offset, the water bottom reflection is quickly followed 
by multiple reflections from the bottom.
Critical angle reflection is defined as where the \bx{head wave}
comes tangent to the reflected wave.
Before (above) $\tau=0$ are the \bx{head wave}s.
There are two obvious slopes,
hence two obvious layer interfaces.
Figure \ref{fig:wzl-32} is much like Figure \ref{fig:wzl-20}
but the water bottom is shallower.

\plot{wzl-20}{width=6.00in,height=3.0in}{
  Marine shot profile (Yilmaz and Cumro) \#20 from the Aleutian Islands.
}
\plot{wzl-32}{width=6.00in,height=3.0in}{
  Marine shot profile (Yilmaz and Cumro) \#32 from the North Sea.
}

\par
Figure~\ref{fig:wglmo} shows data where the first arriving energy
is not along a few straight line segments,
but is along a curve.
This means the velocity increases smoothly with depth
as soft sediments compress.
\plot{wglmo}{width=6.00in,height=2.7in}{
  A common midpoint gather from the Gulf of Mexico
  before (left) and after (right) linear moveout
  at water velocity.
  Later I hope to estimate
  velocity with depth
  in shallow strata.
  Press button for \bx{movie} over midpoint.
}

\subsection{LMO by nearest-neighbor interpolation}
To do \bx{linear moveout} (\bx{LMO}) correction, we need to time-shift data.
Shifting data requires us to interpolate it.
The easiest interpolation method is the nearest-neighbor method.
We begin with a signal given at times {\tt t = t0+dt*it}
where {\tt it} is an integer.
Then we can use equation~(\ref{eqn:lmo}),
namely $\tau=t-px$.
Given the location {\tt tau} of the desired value
we backsolve for an integer, say {\tt itau}.
In C, conversion of a real value to an integer is done by
truncating the fractional part of the real value.
To get rounding up as well as down,
we add
{\tt 0.5}
before conversion to an integer,
namely {\tt itau=0.5+(tau-tau0)/dt}.
This gives the nearest neighbor.
The
way the program works is to identify two points,
one in $(t,x)$-space and one in $(\tau,x)$-space.
Then
the data value at one point in one space is carried to the other.
The adjoint operation copies $\tau$ space back to $t$ space.
%The subroutine used in the illustrations above is
%\texttt{lmo()} \vpageref{lst:lmo}
%with {\tt adj=1}.
%\progdex{lmo}{linear moveout}

\par
Nearest neighbor rounding is crude but ordinarily very reliable.
I discovered a very rare numerical roundoff problem
peculiar to signal time-shifting, a problem
which arises in the linear moveout application
when the water velocity, about 1.48~km/sec is approximated by 1.5=3/2.
The problem arises only where the amount of the time shift
is a numerical value (like 12.5000001 or 12.499999)
and the fractional part should be exactly 1/2 but
numerical rounding pushes it randomly in either direction.
We would not care if an entire signal was shifted
by either 12 units or by 13 units.
What is troublesome, however, is if some random portion
of the signal shifts 12 units while the rest of it shifts 13 units.
Then the output signal has places which are empty while
adjacent places contain the sum of two values.
Linear moveout is the only application
where I have ever encountered this difficulty.
%A simple fix here was to modify the
%\texttt{lmo()} \vpageref{lst:lmo}
%subroutine changing
%the ``1.5'' to ``1.5001''.
The problem disappears if we use a more accurate sound velocity
or if we switch from nearest-neighbor interpolation
to linear interpolation.

\subsection{Muting}
\inputdir{vscan}
Surface waves are a mathematician's delight
because they exhibit many complex phenomena.
Since these waves are often extremely strong,
and since the information they contain about the earth
refers only to the shallowest layers,
typically,
considerable effort is applied to array design in field recording
to suppress these waves.
Nevertheless, in many areas of the earth,
these pesky waves may totally dominate the data.

\par
A simple method to suppress \bx{ground roll} in data processing
is to multiply a strip of data by a near-zero weight (the mute).
To reduce truncation artifacts,
the mute should taper smoothly to zero (or some small value).
Because of the extreme variability from place to place
on the earth's surface,
there are many different philosophies
about designing mutes.
Some mute programs use a data dependent weighting function
(such as automatic gain control).
Subroutine \texttt{mutter()} \vpageref{lst:mutter},
however, operates on a simpler idea: 
the user supplies trajectories defining the mute zone.
\moddex{mutter}{mute}{44}{71}{system/seismic}

\par
Figure~\ref{fig:mutter} shows an example of use
of the routine \texttt{mutter()} \vpageref{lst:mutter} on the shallow water data
shown in Figure~\ref{fig:wglmo}.
\plot{mutter}{width=6.00in,height=2.7in}{
  Jim's first gather before and after muting.
}

\section{DIPPING WAVES}
%\sectionlabel{dipping}

Above we considered waves going vertically
and waves going horizontally.
Now let us consider waves propagating at the intermediate angles.  
For the sake of definiteness, I have chosen
to consider only downgoing waves in this section.
We will later use the concepts developed
here to handle both downgoing and upcoming waves.

\subsection{Rays and fronts}
\inputdir{XFig}
\par
It is natural to begin studies of waves
with equations that describe plane waves
in a medium of constant velocity.

Figure~\ref{fig:front} depicts a ray moving down into the earth
at an angle $ \theta $ from the vertical.
\sideplot{front}{width=3.3in}{
  Downgoing ray and wavefront.
}
%was \activesideplot{front}{width=3.0in}{NR}{
Perpendicular to the ray is a wavefront.
By elementary geometry the angle between the wave\bx{front}
and the earth's surface
is also  $ \theta $.
The \bx{ray} increases its length at a speed  $v$.
The speed that is observable on the earth's surface is the intercept
of the wavefront with the earth's surface.
This speed, namely  $ v / \sin \theta $,  is faster than  $v$.
Likewise, the speed of the intercept of the wavefront and
the vertical axis is  $ v / \cos \theta $.
A mathematical expression for a straight line
like that shown to be the wavefront in Figure~\ref{fig:front} is
\begin{equation}
z \ \ =\ \  z_0 \ -\  x \  \tan \, \theta
\label{eqn:2.1}
\end{equation}
\par
In this expression  $ z_0 $  is the intercept between the wavefront
and the vertical axis.
To make the intercept move downward, replace it by the
appropriate velocity times time:
\begin{equation}
z \ \ =\ \  {v \, t \over  \cos \, \theta } \ -\  x \  \tan \, \theta
\label{eqn:2.2}
\end{equation}
Solving for time gives
\begin{equation}
t(x,z) \ \ =\ \  {z\over v }\ \cos\,\theta \ +\  {x \over v }\  \sin \, \theta
\label{eqn:txz}
\end{equation}
Equation~(\ref{eqn:txz}) tells the time that the wavefront will pass any
particular location  $(x , z)$.
The expression for a shifted waveform
of arbitrary shape is  $ f(t - t_0 ) $.
Using (\ref{eqn:txz}) to define the time shift $ t_0 $ gives an expression for
a wavefield that is some waveform moving on a \bx{ray}.
\begin{equation}
\hbox{moving wavefield} \ \ =\ \ 
f\left( \ t\ -\  {x \over v}\ \sin\,\theta \ -\ {z\over v}\ \cos\,\theta\right)
\label{eqn:mvwv}
\end{equation}

\subsection{Snell waves}
\par
In reflection seismic surveys the velocity
contrast between shallowest and deepest
reflectors ordinarily exceeds a factor of two.
Thus depth variation of velocity is almost always included
in the analysis of field data.
Seismological theory needs to consider waves
that are just like plane waves except that they bend
to accommodate the velocity stratification  $v(z)$.
Figure~\ref{fig:airplane} shows this in an idealized geometry:
waves radiated from the horizontal flight of a supersonic airplane.
\plot{airplane}{width=6.0in}{
  Fast airplane radiating a sound wave into the earth.
  From the figure you can deduce that
  the horizontal speed of the wavefront
  is the same at depth  $z_1$  as it is at depth  $z_2$.
  This leads (in isotropic media) to Snell's law.
}
% was \activeplot{airplane}{height=2.5in}{NR}{
The airplane passes location $x$ at time $t_0(x)$
flying horizontally at a constant speed.
Imagine an earth of horizontal plane layers.
In this model there is nothing to distinguish any point
on the $x$-axis from any other point on the $x$-axis.
But the seismic velocity varies from layer to layer.
There may be reflections, head waves, shear waves, converted waves,
anisotropy, and multiple reflections.
Whatever the picture is, it moves along with the airplane.
A picture of the wavefronts near the airplane moves along with the airplane.
The top of the picture and the bottom of the picture both move laterally at
the same speed even if the earth velocity increases with depth.
If the top and bottom didn't go at the same speed,
the picture would become distorted,
contradicting the presumed symmetry of translation.
This horizontal speed, or rather its inverse  ${\partial t_0}/{\partial x}$,
has several names.
In practical work it is called the
{\em  \bx{stepout}.}
In theoretical work it is called the
{\em  \bx{ray parameter}}.
It is very important to note that  ${\partial t_0}/{\partial x}$
does not change with depth,
even though the seismic velocity does change with depth.
In a constant-velocity medium, the angle of a wave
does not change with depth.
In a stratified medium,
${\partial t_0}/{\partial x}$  does not change with depth.
\par
Figure~\ref{fig:frontz} illustrates the differential geometry of the wave.
Notice that triangles have their
hypotenuse on the $x$-axis and the $z$-axis
but not along the ray.
That's because this figure refers to wave fronts.
(If you were thinking the hypotenuse would measure $v\Delta t$,
it could be you were thinking of the tip of a ray
and its projection onto the $x$ and $z$ axes.)
\plot{frontz}{height=1.8in}{
  Downgoing fronts and rays in stratified medium  $v(z)$.
  The wavefronts are horizontal translations of one another.
}
The diagram shows that
\begin{eqnarray}
{\partial t_0 \over \partial x} \ \ \ &=&\ \ \  { \sin \, \theta  \over v }
\label{eqn:5iei4a}
\\
{\partial t_0 \over \partial z} \ \ \ &=&\ \ \  { \cos \, \theta  \over v }
\label{eqn:5iei4b}
\end{eqnarray}
These two equations define two (inverse) speeds.
The first is a horizontal speed,
measured along the earth's surface,
called the
{\em 
horizontal \bx{phase velocity}.
}
The second is a vertical speed, measurable in a borehole, called the
{\em 
vertical phase velocity.
}
Notice that both these speeds
{\em  exceed}
the velocity  $v$  of wave propagation in the medium.
Projection of wave
{\em  fronts}
onto coordinate axes gives speeds larger than  $v$,
whereas projection of
{\em  rays}
onto coordinate axes gives speeds smaller than $v$.
The inverse of the phase velocities is called the
{\em  \bx{stepout}}
or the 
{\em  \bx{slowness}.}
\par
\bx{Snell's law} relates the angle of a wave
in one layer with the angle in another.
The constancy of equation (\ref{eqn:5iei4a}) in depth is really just
the statement of Snell's law.
Indeed, we have just derived Snell's law.
All waves in seismology propagate in a
velocity-stratified medium.  So they cannot be called
plane waves.  But we need a name for waves that are
near to plane waves.  A %
{\em  \bx{Snell wave} %
} will be defined to be the generalization of a plane wave
to a stratified medium  $v(z)$.
A plane wave that happens to enter a medium
of depth-variable velocity  $v(z)$  gets changed into a Snell wave.
While a plane wave has an angle of propagation, a
Snell wave has instead a %
{\em  \bx{Snell parameter} %
} $p\,=\,{\partial t_0}/{\partial x}$.
\par
It is noteworthy that
Snell's parameter  $p\,=\,{\partial t_0}/{\partial x}$  is directly
observable at the surface,
whereas neither  $v$  nor  $\theta$  is directly observable.
Since  $p\,=\,{\partial t_0}/{\partial x}$  is not only observable,
but constant in depth, it is customary to use it
to eliminate  $\theta$  from equations (\ref{eqn:5iei4a}) and (\ref{eqn:5iei4b}):
\begin{eqnarray}
{\partial t_0  \over \partial x} \ \ \ &=&\ \ \ {\sin\,\theta  \over v }\eq p
\label{eqn:5iei5a}
\\
{\partial t_0  \over \partial z}\ \ \ &=&\ \ \ 
{\cos\,\theta  \over v }\eq \sqrt{
 {1 \over v (z)^2}\ -\ p^2  }
\label{eqn:5iei5b}
\end{eqnarray}

\subsection{Evanescent waves}
Suppose the velocity increases to infinity at infinite depth.
Then equation (\ref{eqn:5iei5b}) tells us that something
strange happens when we reach the depth for which
$p^2$ equals $1/v(z)^2$.
That is the depth at which the ray turns horizontal.
We will see in a later chapter that below this critical depth
the seismic wavefield damps exponentially with increasing depth.
Such waves are called \bx{evanescent}.
For a physical example of an evanescent wave,
forget the airplane and think about a moving bicycle.
For a bicyclist, the slowness $p$ is so large that it dominates $1/v(z)^2$
for all earth materials.
The bicyclist does not radiate a wave,
but produces a ground deformation
that decreases exponentially into the earth.
To radiate a wave,
a source must move faster than the material velocity.

\subsection{Solution to kinematic equations}
\par
The above differential equations will often reoccur in later analysis,
so they are very important.
Interestingly, these differential equations have a simple solution.
Taking the Snell wave to go through the origin at time zero,
an expression for the 
arrival time of the Snell wave at any other location
is given by
\begin{eqnarray}
t_0(x,z) \ \ \ &=&\ \ \ {\sin\,\theta  \over v }\  x\ +\ \int_0^z\ 
{\cos\,\theta  \over v }\ d z 
\label{eqn:5iei6a}
\\
t_0(x,z)\ \ \ &=&\ \ \ p\,x\ +\ \int_0^z\ 
\sqrt{ {1 \over v ( z ) ^2 }\ -\ 
p^2 } \ \ d z 
\label{eqn:5iei6b}
\end{eqnarray}
The validity of equations~(\ref{eqn:5iei6a}) and~ (\ref{eqn:5iei6b})
is readily checked by
computing  $\partial t_0 / \partial x$  and  $\partial t_0 / \partial z $,
then comparing with (\ref{eqn:5iei5a}) and (\ref{eqn:5iei5b}).

\par
An arbitrary waveform  $f(t)$  may be carried by the Snell wave.
Use (\ref{eqn:5iei6a}) and (\ref{eqn:5iei6b}) to {\em  define} the time $t_0$ for
a delayed wave  $f[t-t_0 (x,z)]$  at the location  $(x,z)$.
\begin{equation}
\hbox{SnellWave}(t,x,z)\eq f \, \left( \  t\ -\ 
p\,x\ -\  \int_0^z\ 
\sqrt{ {1 \over v ( z )^2}\ -\ p^2 } \ \ dz \  \right)
\label{eqn:5iei7}
\end{equation}
Equation~(\ref{eqn:5iei7})
carries an arbitrary signal throughout the whole medium.
Interestingly, it does not agree with wave propagation theory
or real life because
equation~(\ref{eqn:5iei7}) does not correctly account for amplitude
changes that result from
velocity changes and reflections.
Thus it is said that
Equation~(\ref{eqn:5iei7})
is ``kinematically'' correct but ``dynamically'' incorrect.
It happens that most industrial data processing only requires
things to be kinematically correct,
so this expression is a usable one.

\section{CURVED WAVEFRONTS}
The simplest waves are expanding circles.
An equation for a circle expanding with velocity  $v$
is
\begin{equation}
v^2 \, t^2  \eq  x^2 \ \ +\ \ z^2 
\label{eqn:circeqn}
\end{equation}
Considering  $t$  to be a constant,
i.e.~taking a snapshot, equation~(\ref{eqn:circeqn}) is that of a circle.
Considering  $z$  to be a constant,
it is an equation in the $(x , t)$-plane for a hyperbola.
Considered in the $(t , x , z)$-volume,
equation~(\ref{eqn:circeqn}) is that of a cone.
Slices at various values of  $t$  show circles of various sizes.
Slices of various values of  $z$  show various hyperbolas.
\par
Converting equation~(\ref{eqn:circeqn})
to traveltime depth $\tau$ we get
\begin{eqnarray}
v^2 \, t^2 &=& z^2 \ +\  x^2 \\
t^2        &=& \tau^2 \ +\ { x^2   \over  v^2 } 
\label{eqn:hyper}
\end{eqnarray}
The earth's velocity typically increases
by more than a factor of two between the earth's surface,
and reflectors of interest.
Thus we might expect that equation~(\ref{eqn:hyper}) would have little
practical use.
Luckily, this simple equation will solve many problems for us
if we know how to interpret the velocity as an average velocity.

\subsection{Root-mean-square velocity}
\sx{root-mean-square}
\sx{RMS velocity}
When a ray travels in a depth-stratified medium,
Snell's parameter $p=v^{-1}\sin\theta$ is constant along the ray.
If the ray emerges at the surface,
we can measure the distance $x$ that it has traveled,
the time $t$ it took, and its apparent speed $dx/dt=1/p$.
A well-known estimate $\hat v$
for the earth velocity contains this apparent speed.
\begin{equation}
\hat v \eq \sqrt{ {x\over t} \ {dx\over dt} }
\label{eqn:vrmsobs}
\end{equation}
To see where this velocity estimate comes from,
first notice that the stratified velocity $v(z)$ can be parameterized
as a function of time and take-off angle of a ray from the surface.
\begin{equation}
v(z) \eq v(x,z) \eq v'(p,t)
\end{equation}
The $x$ coordinate of the tip of a ray with Snell parameter $p$ is
the horizontal component of velocity integrated over time.
\begin{equation}
x(p,t) \eq \int_0^t \ v'(p,t) \ \sin\theta(p,t)\ dt
       \eq p\ \int_0^t v'(p,t)^2\ dt \ 
\end{equation}
Inserting this into equation~(\ref{eqn:vrmsobs})
and canceling $p=dt/dx$ we have
\begin{equation}
\hat v \eq
v_\RMS \eq \sqrt{ {1\over t} \ \int_0^t v'(p,t)^2\ dt\ \ }
\label{eqn:vrmsdefine}
\end{equation}
which shows that the observed velocity is the ``root-mean-square'' velocity.

\par
When velocity varies with depth,
the traveltime curve is only roughly a hyperbola.
If we break the event into many short line segments where the
$i$-th segment has a slope $p_i$ and a midpoint $(t_i,x_i)$
each segment gives a different $\hat v(p_i,t_i)$
and we have the unwelcome chore of assembling the best model.
Instead, we can fit the observational data to the best fitting hyperbola
using a different velocity hyperbola for each apex,
in other words,
find $V(\tau )$ so this equation
will best flatten the data in $(\tau,x)$-space.
\begin{equation}
t^2 \eq \tau^2 + x^2/V(\tau)^2
\end{equation}
Differentiate with respect to $x$ at constant $\tau$ getting
\begin{equation}
2t\, dt/dx \eq 2x/V(\tau)^2
\end{equation}
which confirms that the observed velocity
$\hat v$ in equation (\ref{eqn:vrmsobs}),
is the same as $V(\tau )$ no matter where you measure
$\hat v$ on a hyperbola.

\subsection{Layered media}
\inputdir{vrms}
\par
From the assumption that experimental data
can be fit to hyperbolas
(each with a different velocity and each with a different apex $\tau$)
let us next see how
we can fit an earth model of layers,
each with a constant velocity.
Consider the  horizontal reflector
overlain by a stratified \bx{interval velocity} $v(z)$ 
shown in Figure~\ref{fig:stratrms}.%
\sideplot{stratrms}{width=3.00in}{
  Raypath diagram for normal moveout in a stratified earth.
} %

The separation between the source and geophone,
also called the offset, is $2h$ and the total travel time is $t$.
Travel times are not be precisely hyperbolic,
but it is common practice to find the best fitting hyperbolas,
thus finding the function $V^2(\tau)$.

\begin{equation}
t^2 \eq \tau^2 + \frac{4h^2}{V^2(\tau)}
\label{eqn:vrmshyp}
\end{equation}
where $\tau$ is the zero-offset two-way traveltime.

\inputdir{vscan}
\par
An example of using equation~(\ref{eqn:vrmshyp})
to stretch $t$ into $\tau$
is shown in Figure~\ref{fig:nmogath}.
(The programs that
find the required $V(\tau )$ and do the stretching are coming up in
chapter~\ref{paper:vela}.)
\plot{nmogath}{width=6.00in,height=3.6in}{
  If you are lucky and get a good velocity,
  when you do NMO, everything turns out flat.
  Shown with and without mute.
}


\par
Equation (\ref{eqn:vrmsdefine}) shows that
$V(\tau)$
is
the ``root-mean-square'' or
``\RMS'' velocity defined by
an average of $v^2$ over the layers.
Expressing it for a small number of layers we get
\begin{equation}
V^2(\tau) \eq \frac{1}{\tau}\  \sum_i v^2_i \Delta\tau_i
\label{eqn:vrmsdefn}
\end{equation}
where the zero-offset traveltime $\tau$ is a sum over the layers:
\begin{equation}
\tau \eq \sum_i \ \Delta\tau_i
\label{eqn:onetaudefn}
\end{equation}
The two-way vertical travel time $\tau_i$
in the $i$th layer is related to the
thickness $\Delta z_i$ and the velocity $v_i$  by
\begin{equation}
\Delta\tau_i \eq  \frac{2\ \Delta z_i}{v_i}  \ \ .
\label{eqn:twotaudefn}
\end{equation}
\par
Next we examine an important practical calculation,
getting interval velocities from measured RMS velocities:
Define
in layer $i$,
the interval velocity $v_i$
and the two-way vertical travel time $\Delta\tau_i$.
Define the RMS velocity
of a reflection
from the bottom of the $i$-th layer
to be $V_i$.
Equation (\ref{eqn:vrmsdefn}) tells us that for
reflections from the bottom of the first, second, and third layers we have
\begin{eqnarray}
V_1^2 &=& {v_1^2\Delta\tau_1           
                               \over \Delta\tau_1                 }
\\
\label{eqn:bot2}
V_2^2 &=& {v_1^2\Delta\tau_1+ v_2^2\Delta\tau_2 
                               \over \Delta\tau_1 + \Delta\tau_2        }
\\
\label{eqn:bot3}
V_3^2 &=& {v_1^2\Delta\tau_1+ v_2^2\Delta\tau_2 +v_3^2\Delta\tau_3  
                               \over \Delta\tau_1 + \Delta\tau_2 +\Delta\tau_3}
\end{eqnarray}

Normally it is easy to measure the times of the three hyperbola tops,
$\Delta\tau_1$, 
$\Delta\tau_1 + \Delta\tau_2$
and
$\Delta\tau_1 + \Delta\tau_2 +\Delta\tau_3$.
Using methods in chapter \ref{paper:vela}
we can measure the RMS velocities $V_2$ and $V_3$.
With these we can solve for the interval velocity $v_3$ in the third layer.
Rearrange (\ref{eqn:bot3}) and (\ref{eqn:bot2}) to get

\begin{eqnarray}
\label{eqn:next3}
                                    (\Delta\tau_1 + \Delta\tau_2 +\Delta\tau_3)
V_3^2 &=& v_1^2\Delta\tau_1+ v_2^2\Delta\tau_2 +v_3^2\Delta\tau_3  
\\
\label{eqn:next2}
                                    (\Delta\tau_1 + \Delta\tau_2)
V_2^2 &=& v_1^2\Delta\tau_1+ v_2^2\Delta\tau_2 
\end{eqnarray}
and subtract getting the squared interval velocity $v_3^2$
\begin{equation}
v_3^2 \eq {
        (\Delta\tau_1 + \Delta\tau_2 +\Delta\tau_3) V_3^2  -
        (\Delta\tau_1 + \Delta\tau_2              ) V_2^2
        \over
        \Delta\tau_3}
\label{eqn:estint}
\end{equation}
For any real earth model we would not like an imaginary velocity
which is what could happen if the squared velocity in (\ref{eqn:estint})
happened to be negative.
You see that this means that the RMS velocity we estimate
for the third layer cannot be too much smaller than the one we
estimate for the second layer.

\par
%Experimentalists like equations~\EQN{vrmshyp} and \EQN{vrmsdefn}.
%Suppose data contains two hyperboloids
%(curves that look like hyperbolas)
%one with a top at time $\tau_1$ and the other with a top at $\tau_2$.
%By finding the best fitting hyperbolas,
%they get $V(\tau_1)$ and $V(\tau_2)$.
%If they presume the earth has three constant-velocity layers
%with interfaces between them at $\tau_1$ and $\tau_2$
%then
%they can determine the velocity of each of the top two layers.

\subsection{Nonhyperbolic curves}
\sx{hyperbolic, non}
\sx{nonhyperbolic}
Occasionally data does not fit a hyperbolic curve very well.
Two other simple fitting functions are
\begin{eqnarray}
\label{eqn:fourthorder}
t^2              &=& \tau^2 \ +\ { x^2   \over  v^2 } \ 
                                                +\ x^4 \times {\rm parameter}
\\
\label{eqn:datum}
(t-t_0)^2        &=& (\tau-t_0)^2 \ +\ { x^2   \over  v^2 } 
\end{eqnarray}
Equation~(\ref{eqn:fourthorder}) has an extra adjustable parameter
of no simple interpretation other than the beginning of a power series in $x^2$.
I prefer Equation~(\ref{eqn:datum}) where the extra adjustable parameter
is a time shift $t_0$ which has a simple interpretation,
namely, a time shift
such as would result from a near-surface low velocity layer.
In other words, a datum correction.

\todo{ XXXXXX 
\subsection{Can I abandon the material in this section?}
\par
We could work out the mathematical problem
of finding an analytic solution for
the travel time
as a function of distance in an earth with stratified $v(z)$,
but the more difficult problem is
the practical one which is the reverse,
finding $v(\tau)$ from the travel time curves.
Mathematically we can
express the travel time (squared)
as a power series in distance $h$.
Since everything is symmetric in $h$,
we have only even powers.
The practitioner's approach is to look at small offsets
and thus ignore $h^4$ and higher powers.
Velocity then enters only as the coefficient of $h^2$.
Let us why it is the \RMS\ velocity,
equation~(\ref{eqn:vrmsdefn}),
that enters this coefficient.
%
% above is new
%
\par
The hyperbolic form of equation~(\ref{eqn:vrmshyp}) will generally not be exact
when $h$ is very large.
For ``sufficiently'' small $h$,
the derivation of the hyperbolic shape follows
from application of Snell's law at each interface.
Snell's law implies that the Snell parameter $p$, defined by
\begin{equation}
p \eq \frac{\sin\theta_i}{v_i}
\label{eqn:pdefine}
\end{equation}
is a constant along both raypaths in Figure~\ref{fig:stratrms}.
%This law implies that $\cos\theta_i$
%can be written in terms of $p$ as
%\begin{equation}
%\cos\theta_i \eq \sqrt{1-(pv_i)^2} \ \ \ .
%\end{equation}
Inspection of Figure~\ref{fig:stratrms} shows that 
in the $i$th layer
the raypath horizontal distance $\Delta x_i$ and travel time $\Delta t_i$
are given on the left below by
\begin{eqnarray}
\label{eqn:xrms}
\Delta x_i &=& \Delta z_i \tan\theta_i
           \eq
%          \frac{p}{2}
%          \frac{\Delta\tau_i v_i^2}
           \frac{v_i\Delta\tau}{2} \ 
           \frac{p v_i}
           {\sqrt{1-p^2v_i^2}}
           \eq \frac{p}{2} \Delta\tau_i v_i^2 + O(p^3)    \\
\label{eqn:trms}
\Delta t_i &=& \frac{2\,\Delta z_i}{v_i \cos\theta_i} 
           \eq \frac{\Delta\tau_i}{\sqrt{1-p^2v_i^2}}
           \eq \Delta\tau_i \left(1+
                {\tiny 1 \over 2}  p^2 v_i^2 \right) + O(p^4) \ \ .
\end{eqnarray}
The center terms above arise by using equation~(\ref{eqn:pdefine})
to represent $\tan\theta$ and $\cos\theta$
as a function of $\sin\theta$ hence $p$,
and the right sides above come from expanding in powers of $p$.
Any terms of order $p^3$ or higher will be discarded,
since these become important only at large values of $h$.
Summing equation~(\ref{eqn:xrms}) and~(\ref{eqn:trms}) over all layers 
yields the half-offset $h$ separating the midpoint
from the geophone location and the total travel time $t$.
\begin{eqnarray}
\label{eqn:hrms}
h &=&\frac{p}{2}\ \tau \  V^2(\tau) + O(p^3) \\
\label{eqn:310}
t &=&\tau \left( 1 + \frac{1}{2} p^2 V^2(\tau) \right) + O(p^4) \ \ \ .
\end{eqnarray}
Solving equation~(\ref{eqn:hrms}) for $p$ gives $p=2h/(\tau V^2)$,
justifying the neglect of the $O(p^3)$ terms when $h$ is small.
Substituting this value of $p$ into equation~(\ref{eqn:310}) yields
\begin{equation}
t \eq \tau \left( 1 + \frac{2h^2}{\tau^2 V^2(\tau)} \right) + O(p^4) \ \ \ .
\end{equation}
Squaring both sides and discarding terms of order $h^4$ and $p^4$
yields the advertised result, equation~(\ref{eqn:vrmshyp}).

 XXXXXXXX
 }

\subsection{Velocity increasing linearly with depth}
Theoreticians are delighted by velocity increasing linearly with depth
because it happens that many equations work out in closed form.
For example, rays travel in circles.
We will need convenient expressions for velocity
as a function of traveltime depth
and \RMS\ velocity as a function of traveltime depth.
Let us get them.
We take the \bx{interval velocity} $v(z)$ increasing linearly with depth:
\begin{equation}
v(z) \eq v_0 + \alpha z
\end{equation}
%
This presumption can also be written as a differential equation:
\begin{equation}
\frac{dv}{dz} \eq \alpha  .
\end{equation}
%
The relationship between $z$ and vertical two-way traveltime $\tau(z)$
(see equation~(\ref{eqn:twotaudefn})) is also given by a differential equation:
\begin{equation}
\frac{d \tau}{dz} \eq \frac{2}{v(z)}.
\end{equation}
%
Letting $v(\tau)=v(z(\tau))$
and applying the chain rule
gives the differential equation for $v(\tau)$:
\begin{equation}
\frac{dv}{dz}
\frac{dz}{d \tau}
\eq
\frac{dv}{d \tau}
\eq
\frac{v \alpha}{2},
\end{equation}
whose solution gives us the desired expression for \bx{interval velocity}
as a function of traveltime depth.
\begin{equation}
v(\tau) \eq v_0 \ e^{\alpha \tau / 2 }  .
\label{eqn:priorint}
\end{equation}
%

\subsection{Prior RMS velocity}
Substituting the theoretical interval velocity $v(\tau)$
from equation~(\ref{eqn:priorint})
into the definition of
\RMS\ velocity $V(\tau)$
(equation~(\ref{eqn:vrmsdefn}))
yields:
\begin{eqnarray}
\tau \ V^2(\tau) &=& \int_{0}^{\tau} v^2(\tau') \ d \tau'
\\
               &=& v_0^2 \ \frac {e^{\alpha \tau} - 1} {\alpha} .
\end{eqnarray}
Thus the desired expression for \RMS\ velocity
as a function of traveltime depth is:
\begin{equation}
V(\tau) \eq v_0 \ 
        \sqrt{
        \frac{e^{\alpha \tau} - 1 }{\alpha \tau}
        }
\label{eqn:Vrms}
\end{equation}
For small values of $\alpha \tau$,
this can be approximated as
\begin{equation}
V(\tau) \quad\approx \quad v_0\ \sqrt{1 + \alpha \tau / 2}  .
\end{equation}

%\iex{Exer/Intro}{assignment}
%\iex{Exer/Shift}{assignment}
%\iex{Exer/2Dft}{assignment}
%\iex{Exer/Phasemod}{assignment}
%\iex{Exer/Phasedown}{assignment}
%\iex{Exer/Phasemig}{assignment}
%\iex{Exer/Stolt1}{assignment}
%\iex{Exer/Stolt2}{assignment}










































                                                                        

\subsection{Dix Inversion}

Applying the chain rule, we \old{can} rewrite the Dix inversion formula
\citep{dix:68} as follows:
\begin{equation}
\hat{\mu}=\frac{d}{{d\tau }}\left[ {\tau _{0}(\tau )\mu (\tau )}\right] %
\left[ \frac{d{\tau }_{0}}{{d\tau }}\right] ^{-1}.  \label{eqn:dixchain}
\end{equation}
where $\hat{\mu}$ is a vertically-variable interval parameter, ${\mu}$
is the corresponding effective parameter, and $\tau_{0}(\tau)$ is the
zero-slope time mapping function. Substituting the LHS of
\reqs{2ndmoment} and \ren{4thmoment} as $\mu$ and \req{tau0mapping} as
$\tau_0$, we \old{can} deduce expressions for the interval NMO velocity
$\hat{V}_{N}$, horizontal velocity $\hat{V}_{H}$, and the
anellipticity parameter $\hat{\eta}$. The derivations and the final formulas
are detailed in appendix B. In order to retrieve interval parameters
by slope-based Dix inversion, one needs as inputs the slope $R$ and
\new{the} curvature $Q$ fields \old{and} \new{as well as} their derivatives along the time axis $\tau$
(see Table \ref{tbl:velocities}). This confirms that, even in \taup,
an application of Dix's formula requires the knowledge of the
effective quantities which, in this context, are mathematically
represented by the slope $R$ and curvature $Q$. The $\tau_0$ mapping
field is also needed to map the estimated VTI parameters to the
correct imaging time. The Dix inversion route does not seem very
practical in the \taup domain because the equations (derived in
appendix B) appear cumbersome.
	
%\rEq{dixchain} needs the $\tau$ derivative $R_{\tau}=\partial R / \partial \tau$ and $Q_{\tau}=\partial Q %/ \partial \tau$ of the slope   and curvature fields.




%Employing the Dix inversion approach \citep{dix:68} and defining $R_{\tau
%}=\partial R/\partial \tau $ we can deduce from equations \ref%
%{eqn:tau0mappingISO} and \ref{eqn:vNmapISO} an expression for the interval
%moveout\ velocity $\hat{V}_{N}$ in the isotropic or elliptical anisotropy
%case. Hence interval velocity $\hat{V}_{N}$ \ becomes another attribute we
%can directly extract from the data through slope estimation, as follows:
%
%\begin{equation}
%\hat{V}_{N}=\frac{1}{{p(pR-\tau )}}\frac{{pR(R+\tau R_{\tau })-2R_{\tau
%}\tau ^{2}}}{{{2\tau -p(R+\tau R_{\tau })}}}.  \label{eqn:vNdixISO}
%\end{equation}
%
%This equation is the $\tau -p$ analogous of the equation 15 already
%published in \cite{fomel:2046}. The derivation is in appendix A.
%
%In the case of non-elliptic $(\eta \neq 0) $ VTI anisotropy , although the algebra becomes rather involved, it is still possible to obtain two equations that relate effective $V_{N}^{\text{ \ \ }}$and $\eta $
%parameters to the interval $\hat{V}_{N}$ and $\hat{\eta}$ ones. Again,  starting from the 
%Dix formula and defining $%
%Q_{\tau }=\partial Q/\partial \tau ,N_{\tau }=\partial N/\partial \tau $ and
%$D_{\tau }=\partial D/\partial \tau ,$ we can deduce from equation \ref%
%{eqn:tau0mapping} and \ref{eqn:vNmap} the VTI interval moveout velocity $%
%\hat{V}_{N}:$
%

%
%The formula for interval $\hat{\eta}$ instead is entirely detailed in the
%appendix a. 



%Figure \ref{fig:dataSynthDix} shows the interval moveout $\hat{V}%
%_{N}$ (a) and horizontal $\hat{V}_{N}$ (b) velocity toghether with the
%interval anellipticity $\hat{\eta}$ (c) obtained using the relations \ref%
%{eqn:vNdixVTI} and \ref{eqn:etaDIX} on the synthetic data example shown in
%figure \ref{fig:dataSynth} (a). These profiles comes after the oriented
%mapping in equation \ref{eqn:tau0mapping}. The exact interval profiles (blue
%dotted lines) are recovered perfectly although the resolution worsens with
%respect to the effective profiles shown in figure \ref{fig:dataSynthOriented}%
%. Again, the main reason is the instability of the numerical differentiation
%along the $\tau $ direction. We noticed that the estimates of $\hat{V}_{H}$
%and $\hat{\eta},$the parameters that controls the large offset, are noisier
%than $\hat{V}_{N}$ and this fact agrees with the observation by \cite%
%{ilyabook2006} that states that high-order moveout parameters are in general
%less constrained than the short-spread normal moveout velocity $\hat{V}_{N}$%
%. The field data results in fugure \ref{fig:dataCongoFX} (a) are so much
%noisier such that only $\hat{V}_{N}$ velocity profile is meaningful and
%worthy of being presented.

\published{Geophysics, 2020, 85, no. 2, V223–V232}

\title{Relative time seislet transform}

\author{% Put authors and institues here
    Zhicheng Geng$^1$, 
    Xinming Wu$^1$,  
    Sergey Fomel$^1$, and
    Yangkang Chen$^2$
\\
$^1$Bureau of Economic Geology, John A. and Katherine G. Jackson School of Geosciences, The University of Texas at Austin, Austin, Texas, USA \\
$^2$Zhejiang University, School of Earth Sciences, Hangzhou, Zhejiang, China}

\maketitle

\righthead{RT-seislet transform}
\lefthead{Geng et al.}
\footer{TCCS}

%-----------------------------------------------------------------
\begin{abstract}
    The seislet transform utilizes the wavelet-lifting scheme and local slopes 
    to analyze the seismic data. 
    In its definition, the designing of prediction operators specifically for 
    seismic images and data is an important issue. 
    We propose a new formulation of the seislet transform based on the relative 
    time (RT) attribute. 
    This method uses RT volume to construct multiscale prediction operators. 
    With the new prediction operators, the seislet transform gets accelerated 
    since distant traces get predicted directly. 
    We apply the proposed method to synthetic and real data to demonstrate that 
    the new approach reduces computational cost and obtains excellent sparse
    representation on test datasets. 
\end{abstract}

%-----------------------------------------------------------------
\section{Introduction}
    Sparse transforms aim to represent the most important information of an 
    image with few coefficients in the transform domain while obtaining a good 
    quality approximation of the original image. 
    Over the past several decades, different types of wavelet-like transforms 
    have been proposed and successfully applied in image compression and 
    denoising~\cite[]{hennenfent2006seismic, li2013contourlet}, including 
    curvelets~\cite[]{starck2002curvelet, ma2010curvelet}, 
    contourlets~\cite[]{do2005contourlet}, shearlets~\cite[]{labate2005sparse}, 
    directionlets~\cite[]{velisavljevic2006directionlets}, 
    bandelets~\cite[]{le2005sparse}. 
    The strong anisotropic selectivity of these wavelet-like transforms helps 
    achieve excellent data compression and accurate reconstruction for seismic 
    images.

    \cite{fomel2010seislet} introduced the seislet transform, which is a digital 
    wavelet-like transform designed specifically for seismic data. 
    Based on the lifting scheme used in digital wavelet transform (DWT) 
    construction~\cite[]{sweldens1995lifting}, the seislet transform follows 
    dominant local slopes obtained by plane-wave destruction 
    (PWD)~\cite[]{claerbout2000basic, fomel2002applications} to predict seismic 
    events. 
    Instead of using PWD, \cite{liu2010oc} used offset continuation (OC) to 
    construct updating and prediction operators. 
    The OC-seislet transform has better performance than the PWD-seislet 
    transform in characterizing and compressing structurally complex prestack 
    reflection data. 
    To reduce sensitivity to strong noise interference, \cite{liu2015signal} 
    proposed a velocity-dependent (VD) seislet transform. 
    In this method, the normal moveout equation was introduced to serve as a 
    connection between local slope and scanned velocities. 
    \cite{chen2018emd} utilized empirical mode decomposition (EMD) to obtain 
    smoothly non-stationary data, and then applied the 1D non-stationary seislet 
    transform to the data. This new approach was called the EMD-seislet 
    transform and has shown excellent performance in attenuating random noise. 
    Recently, several studies show the superiority of the seislet transform
    on sparse representation of seismic data over the Fourier transform, the 
    wavelet transform and the curvelet transform \cite[]{fomel2010seislet, 
    gan2015dealiased, chen2018emd}.

    With the ability to compress and reconstruct seismic images, the 
    seislet transform has been successfully applied in seismic data processing 
    such as noise attenuation~\cite[]{chen2016dip, chen2016double}, 
    deblending~\cite[]{chen2014iterative, gan2016simultaneous} and data 
    interpolation~\cite[]{liu2013iterative, gan2016simultaneous}.
    However, the original implementation of PWD-seislet transform requires
    recursively computation when predicting distant traces, which increases its 
    computation cost, especially when the dataset is very large.
    Additionally, the smooth slopes from PWD in the original seislet transform 
    can fail to correctly follow reflections across discontinuities including 
    faults and unconformities. 
    Therefore, geologically meaningful discontinuities may not be optimally 
    compressed. 

    The relative time (RT) volume, $\tau(x,y,t)$, has different meanings in
    different domains. 
    In seismic image domain, the RT is the same as the relative geologic 
    time~\cite[]{stark2003unwrapping, stark2004relative} and each RT contour 
    corresponds to a geologic horizon.
    However, RT volumes of seismic gathers do not necessarily have geologic 
    meaning, since in this case, constant RT contours align with seismic events.
    There are several ways to generate an RT volume. 
    One can always pick out as many horizons or events as possible to obtain it, 
    which is simple but laborious. 
    The RT volume can also be generated by unwrapping seismic instantaneous 
    phase~\cite[]{stark2003unwrapping, wu2012generating}.
    \cite{fomel2010predictive} used local slopes of seismic events estimated by 
    PWD~\cite[]{fomel2002applications} to generate the RT volume, which has 
    superior computational performance. 
    The RT volume already has been successfully applied in missing well-log 
    data prediction \cite[]{bader2018missing} and seismic horizons construction 
    \cite[]{wu2015horizon}.
    
    In this paper, we propose a new implementation of the seislet transform, 
    called RT-seislet transform, by using the RT obtained from the predictive 
    painting~\cite[]{fomel2010predictive} to construct prediction operators for 
    the seislet transform. 
    In this way, prediction of one trace from distant traces is computed 
    directly and accurately, which saves a lot of computation and has better 
    performance with faults and unconformities.
    By applying the new formulation of the seislet transform to synthetic and 
    real data, we demonstrate that the proposed method is efficient and able to 
    preserve discontinuities, while achieving excellent sparse representation of 
    seismic data. 

    This paper is organized as follows: we first review the seislet 
    transform and the estimation of an RT volume. 
    Then, we incorporate RT attribute to construct new formulation of prediction 
    and update operators for the seislet transform.
    Finally, we test the proposed RT-seislet transform on both synthetic and 
    real datasets and compare its performance with the PWD-seislet transform.
    
%-----------------------------------------------------------------
\section{Theory}
\subsection{Reviews of the 2D seislet transform}
    As proposed by \cite{fomel2010seislet}, the seislet transform can be 
    constructed using the lifting scheme \cite[]{sweldens1995lifting}. 
    The procedure of the lifting scheme is as follows:
    \begin{enumerate}
        \item Arrange the input data as a sequence of records. 
            In the seislet transform, the input data are common shot gather or 
            seismic image.
        \item Divide the arranged records into even and odd components 
            $\mathbf{e}$ and $\mathbf{o}$. For the 2D seislet transform, each 
            trace is a component and the seismic data are split into two parts 
            according to trace indices.
        \item Find the residual $\mathbf{r}$ between the odd component and 
            its prediction from the even component
            \begin{equation}
                \mathbf{r} = \mathbf{o} - \mathbf{P[e]},
                \label{eq:a}
            \end{equation}
            where $\mathbf{P}$ is a \emph{prediction} operator.
        \item Using the difference from the previous step to get a coarse 
            approximation $\mathbf{c}$ of the data by updating the even 
            component
            \begin{equation}
                \mathbf{c} = \mathbf{e} + \mathbf{U[r]},
                \label{eq:b}
            \end{equation}
            where $\mathbf{U}$ is an \emph{update} operator.
        \item The coarse approximation $\mathbf{c}$ obtained from last step 
            becomes the new data, and previous steps are applied to the new data 
            at the next scale level.
    \end{enumerate}

    The coarse approximation $\mathbf{c}$ from the final scale and residual
    difference $\mathbf{r}$ from all scales form the result of the seislet 
    transform.

    The prediction and update operators are defined, for example, by modifying 
    operators in the construction of CDF 5/3 biorthogonal 
    wavelets~\cite[]{cohen1992biorthogonal} as follows:
    \begin{equation}
        \label{P}
        \mathbf{P[e]}_k = \left(\mathbf{S}_k^{(+)}[\mathbf{e}_{k-1}]+
        \mathbf{S}_k^{(-)}[\mathbf{e}_{k}] \right) / 2
    \end{equation}
    and
    \begin{equation}
        \label{U}
        \mathbf{U[r]}_k = \left(\mathbf{S}_k^{(+)}[\mathbf{r}_{k-1}]+
        \mathbf{S}_k^{(-)}[\mathbf{r}_{k}] \right) / 4,
    \end{equation}
    where $\mathbf{S}_k^{(+)}$ and $\mathbf{S}_k^{(-)}$ can accomplish the 
    prediction of a trace from its left and right neighbors, respectively. 
    The predictions need to be applied at different scales. 
    For 2D seislet transform, different scales mean different distances 
    between traces. 
    More accurate higher-order formulations are possible. 
    Following the inverse lifting scheme from coarse scale to fine scale, 
    the inverse seislet transform can be constructed.

\subsection{RT volume estimation}
    In the proposed method, the predictive painting~\cite[]{fomel2010predictive}, 
    based on the prediction operator arising from 
    PWD~\cite[]{fomel2002applications}, is utilized to generate the RT volume. 
    In the predictive painting, after local slopes are estimated, an RT volume 
    is computed by spreading time values of reference traces along dominant 
    event slopes throughout the whole seismic volume.

    Using linear operator notation, PWD operation is defined as
    \begin{equation}
        \mathbf{d} = \mathbf{D}{}\mathbf{s},
    \end{equation}
    where $\mathbf{s} =[\mathbf{s}_1 \,\mathbf{s}_2 \,\cdots\, \mathbf{s}_N]^T$ 
    is a collection of seismic traces and $\mathbf{d}$ denotes the residual from 
    destruction. $\mathbf{D}$ is the destruction operator defined as 
    \begin{equation}
        \label{pred}
        \mathbf{D}=
        \begin{bmatrix}
            \mathbf{I} & 0 & 0 & \cdots & 0 \\
            -\mathbf{P}_{1,2} & \mathbf{I} & 0 & \cdots & 0 \\
            0 & -\mathbf{P}_{2,3} & \mathbf{I} & \cdots & 0 \\
            \vdots & \ddots & \ddots & \ddots & \vdots \\
            0 & \cdots & 0 & -\mathbf{P}_{N-1,N} & \mathbf{I}
        \end{bmatrix},
    \end{equation}
    where $\mathbf{I}$ denotes the identity matrix and $\mathbf{P}_{i,i+1}$ is 
    an operator that predicts the $(i+1)$-th trace from the $i$-th trace 
    according to the local slope of seismic event or seismic horizon.

    By a simple recursion, a trace can be predicted utilizing a distant trace. 
    For example, $\mathbf{P}_{r,k}$, predicting the $k$-th trace from the $r$-th 
    trace, is defined as
    \begin{equation}
        \label{predict}
        \mathbf{P}_{r,k}=\mathbf{P}_{k-1,k}\mathbf{P}_{k-2,k-1}\cdots
        \mathbf{P}_{r+1,r+2}\mathbf{P}_{r,r+1}.
    \end{equation}
    Then, the prediction of trace $\mathbf{s}_r$, given a reference trace
    $\mathbf{s}_r$ is accomplished by
    \begin{equation}
        \label{pp}
        \mathbf{s}_k^{\text{pre}}=\mathbf{P}_{r,k}\mathbf{s}_r.
    \end{equation}
    After determining the prediction operators in equation~\ref{pred} by PWD, 
    the predictive painting algorithm recursively spreads information from a 
    reference trace to the whole seismic data volume by following local slopes
    using equation \ref{pp}. 
    If the information spread by the predictive painting are the time 
    coordinates, an RT volume is easily computed.

    Figure~\ref{fig:sigmoid} shows a synthetic image with several complex 
    geologic structures. 
    Local slopes corresponding to Figure~\ref{fig:sigmoid} are shown in 
    Figure~\ref{fig:dip-pad}, which are estimated by PWD. 
    Figure~\ref{fig:pick-1} is the RT volume calculated by the predictive 
    painting using local slopes in Figure~\ref{fig:dip-pad}. 
    The reference trace of this RT volume is in the center of the image, which 
    is marked as a vertical line in Figure~\ref{fig:pick-1}. 
    This single reference RT volume is used to flatten the input seismic 
    image (Figure \ref{fig:sigmoid}) by unshifting each traces using the 
    information it contains, since RT indicates how much a given trace is 
    shifted with respect to the reference trace.
    The flattened image is shown in Figure~\ref{fig:flat-1}.
    Another RT volume with multiple references, shown in 
    Figure~\ref{fig:pick-5}, is generated by selecting five reference traces and 
    calculating the average of all estimated RT volumes.
    Figure~\ref{fig:flat-5} shows the corresponding flattened image.
    The comparison between two flattened seismic images (Figure~\ref{fig:flat-1} 
    and \ref{fig:flat-5}) indicates that the RT volume with only one reference 
    trace does not necessarily have information about all seismic events since 
    there are discontinuities such as faults and unconformities, while the RT 
    volume computed with more reference traces contains more information.

    \inputdir{rt}
    \plot{sigmoid}{width=0.8\columnwidth}{Synthetic seismic image from 
    \cite{claerbout2000basic}.}
    \plot{dip-pad}{width=0.8\columnwidth}{Local slopes estimated from 
    Figure~\ref{fig:sigmoid}.}
    \multiplot{2}{pick-1,pick-5}{width=0.6\columnwidth}{RT estimated by 
    predictive painting from Figure~\ref{fig:sigmoid} using (a) a single 
    reference trace and (b) multiple reference traces.}
    \multiplot{2}{flat-1,flat-5}{width=0.6\columnwidth}{Flattened images using
    RT volumes from (a) Figure \ref{fig:pick-1} and (b) Figure~\ref{fig:pick-5}.
    }

\subsection{RT formulation of prediction and update operators}
    RT contains the relationship between traces. 
    From RT volume, it is straightforward to obtain the shift of one trace with 
    respect to the reference. 
    Therefore, the trace prediction using RT volumes is formulated as follows:
    \begin{equation}
        \mathbf{s}_k^{\text{pre}}(t)=\mathbf{s}_r(\mathbf{\tau}_{k,r}(t))
        \label{new_form}
    \end{equation}
    where $\mathbf{s}_k$ and $\mathbf{s}_r$ denote the $k$-th and the $r$-th 
    trace, respectively. 
    $\tau_{k,r}(t)$ is the RT value of the $k$-th trace with the $r$-th trace as 
    the reference trace, indicating the shift of the $k$-th trace with
    respect to the $r$-th trace. 
    According to equation \ref{new_form}, the prediction of the $k$-th trace 
    from the $r$-th trace is easily accomplished by simply applying forward 
    interpolation to the $r$-th trace using corresponding RT values.
    In the proposed method, we use equation \ref{new_form} to define operators 
    $\mathbf{S}$ in equations \ref{P} and \ref{U} to construct corresponding 
    prediction and update operators. 
    In this way, a trace is predicted from a distant trace directly by the 
    RT attribute instead of the recursive computation used in the PWD-seislet 
    transform. 
    Also, because an accurate RT volume stores information about all horizons 
    and structural discontinuities including faults and unconformities, 
    predictions of traces around discontinuities using equation 
    \ref{new_form} are accurate. 
    Therefore, a better delineation of faults and unconformities can be 
    achieved.

    In computing an RT volume, we need to first choose one or multiple reference 
    traces. 
    After computing the RT volume, the RT attribute between any two traces is 
    easily obtained from one RT volume using:
    \begin{equation}
        \label{rt}
        \mathbf{\tau}_{k,r}(t)=\mathbf{\tau}^{-1}_{r,l}(\mathbf{\tau}_{k,l}(t)),
    \end{equation}
    where $\mathbf{\tau}_{j,i}(t)$ represents the RT value of the $j$-th trace 
    with the $i$-th trace as the reference trace, and 
    $\mathbf{\tau}^{-1}_{r,l}(t)$ is the time 
    warping~\cite[]{burnett2009moveout} of $\tau_{r,l}(t)$. 
    According to equation \ref{rt}, given the shift information 
    $\tau_{r,l}(t)$ and $\tau_{k,l}(t)$, the shift relationship between the 
    $k$-th trace and the $r$-th trace can be obtained by applying an inverse 
    interpolation to $\tau_{r,l}(t)$ using $\tau_{k,l}(t)$. 
    Then, equation \ref{new_form} is applied to implement the prediction of the 
    $k$-th trace from the $r$-th trace.
    Therefore, only one RT volume is needed for the proposed implementation of 
    the seislet transform. 

    Figure~\ref{fig:pick-150,invint,pick-50,pick-50-true} shows an example of
    implementing equation~\ref{rt} to compute an RT volume from another one with 
    different reference traces. 
    The RT volume in Figure~\ref{fig:pick-150} is obtained by the predictive 
    painting with the 150th trace as the reference trace. 
    This RT volume can be represented by $\mathbf{\tau}_{k,150}(t), 
    k=1,2,\ldots,n$. 
    Figure~\ref{fig:invint} is the time warping of this RT volume. 
    We extract the 50th trace from Figure~\ref{fig:invint}, which is denoted by 
    $\mathbf{\tau}^{-1}_{50,150}(t)$. 
    Then, by implementing equation~\ref{rt}, i.e., inverse interpolating 
    $\mathbf{\tau}^{-1}_{50,150}(t)$ using the whole RT volume 
    ($\mathbf{\tau}_{k,150}(t), k=1,2,\ldots,n$), we can get a new 
    RT volume, as shown in Figure~\ref{fig:pick-50} ($\mathbf{\tau}_{k,50}(t), 
    k=1,2,\ldots,n$). Here, we use the 50th trace as the reference trace. 
    To evaluate the effectiveness of equation \ref{rt}, RT volumes from 
    Figure~\ref{fig:pick-50} and \ref{fig:pick-50-true} are used to flatten 
    Figure~\ref{fig:sigmoid}. 
    Flattened images are shown in Figure~\ref{fig:flat1,flat2}. 
    Small differences between Figure~\ref{fig:flat1} and~\ref{fig:flat2} 
    indicate that these two RT volumes (Figure~\ref{fig:pick-50} 
    and~\ref{fig:pick-50-true}) contain similar information, which demonstrate 
    that equation~\ref{rt} can be used to get the relationship between any two 
    traces from one single RT volume.
    
    \multiplot{4}{pick-150,invint,pick-50,pick-50-true}{width=0.4\columnwidth}{
    (a) The RT volume obtained by the predictive painting and the reference 
    trace is the 150th trace. (b) Time warping of (a). (c) RT volume obtained by 
    equation~\ref{rt}. (d) The RT volume obtained by the predictive painting and 
    the reference trace is the 50th trace.}
    \multiplot{3}{flat1,flat2}{width=0.7\columnwidth}{Flattened images using RT 
    volumes from (a) Figure \ref{fig:pick-50} and (b) 
    Figure~\ref{fig:pick-50-true}.}

\subsection{Workflow}
    The workflow of the proposed method is shown in Figure~\ref{fig:workflow}. 
    As we can see, the proposed approach is convenient to implement since only 
    one RT volume is needed before applying the new formulation of the seislet 
    transform. 
    In the proposed approach, we need first use PWD to compute the local slopes 
    of the input seismic data such that the RT volume can be computed by the 
    predictive painting from the estimated slopes, as discussed in the previous 
    section. 
    With the help of the generated RT volume, the RT-seislet transform is then 
    applied to the input data to get coefficients in the RT-seislet transform 
    domain. 
    With the transformed data, signal processing tasks, including data 
    reconstruction and noise attenuation, can be easily applied.

    \inputdir{.}
    \plot{workflow}{width=0.8\columnwidth}{Workflow of the proposed RT-seislet
    transform.}

%-----------------------------------------------------------------
\section{Synthetic Data Example}
    We use the same data example from Figure~\ref{fig:sigmoid} to demonstrate 
    the performance of the proposed method in ef{}ficient data reconstruction. 
    Figure~\ref{fig:rt} shows the RT calculated from the image using the 
    predictive painting with multiple reference traces. 
    We applied both the original PWD-seislet transform and the RT-seislet 
    transform to this image. 
    Then, we threshold coefficients and keep only the largest ones from both the 
    transforms to reconstruct the same original data. 
    By using the same percent (1\%) of large coefficients, the reconstruction 
    from the new formulation (Figure~\ref{fig:rtseisrec1}) recovers more 
    accurate amplitude values and better preserves structural discontinuities 
    near faults and unconformities than the one from PWD-seislet transform 
    (Figure~\ref{fig:pwdseisrec1}).
    
    \inputdir{sigmoid}
    \plot{rt}{width=0.8\columnwidth}{RT volume estimated by the predictive 
    painting using multiple reference traces.}
    \multiplot{2}{pwdseisrec1,rtseisrec1}{width=0.6\columnwidth}{Synthetic 
    seismic image reconstruction by thresholding coefficients and keeping only 
    1\% of significant coefficients using (a) inverse PWD-seislet transform and 
    (b) inverse RT-seislet transform.}

    The comparison between coefficients from the PWD-seislet transform and 
    the RT-seislet transform is shown in Figure~\ref{fig:coef}. 
    The minor difference of their decay rate indicates that the PWD-seislet 
    transform and the proposed transform obtain the same sparsity for the input 
    data.
    Figure~\ref{fig:tim} shows the plot of CPU time against the number of 
    traces. 
    The proposed formulation of the seislet transform shows promising 
    computation efficiency since the prediction of a trace using distant 
    traces is accomplished directly instead of recursively as in the original 
    PWD-seislet transform.
    
    \plot{coef}{width=0.8\columnwidth}{Normalized coefficients from the 
    PWD-seislet transform (red dashed line) and the RT-seislet transform (blue 
    solid line) sorted from large to small on a decibel scale.}

    \inputdir{time}
    \plot{tim}{width=0.80\columnwidth}{CPU time of the PWD-seislet transform 
    (red dashed line) and the RT-seislet transform (blue solid line) versus the 
    number of traces.}

    For the purpose of testing the sensitivity of the RT-seislet transform to 
    noise, we add random noise to the synthetic model and utilize 
    signal-to-noise ratio (SNR) as measurement: 
    \begin{equation}
        \label{snr}
        SNR=10\log_{10}\frac{\|\mathbf{s}_{true}\|^2_2}
        {\|\mathbf{s}_{true}-\mathbf{s}\|^2_2}\,,
    \end{equation}
    where $\mathbf{s}_{true}$ is the original data, and $\mathbf{s}$ denotes
    the reconstructed data (e.g., Figure~\ref{fig:rtseisrec1}) or noisy data.
    We plot SNRs of reconstructed data using the inverse PWD-seislet transform 
    and the inverse RT-seislet transform with respect to the noise variance in 
    Figure~\ref{fig:snr}. 
    Apparently, as the noise level increases, the performance of the RT-seislet 
    gets worse, which shows that the proposed method is sensitive to noise. 
    But, compared with the original PWD-seislet transform, the new method has 
    superior performance that is less sensitive to noise.
    Figure~\ref{fig:sigmoid-noise0,rtseisrec-noise0} shows one example of the
    noisy data and the reconstructed data. The SNR of 
    Figure~\ref{fig:rtseisrec-noise0} is 17.19 dB.

    \inputdir{noise}
    \plot{snr}{width=0.80\columnwidth}{SNR diagram of the PWD-seislet transform 
    (red dashed line) and the RT-seislet transform (blue solid line) with 
    respect to the noise level (variance value).}
    \multiplot{2}{sigmoid-noise0,rtseisrec-noise0}{width=0.6\columnwidth}{(a)
    Noisy data. (b) Reconstructed data using 10\% coefficients (SNR=17.19).}

    Figure~\ref{fig:para,para-rtseisrec1,para-diff} presents another synthetic 
    example. 
    Figure~\ref{fig:para} and \ref{fig:para-rtseisrec1} shows the synthetic shot 
    gather with three hyperbolic events and Figure~\ref{fig:para-rtseisrec1} is 
    the data reconstruction by the RT-seislet transform using only 1\% of the 
    most significant coefficients, respectively. 
    We use the first, the middle and the last traces as the reference traces to 
    compute the RT volume.
    The residual between the synthetic shot gather and the reconstruction
    result is shown in Figure~\ref{fig:para-diff}.
    For this simple synthetic shot gather, the proposed method achieves an 
    excellent data reconstruction result.

    \inputdir{synth}
    \multiplot{3}{para,para-rtseisrec1,para-diff}{width=0.45\columnwidth}{(a) 
    Synthetic shot gather. (b) Data reconstruction by the inverse RT-seislet 
    transform using only 1\% of the most significant coefficients. (c) 
    Difference between (a) and (b).}
    
\section{2D Field Data Example}
    To further test the proposed method, we use a common-midpoint gather from 
    the North Sea dataset (Figure~\ref{fig:gath0}). 
    Figure~\ref{fig:seis} shows the data in the seislet transform domain. 
    We observe that significant coefficients are apparent in only a narrow area 
    on the left, which indicates the excellent compression ability of the 
    proposed RT formulation of the seislet transform. 
    The field data reconstruction using only 1\% of the most significant 
    RT-seislet coefficients is shown in Figure~\ref{fig:seisrec1}. 
    The reconstructed image retains the most important seismic events in the 
    original seismic data.

\inputdir{gath}
    \plot{gath0}{width=0.80\columnwidth}{Common-midpoint gather.}
    \multiplot{2}{seis,seisrec1}{width=0.60\columnwidth}{(a) The field data in 
    the RT-seislet transform domain and (b) reconstruction using only 1\% of 
    significant coefficients by the inverse RT-seislet transform.}

    If we mute the RT-seislet coefficients at fine scales while keeping the 
    significant coefficients at coarse scales (Figure~\ref{fig:seisdeno}), the 
    inverse RT-seislet transform enables effective denoising, removing 
    incoherent noise from the gather. 
    The result of denoising are shown in Figure~\ref{fig:seis16}. 
    Figure~\ref{fig:diff} shows the removed noise section using the proposed 
    implementation of the seislet transform.

\multiplot{4}{gath1,seisdeno,seis16,diff}{width=0.45\columnwidth}{(a) 
    Original gather. (b) Denoised result by muting RT-seislet coefficients at 
    fine scales (a) and (c) the removed noise section using the RT-seislet 
    transform.}

    The RT-seislet is also able to perform interpolation. 
    We add two more scales with small random noise to Figure~\ref{fig:seis} and 
    interpolate the RT volume by a factor of four. 
    Then the interpolated shot gather is obtained and the number of traces is 
    increased by four. 
    The extended data in the seislet transform domain is shown in 
    Figure~\ref{fig:seisrand2}. 
    Figure~\ref{fig:gath4} is the interpolated shot gather. 
    Figure \ref{fig:fft,fft2} shows the F-K spectrum of the zero-padded 
    field data and the interpolated gather. 
    The zero-padded shot gather is generated by manually padding zero traces 
    between two traces.
    In this example, the seislet transform domain is extended with random noise 
    because there is unpredictable noise in the real case.

    \multiplot{2}{seisrand2,gath4}{width=0.6\columnwidth}{(a) Extended seislet 
    transform domain. (b) Interpolated shot gather by the inverse RT-seislet 
    transform.}
    \multiplot{2}{fft,fft2}{width=0.6\columnwidth}{F-K spectra of (a) 
    zero-padded  and (b) the interpolated field data.}

\section{3D Field Data Example}
    We also test the proposed method on a 3D dataset. 
    Figure~\ref{fig:cuber} is the Teapot Dome seismic dataset. 
    After inline and crossline slopes (Figure~\ref{fig:dip1,dip2}) are 
    estimated, the RT volume (Figure~\ref{fig:pick}) is easily obtained from the 
    predictive painting with multiple reference traces. 
    With the RT volume, the 3D seislet transform can be constructed by 
    cascading the 2D seislet transform along inline and crossline directions. 
    To be specific, we first apply the 2D RT-seislet transform along inline 
    direction, and then another 2D transform along crossline is applied on the 
    coefficients from the previous step to obtain the coefficients in the 3D 
    transform domain.
    The coefficients of the 2D seislet transform along inline direction and the 
    3D seislet transform are shown in Figure~\ref{fig:rtseis-inline} 
    and~\ref{fig:rtseis}.
    In the seislet domain, most of the coefficients concentrate in the coarse 
    scales, which shows the excellent compression ability of the RT-seislet 
    transform. 
    Figure~\ref{fig:teapot-rtseisrec5} is the data reconstruction result using 
    only 5\% of the most significant coefficients. 
    The difference between Figure \ref{fig:cuber} and the reconstructed data, 
    contains mainly noise, is shown in Figure \ref{fig:teapot-diff}.
    The proposed method has good performance for reconstruction and preserving 
    the structures of the dataset.

    \inputdir{teapot}
    \multiplot{2}{cuber,pick}{width=0.6\columnwidth}{(a) Teapot dome dataset. 
    (b) RT volume estimated by the predictive painting.}
    \multiplot{2}{dip1,dip2}{width=0.6\columnwidth}{Inline (a) and Crossline (b)
    Dip.}
    \multiplot{4}{rtseis-inline,rtseis,teapot-rtseisrec5,teapot-diff}
    {width=0.4\columnwidth}{(a) The coefficients in 2D seislet domain along 
    inline only. (b) The coefficients in 3D seislet domain. (c) Data 
    reconstruction using only 5\% of significant coefficients by the inverse 
    RT-seislet transform. (d) Difference between (c) and the original seismic 
    volume (Figure~\ref{fig:cuber}).}

%-----------------------------------------------------------------
\section{Discussion}
    The proposed formulation of the seislet transform attempts to construct 
    multiscale prediction operators, to enhance the performance of the 
    seislet transform. 
    The prediction operators in the original seislet transform are defined by 
    PWD, so the prediction of distant traces needs recursive computation, which 
    is very time-consuming when it comes to large size dataset. 
    In the proposed RT-seislet transform, an RT volume is used to define a 
    multiscale prediction operator. 
    As shown in the previous sections, the relationship between any two traces 
    can be obtained from the RT volume, which helps avoid the recursive 
    computation for the prediction of distant traces. 
    As a result, the RT-seislet transform is much more efficient than the 
    original seislet transform.

    Besides, in the implementation of the proposed method, the choice of 
    reference traces plays an important role in the generation of the RT volume. 
    More reference traces usually mean a better RT volume. 
    The location of reference traces also needs to be considered carefully. 
    In practice, evenly distributed reference traces are recommended.
    Besides, complex geologic structures need to be treated differently.
    For example, it is better to put reference traces at each side of faults. 
    Another important issue is about the smoothing radius when estimating local 
    slopes by PWD. 
    For noise-free data, small smoothing radius is enough. 
    The smoothing radius is three for the estimation of 
    Figure~\ref{fig:dip-pad}. 
    And with the increasing of the noise level, larger smoothing radius is 
    needed. 

    The proposed RT-seislet transform highly depends on the RT volume. 
    In our method, local slopes by PWD, which are easily affected by noise, are 
    utilized to generate the RT volume, so the new formulation is sensitive to 
    the noise. 
    However, this problem certainly can be solved by improving the PWD or using 
    other methods, which are not sensitive to noise, to generate the RT volume.

%-----------------------------------------------------------------
\section{Conclusion}
    We have proposed a new formulation of the seislet transform, a novel 
    approach to sparse seismic data. 
    In our approach, we use an RT volume to construct multiscale prediction and 
    update operators for the seislet transform. 
    This proposed approach is computationally more efficient and better 
    preserves geologically meaningful discontinuities than the original 
    formulation of the seislet transform in reconstructing the raw seismic data. 
    It also has excellent performance in obtaining the sparse representation of
    seismic data.
    Synthetic and field data examples show that the proposed new formulation is 
    efficient and effective in data reconstruction and noise attenuation.

%-----------------------------------------------------------------
\section{Acknowledgements}
    The research is partially supported by Texas Consortium for Computation 
    (TCCS), the “Thousand Youth Talents Plan”, and the Starting Funds from 
    Zhejiang University. 
    The computational examples reported in this paper are reproducible using 
    the \emph{Madagascar} open-source software 
    package~\cite[]{fomel2013madagascar}. 
    
\section{DATA AND MATERIALS AVAILABILITY}
    Data associated with this research are available and can be obtained by 
    contacting the corresponding author. 

\onecolumn
\bibliographystyle{seg}
\bibliography{rtseis}

\published{Geophysics, 75, no. 4, A25-A30, (2010)}
\lefthead{Fomel}
\righthead{Predictive painting}

\title{Predictive painting of 3-D seismic volumes}
\author{Sergey Fomel}

\address{
sergey.fomel@beg.utexas.edu \\
John A. and Katherine G. Jackson School of Geosciences \\
The University of Texas at Austin \\
University Station, Box X \\
Austin, TX 78713-8924}

\ms{GEO-2010-0034}

\maketitle

\begin{abstract}
  Structural information is the most important content of seismic
  images. I introduce a numerical algorithm for spreading information
  in 3-D volumes according to the local structure of seismic
  events. The algorithm consists of two steps. First, local
  spatially-variable inline and crossline slopes of seismic events are
  estimated by the plane-wave-destruction method. Next, a seed trace
  is inserted in the volume, and the information contained in that
  trace is spread inside the volume, thus automatically ``painting''
  the data space. Immediate applications of this technique include
  automatic horizon picking and flattening in applications to both
  prestack and post-stack seismic data analysis. Synthetic and field
  data tests demonstrate the effectiveness of predictive painting.
\end{abstract}

\section{Introduction}

Structural information is the most important content of seismic
images. One way to characterize structure is to assign a dominant
local slope attribute to all elements in a volume. \cite{pvi} proposed
the method of plane-wave destruction for detecting local slopes of
seismic events. Closely related ideas were developed in the
differential semblance optimization framework
\cite[]{symes,kim}. Plane-wave destruction finds many important
applications in seismic data analysis, including data regularization,
noise attenuation, and velocity-independent imaging
\cite[]{GEO67-06-19461960,pmig,diffr,will}.

The main principle of plane-wave destruction is prediction: each
seismic trace gets predicted from its neighbors that are shifted along
the event slopes, and the prediction error gets minimized to estimate
optimal slopes. In this paper, I propose to extract the prediction
operator from the plane-wave destruction process and to use it for
recursive spreading of information inside the seismic volume. I call
this spreading \emph{predictive painting}.

One particular kind of information that becomes meaningful when spread
in a volume is \emph{relative geologic age}, in the terminology of
\cite{stark}: seismic layers arranged according to the relative age of
sedimentation. Once relative geological age is established everywhere
in the volume, it is possible to flatten seismic images by extracting
\emph{stratal slices} \cite[]{GEO63-02-05020513} without manual
picking of horizons. Even though flattened seismic horizons do not
necessarily correspond to equivalent true geologic ages, flattening
improves the interpreter's ability to understand and quantify the
structural architecture of sedimentary layers \cite[]{zeng}.  The idea
of using local shifts for automatic picking was introduced by
\cite{bienati} and \cite{lomask}. \cite{SEG-2003-50195022}
presents an alternative approach involving instantaneous
phase unwrapping. Analogous techniques are implemented by \cite[]{ssis,ssis2}.

Flattening and automatic picking of horizons are important not only
for final structural interpretation but also for prestack imaging and
data analysis and for extracting prestack amplitude attributes. The
idea of prestack gather flattening using local
cross-corre\-la\-ti\-ons was developed previously by a number of
authors \cite[]{exxon,sintef,gulunay2,gulunay}. 

The predictive painting method, introduced in this paper, provides a
new approach for extracting and applying structural patterns, with
superior computational performance. The advantages of the
proposed method include both conceptual simplicity and computational
efficiency. In the next sections, I describe the basic algorithm for
automatic painting and demonstrate its performance with synthetic and
field data examples.

\section{Destruction and prediction of plane waves}

Plane-wave destruction originates from a local plane-wave model for
characterizing seismic data \cite[]{GEO67-06-19461960}. The
mathematical basis is the local plane differential equation
\begin{equation}
  \frac{\partial P}{\partial x} + 
  \sigma\,\frac{\partial P}{\partial t} = 0\;,
  \label{eqn:pde}
\end{equation}
where $P(t,x)$ is the wave field and $\sigma$ is the local slope,
which may also depend on $t$ and $x$ \cite[]{pvi}. In the case of a
constant slope, equation~\ref{eqn:pde} has the simple general solution
\begin{equation}
  P(t,x) = f(t - \sigma x)\;,
  \label{eqn:plane}
\end{equation}
where $f(t)$ is an arbitrary waveform. Equation~\ref{eqn:plane} is
nothing more than a mathematical description of a plane wave. Assuming
that the slope $\sigma(t,x)$ varies in time and space, one can design
a local operator to propagate each trace to its neighbors.

Let $\mathbf{s}$ represent a seismic section
as a collection of traces: $\mathbf{s} =
\left[\mathbf{s}_1 \; \mathbf{s}_2 \; \ldots \;
\mathbf{s}_N\right]^T$, where $\mathbf{s}_k$ corresponds to $P(t,x_k)$
for $k=1,2,\ldots$ A plane-wave destruction operator
\cite[]{GEO67-06-19461960} effectively predicts each trace from its
neighbor and subtracts the prediction from the original trace.  In the
linear operator notation, the plane-wave destruction operation can be
defined as
\begin{equation}
  \label{eq:pwd}
  \mathbf{r} = \mathbf{D\,s}\;,
\end{equation}
where $\mathbf{r}$ is the destruction residual, and $\mathbf{D}$ is the
destruction operator defined as
\begin{equation}
  \label{eq:d}
  \mathbf{D} = 
  \left[\begin{array}{ccccc}
      \mathbf{I} & 0 & 0 & \cdots & 0 \\
      - \mathbf{P}_{1,2} & \mathbf{I} & 0 & \cdots & 0 \\
      0 & - \mathbf{P}_{2,3} & \mathbf{I} & \cdots & 0 \\
      \cdots & \cdots & \cdots & \cdots & \cdots \\
      0 & 0 & \cdots & - \mathbf{P}_{N-1,N} & \mathbf{I} \\
    \end{array}\right]\;,
\end{equation}
where $\mathbf{I}$ stands for the identity operator, and
$\mathbf{P}_{i,j}$ describes prediction of trace $j$ from trace
$i$. Prediction of a trace consists of shifting the original trace
along dominant event slopes. The prediction operator is a numerical
solution of equation~\ref{eqn:pde} for local plane wave propagation
in the $x$ direction. The dominant slopes are estimated by minimizing
the prediction residual $\mathbf{r}$ using regularized least-squares
optimization. I employ shaping
  regularization \cite[]{shape} for controlling the smoothness of the
  estimated slope fields. In the 3-D case, a pair of inline and
crossline slopes, $\sigma_x(t,x,y)$ and $\sigma_y(t,x,y)$, and a pair
of destruction operators, $\mathbf{D}_x$ and $\mathbf{D}_y$, are
required to characterize the 3-D structure. Each prediction in 3-D occurs in
  either inline or crossline direction and thus conforms to
  equation~\ref{eq:d}. However, as explained below in the discussion
  of Dijkstra's algorithm, it is possible to arrange all 3-D traces
  in a sequence for further processing.

Prediction of a trace from a distant neighbor can be accomplished by
simple recursion. Predicting trace $k$ from trace $1$ is simply
\begin{equation}
  \label{eq:pred}
  \mathbf{P}_{1,k} = \mathbf{P}_{k-1,k}\,
  \cdots\,\mathbf{P}_{2,3}\,\mathbf{P}_{1,2}\;.
\end{equation}
If $\mathbf{s}_r$ is a reference trace, then the prediction of
  trace $\mathbf{s}_k$ is $\mathbf{P}_{r,k}\,\mathbf{s}_r$. I call
the recursive operator $\mathbf{P}_{r,k}$
\emph{predictive painting}. Once the elementary prediction operators
in equation~\ref{eq:d} are determined by plane-wave destruction,
predictive painting can spread information from a given trace to its
neighbors recursively by following the local structure of seismic
events. The next section illustrates the painting concept using 2-D
examples.

\section{Predictive painting in 2-D}

\inputdir{flat}
\multiplot{2}{sigmoid,sdip}{width=0.7\textwidth}{(a) Synthetic image from \cite{bei}. (b) Local dip estimate.}
\multiplot{2}{spaint,spick}{width=0.7\textwidth}{Predictive painting using synthetic image from Figure~\ref{fig:sigmoid}. (a) Painted horizons. (b) Painted relative age. The reference trace is in the center of the image.}
\multiplot{2}{sflat,flat}{width=0.7\textwidth}{Synthetic image from Figure~\ref{fig:sigmoid} flattened using a single reference trace (a) and multiple references (b). The reference traces are marked by vertical lines.}

The input for the first example is borrowed from \cite{bei} and shown
in Figure~\ref{fig:sigmoid}. It is a synthetic seismic image
containing dipping beds, an uncomformity, and a
fault. Figure~\ref{fig:sdip} shows local dips measured by the
plane-wave destruction algorithm. The slope estimate correctly depicts
the constant dip in the top part of the image and the sinusoidal
variation of the dip in the bottom.  Figure~\ref{fig:spaint} shows the
output of predictive painting: I assign the reference trace, selected
in the middle of the image, with several horizon picks, which are then
automatically spread into the image space by using prediction
operators from equation~\ref{eq:pred}. Figure~\ref{fig:spick} shows
another kind of painting: This time, the reference trace contains
simply the time values along this trace. When spread by
predictive painting, it turns into the \emph{relative geologic age}
attribute, as defined by \cite{stark}.  Relative age indicates how
much a given trace is shifted with respect to the reference
trace. Unshifting each trace accomplishes automatic flattening. The
result is shown in Figure~\ref{fig:sflat}. Most of the horizons are
successfully flattened, although the algorithm fails to ``heal'' some
of events across the fault because of significant structural
discontinuities. In cases like that, the geological
insight of the interpreter is invaluable and cannot be easily replaced
by an automatic algorithm.

One reference trace is not necessarily structurally connected to all
events in the volume. By using multiple references and averaging the
relative age among all of them, one can obtain a more accurate
extraction of stratal slice information. The result of using multiple
references, shown in Figure~\ref{fig:flat}, contains more detailed
information about horizons that are not structurally visible from the
single reference trace.

As mentioned in the introduction, flattening and automatic picking of
horizons are useful not only for post-stack structural interpretation
but also for prestack data analysis. Figure~\ref{fig:pgath} shows an
application of 2-D predictive painting to a marine CMP (common
midpoint) gather. After the field of local slopes has been found
(Figure~\ref{fig:pgath}b), predictive painting is applied to mark
individual events (Figure~\ref{fig:pgath}c) or to flatten them
(Figure~\ref{fig:pgath}d), which effectively accomplishes moveout
correction. This processing is automatic and does not require manual
picking or any prior assumption on the moveout shape. After extracting
the moveout information, the moveout parameters can be estimated by
least-squares fitting, as described by \cite{will}. It is
  important to note that the gather flattening method is prone to
  errors in the presence of crossing events, such as multiple
  reflections, since only the dominant slopes of the strongest events
  are going to be picked up by the slope estimation procedure.

\inputdir{flatelf} 
\plot{pgath}{width=\columnwidth}{A marine CMP gather (a), estimated
local slopes (b), events marked by predictive painting (c), and
flattening (d).}

\section{Predictive painting in 3-D}

The challenge of predictive flattening in 3-D is in selecting a
recursive path that the reference trace should follow to paint its
neighbors. For choosing this path, I adopt a version of Dijkstra's
shortest path algorithm \cite[]{dijkstra0,dijkstra}. Dijkstra's
algorithm finds the path between two nodes in a network of nodes,
where there is a cost associated in connecting each node with its
neighbors. In our case, the nodes are seismic traces in a 3-D cube.
I use the semblance between neighboring traces as a cost
function. Dijkstra's algorithm finds the shortest (minimum-cost)
path by effectively arranging all nodes in a sequence from low to
high cost and evaluating each new node using the information from
previous nodes. I run the shortest-path algorithm starting from the
reference trace and paint other traces in a recursive sequence using
the information from previously painted traces. 
Using semblance as a cost function helps avoiding 3-D misties by
  forcing the shortest path to go around possible fault areas.

%Figure~\ref{fig:wtime}
%shows some of the shortest paths for the 3-D data test displayed in
%Figure~\ref{fig:win,wflat,wdip1,wdip2,wpaint,wcont}.

\inputdir{comaz} 

The 3-D data test is reproduced from \cite{lomask}. It uses a portion
of a depth-migrated 3-D image with structural folding and angular
unconformities (Figure~\ref{fig:win}). Inline and crossline dips are
measured automatically from the image using plane-wave destruction
(Figures~\ref{fig:wdip1} and~\ref{fig:wdip2}). Figure~\ref{fig:wpaint}
shows painting of individual strong horizons in the volume.
Figure~\ref{fig:wflat} displays automatic flattening using predictive
painting of the relative geologic age. % Since several reference 
Figure~\ref{fig:wcont} shows
some of the corresponding equal-relative-age horizons displayed on top
of the original image. Predictive painting is able to correctly
identify the most significant three-dimensional structural features.

%\plot{wtime}{width=\columnwidth}{Shortest time and selected paths
%  from the reference trace calculated by Dijkstra's shortest-path
%  algorithm.}

\multiplot{2}{win,wflat}{width=0.7\textwidth}{A North Sea image
  from~\cite{lomask} (a) and its predictive flattening (b).}

\multiplot{2}{wdip1,wdip2}{width=0.7\textwidth}{Inline (a) and
crossline (b) slopes in the North Sea Image estimated by plane-wave
destruction. Blue colors indicate negative slope; red colors, positive
slope.}

\multiplot{2}{wpaint,wcont}{width=0.7\textwidth}{Predictive painting (a)
  and automatic picking (b) of major horizons in the North Sea Image.}

%\inputdir{chev} 
%\multiplot{2}{chev,flats}{width=\columnwidth}{A Gulf
%of Mexico image from~\cite[]{lomask} before (a) and after (b)
%predictive flattening.}

\section{Conclusions}
I have introduced predictive painting, a numerical algorithm for
automatic spreading of information in 3-D seismic volumes according to
the local structure of seismic events. The structure is extracted
using plane-wave destruction, which operates by predicting each trace
in the volume from its inline and crossline neighbors. In the second
step, prediction operators are used to spread information inside the
volume. This procedure is automatic and computationally fast because
it requires only a small fixed number of operations per each trace or
data sample. Synthetic and field data tests demonstrate the
effectiveness of predictive painting in accomplishing such tasks as
automatic flattening and horizon picking.

Further research should concentrate on combining automatic tools with
interactive interpretation to allow the information extracted from
seismic data to be integrated with the geological insight of the
interpreter.

\section{Acknowledgments}

I would like to thank William Burnett, Dave Hale, Yang Liu, Jesse
Lomask, James Rickett, and Hongliu Zeng for inspiring discussions. I
am especially grateful to late Tury Taner for his encouragement and
support of this work.

This publication is authorized by the Director, Bureau 
of Economic Geology, The University of Texas at Austin.

\bibliographystyle{seg}
\bibliography{SEG,flat}

%%% Local Variables: 
%%% mode: latex
%%% TeX-master: t
%%% End: 

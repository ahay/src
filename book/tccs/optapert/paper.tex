\makeatletter
\renewcommand{\thesubfigure}{\alph{subfigure}}
\renewcommand{\@thesubfigure}{(\alph{subfigure})\hskip\subfiglabelskip}
\renewcommand{\@@thesubfigure}{(\alph{subfigure})}
\makeatother

\title{Selecting an optimal aperture in Kirchhoff migration using dip-angle images}

\renewcommand{\thefootnote}{\fnsymbol{footnote}} 

\lefthead {Klokov and Fomel}
\righthead {Optimal migration aperture}

\author{Alexander Klokov\footnotemark[1] and Sergey Fomel\footnotemark[1]}

\address {
\footnotemark[1]Bureau of Economic Geology, 
Jackson School of Geosciences, 
The University of Texas at Austin, 
University Station, Box X, 
Austin, TX 78713-8924, 
USA 
}

\begin{abstract}
We present a method \old{of}\new{for} selecting a migration aperture \old{for}\new{in} Kirchhoff \old{imaging}\new{migration}.
We first split migrated data into constant-dip-angle partial images. Then,
in every partial image, we estimate the consistency between each event and the constant dip of the analyzed
section. We filter out events whose slope is far from the corresponding dip. Stacking of the filtered
partial images corresponds to migration having an optimal aperture. Synthetic and real data examples
demonstrate that the proposed approach to migration-aperture optimization \new{is able to} reduce\old{s} migration noise \old{but}\new{while}
\old{preserves}\new{preserving} diffraction energy, which characterizes small geological objects and brings additional resolution to the image. 
\end{abstract}

\section{Introduction}
Kirchhoff migration remains a popular tool for seismic imaging (\citealp[]{leveille11}).
It handles every input trace separately, enabling the user to work with data sets of any configuration (including
irregular geometry) and to apply target-oriented analysis. One of the \old{weaknesses}\new{features} of Kirchhoff migration 
is its global aperture. The migration operator distributes energy throughout a wide area described by 
an impulse-response surface. However, a seismic image is constructed only by a local part of the surface that
touches a reflector. The remainder does not make a constructive contribution and, even worse, may generate
image artifacts (\citealp[]{sun98,hertweck03}).

The artifact problem can be solved by limiting migration aperture. \citet[]{schleicher97} analyzed 
influence of aperture parameters on imaging results and proposed a technique\old{that is} based on projected Fresnel zones 
for shortening the migration curve. After preliminary picking of target 
reflections and computing diffraction curves, \new{they estimated} the migration-aperture size\old{was estimated} by
analytical equations. \citet[]{sun98} analyzed the structure of a seismic image and proposed rules for its 
optimal construction. Two main principles were identified: (1) the tangent point between a diffraction curve and
a reflector must be located in the central part of the aperture and (2) the aperture must be at least as large as the first
Fresnel zone. Several methods were proposed to achieve these two objectives. \citet[]{tillmanns99} detected
instantaneous slowness in the data domain. \citet[]{sun01} described wave-path migration smearing energy 
within a small zone centered on the specular reflection point and defined under a stationary-phase
approximation. \citet[]{luth05} smeared migrated multicomponent data only inside a Fresnel volume built
around rays computed using polarization of the wavefield. \citet[]{buske09} constructed a Fresnel volume for
single-component seismic data, determining the emergent angle by local slowness analysis.
\citet[]{tabti04} worked with diffraction-operator panels and picked Fresnel apertures 
after preliminary low-pass filtering. \citet[]{kabbej07} introduced an attribute that characterizes the distance 
between a common-midpoint position and a currently imaged depth point then migrated the attribute and used it for
weighting-function construction. \citet[]{spinner07} used the common-reflection-surface operator to determine
parameters for the optimal migration aperture. \citet[]{alerini09} estimated horizon slopes in the image domain.

The migrated dip-angle domain has properties that are favorable for taking both issues --- the aperture 
position and its width --- into consideration. In this domain, a reflection event has a concave shape with
an apex whose position corresponds to the reflector dip (\citealp[]{audebert02,landa08,klokov12}). 
The event summation in the dip-angle direction produces an image in accordance with the stationary-point
principle (\citealp[]{bleistein01}). To limit aperture in the dip-angle domain \new{and to restrict summation to stationary points}, 
\citet[]{bienati09} applied an automatic slope estimation followed by muting. Aperture size was defined on the basis of wavelet bandwidth.
\citet[]{dafni12} proposed analyzing migrated gathers simultaneously in dip-angle and scattering-angle directions.

A seismic wavefield may contain reflections and diffractions. There are a number of important 
differences between these two components \cite[]{klem-musatov94}.
One of the differences is that reflections require a narrow migration aperture, whereas diffractions require an aperture as
wide as possible (while allowing matching of a diffraction curve).
Therefore, all methods that imply migration-aperture limiting are oriented
toward optimal imaging of reflection boundaries \new{and not diffraction objects}. Diffractions characterize small but important geological
objects and play a significant role in imaging of rough reflection boundaries \cite[]{khaidukov04}.
Their attenuation may cause a significant loss of resolution (\citealp[]{neidell97}). For image resolution to be preserved,
a migration-optimization method should \new{aim to} protect the diffraction component. In this paper, we demonstrate an approach
that allows us to achieve an optimal aperture size for reflection boundaries while \new{also} protecting the diffraction component.
The main idea is analyzing slope information in constant-dip partial images.

\section{Diffractions and reflections in constant-dip partial images}

Common-reflection-angle migration \cite[]{xu01} can produce a set of 
dip-angle gathers in which traces are partial images of a fixed lateral position for different migration dips.
At the same time, the migrated volume can be considered as a set of constant-dip partial images. Integration of the 
partial images along the dip angle provides a seismic image:
\begin{equation}
\label{eq:imaging}
I(\mathbf{\bar{x}},z)=\int\limits\hat{I}(\mathbf{\bar{x}},z,\alpha)d\alpha\;,
\end{equation}
where $\hat{I}(\mathbf{\bar{x}},z,\alpha)$ is a true-amplitude partial image for migration dip $\alpha$ (\citealp[]{bleistein2002}).

A dip-angle gather exposes the difference between reflections and diffractions (\citealp[]{audebert02,landa08,klokov12}). 
Reflection events have a concave shape, and an image of a reflection boundary is constructed by a limited range of dip
angles near the event apex. The effective range contains migrated impulses whose shift from the event apex does not 
exceed half \old{of a}\new{the prevailing} period. This area is equivalent to the Fresnel zone in the dip-angle domain.
Diffraction events \old{are}\new{appear} flat, however, which means that the full range of migration-dip angles \old{could offer}\new{provides}
an effective contribution to the image (Figure~\ref{fig:smile}). Thus, \new{an} optimal migration aperture should simultaneously be limited for reflections
and expanded for diffractions.

\inputdir{.}
\plot{smile}{width=0.8\columnwidth}{Reflection event and diffraction event in the 2D dip-angle gather.}

In the constant-dip partial-image domain, reflections and diffractions appear different as well, although they
have some analogous features. To examine this, let us start with the simple case of zero-offset migration
having the correct constant velocity. Similar derivations can be considered for prestack data.

A diffraction event behaves hyperbolically (Figure~\ref{fig:scheme-diffr}). Its shape is defined by
the following equation (see derivation in Appendix):
\begin{equation}
\label{eq:diffevent}
z(x_m)=(x_m - x_0) \tan \alpha + \sqrt {(x_m - x_0)^2 / \cos^2 \alpha + z^2_0}\;,
\end{equation}
where $\alpha$ is the constant-dip of the partial image, $x_m$ is an imaging position, and $x_0$ and $z_0$ are position and real depth 
of the diffraction point, respectively.

At the position of the diffraction point, when $x_m~=~x_0$, the event has the correct depth $z(x_m)=z_0$.
The slope of the event at this point corresponds to the derivative:
\begin{equation}
\label{eq:diffeventderiv}
\frac {\partial z }{\partial x_m} \vert_{x_m=x_0} = \tan \alpha\;.
\end{equation}
Equation~\ref{eq:diffeventderiv} indicates that an effective part of a diffraction event, which corresponds to the
correct position \new{and contributes to the image}, \old{had}\new{has} the same slope as the migration dip, or the constant-dip of the corresponding partial image.

The response of a plane reflector in the constant-dip partial image is a segment. Let us consider
a small part of the reflection boundary, which has a dip $\alpha_0$ and lateral length $d$ (Figure~\ref{fig:scheme-refl}).
In the correct-constant-velocity case, coordinates of the segment edges after constant-dip migration can be defined by
rotation around escape points $y_1$ and $y_2$:
\begin{eqnarray}
\label{eq:p1x}
  \bar{x_1} & = & z_1 \frac{\sin \alpha_0 - \sin \alpha}{\cos \alpha_0}\;, \\
\label{eq:p1z}
  \bar{z_1} & = & z_1 \frac{\cos \alpha}{\cos \alpha_0}\;,
\end{eqnarray}
and
\begin{eqnarray}
\label{eq:p2x}
  \bar{x_2} & = & z_2 \frac{\sin \alpha_0 - \sin \alpha}{\cos \alpha_0} + d\;, \\
\label{eq:p2z}
  \bar{z_2} & = & z_2 \frac{\cos \alpha}{\cos \alpha_0}\;.
\end{eqnarray}
The migrated edge points define a segment, whose slope is
\begin{equation}
\label{eq:reflslope}
\tan \beta = \frac {\bar{z_2}-\bar{z_1}}{\bar{x_2} - \bar{x_1}} = \frac{\cos \alpha \sin \alpha_0}{1 - \sin \alpha \sin \alpha_0}\;.
\end{equation}
Equation~\ref{eq:reflslope} illustrates some features of the migrated reflection boundary. For instance, a migrated horizontal
reflector ($\alpha_0 = 0$) keeps the zero-slope independent of the migration dip.

\inputdir{.}
\multiplot{2}{scheme-diffr,scheme-refl}{width=0.8\columnwidth}{Constant-dip migration (scheme) of (a) a diffraction point and (b) a reflection boundary.}

Equation~\ref{eq:reflslope} also allows us to find the condition in which the migrated slope is consistent with the constant-dip
partial image ($\beta=\alpha$). In this case,
\begin{equation}
\label{eq:cond}
\frac{\cos \alpha \sin \alpha_0}{1 - \sin \alpha \sin \alpha_0} = \frac {\sin \alpha}{\cos \alpha}\;.
\end{equation}
If $\alpha$ is derived from equation~\ref{eq:cond}, note that the equation has only an $\alpha=\alpha_0$ solution. In other words,
the slope of a migrated reflection event equals that of the migration dip only after migration with the correct dip. Such is the case when
the segment corresponds to the stationary point and contributes to the image. Thus, an effective reflection event is consistent
with the constant-dip partial image, as it is for diffractions. The derivation in the Appendix shows that this fact is true for curved reflectors
as well.

To verify the assertions presented, we ran the following experiment. We put two reflectors and one scattering point in the constant-velocity field
(Figure~\ref{fig:theo-model}). One reflector is flat, and the second has a dip of 15~degrees. We assume that reflectivity is generated
by density variations. The modeled zero-offset section is presented in Figure~\ref{fig:data0}. We migrated the data using Kirchhoff common-angle
migration with the correct velocity. Figure~\ref{fig:zo-res-init} displays the image and a dip-angle gather extracted from 
the position where the scattering point is located. The diffraction event appears flat in the dip-angle gather, as expected, \old{and}\new{while} the reflection
events have concave shapes. Figure~\ref{fig:zodpis0,zodpis15} shows two partial images corresponding to 0~and~15~degrees.
Note that reflection boundaries and an effective part of the diffraction event are consistent with the partial image --- 0~and~15 degrees, 
respectively (the slope is indicated by dashed lines).

To represent a realistic situation, we placed the same reflectors and scattering point within a smoothed velocity field from the Marmousi model.
Both reflection and diffraction events have complicated shapes in the data domain (Figure~\ref{fig:dm-zo}). Figure~\ref{fig:res-init}
displays the image and the dip-angle gather after migration with the correct velocity. \new{The traveltimes for Kirchhoff angle-domain migration
were computed by Huygens wavefront-tracing (\citealp{sava01}), which allowed us to handle multiple arrivals.} The diffraction event is flat, as it was in the
constant-velocity example, but the shape of the reflection events appears much more complicated. The gather is contaminated by strong
migration artifacts that produce noise in the image. Figure~\ref{fig:dpis0,dpis15} shows two partial images corresponding to 0~and~15 degrees.
Both images appear to be quite noisy. However, note that, even after migration with a complicated-velocity model, \old{reflection event remaining at the stationary point has a slope consistent with the partial image}\new{effective reflection events are consistent with the dips of partial
images.} The diffraction event at the scattering-point position has the appropriate slope as well \new{(the slopes are indicated by dashed lines)}. 

An event-slope-consistency analysis may provide information about separation between constructive
and nonconstructive parts of migrated data. \new{The key observation is that} constructive reflection and diffraction events have slopes
that are close to the constant-dip of the partial image. Slopes of nonconstructive events differ significantly from those of partial-image
dips. Integration of constructive events only corresponds to migration having an optimal
aperture.

\inputdir{theo}
\plot{theo-model}{width=0.8\columnwidth}{Theoretical model with two plane reflectors and one diffractor.}
\plot{data0}{width=0.8\columnwidth}{Zero-offset section corresponding to the constant-velocity model.}
\plot{zo-res-init}{width=0.8\columnwidth}{Depth-migrated image and common-image-gather in dip-angle domain corresponding to distance of 7~km.}
\multiplot{2}{zodpis0,zodpis15}{width=0.7\columnwidth}{(a) Zero-degree partial image and (b) fifteen-degree partial image for the constant-velocity model.}

\inputdir{depthMig}
\plot{dm-zo}{width=0.8\columnwidth}{Zero-offset section corresponding to the Marmousi-velocity model.}
\plot{res-init}{width=0.8\columnwidth}{Depth-migrated image and common-image-gather in dip-angle domain corresponding to distance of 7~km.}
\multiplot{2}{dpis0,dpis15}{width=0.7\columnwidth}{(a) Zero-degree partial image and (b) fifteen-degree partial image for the Marmousi-velocity model.}

\section {Slope-consistency analysis}

One approach to analyzing constant-dip partial-image consistency
is to measure the local slope in every partial image and to construct a taper function that is based on this measurement. The 
taper should protect events whose slope is close to the corresponding constant dip. Slopes that differ significantly from that of
the constant dip should be attenuated. \old{The weak point of this approach is that it may fail if conflicting dips occur in the partial image.}
\new{However, if conflicting dips occur in the partial image, dips estimation becomes a non-trivial task.}

\old{An alternative way of estimating}\new{To avoid this issue, we estimate} segment conformity with a partial image \old{is}\new{by}
stacking along a local trajectory defined by partial-image constant dip. Constructive segments match the trajectory that yields high values of coherency.
On the other hand, destructive segments get stacked in the wrong direction and provide low values of coherence.
Classic semblance (\citealp[]{taner69}), as a coherence measure, can be used for \old{construction of}\new{defining} a weight
function. Weighting of migrated data \old{is equivalent}\new{corresponds} to migration aperture optimization.

For illustration, we use the constant-dip partial images shown in Figure~\ref{fig:dpis0,dpis15}. They correspond to migration dips of 0~and~15 degrees, 
and we first apply local stacking along directions corresponding to 0~and~15~degrees, respectively. We extend stacking to the dip-angle direction
as well. Effective contributions are locally plane in this direction, and local stacking along migration dips allows for signal enhancement.
We measure coherency of the stacked data and get the semblance function $S(\alpha, z, x)$ presented in Figure~\ref{fig:semb-0,semb-15}. 
High semblance values correspond to events consistent with the dips of partial images. The next step is to transform the semblance function
to a weight function $W(\alpha, z, x)$, which should optimally handle migrated amplitudes. The weight should not change a plane diffraction
event or an effective apex area of reflections. At the same time, the weight should smoothly attenuate away from the apexes so that 
edge effects can be eliminated (\citealp[]{hertweck03}). We define \old{optimal}\new{our} weight function using the following thresholding rule:
\begin{equation}
W(\alpha, z, x) = 
\left\{ \begin{array}{rcl}
1 & \mbox{for} & S(\alpha, z, x) > s_2\;, \\ 
\frac {S(\alpha, z, x) - s_1}{s_2 - s_1} & \mbox{for} & s_1 < S(\alpha, z, x) < s_2\;, \\
0 & \mbox{for} & S(\alpha, z, x) < s_1\;,
\end{array}\right.
\end {equation}
where $s_1$ and $s_2$ define \old{the}\new{two} thresholds. Thus, weight function is defined by summation bases and thresholding parameters. In complex inhomogeneous media, \new{these} parameters depend on migrated data and can be estimated by trial and error. Local summation in the dip-angle direction can
be performed along constructive parts of the reflection concave event, whose size is defined after analysis of migrated dip-angle gathers.
An optimal lateral window should suppress migration artifacts at a minimal computational cost.\old{Threshold parameters are then found heuristically.}
\new{The semblance function in equation~10 appears to have smooth features that simplifies a heuristic choice of the threshold parameters.}

\inputdir{depthMig}
\multiplot{2}{semb-0,semb-15}{width=0.45\columnwidth}{Semblance function for (a) zero-degree partial image and (b) 
fifteen-degree partial image.}

In our synthetic example, we used a dip-angle window of~10~degrees\old{and}\new{,} a lateral window of~150~m\new{, and parameters $s_1$ and $s_2$ equal to 0.2 and 0.4, respectively}. Figure~\ref{fig:dpis-cln-0,dpis-cln-15} shows the weighted
partial images. Reflectors consistent with the partial images, as well as appropriate parts of the diffraction event, are preserved. 
At the same time, a large amount of noneffective contribution\new{, which may produce migration noise,} has been eliminated. 
Summation of the weighted partial images provides the stacked image, which has the migration artifacts being suppressed significantly
(Figure~\ref{fig:res-clean}).

\multiplot{2}{dpis-cln-0,dpis-cln-15}{width=0.7\columnwidth}{(a) Zero-degree partial image and (b) fifteen-degree partial image after
migration-aperture optimization.}
\plot{res-clean}{width=0.8\columnwidth}{Image and dip-angle gather after migration-aperture optimization by slope-consistency analysis.}

\old{Local summation in the dip-angle direction by itself attenuates}\new{A weight function constructed by using the dip-angle direction only
allows for attenuating} reflection events around their apex areas while preserving diffraction events \new{as well}. 
Figure~\ref{fig:res-clean1} shows the result of migrated data weighting using local summation \old{along migration dips only}\new{along dip-angle
window of 10 degrees}.
Reflection events are localized around their effective areas, and the diffraction event has its full width being preserved. However,
the image remains contaminated by migration artifacts, which happen to be stationary in the dip-angle direction.
\plot{res-clean1}{width=0.8\columnwidth}{Image and dip-angle gather after migration-aperture optimization by local summation in the dip-angle
 direction.}

We next address the situation when migration velocity is not correct. Figure~\ref{fig:fres-init} shows the image and the dip-angle 
gather after migration with 10\% lower velocity. Reflection boundaries look distorted. The diffractor is undermigrated and has a complicated
shape in the dip-angle gather. The image is contaminated by strong migration artifacts. For migration-aperture optimization by
slope-consistency analysis, we used the same parameters as in the previous example (Figure~\ref{fig:fres-clean}).
Reflection and diffraction events both are optimized to their effective parts, eliminating migration noise in the image.

\plot{fres-init}{width=0.8\columnwidth}{Initial image and common-image-gather in dip-angle domain corresponding to distance of 7~km (wrong velocity case).}
\plot{fres-clean}{width=0.8\columnwidth}{Image and dip-angle gather after migration-aperture optimization by slope-consistency analysis (wrong velocity case).}

These experiments demonstrate that slope-consistency filtering can be used after migration even \new{when} using a complicated velocity model or an
incorrect model. \new{In the previous section, we considered theoretically the case of the zero-offset migration. 
The proposed principles may apply also to prestack migration in which stacking of the full offset range provides accumulation of
effective migrated data parts.}
The experiments also reveal that noneffective segments may have an appropriate slope as well. For instance, the optimized fifteen-degree
partial image (Figure~\ref{fig:dpis-cln-15}) has some events at a depth of 1.5~km. These events appear to have a correct slope that is 
close to 15~degrees, and they therefore evade nonconsistent-slope filtering and bring noise to the image of the flat reflector. This kind
of noise is inherent in Kirchhoff migration and is produced by the complexity of the velocity model \new{and incompleteness of the migrated 
data (\citealp[]{stolk04})}. In practice, it can be eliminated by \old{prestack migration using}\new{stacking} the full range of offsets.

The proposed approach can be used easily in three dimensions, in which migrated events are defined by two orthogonal \old{angles}\new{directions}
(\citealp[]{klokov12}). Migrated constant-dip traces compose a 3D~volume, and slope-consistency analysis should be performed along a plane
surface whose slope is defined
by the two migration dips. The method does not require slope estimation and has a relatively low computational cost.

\new{Note that the effectiveness of the method is conditioned to that of semblance estimation. In the case of conflicting events, semblance
operator effectively deals with events that are relatively strong to be detected. In practice, weak diffraction events might get suppressed
by reflections and noise, which might cause difficulties in the slope-consistency estimation and, consequently, in the events protection.
These weak events can be enhanced by diffraction imaging techniques (\citealp[]{fomel07,klokov12})}.

\section{Synthetic data examples}

We next tested the presented approach on \new{prestack migration of} the Sigsbee2B \old{model}\new{data set} \cite[]{paffenholz02}.
The model contains a number of artificial point scatterers. The top of the salt \new{body} has a high curvature that acts as a transition between 
reflections and diffractions. There is a number of faults, which produce diffraction energy as well. Therefore, the data set is 
appropriate for testing of \old{migration-aperture}diffraction protection.

Figure~\ref{fig:sigs-input} shows a seismic image of the left part of the model and one dip-angle gather 
extracted from the position
in which two strong diffraction points are present. The gather contains two flat diffraction events at depths of 5.2 and 7.6~km,
which correspond to the artificial scatterers. \old{Also,}Some diffraction energy is \new{also} found at a depth of 4.3~km scattered by the fault.

Figure~\ref{fig:sigs-ap} shows the image and dip-angle gather after migration-aperture limiting. \new{First}, we detected reflection apex positions,
estimated effective dip ranges around them, and rejected migrated amplitudes that did not correspond to effective dip intervals. This procedure enables
elimination of migration noise with correct imaging of reflection boundaries \cite[]{bienati09}. However, aperture limiting causes significant shortening 
of diffraction events and, as a result, weakening, or even disappearance, of diffraction objects in the image.

To optimize the migration aperture using the proposed slope-consistency approach, we performed semblance analysis using a dip-angle
aperture of 10 degrees \old{and}\new{,} a lateral base of 200~m\new{, and parameters $s_1$ and $s_2$ equal to 0.2 and 0.4, respectively.}
(Figure~\ref{fig:sigs-semb}). Reflection events become limited to the vicinity of their apexes, as in the
previous example. At the same time, the procedure protects diffraction events in the gather and, hence, the corresponding 
diffraction objects in the image. Artificial point scatterers and faults are imaged correctly.

Figure~\ref{fig:pimage15-input-l}
demonstrates one constant-dip partial image corresponding to 15 degrees. It resembles the seismic
image of the model, with some differences. \new{As expected, dipping}\old{Some of the} reflection boundaries are shifted from the correct positions\old{, for example,} and diffraction objects
are represented by hyperbolic events. After slope-consistency analysis, we construct a filtering mask (Figure~\ref{fig:pimage15-mask})
to protect events whose slope is close to that of the dip of the partial image (Figure~\ref{fig:pimage15-semb-l}).
\new{As a result,} diffraction hyperbolas \old{were shortened}\new{became confined} to the vicinities of their correct positions, where they have
the appropriate slope. Faults at a depth of 4 km and \old{some}\new{other} reflection boundaries were protected as well. Remaining migrated energy was eliminated 
significantly.

\inputdir{sigsbee}
\plot{sigs-input}{width=0.8\columnwidth}{Initial image and common-image gather in dip-angle domain corresponding to distance of 6.1 km.}
\plot{sigs-ap}{width=0.8\columnwidth}{Image and dip-angle gather after migration-aperture limiting.}
\plot{sigs-semb}{width=0.8\columnwidth}{Image and dip-angle gather after migration-aperture optimization by slope-consistency analysis.}

\multiplot{3}{pimage15-input-l,pimage15-mask,pimage15-semb-l}{width=0.45\columnwidth}{(a) Fifteen-degree constant-dip partial image, 
(b) weight function produced by slope-consistency analysis, and (c) partial image after migration-aperture optimization. 
Dashed line indicates position of considered artificial scattering points.}

The salt body complicates Kirchhoff imaging in the right part of the Sigsbee model and increases the amount of migration noise.
Figure~\ref{fig:init} shows a seismic image and one dip-angle gather extracted from the position in which strong
diffraction points are present. The scatterer at 7.6~km is illuminated well, and the dip-angle gather contains the corresponding plane
diffraction event. A fault can be identified at the 15-km position and depth of 7~km. \old{A large}\new{Some} amount of migration noise is concentrated just
below the salt and in the salt cavities above.

Migration-aperture optimization by slope-consistency analysis allows us to significantly decrease migration noise and, at the same time, 
to protect diffraction objects --- artificial point scatterers, the fault, and the salt boundary are \new{all} imaged correctly (Figure~\ref{fig:sca}). 
\new{In comparison, simple} aperture limiting\old{, in turn,} leads to noise suppression as well; however, the diffraction objects are lost, and the salt body appears
distorted (Figure~\ref{fig:ap}).

\inputdir{rightSigsbee}
\plot{init}{width=0.8\columnwidth}{Initial image and common-image gather in dip-angle domain corresponding to distance of~16.8~km.}
\plot{sca}{width=0.8\columnwidth}{Image and dip-angle gather after migration-aperture optimization by slope-consistency analysis.}
\plot{ap}{width=0.8\columnwidth}{Image and dip-angle gather after migration-aperture limiting.}

\section{Field data application}

We next applied our aperture-optimization technique to a field 3D data set obtained from the Piceance Basin area in Colorado, USA.
Figure~\ref{fig:image3} shows a part of the Williams Fork Formation, which is a well-known reservoir that
contains a number of channels and point-bar sand bodies (\citealp{pranter07}). \cite{will} and \cite{klokov12} previously analyzed
the data by diffraction imaging and showed that a number of diffraction objects were clustered in the top part of the volume.
The image is corrupted by migration artifacts, and diffraction objects are buried under the noise.

We performed slope-consistency analysis using a 10-degree dip-angle windows, a 50-m-square stacking base\new{, and parameters $s_1$ and $s_2$ equal to
0.1 and 0.3, respectively} (Figure~\ref{fig:image-clean}). 
Migration noise appears to be filtered out, whereas point objects in the shallow part are preserved. Diffraction objects are clearly observed
in the slice. 

For comparison, we migrated the data using a limited aperture of 10 degrees in two principal directions (Figure~\ref{fig:image3s}).
Although most of the migration artifacts have been successfully eliminated, diffraction objects from the top part get smeared. 
These objects have low energy caused by scarcity of migration dips in the aperture.

\inputdir{gibson}
\plot{image3}{width=0.99\columnwidth}{Initial 3D image for Piceance Basin data set.}
\plot{image-clean}{width=0.99\columnwidth}{Image after migration-aperture optimization by slope-consistency analysis.}
\plot{image3s}{width=0.99\columnwidth}{Image obtained after migration with limited aperture.}

\section {Conclusions}

Migration-aperture optimization in Kirchhoff angle-domain migration can significantly enhance
the quality of a seismic image. The aperture-optimization
method needs to eliminate migration artifacts without distorting the useful signal. The diffraction component of the seismic wavefield
characterizes small but important geological objects and brings extra resolution to seismic imaging. For an optimal image of diffraction
objects, as wide a migration aperture as possible should be used. However, imaging of reflections requires an aperture narrowed
around the tangent point. 

To meet both requirements, we propose utilizing constant-dip partial images. An analysis of the consistency between local slopes and
constant dips of a partial image makes evaluating the contribution of any part of the migrated data possible. This analysis allows us
to extract constructive events. Stacking these events is equivalent to migration using an optimal aperture.

\section{Acknowledgments}

We thank Igor Ravve, three anonymous reviewers, and the associate editor \new{J\"{o}rg Schleicher} for constructive suggestions that improved the paper.

Partial funding for this project was provided by RPSEA through the “Ultra-Deepwater and Unconventional 
Natural Gas and Other Petroleum Resource” program authorized by the US Energy Policy Act of 2005. 
RPSEA (http://www.rpsea.org/) is a nonprofit corporation whose mission is to provide a stewardship 
role in ensuring the focused research, development and deployment of safe and environmentally
responsible technology that can effectively deliver hydrocarbons from domestic resources to the 
citizens of the United States. RPSEA, operating as a consortium of premier US energy research universities, 
industry, and independent research organizations, manages the program under a contract with the US Department 
of Energy’s National Energy Technology Laboratory.

\appendix
\section{Appendix: Hyperbolic reflector in the dip-angle domain}

For insight into the appearance of reflector images in the dip-angle domain, let us consider the case of a hyperbolic reflector \cite[]{fomel2013}. \new{A special property of hyperbolic reflectors is that they can transform to plane dipping reflectors or point diffractors with an appropriate choice of parameters.} 
Reflector depth is given by the function
\begin{equation}
\label{eq:zx}
z(x)=\sqrt{z_0^2+(x-x_0)^2\,\tan^2\beta}\;,
\end{equation}
and zero-offset reflection traveltime is given by
\begin{equation}
\label{eq:ty}
  t(y)=\frac{2}{v}\,\sqrt{z_0^2+(y-x_0)^2\,\sin^2\beta}\;.
\end{equation}
When the reflector is imaged by time migration in the dip-angle domain
\cite[]{sava03} using velocity $v_m$, point $\{y,t\}$ in
the data domain migrates to $\{x_m,t_m\}$ in the image domain
according to 
\begin{eqnarray}
\label{eq:xm}
x_m & = & y - \frac{v_m}{2}\,t\,\sin{\alpha} = \displaystyle y - \frac{v_m\,\sin{\alpha}}{v}\,\sqrt{z_0^2+(y-x_0)^2\,\sin^2\beta}\;, \\
\label{eq:tm}
t_m & = & t\,\cos{\alpha} = \displaystyle \frac{2\,\cos{\alpha}}{v}\,\sqrt{z_0^2+(y-x_0)^2\,\sin^2\beta}\;,
\end{eqnarray}
where $\alpha$ is the migration dip angle. Eliminating $y$ from
equations~\ref{eq:xm} and \ref{eq:tm}, we arrive at the equation
\begin{equation}
\label{eq:tmxm}
  t_m(x_m) = \frac{2\,\cos{\alpha}}{v}\,\frac{\gamma\,(x_m-x_0)\,\sin{\alpha}\,\sin^2{\beta} + \sqrt{(x_m-x_0)^2\,\sin^2{\beta}+z_0^2\,D}}{D}\;,
\end{equation}
where $\gamma=v_m/v$ and
$D=1-\gamma^2\,\sin^2{\alpha}\,\sin^2{\beta}$. Equation~\ref{eq:tmxm}
describes the shape of the image of the hyperbolic
reflector~(\ref{eq:zx}) in the dip-angle domain.

When the dip of the migrated event, imaged at a correct velocity ($\gamma=1$),
\begin{equation}
  \tan{\alpha_m} = \displaystyle \frac{v}{2}\,t_m'(x_m) = 
  \frac{\cos{\alpha}\,\sin^2{\beta}}{D}\,\left[\sin{\alpha} 
    + \frac{x_m-x_0}{\sqrt{(x_m-x_0)^2\,\sin^2{\beta}+z_0^2\,D}}\right]\;,
  \label{eq:alfam}
\end{equation}
is equal to the dip of the image ($\alpha_m=\alpha$), it also becomes
equal to the true dip of the reflector ($\alpha_m=\alpha_0$), where 
\begin{equation}
\label{eq:alfa0}
\tan{\alpha_0} = z'(x_m) = \frac{(x_m-x_0)\,\tan^2{\beta}}{\sqrt{z_0^2+(x-x_0)^2\,\tan^2\beta}}\;.
\end{equation}
We can specify these conditions for two special cases described next.

\subsection{Point diffractor}

The hyperbolic reflector in equation~\ref{eq:zx} creates a point diffractor at coordinates $\{x_0,z_0\}$ when $\beta=\pi/2$.  In this case, equation~\ref{eq:tmxm}
simplifies to \cite[]{klokov12}
\begin{equation}
\label{eq:dtmxm}
t_m(x_m) = \frac{2\,\cos{\alpha}}{v}\,\frac{\gamma\,(x_m-x_0)\,\sin{\alpha} + \sqrt{(x_m-x_0)^2+z_0^2\,\left(1-\gamma^2\,\sin^2{\alpha}\right)}}{1-\gamma^2\,\sin^2{\alpha}}\;.
\end{equation}
At a correct velocity ($\gamma=1$),
\begin{equation}
\label{eq:dtmxmc}
t_m(x_m) = \frac{2}{v}\,\frac{(x_m-x_0)\,\sin{\alpha} + \sqrt{(x_m-x_0)^2+z_0^2\,\cos^2{\alpha}}}{\cos{\alpha}}\;,
\end{equation}
which is equivalent to equation~\ref{eq:diffevent} in the main text. The dip of the image is
\begin{equation}
  \label{eq:dalfam}
  \tan{\alpha_m} = \frac{v}{2}\,t_m'(x_m) = \tan{\alpha} + \frac{x_m-x_0}{\cos{\alpha}\,\sqrt{(x_m-x_0)^2+z_0^2\,\cos^2{\alpha}}}\;.
\end{equation}
It is easy to verify that, above the diffraction point ($x_m=x_0$), $\alpha_m=\alpha$.

\subsection{Plane dipping reflector}

The hyperbolic reflector in equation~\ref{eq:zx} becomes a plane
dipping reflector when $z_0=0$. In this case, equation~\ref{eq:tmxm}
simplifies to \cite[]{klokov12}
\begin{equation}
  \label{eq:ptmxm}
  t_m(x_m) = \frac{2}{v}\,\frac{(x_m-x_0)\,\cos{\alpha}\,\sin{\beta}}{1-\gamma\,\sin{\alpha}\,\sin{\beta}}\;.
\end{equation}
The dip of the image at a correct velocity is
\begin{equation}
  \label{eq:palfam}
  \tan{\alpha_m} = \frac{v}{2}\,t_m'(x_m) = \frac{\cos{\alpha}\,\sin{\beta}}{1-\sin{\alpha}\,\sin{\beta}}\;,
\end{equation}
which is equivalent to equation~\ref{eq:reflslope} in the main text. The dip of the reflector in this case is simply $\alpha_0=\beta$. It
is easy to verify that when $\alpha=\beta$, $\alpha_m=\alpha$.

\bibliographystyle{seg}
\bibliography{SEG,optApert}

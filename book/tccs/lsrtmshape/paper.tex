\title{Seismic imaging of incomplete data and simultaneous-source data using least-squares reverse-time migration with shaping regularization}
\author{Zhiguang Xue, Yangkang Chen, Sergey Fomel, and Junzhe Sun}
\maketitle

\address{
Bureau of Economic Geology \\
John A. and Katherine G. Jackson School of Geosciences \\
The University of Texas at Austin \\
University Station, Box X \\
Austin, TX 78713-8924 \\
}

\lefthead{Xue et al.}
\righthead{LSRTM with shaping regularization}
\footer{TCCS-10}

\begin{abstract}
Simultaneous-source acquisition improves the efficiency of the seismic data acquisition process.
However, direct imaging of simultaneous-source data may introduce crosstalk artifacts in the final image.
Likewise, direct imaging of incomplete data avoids the step of data reconstruction, but can suffer from migration artifacts.
We propose to incorporate shaping regularization into least-squares reverse-time migration (LSRTM) and 
to use it for suppressing interference noise caused by simultaneous-source data or migration artifacts caused by incomplete data.
To implement LSRTM, we apply lowrank one-step reverse-time migration and its adjoint iteratively in the conjugate-gradient algorithm 
to minimize the data misfit.
A shaping operator imposing structure constraints on the estimated model is applied at each iteration.
We construct the shaping operator as structure-enhancing filtering 
in order to attenuate migration artifacts and crosstalk noise while preserving structural information.
Numerical tests on synthetic models demonstrate that 
the proposed method exhibits a fast convergence rate 
and is effective in attenuating migration artifacts and crosstalk noise.
\end{abstract}

\section{Introduction}
Seismic migration can be regarded as a linear inverse problem, 
and its solution can be obtained by iteratively seeking a model that best fits the observed data. 
This approach is often called least-squares migration (LSM) \cite[]{lsdmo}. 
Its original implementation was with Kirchhoff migration \cite[]{nemeth,duquet}, 
then LSM was developed for one-way wave-equation migration \cite[]{kuehl,juefu2004,clapp05}, 
and more recently it is implemented with reverse-time migration (RTM) \cite[]{multi12,yuzhang,yujin,zeyu15,jie15,junzhe15}.
It has been demonstrated that LSM helps attenuate migration artifacts and provides a sharper, 
better frequency distribution of estimated reflectivity \cite[]{nemeth,kuehl,yaxun07,dutta2014}.
LSM can also be applied to simultaneous-source data \cite[]{beasley,hampson2008} or blended data \cite[]{berkhout} 
where it not only enhances the resolution of the image, 
but also suppresses the crosstalk without any pre-separation of the blended data \cite[]{yaxun09,daiwei09}.
Although LSM using gradient-type iterative methods is able to attenuate migration artifacts and crosstalk, 
its expensive computational cost necessitates a regularization approach that can help get a noise-free imaging result more efficiently.

Angle-domain common-image gathers generated with extended imaging condition \cite[]{sava03,biondi04,sava06}
is perhaps the most straightforward dimension for regularizing LSM 
because the reflection angle dimension is generally continuous.
\cite{prucha02} and \cite{kuehl} applied smoothing regularization along the ray parameter axis in LSM to supress migration artifacts caused by irregular sampling.
\cite{salomons14} added 
a penalization term in LSM for the difference between adjacent angle traces in common-image gathers 
to supress artifacts caused by incomplete illumination.
\cite{stanton15} regularized LSM of elastic data by dip filtering in the angle domain to reduce the effect of source/receier sampling,
noise, and PP/PS crosstalk artifacts.
Although common-image gathers provide a more straightforward domain to impose a regularization for LSM, 
it requires more computation to
construct common-image gathers as well as significantly larger memory to store the image volume.

\cite{sergeyshape1} proposed to incorporate a shaping operator enforcing model compliance to the 
specified shape at each iteration into a conjugate-gradient algorithm 
and showed that shaping regularization provides a better control on the estimated model and often leads to a faster iterative convergence \cite[]{fomel09}.
In this paper, we propose to incorporate shaping regularization into LSRTM. 
We choose structure-enhancing filtering \cite[]{liuyang,swindeman15} as a shaping regularization operator 
for effectively removing noise while preserving structural information. 
The extra computational cost caused by this model constraint is negligible when compared to the cost of LSM.
For our linear modeling/migration operator pair, we choose lowrank one-step modeling and its adjoint \cite[]{junzhe14}.

The paper is organized as follows: First, we present the theory of LSRTM with shaping regularization 
(LSRTM-SR) in application to incomplete data and blended data.
Next, we describe two synthetic examples of supressing migration artifacts introduced by incomplete data and interference caused by blended data.
These experiments demonstrate that the proposed method is highly effective in removing migration artifacts and crosstalk noise.
 
\section{Theory}
Linear seismic modeling can be expressed in a compact linear-algebra notation as:
\begin{equation}
\label{eq:bornmodel}
\mathbf{Lm} = \mathbf{d} \; ,
\end{equation}
where $\mathbf{L}$ is the modeling operator, $\mathbf{m}$ is the reflectivity model, and $\mathbf{d}$ is the observed seismic reflection data.

Conventional migration corresponds to approximating the inverse of $\mathbf{L}$ with its adjoint \cite[]{jon92}:
\begin{equation}
\label{eq:cam}
\hat{\mathbf{m}} = \mathbf{L}^T\mathbf{d} \; ,
\end{equation}  
where $\mathbf{L}^T$ represents the migration operator. 
To implement $\mathbf{L}$ and $\mathbf{L}^T$, we use lowrank one-step modeling and RTM \cite[]{junzhe14,sun15}.

The objective function of regularized LSRTM can be expressed as
\begin{equation}
\label{eq:lsrtm}
J(\mathbf{m})=\mathbf{||(Lm-d)||}^2_2+\epsilon^2||\mathbf{Dm}||^2_2 \; ,
\end{equation}
whose formal minimum is
\begin{equation}
\label{eq:solution}
\hat{\mathbf{m}}=(\mathbf{L}^T\mathbf{L}+{\epsilon}^2 \mathbf{D}^T \mathbf{D})^{-1} \mathbf{L}^T \mathbf{d} \; ,
\end{equation}
where $\mathbf{D}$ is a regularization operator.

Alternatively, \cite{sergeyshape1} defines the least-squares solution with shaping regularization as follows:
\begin{equation}
\label{eq:lsshape}
\hat{\mathbf{m}} = \left[\lambda^2\mathbf{I} + \mathbf{S}(\mathbf{L}^T\mathbf{L}-\lambda^2\mathbf{I})\right]^{-1}\mathbf{S}\mathbf{L}^T\mathbf{d} \; ,
\end{equation}
where $\mathbf{S}$ is a model shaping operator, $\lambda$ is a scaling parameter
and $\mathbf{I}$ is the identity matrix.

The goal of regularization is to impose constraints on the estimated model.
Shaping regularization implements it by explicit mapping of the estimated model to the space of admissible model \cite[]{sergeyshape1,xue15}.
In our approach to image simultaneous-source data, the estimated model contains crosstalk.
The admissible model is the one that has reduced crosstalk.
The two models can be bridged by a shaping regularization operator, which we construct as a structure-enhancing filter
that attenuates noise along the dominant dip direction.
The connection between equations~\ref{eq:solution} and~\ref{eq:lsshape} can be represented as
\begin{equation}
\label{eq:connection}
{\epsilon}^2 \mathbf{D}^T \mathbf{D}={\mathbf{S}}^{-1}-\mathbf{I} \; .
\end{equation}

When the shaping operator is symmetric and representable in the form $\mathbf{S}=\mathbf{HH}^T$, 
equation \ref{eq:lsshape} can be rewritten as
\begin{equation}
\label{eq:shapefinal}
\hat{\mathbf{m}} = \mathbf{H} (\lambda^2\mathbf{I} + \mathbf{H}^T(\mathbf{L}^T\mathbf{L}-\lambda^2\mathbf{I})\mathbf{H} )^{-1}\mathbf{H}^T\mathbf{L}^T\mathbf{d} \; .
\end{equation}
Following \cite{sergeyshape1}, we perform inversion in equation~\ref{eq:shapefinal} iteratively using the conjugate-gradient algorithm \cite[]{hestenes52}.

\subsection{LSRTM-SR for incomplete data}
Due to many physical and economic constraints, seismic data may have limited aperture, coarse wavefield sampling, and gaps \cite[]{nemeth}. 
Direct imaging of incomplete data avoids the steps of interpolation but may suffer from intense artifacts.

In the case of irregularly sampled data, the forward operator in equation \ref{eq:bornmodel} is a combination of forward modeling operator $\mathbf{L}$ and sampling operator $\mathbf{M}$. The new form of equation \ref{eq:bornmodel} is
\begin{equation}
\label{eq:mask}
\mathbf{MLm}=\mathbf{d} \; .
\end{equation}
In our synthetic experiments, we define $\mathbf{M}$ as a simple masking operator.

\subsection{LSRTM-SR for simultaneous-source data}
Simultaneous-source acquisition allows more than one source to be activated at the same time 
regardless of the interference between different sources. 
This approach has attracted much attention from the industry 
because it can help reduce the acquisition cost and improve the quality of acquired seismic data \cite[]{berkhout}. 

For a constant-receiver survey, the simultaneous-source data can be expressed as:
\begin{equation}
\label{eq:blend}
\mathbf{d} = \sum_{j=1}^{M} \mathbf{T}_j\mathbf{d}_j \; ,
\end{equation}
where $\mathbf{d}$ is the blended data, $M$ is the number of simultaneous sources, 
$\sum_{j=1}^{M} \mathbf{T}_j$ denotes the blending operator \cite[]{berkhout}, 
also known as encoding function \cite[]{romero00} or dithering \cite[]{chen14}, 
and $\mathbf{d}_j$ represents an individual unblended shot gather, 
which can be modeled with the modeling operator $\mathbf{L}_j$ associated with the $j$th shot:
\begin{equation}
\label{eq:born}
\mathbf{d}_j = \mathbf{L}_j \mathbf{m} \; .
\end{equation}
Inserting equation \ref{eq:born} into equation \ref{eq:blend}, we obtain
\begin{equation}
\label{eq:blendfinal}
\mathbf{d} = \sum_{j=1}^{M} \mathbf{T}_j\mathbf{L}_j \mathbf{m} \; .
\end{equation}
Thus, in the case of simultaneous-source data, the forward operator $\mathbf{L}$ in equation \ref{eq:bornmodel} 
becomes a combination of modeling operators $\mathbf{L}_j$ and the blending operator $\sum_{j=1}^{M} \mathbf{T}_j$.

% First example
\inputdir{synth}
\subsection{Review of structure-enhancing filtering}
Structure-enhancing filtering is a nonstationary smoothing operator that follows local structural dips. 
\cite{liuyang} realized it by applying an averaging filter along an extended dimension to remove the structurally non-conforming noise. 
The extended dimension is acquired by plane-wave prediction, which predicts a trace from its neighbours according to 
the local slopes of seismic events \cite[]{sergeypaint}.

Figure~\ref{fig:vel-syn,dip,spikes,ptris20,ptris40,ptris90} illustrates the shaping process of structure-enhancing filtering. 
Figure~\ref{fig:vel-syn} shows a simple velocity model.
Figure~\ref{fig:dip} is a local dip estimate obtained using plane-wave destruction \cite[]{fomel2002pwd}.
Figure~\ref{fig:spikes} shows 16 points picked from velocity interfaces. 
We use the local dips shown in Figure~\ref{fig:dip} to smooth the points with structure-enhancing filtering and 
to generate the corresponding impulse responses.
Figures~\ref{fig:ptris20}-\ref{fig:ptris90} show three groups of impulse responses with increasing shaping strength 
controlled by the length of prediction ($ns$ in Figure~\ref{fig:vel-syn,dip,spikes,ptris20,ptris40,ptris90}).
As the length increases, the impulse responses become longer, and finally they blend together and reconstruct the structure of the velocity model.
The directions of the impulse responses follow exactly the structural directions at the corresponding locations in Figure~\ref{fig:vel-syn},
which justifies the effectiveness of structure-enhancing filtering as a shaping regularization operator.

\multiplot{2}{vel-syn,dip,spikes,ptris20,ptris40,ptris90}{width=0.45\textwidth}{Shaping by structure-enhancing filtering. (a) An example velocity; (b) local dip map; (c) 16 points selected as input; (d) impulse responses of structure-enhancing filtering for the 16 points ($ns=20$); (e) impulse responses for a larger prediction radius ($ns=40$); (f) impulse responses for an even larger prediction radius ($ns=90$).}

In application to 3D imaging,
dips of both inline and crossline directions can be estimated, 
and 3D shaping regularization becomes a chain of shaping operations in inline and crossline directions \cite[]{sergeyshape1}.

\section{Numerical Examples}
Our first example is based on the velocity model shown in Figure~\ref{fig:vel-syn}.
We use forward modeling to generate a synthetic data set under both conventional and simultaneous-source acquisitions.
For simultaneous-source acquisition, two shots were excited at nearly the same time. 
Figure~\ref{fig:synshots16} shows the incomplete data obtained after we randomly removed $80\%$ traces.
Figure~\ref{fig:syn2shots16} indicates the simultaneous-source data.

\multiplot{2}{synshots16,syn2shots16}{width=0.45\textwidth}{
Synthetic data for model in Figure~\ref{fig:vel-syn}. 
(a) Incomplete data with $80\%$ traces missing; (b) blended shot gathers.}

In Figure~\ref{fig:syninimg16,synincg20,syninshape5,syninshape20}, 
we compare the effects of different methods on migrating incomplete data.
Figure~\ref{fig:syninimg16} shows the RTM result, 
which has poor illumination and a number of migration artifacts caused by insufficient data.
The image after 20 iterations of LSRTM without any constraints is shown in Figure~\ref{fig:synincg20}.
It contains less artifacts, and the effect of poor illumination has been partially compensated.
Figure~\ref{fig:syninshape5} shows the fifth iteration result of LSRTM-SR.
After five iterations, most of the noise has been removed, however the amplitude of the events is still imbalanced.
The 20th iteration result is shown in Figure~\ref{fig:syninshape20}.
We can find that the image becomes cleaner, and the energy of the weak reflectors has been compensated.

\multiplot{2}{syninimg16,synincg20,syninshape5,syninshape20}{width=0.45\textwidth}{Migration results of incomplete data. (a) RTM image; (b) LSRTM image after 20 iterations; (c) LSRTM-SR image after 5 iterations; (d) LSRTM-SR image after 20 iterations.}

The structure-enhancing filtering employed in LSRTM-SR requires local dip information.
In our experiments, the dip is iteratively estimated at each iteration from the result of the previous iteration.
Figures~\ref{fig:synindip0} and~\ref{fig:synindip19} show the dips used at the first and last iterations of LSRTM-SR.
The initial dip has some large values caused by migration artifacts on the left and right top corners.
These erroneous values gradually disappeared over iterations, which is verified by the fact that
the dip for the last iteration has the same pattern as the dip directly estimated from the exact velocity model, shown in Figure~\ref{fig:dip}.
Figure~\ref{fig:syninfilimg} shows the direct application of shaping operator to the adjoint RTM.
Remaining artifacts are easily observed.
Figure~\ref{fig:synincurve} shows the convergence curves of normalized data error against the iteration number for LSRTM and LSRTM-SR.
The two curves appear similar and both exhibit a fast convergence rate.

\multiplot{2}{synindip0,synindip19,syninfilimg,synincurve}{width=0.45\textwidth}{
Dip estimation and convergence curve for incomplete data case.
(a) Initial dip estimated from RTM image (Figure~\ref{fig:syninimg16}), which is used at the first iteration of LSRTM-SR;
(b) dip used at the last iteration of LSRTM-SR;
(c) image obtained by directly applying shaping operator to RTM image with the initial dip in Figure~\ref{fig:synindip0}; 
(d) data-misfit convergence curves of LSRTM (dashed line) and LSRTM-SR (solid line).}

Figure~\ref{fig:syn2img16,synsimcg20,synsimshape5,synsimshape20} indicates a series of results for the blended data case.
Blended data cause strong interference between layers shown in Figure~\ref{fig:syn2img16}. 
LSRTM without shaping regularization can remove part of the interference and compensate the amplitude of reflectivity (Figure~\ref{fig:synsimcg20}).
A crosstalk-free profile is obtained by LSRTM-SR after 20 iterations (Figure~\ref{fig:synsimshape20}). 
Its convergence curve (Figure~\ref{fig:synsimcurve}) is similar to that of LSRTM 
and reveals that the data error has decreased by more than 90\% after 20 iterations.
Figures~\ref{fig:synsimdip0} and~\ref{fig:synsimdip19} show the dips used at the first and last iterations of LSRTM-SR, respectively.
In comparison, the image obtained by simply applying shaping operator to the output of RTM
still has strong remaining crosstalk artifacts
and cannot get any benefits of LSRTM (Figure~\ref{fig:synsimfilimg}).

\multiplot{2}{syn2img16,synsimcg20,synsimshape5,synsimshape20}{width=0.45\textwidth}{
Migration results of blended data. (a) RTM image; (b) LSRTM image after 20 iterations; 
(c) LSRTM-SR image after 5 iterations; (d) LSRTM-SR image after 20 iterations.}
\multiplot{2}{synsimdip0,synsimdip19,synsimfilimg,synsimcurve}{width=0.45\textwidth}
{ Dip estimation and convergence curve for simultaneous-source case.
(a) Initial dip estimated from RTM image (Figure~\ref{fig:syn2img16}), which is used at the first iteration of LSRTM-SR;
(b) dip used at the last iteration of LSRTM-SR;
(c) image obtained by directly applying shaping operator to RTM image with the initial dip in Figure~\ref{fig:synsimdip0};
(d) convergence curves of LSRTM (dashed line) and LSRTM-SR (solid line);
}

% Second Example
\inputdir{sigsbeesim}
Next, we use a more realistic Sigsbee 2A model \cite[]{paffenholz2002} to test the proposed method. 
Figure~\ref{fig:8shots16-sig} shows the blended data set which has 16 super shot gathers,
with each super shot gather generated by firing 8 sources at the same time.
If directly using RTM to migrate the data, the result contains significant crosstalk, as shown in Figure~\ref{fig:8img16-sig}.
Figure~\ref{fig:simcg10-sig} shows the LSRTM result after 10 iterations, which has strong remaining crosstalk.
Then we switch to use LSRTM-SR. 
The 10th iteration result (Figure~\ref{fig:simshape10-sig})
shows that the crosstalk has been greatly eliminated, while the structural information has been preserved.
Figure~\ref{fig:sigsimdip0} shows the dip estimated from RTM image, and used for the first iteration of LSRTM-SR.
\plot{8shots16-sig}{width=0.9\textwidth}{Synthetic simultaneous-source data for Sigsbee model.}
\multiplot{2}{8img16-sig,sigsimdip0}{width=0.9\textwidth}{RTM result (a) and initial dip (b) of blended data for Sigsbee model.}
\multiplot{2}{simcg10-sig,simshape10-sig}{width=0.9\textwidth}{Inversion results of blended data. (a) LSRTM 10th iteration; (c) LSRTM-SR 10th iteration.}

In this experiment, we enforce different shaping strengths through the inversion, 
a stronger shaping (larger radius) at the first two iterations and a weaker shaping (smaller radius) at the later iterations.
We do that in order to remove the strong crosstalk first and then use the subsequent inversion process to 
reduce data misfit and to recover
small-scale features, such as faults and diffraction points, that may get smeared by the shaping operator at early iterations.
Figure~\ref{fig:part2-8img16-sig,part2-shape2-sig,part2-shape10-sig} shows how this idea works.
Comparing Figures~\ref{fig:part2-8img16-sig} and~\ref{fig:part2-shape2-sig}, 
we can find the crosstalk was already totally suppresed at the 2nd iteration, 
however shaping somewhat smoothed out the diffraction points.
As shown in Figure~\ref{fig:part2-shape10-sig}, this issue has been partially resolved by the subsequent inversion process.
The diffraction points became focused at around the 10th iteration.
Figure~\ref{fig:simcurve-sig} shows the convergence history of normalized data misfit against iteration number for LSRTM and LSRTM-SR.
The two curves are very close, which means the two methods have similar data misfit decrease rate.
However, we can deduce that LSRTM-SR has a faster model misfit convergence 
from the comparison shown in Figure~\ref{fig:simcg10-sig,simshape10-sig}.

\multiplot{3}{part2-8img16-sig,part2-shape2-sig,part2-shape10-sig}{width=0.85\textwidth}{Zoomed sections corresponding to (a) RTM, (b) LSRTM-SR 2nd iteration and (c) LSRTM-SR 10th iteration images respectively.}

\inputdir{sigsbeein}
Figure~\ref{fig:zero46-sig} shows the incomplete data with 60\% of its traces missing.
Just as in the case of simultaneous-source data, 
the same conclusions can be reached for the incomplete data case.
A comparison between Figures~\ref{fig:img46-sig} and~\ref{fig:inshape10-sig} demonstrates that LSRTM-SR can effectively
suppress artifacts caused by the incomplete data.
The convergence history curve (Figure~\ref{fig:incurve-sig}) indicates that the data misfit decreased by almost 90\% after 10 iterations.
The convergence curve of LSRTM is also included in Figure~\ref{fig:incurve-sig}.
\plot{zero46-sig}{width=0.9\textwidth}{Synthetic data for Sigsbee model with 60\% traces missing.}
\multiplot{2}{img46-sig,inshape10-sig}{width=0.9\textwidth}{Comparison between RTM (a) and LSRTM-SR with 10 iterations (b) for incomplete data.}
\multiplot{2}{simcurve-sig,incurve-sig}{width=0.7\textwidth}{Data-misfit convergence history for the simultaneous-source (a) and incomplete (b) data. 
(Dashed line: LSRTM; Solid line: LSRTM-SR.)}

\section{Conclusions}
We propose to incorporate shaping regularization imposing structural constraints on the estimated reflectivity in the process of least-squares migration.
Numerical tests on synthetic models confirm the applicability of the proposed method and show that
least-squares RTM with shaping regularization, while having a similar convergence rate to LSRTM, 
can effectively suppress both migration artifacts caused by incomplete data and interference effects caused by simultaneous-source data.

\section{Acknowledgements}
We thank Aaron Stanton, Daniel Trad, Mauricio Sacchi, the associate editor, and one anonymous reviewer for constructive suggestions.
We thank sponsors of the Texas Consortium for Computational Seismology (TCCS) for financial support. 

\bibliographystyle{seg}
\bibliography{paper}

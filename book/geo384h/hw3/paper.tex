\author{Joseph Fourier}
%%%%%%%%%%%%%%%%%%%%%%%
\title{Homework 3}

\begin{abstract}
  This homework has four parts. 
  \begin{enumerate}
  \item Theoretical questions related to Fourier transform and digital filters.
  \item Data compression using 2-D Fourier transform.
  \item Missing data interpolation using compressive properties of the Fourier transform.
  \item Analyzing your own data.
  \end{enumerate}
\end{abstract}

\section{Prerequisites}

Completing the computational part of this homework assignment requires
\begin{itemize}
\item \texttt{Madagascar} software environment available from \\
\url{http://www.ahay.org/}
\item \LaTeX\ environment with \texttt{SEGTeX} available from \\ 
\url{http://www.ahay.org/wiki/SEGTeX}
\end{itemize}
To do the assignment on your personal computer, you need to install
the required environments. Please ask for help if you don't know where
to start.

The homework code is available from the \texttt{Madagascar} repository
by running
\begin{verbatim}
svn co https://github.com/ahay/src/trunk/book/geo384h/hw3
\end{verbatim}

\section{Theoretical part}

You can either write your answers to theoretical questions on paper or
edit them in the file \texttt{hw3/paper.tex}. Please show all the
mathematical derivations that you perform.

\begin{enumerate}

\item Show that, using the helix transform and imposing helical boundary conditions, it is possible to compute a 2-D digital Fourier transform using 1-D FFT program. Assuming that the input data is of size $N \times N$, would this approach have any computational advantages?

\item The Taylor series expansion of the inverse sine function around zero is
\begin{equation}
  \label{eq:arcsin}
  \arcsin{x} = x + \frac{1}{2}\,\frac{x^3}{3} + 
  \frac{1 \cdot 3}{2 \cdot 4}\,\frac{x^5}{5} + 
  \frac{1 \cdot 3 \cdot 5}{2 \cdot 4 \cdot 6}\,\frac{x^7}{7} + 
  \cdots
\end{equation}
\begin{enumerate}
\item Show how one can use expansion~(\ref{eq:arcsin}) to design a
  digital filter that approximates the derivative
  operator. 

  \textbf{Hint:} Use the identity $1/Z-Z = 2\,i\,\sin(\omega\,\Delta t)$.

\item In particular, find a seven-point derivative filter of the form
\begin{equation}
  \label{eq:d6}
  D(Z) = d_{-3}/Z^{3} + d_{-2}/Z^{2} + d_{-1}/Z + d_0 + 
  d_1\,Z + d_2\,Z^2 + d_3\,Z^3\;.
\end{equation}
\end{enumerate}

\item The triangle smoothing filter of radius $N$ has the $Z$-transform expression
\begin{equation}
\label{eq:triangle}
T_N(Z) = \displaystyle \frac{(1-Z^N)\,(1-Z^{-N})}{N^2\,(1-Z)\,(1-1/Z)}
\end{equation}

\begin{enumerate}
\item Find the Fourier transform of the digital triangle smoothing filter.
\item To approximate triangle smoothing with a non-integer radius $R$,
  we can linearly interpolate between two triangles
\begin{equation}
\label{eq:triangle}
T_R(Z) \approx w\,T_N(Z) + (1-w)\,T_{N+1}(Z)\;, 
\end{equation}
where $N \le R \le N+1$. Find an appropriate expression for the weight
$w$ in terms of $R$ and $N$.
\end{enumerate}

\end{enumerate}

\section{Fourier compression}
\inputdir{compress}

In this exercise, we will use a depth slice selected from a 3-D
seismic volume and shown in Figure~\ref{fig:data} \cite[]{hall}. Notice
a channel structure.

\sideplot{data}{width=\textwidth}{Seismic depth slice with a channel structure.}

\sideplot{fft}{width=\textwidth}{Absolute value of the Fourier transform of 
the seismic slice from Figure~\ref{fig:data}. The circle inside shows
a window selected for compression.}

The goal of your assignment is to find a compressed representation of
the data in the Fourier transform domain. Figure~\ref{fig:fft} shows
the Fourier transform of the data from Figure~\ref{fig:data}. We can
see that most of the energy gets concentrated near the center (zero
frequency).

There are two alternative ways to compress data in the Fourier domain:
\begin{itemize}
\item One approach is to
select a range of frequencies that contain the most important
information. An advantage of this approach is the ability to subsample
the original data by transforming back from a windowed range of frequencies.
The results from this method are shown in Figure~\ref{fig:sig,cut}.
\item Another approach is to zero all Fourier coefficients below a certain threshold value, regardless of which frequencies they represent.  
 The results from this method are shown in Figure~\ref{fig:thr,noi}. 
Figure~\ref{fig:hist} shows the selected threshold plotted against the histogram of Fourier coefficients.
\end{itemize}

\multiplot{2}{sig,cut}{width=0.45\textwidth}{Data separated into signal (a) and noise (b) by applying Fourier compression with windowing.}
\multiplot{2}{thr,noi}{width=0.45\textwidth}{Data separated into signal (a) and noise (b) by applying Fourier compression with thresholding.}

\sideplot{hist}{width=0.8\textwidth}{Normalized histogram of Fourier coefficients (by absolute value). The vertical line shows the selected threshold.}

\begin{enumerate}
\item Change directory to \texttt{hw3/compress}.
\item Run 
\begin{verbatim}
scons view
\end{verbatim}
to reproduce the figures on your screen.
\item Modify the \texttt{SConstruct} file to decrease the size of the window so that the noise level increases in Figure~\ref{fig:cut}. How do you measure the noise level? Find a level that you find negligibly small.
\item Modify the \texttt{SConstruct} file to increase the threshold value so that the compression achieves the same quality as in the previous case. The noise level in Figure~\ref{fig:noi} should match that in Figure~\ref{fig:cut}.
\item Compare the number of nonzero Fourier coefficients in both cases. Which method achieves a better compression?
\item \textbf{EXTRA CREDIT} for finding a way for a better compression of the data in the Fourier domain. Your data reconstruction should have 
the same noise level, yet the number of non-zero coefficients in the Fourier domain should be smaller.
\end{enumerate}

\lstset{language=python,numbers=left,numberstyle=\tiny,showstringspaces=false}
\lstinputlisting[frame=single]{compress/SConstruct}

\section{Projection onto convex sets}
\inputdir{pocs}

\sideplot{hole}{width=\textwidth}{Seismic depth slice
  after removing selected parts of the data.}

The goal of the next exercise is to figure out if one can use compactness
of the Fourier transform to reconstruct missing data. The missing
parts are created artificially by cutting holes in the original data
(Figure~\ref{fig:hole}).

\multiplot{2}{fft-data,fft-hole}{width=0.4\textwidth}{Fourier transform of the original data (a) and data with holes
  with holes (b). The absolute value is displayed}

\sideplot{fft-mask}{width=0.8\textwidth}{Fourier-domain mask for selecting a convex set.}

Figures~\ref{fig:fft-data} and~\ref{fig:fft-hole} show the digital
Fourier transform of the original data and the data with holes. We
observe again that the support of the data in the Fourier domain is
compact thanks to the data smoothness. Cutting holes in the physical
domain creates discontinuities that make the Fourier response spread
beyond the original support. Figure~\ref{fig:fft-mask} shows a
Fourier-domain mask designed to contain the support of the original
data.

To accomplish the task of missing data interpolation, we will use an
iterative method known as POCS (\emph{projection onto convex
sets}). By definition, a convex set $\mathcal{C}$ is a set of
functions such that, for any $f_1(\mathbf{x})$ and $f_2(\mathbf{x})$
from the set, $g(\mathbf{x}) = \lambda\,f_1(\mathbf{x}) +
(1-\lambda)\,f_2(\mathbf{x})$ (for $0 \le \lambda \le 1$) also belongs
to the set. A projection onto a convex set means finding a function in
the set that is of the shortest distance to the given function. The
POCS theorem states that if one wants to find a function that belongs
to the intersection of two convex sets $C_1$ and $C_2$, the task can
be accomplished iteratively by alternating projections onto the two
sets \cite[]{youla}.

In our example, $C_1$ is the set of all functions that are equal to
the known data outside of the holes. $C_2$ is the set of all functions
that have a predifined compact support in the Fourier domain (and
therefore are smooth in the physical domain). The algorithm consists
of the following steps:
\begin{enumerate}
\item Apply 2-D Fourier transform. 
\item Multiply by a Fourier-transform mask to enforce compact support.
\item Apply inverse 2-D Fourier transform.
\item Replace data outside of the holes with known data.
\item Repeat.
\end{enumerate}
The output after 5 iterations is shown in Figure~\ref{fig:pocs}.

\sideplot{pocs}{width=\textwidth}{Missing data interpolated by
  iterative projection onto convex sets.}

\lstinputlisting[frame=single]{pocs/SConstruct}

Your task:
\begin{enumerate}
\item Change directory to \texttt{hw3/pocs}
\item Run 
\begin{verbatim}
scons view
\end{verbatim}
to reproduce the figures on your screen.
\item Additionally, you can run
\begin{verbatim}
scons pocs.vpl
\end{verbatim}
to see a movie of different iterations.
\item By modifying appropriate parameters in the \texttt{SConstruct} file and repeating computations,
find out
\begin{enumerate}
\item How many iterations are required for convergence?
\item How large can you make the holes and still be able to achieve a reasonably good reconstruction?
\end{enumerate}
\item \textbf{EXTRA CREDIT} for finding a different convex set for either faster or more accurate missing data reconstruction.
\end{enumerate}


\section{Your own data}

Your final task is to apply the Fourier compression or interpolation technique
to your own data:
\begin{enumerate}
\item Select a dataset suitable for compression. 
\item Apply one of the algorithms of the previous two sections and choose appropriate parameters.
\item Include the results in your homework.
\end{enumerate}

\section{Completing the assignment}

\begin{enumerate}
\item Change directory to \texttt{hw3}.
\item Edit the file \texttt{paper.tex} in your favorite editor and change the
  first line to have your name instead of Fourier's.
\item Run
\begin{verbatim}
sftour scons lock
\end{verbatim}
to update all figures.
\item Run
\begin{verbatim}
sftour scons -c
\end{verbatim}
to remove intermediate files.
\item Run
\begin{verbatim}
scons pdf
\end{verbatim}
to create the final document.
\item Submit your result (file \texttt{paper.pdf}) on paper or by
e-mail.
\end{enumerate}

\bibliographystyle{seg}
\bibliography{hw3}

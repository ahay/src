\author{Joseph Fourier} 
%%%%%%%%%%%%%%%%%%%%%%%%%
\title{Homework 5}

\begin{abstract}
  This homework has only a computational part but it will require you
  to make some theoretical developments as well.  The theme of the
  homework is one-way wavefield extrapolation. You will develop
  efficient approximations for one-way wave extrapolators for a
  constant velocity and for a highly variable velocity.
\end{abstract}

\section{Computational part}

\lstset{language=python,numbers=left,numberstyle=\tiny,showstringspaces=false}

\begin{enumerate}

\item The first example is the familiar model with a hyperbolic
  reflector under a constant velocity layer.  The model is shown in
  Figure~\ref{fig:model}. Figure~\ref{fig:shot} shows a shot gather
  modeled at the surface and extrapolated to a level of 1 km in depth
  using two extrapolation operators: 
  \begin{enumerate}
  \item The exact phase shift filter
    \begin{equation}
      \label{eq:ps}
      \hat{U}(z,k,\omega) = \hat{U}(0,k,\omega)\,e^{i\,\sqrt{S^2\,\omega^2 - k^2}\,z}\;.
    \end{equation}
  \item Its approximation
    \begin{equation}
      \label{eq:psa}
      \hat{U}(z,k,\omega) \approx \hat{U}(0,k,\omega)\,
      e^{i\,S\,\omega\,z}\,
      \frac
      {\displaystyle S\,\omega + \frac{i\,\left(\cos{(k\,\Delta x)}-1\right)\,z}{2\,(\Delta x)^2}}
      {\displaystyle S\,\omega - \frac{i\,\left(\cos{(k\,\Delta x)}-1\right)\,z}{2\,(\Delta x)^2}}\;.
    \end{equation}
    Approximation~(\ref{eq:psa}) is suitable for an implementation in the space domain
    with a digital recursive filter. However, its accuracy is limited,
    which is evident both from Figure~\ref{fig:shot} and from
    Figure~\ref{fig:phase}, which compares the phases of the exact and
    the approximate extrapolators. We can see that
    approximation~(\ref{eq:psa}) is accurate only for small angles
    from the vertical~$\theta$ (defined by $\sin{\theta} = k\,S/\omega$). 
  \end{enumerate}

  \textbf{Your task:} Design an approximation that would be more accurate
    than approximation~(\ref{eq:psa}). Your approximation should be
    suitable for a digital filter implementation in the space
    domain. Therefore, it can involve $k$ only through $\cos{(k\,\Delta
      x)}$ functions.
    \begin{enumerate}
    \item Change directory 
\begin{verbatim}
> cd ~/geo384w/hw5/hyper
\end{verbatim}
    \item Run
\begin{verbatim}
> scons view
\end{verbatim}
      to generate figures and display them on your screen.  
    \item Edit the \texttt{SConstruct} file to change the approximate extrapolator.
    \item Run
\begin{verbatim}
> scons view
\end{verbatim}
      again to observe the differences.
    \end{enumerate}

\inputdir{hyper}

\plot{model}{width=\textwidth}{Synthetic velocity model with a hyperbolic reflector.}

\plot{shot}{width=\textwidth}{Synthetic shot gather. Left: Modeled for receivers at the surface. Middle: Receivers extrapolated to 1 km in depth with an exact phase-shift extrapolation operator. Right: Receivers extrapolated to 1 km in depth with an approximate extrapolation operator.}

\sideplot{phase}{width=\textwidth}{Phase of the extrapolation operator at 5 Hz frequency as a function of the wave propagation angle. Solid line: exact extrapolator. Dashed line: approximate extrapolator.}

{\small
  \lstinputlisting[frame=single]{hyper/SConstruct}}

\item The second example is the Sigsbee synthetic velocity model that
  we also encountered previously. The model is shown in
  Figure~\ref{fig:vel}. We will select one slice of the model (at 15
  kft depth) to analyze different wavefield
  extrapolators. Figure~\ref{fig:phase2} compares the phases of two one-way wave
  extrapolation operators:
  \begin{enumerate}
  \item The exact non-stationary phase shift filter
    \begin{equation}
      \label{eq:nps}
      E(k,x) = e^{i\,\sqrt{S^2(x)\,\omega^2 - k^2}\,\Delta z}
    \end{equation}
  \item Its split-step approximation
    \begin{equation}
      \label{eq:npsa}
      E(k,x) \approx e^{i\,S(x)\,\omega\,\Delta z}\,e^{i\,\sqrt{S_0^2\,\omega^2 - k^2}\,\Delta z-i\,S_0\,\omega\,\Delta z}
    \end{equation}
    Approximation~(\ref{eq:npsa}) can be implemented efficiently, because it involves
    only diagonal matrix multiplications in space and wavenumber domains. However, its accuracy is limited.
  \end{enumerate}   
  
  \textbf{Your task:} Design an approximation that would be more accurate
  than approximation~(\ref{eq:npsa}). Your approximation should be
  suitable for an efficient implementation. It can involve products of functions
  of $x$  and functions
  of $k$ but not any functions that mix $x$ and $k$. A possible exception is $\cos{(k\,\Delta x)}$, which can be mixed
  with functions of $x$ for a non-stationary digital filter implementation in the space domain.
    \begin{enumerate}
    \item Change directory 
\begin{verbatim}
> cd ~/geo384w/hw5/sigsbee
\end{verbatim}
    \item Run
\begin{verbatim}
> scons view
\end{verbatim}
      to generate figures and display them on your screen.  
    \item Edit the \texttt{SConstruct} file to change the approximate extrapolator.
    \item Run
\begin{verbatim}
> scons view
\end{verbatim}
      again to observe the differences.
    \end{enumerate}

\inputdir{sigsbee}

\plot{vel}{width=\textwidth}{Sigsbee velo(city model. A slice of the model at 15 kft depth is selected for analyzing wavefield extrapolation operators.}

\plot{phase2}{width=\textwidth}{Phase of the wavefield extrapolation operators for a depth slice at 15 kft and frequency of 5 Hz as a function of space $x$ and wavenumber $k$. Top: Exact mixed-domain extrapolator. Bottom: Split-step approximation.}

{\small
  \lstinputlisting[frame=single]{sigsbee/SConstruct}}

\end{enumerate}

\section{Completing the assignment}

\begin{enumerate}
\item Change directory to \verb#~/geo384w/hw5#.
\item Edit the file \texttt{paper.tex} in your favorite editor and change the
  first line to have your name instead of Fourier's.
\item Run
\begin{verbatim}
  > sftour scons lock
\end{verbatim}
to update all figures.
\item Run
\begin{verbatim}
  > sftour scons -c
\end{verbatim}
  to remove intermediate files.
\item Run
 \begin{verbatim} 
  > scons pdf
\end{verbatim}
  to create the final document.
\item Submit your result (file \texttt{paper.pdf}) on paper or by
  e-mail. If you do your assignment on one of the computers in the
  Unix lab, you can simply leave the file in your directory.
\end{enumerate}

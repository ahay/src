\author{Charles Hewitt Dix} 
%%%%%%%%%%%%%%%%%%%%%%%%%
\title{Homework 6}

\begin{abstract}
  In this (after class) homework we return to the Blake Outer Ridge
  dataset and process it to create an image of the subsurface. Your
  task is open-ended: find a way to improve the image by modifying the
  data processing flow.
\end{abstract}

\lstset{language=python,numbers=left,numberstyle=\tiny,showstringspaces=false}

\section{Computational part}
\inputdir{blake}

The dataset is a 2-D line from the Blake Outer Ridge area
offshore Florida. It was collected by USGS in order to study the
occurrence of methane hydrates. The dataset and its analysis for gas
hydrate detection are described by
\cite{GEO63-05-16591669,GEO65-02-05650573}.

The following figures show the dataset at different stages of seismic
data processing: from initial data to an image in depth. \textbf{Your
  task:} Modify the data processing sequence to create a justifiably
better image.

\plot{nmo}{width=\textwidth}{Common midpoint gather (left),
  velocity analysis panel using normal moveout (middle), and common
  midpoint gather after normal moveout (right). A curve in the middle plot
  indicates an automatically picked velocity trend.}

\multiplot{2}{noff,stack}{width=0.8\textwidth}{(a) Near-offset
  section. (b) Normal moveout stack.}

\multiplot{2}{picks,vel}{width=0.8\textwidth}{(a) Picked stacking
  velocity. (b) Interval velocity estimated by Dix inversion.}

\plot{image}{width=\textwidth}{Seismic image created by converting
  stacked data from time to depth. Can you identify geological
  features that are not properly imaged?}

{\small
  \lstinputlisting[frame=single]{blake/SConstruct}}

\begin{enumerate}
\item Change directory 
\begin{verbatim}
> cd ~/geo384w/hw6/blake
\end{verbatim}
\item Run
\begin{verbatim}
> scons view
\end{verbatim}
to generate figures and display them on your screen.  
\item Edit the \texttt{SConstruct} file.
Check your result by running
\begin{verbatim}
> scons view
\end{verbatim}
again.
\end{enumerate}

\section{Completing the assignment}

\begin{enumerate}
\item Change directory to \verb#~/geo384w/hw6#.
\item Edit the file \texttt{paper.tex} in your favorite editor and change the
  first line to have your name instead of Huygens's.
\item Run
\begin{verbatim}
  > sftour scons lock
\end{verbatim}
to update all figures.
\item Run
\begin{verbatim}
  > sftour scons -c
\end{verbatim}
  to remove intermediate files.
\item Run
 \begin{verbatim} 
  > scons pdf
\end{verbatim}
  to create the final document.
\item Submit your result (file \texttt{paper.pdf}) on paper or by
  e-mail. If you do your assignment on one of the computers in the
  Unix lab, you can simply leave the file in your directory.
\end{enumerate}

\bibliographystyle{seg} 
\bibliography{SEG}


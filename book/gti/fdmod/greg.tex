\title{Multiples in imaging}
\author{Greg Wimpey}
\email{gwimpey@denver.westerngeco.slb.com}

\maketitle

% redo model (no overlaping P and M)
% migrate (dens=n) and analyze and think
% migrate (dens=y) and analyze and think
% look at wavefields

\section{Literature Review}

``Migration by extrapolation of time-dependent boundary values'' In
1983, McMechan introduced a migration method he termed ``boundary
value migration''.  He recognized that exploding-reflector zero-offset
migration can be treated as the reverse of data modeling.  The
recorded data are treated as boundary conditions of the modeling
problem.  The modeling itself is performed using the 2-way wave
equation.  As in other exploding reflector methods, the velocity used
for migration is one-half the true velocity.  Numerical examples of
migration of synthetic data generated by a point scatterer, sloping
plane, vertical plane, and sinusoidal reflector are
shown. \citep{mcmechan83}

``Reverse time migration'' \citeauthor{baysaletal1983} present
reverse-time extrapolation as an alternative to depth extrapolation
for zero-offset or poststack depth migration.  Under the assumption
that stacked data should contain no multiples, Gazdag's
(\citeyear{gazdag81}) 90-degree wave equation is used to perform the
extrapolation:
\[\pm \left [ \frac{\partial^2}{\partial x^2} +
\frac{\partial^2}{\partial y^2} \right ]^{1/2} P = \frac{1}{c(x,z)}
\frac{\partial P}{\partial t}\] The extrapolation is performed by
Fourier transforming the data at time $T$ into the wavenumber domain,
multiplying by the square-root derivative operator, a reverse FFT, and
multiplication by the spatially-variant velocity. The resulting
estimate of the time derivative is used, along with the wavefield at
$T+ \Delta T$ to estimate the wavefield at $T- \Delta T$. The
operation is repeated for each time step. The authors state that the
method avoids numerical dispersion and exponentially growing
evanescent waves. Examples of migration of synthetic datasets are
provided.  \citep{baysaletal1983}

``Reverse-time migration of offset vertical seismic profiling data
using the excitation-time imaging condition.'' Chang and McMechan
introduce the ``excitation time'' imaging condition for finite-offset
data, specifically VSP data. Travel times are calculated from the
source location to each point in the image space using
raytracing. Receiver data are time-reversed and used as input to
forward modeling. Then, at each time step, the travel times are used
as an index into the extrapolated time-reversed wavefield.  The
corresponding amplitudes are summed into the output image.  A minimum
time criterion is used to select from multiple arrivals at an image
point on the source side. The authors note that the direct arrival
must be removed from the data before imaging.  Since muting the direct
arrival also results in muting some reflected data, it is impossible
in the VSP case to image right up to the borehole.  Examples are
provided of the technique using both synthetic and field datasets.
\citep{changmechmechan1986}

``3-D acoustic prestack reverse-time migration'' In 1989, Chang and
McMechan present a method for 3-D prestack migration.  They combine
the boundary-value method with the excitation-time prestack imaging
condition for each point in a 3-D volume.\citep{changandmcmechan1989b}

\section{Theory}

\section{Examples}

\inputdir{greg}

\plot{velo}   {width=\textwidth}{Velocity}
\plot{sour}   {width=\textwidth}{Source data}

\plot{dens1}  {width=\textwidth}{Density}
\plot{rcvr1}  {width=\textwidth}{Receiver data}
\plot{image1n}{width=\textwidth}{Image w/o density}
\plot{image1y}{width=\textwidth}{Image w/  density}

\plot{dens2}  {width=\textwidth}{Density}
\plot{rcvr2}  {width=\textwidth}{Receiver data}
\plot{image2n}{width=\textwidth}{Image w/o density}
\plot{image2y}{width=\textwidth}{Image w/  density}

%\multiplot{4}{vel_model,dens_model,rtm_cdens,rtm_vdens}
%{angle=90,width=0.30\textwidth}{Model and Images}

\nocite{*}

\bibliographystyle{seg}
\bibliography{greg}
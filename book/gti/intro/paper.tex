\documentclass[12pt]{handout}

\begin{document}

\author{Sergey Fomel}
\title{Introduction}

Seismic imaging is responsible for some of the largest petroleum discoveries
in many regions worldwide. It is a glorious field that has been attracting
ingenious contributions from many bright geoscientists, physicists, and
mathematicians.  Analogously to medical X-ray imaging, which is indispensable
for diagnosing internal damages and diseases in living organisms, seismic
imaging is an indispensable tool for studying and ``diagnosing'' the 
interior of the Earth.

In the industrial practice, seismic reflection data pass through
several intermingled stages: field acquisition, processing, and
interpretation.  Somewhere between processing and geological
interpretation (Figure~\ref{fig:imaging}), an important step occurs. A
seismic image is constructed.

In this class, we will examine the main principles, assumptions, and
limitations that lead to this construction. You will learn about
fundamental differences between time and depth imaging, post-stack and
prestack imaging, Kirchhoff and wave extrapolation imaging, and why in
certain geographical areas one is preferred to the other. You will be
given computer exercises to get a hands-on experience with seismic
image generation.

\begin{figure}[h]
\centering
\setlength{\unitlength}{1in}
\begin{picture}(4,1.5)
\put(0,1.0){\makebox(4,0.5){Acquisition}}
\put(2,1.15){\vector(0,-1){0.3}}
\put(0,0.5){\makebox(4,0.5){Processing}}
\put(2,0.65){\vector(0,-1){0.3}}
\put(0,0){\makebox(4,0.5){Interpretation}}
\thicklines
\put(3.10,0.5){\vector(-1,0){1}}
\put(3.25,0.25){\makebox(3,0.5)[l]{\textbf{Seismic Image}}}
\label{fig:imaging}
\end{picture}
\caption{Schematic flow of seismic reflection data.} 
\end{figure}

\section{Geometrical approach}

What is the main objective of seismic imaging? There are several alternative
definitions and approaches to the seismic imaging problem. In this class, we
will use a \emph{geometrical approach}. 

Observe the seismic image in Figure~\ref{fig:elffmg}. If you show this image
to a person unfamiliar with reflection seismology and ask her to describe the
picture, what features she will notice first? The most prominent features are
curved lines (surfaces in 3-D) running across the image. The seismic
interpreter associates those lines and surfaces with geological boundaries
that reflect seismic energy.

\plot{elffmg}{width=6in}{Seismic image from the North Sea. What are
  the main features? Can you identify the salt body? The shallow
  channel? Faults and uncomformities?}

In the seismic reflection experiment, reflection shot records are
recorded for a given shot (seismic energy source) and a collection of
geophones (seismic receivers), typically placed in different locations
on the surface.  Figure~\ref{fig:shot} shows a 2-D synthetic shot
record generated for the famous Marmousi model, shown in
Figure~\ref{fig:model}\footnote{The Marmousi model and the
corresponding synthetic data were generated by IFP in France and
inspired by the complex geology offshore Africa
\cite[]{TLE13-09-09270936}. The data have been used in numerous
seismic imaging experiments.}. The most prominent features in this
picture are again bending curved lines. For the reflection
seismologists, these curves are reflection or diffraction ``events''.

The geometrical approach to seismic imaging associates reflection events,
observed in the recorded data, with seismic reflectors existing in the Earth's
interior. According to this approach, the main goal of seismic imaging is to
transform reflection seismic data in such a way that reflection events take
the position of reflectors. There can be other goals (estimation of the
physical properties of the Earth's material, detecting reflectivity versus
angle, etc.) but all of them are subject to the success of the main goal. 

\boxit{\textbf{The main goal of seismic imaging} is to transform reflection
  data in such a way that reflection events take the position of reflectors.}

\plot{shot}{width=6in}{2-D shot gather generated synthetically for the
  Marmousi velocity model. The shot location is 6000~m. Can you identify
  reflection events? Diffraction events?}

\plot{model}{width=6in}{Marmousi velocity model.}

The term ``imaging'' is used in different scientific and engineering
disciplines (such as medicine and astronomy) and sometimes has a broader
meaning. SIAM (Society for Industrial and Applied Mathematics) formed an
activity group on Imaging Science in 2002. This group defines ``imaging''
broadly as ``reconstruction, enhancement, segmentation, analysis,
registration, compression, representation, and tracking of two and three
dimensional images''.

In reflection seismology, ``imaging'' is often synonymous with ``seismic
migration'', which explicitly refers to the process of ``migrating'' seismic
reflection records to the subsurface reflector positions. ``Imaging'' has a
somewhat broader applicability, because it can describe image construction
even in the cases where the migration procedure is not required.

As we will see later in the class, it is not necessary of even advantageous to
extract the geometry of reflections and reflectors explicitly. The modern
imaging methods operate with recorded waves with the understanding of the
geometry implicitly embedded in them. The relationship between reflection
geometry and wave propagation will be discussed in the next lecture.

\section{Acknowledgments and references}

I learned many ideas in the geometrical approach to
seismic imaging from Sergey Goldin and Jon Claerbout. The fundamental
principles in the general theory of seismic imaging are developed and
presented in the books by \cite{fgdp,iei}, \cite{berkhout}, and
\cite{stolt}. Some of the conventional practices are reviewed by
\cite{scales} and \cite{IG202-00-10012027}.

\bibliographystyle{sep}
\bibliography{intro,SEG}

\end{document}

%%% Local Variables: 
%%% mode: latex
%%% TeX-master: t
%%% TeX-master: t
%%% TeX-master: t
%%% End: 

\published{Geophysics, 75, WB235-WB245, (2010)}
\title{OC-seislet: seislet transform construction with differential offset continuation}

\renewcommand{\thefootnote}{\fnsymbol{footnote}}

\ms{GEO-2009-0420}

\address{
\footnotemark[1] College of Geo-exploration Science and Technology,\\
Jilin University \\
No.6 Xi minzhu street, \\
Changchun, China, 130026 \\
\footnotemark[2] Bureau of Economic Geology,\\
John A. and Katherine G. Jackson School of Geosciences \\
The University of Texas at Austin \\
University Station, Box X \\
Austin, TX, USA, 78713-8924}

\author{Yang Liu\footnotemark[1]\footnotemark[2] and Sergey Fomel\footnotemark[2]}

\footer{GEO-2009-0420}
\lefthead{Liu and Fomel}
\righthead{OC-seislet transform}
\maketitle

\begin{abstract}

 Many of the geophysical data analysis problems, such as signal-noise
 separation and data regularization, are conveniently formulated in a
 transform domain, where the signal appears sparse. Classic transforms
 such as the Fourier transform or the digital wavelet transform, fail
 occasionally in processing complex seismic wavefields, because of the
 nonstationarity of seismic data in both time and space dimensions. We
 present a sparse multiscale transform domain specifically tailored to
 seismic reflection data. The new wavelet-like transform -- the
 \emph{OC-seislet transform} -- uses a differential
 offset-continuation (OC) operator that predicts prestack reflection
 data in offset, midpoint, and time coordinates. It provides high
 compression of reflection events. In the transform domain, reflection
 events get concentrated at small scales. Its compression properties
 indicate the potential of OC-seislets for applications such as
 seismic data regularization or noise attenuation. Results of applying
 the method to both synthetic and field data examples demonstrate that
 the OC-seislet transform can reconstruct missing seismic data and
 eliminate random noise even in structurally complex areas.
\end{abstract}

\section{Introduction}

Digital wavelet transforms are excellent tools for multiscale data
analysis. The wavelet transform is more powerful when compared with
the classic Fourier transform, because it is better fitted for
representing non-stationary signals. Wavelets provide a sparse
representation of piecewise regular signals, which may include
transients and singularities \cite[]{mallat}. In recent years, many
wavelet-like transforms that explore directional characteristics of
images \cite[]{Starck00,Do05,Pennec05,Velisavljevic05} were
proposed. The curvelet transform in particular has found important
applications in seismic imaging and data analysis
\cite[]{douma,chauris,Herrmann2007Let}. \cite{Fomel06} and
\cite{Fomel10} investigated the possibility of designing a
wavelet-like transform tailored specifically to seismic data and
introduced it under the name of the \emph{seislet transform}. Based on
the digital wavelet transform (DWT), the seislet transform follows
patterns of seismic events (such as local slopes in 2-D and
frequencies in 1-D) when analyzing those events at different
scales. The seislet transform's compression ability finds applications
in common data processing tasks such as data regularization and noise
attenuation. However, the problem of pattern detection limits its
further applications. In 2-D, conflicting slopes at a single data
point are difficult to detect reliably even using advanced methods
\cite[]{Fomel02}. It is also difficult to estimate local slopes in the
presence of strong noise. A similar situation occurs in the 1-D case,
in which it is difficult to exactly represent a known seismic signal
using a limited set of frequencies.
\par
Offset continuation is a process of seismic data transformation
between different offsets
\cite[]{Deregowski81,Bolondi82,Salvador82,GEO68-02-07180732}.
Different types of dip moveout (DMO) operators \cite[]{Hale91} can be
regarded as continuation to zero offset and derived as solutions to
initial-value problems with the offset-continuation differential
equation. In the shot-record domain, offset continuation transforms to
shot continuation, which describes the process of transforming
reflection seismic data along shot location
\cite[]{GEO61-06-18461858,Spagnolini96,Fomel03a}. The 3-D analog is
known as azimuth moveout or AMO
\cite[]{GEO63-02-05740588,Fomel03b}. \cite{Bleistein00} developed
 a general platform for Kirchhoff data mapping, which includes offset
 continuation as a special case.
\par
In this paper, we propose to incorporate offset continuation as the
prediction operator into the seislet transform. We design the
transform in the log-stretch-frequency domain, where each frequency
slice can be processed independently and in parallel. We expect the
new seislet transform to perform better than the previously proposed
seislet transform by plane-wave destruction, PWD-seislet transform
\cite[]{Fomel10}, in cases of moderate velocity variations and complex
structures that generate conflicting dips in the data.

\section{Theoretical basis}

OC-seislet transform follows the same construction strategy, the
lifting scheme, as the PWD-seislet transform but employs the
offset-continuation operator for prediction at each transform scale.

\subsection {The lifting scheme for the seislet transform}

The lifting scheme \cite[]{Sweldens95} provides a convenient approach
for designing digital wavelet transforms. The general recipe is as
follows:
\begin{enumerate}
\item Organize the input data as a sequence of records. For 
      OC-seislet transform of 2-D seismic reflection data, the input
      is in the `frequency'-`midpoint wavenumber'-`offset' domain
      after the log-stretched NMO correction \cite[]{Bolondi82},
      and the transform direction is offset.  
\item Divide the data records (along the offset axis in the case of the 
      OC-seislet transform) into even and odd components $\mathbf{e}$
      and $\mathbf{o}$. This step works at one scale level.
\item Find the residual difference $\mathbf{r}$ between the odd
      component and its prediction from the even component: 
   \begin{equation} 
   \label{eq:c}
  \mathbf{{r}} = \mathbf{{o}} - \mathbf{P[{e}]}\;, 
  \end{equation}
  where $\mathbf{P}$ is a \emph{prediction} operator. 
  For example, one can obtain Cohen-Daubechies-Feauveau (CDF) 5/3
  biorthogonal wavelets \cite[]{Cohen92} by defining the
  prediction operator as a linear interpolation between two neighboring
  samples,
  \begin{equation}
  \label{eq:p}
  \mathbf{P[e]}_k = \left(\mathbf{e}_{k-1} + \mathbf{e}_{k}\right)/2\;,
  \end{equation}
   where $k$ is an index number at the current scale level.
\item Find an approximation $\mathbf{c}$ of the data by updating
  the even component:
\begin{equation}
    \label{eq:r}
    \mathbf{{c}}  = \mathbf{{e}} + \mathbf{U[{r}]}\;,
  \end{equation}
where $\mathbf{U}$ is an \emph{update} operator. Constructing the
update operator for CDF 5/3 biorthogonal wavelets aims at preserving 
the running average of the signal
  \cite[]{athome}:
\begin{equation}
  \label{eq:u}
  \mathbf{U[r]}_k = \left(\mathbf{r}_{k-1} + \mathbf{r}_{k}\right)/4\;.
\end{equation}
\item The coarse approximation $\mathbf{{c}}$ becomes the new data,
  and the sequence of steps is repeated on the new data
    to calculate the transform coefficients at a coarser scale level.
\end{enumerate}
 
Next, we define new prediction and update operators using
offset-continuation operators.

\subsection{OC-seislet structure}

We define the OC-seislet transform by specifying prediction and update
operators with the help of the offset-continuation
operator. Prediction and update operators for the OC-seislet transform
are specified by modifying the biorthogonal wavelet construction in
equations~\ref{eq:p} and \ref{eq:u} as follows
\cite[]{Fomel06,Fomel10}:
\begin{eqnarray}
  \label{eq:sp}
  \mathbf{P[e]}_k & = & 
    \left(\mathbf{S}_k^{(+)}[\mathbf{e}_{k-1}] + 
  \mathbf{S}_k^{(-)}[\mathbf{e}_{k}]\right)/2 \\
  \label{eq:su}
  \mathbf{U[r]}_k & = & 
     \left(\mathbf{S}_k^{(+)}[\mathbf{r}_{k-1}] + 
    \mathbf{S}_k^{(-)}[\mathbf{r}_{k}]\right)/4\;,
\end{eqnarray}
where $\mathbf{S}_k^{(+)}$ and $\mathbf{S}_k^{(-)}$ are operators that
predict the data record (a common-offset section) by differential
offset continuation from its left and right neighboring common-offset
sections with different offsets. Offset continuation operators provide
the physical connection between data records. The theory of offset
continuation is reviewed in Appendix A.

One can also employ a higher-order transform, for example, by using
the template of the CDF 9/7 biorthogonal wavelet transform, which is
used in JPEG-2000 compression \cite[]{Lian01}. There is only one
stage (one prediction and one update) for the CDF 5/3 wavelet
transform, but there are two cascaded stages and one scaling operation
for CDF 9/7 wavelet transform. Prediction and update operators for a
high-order OC-seislet transform are defined as follows: 
\begin{equation}
  \label{eq:seis01}
  \mathbf{P}_1[\mathbf{e}]_k=(\mathbf{S}_k^{(+)}[\mathbf{e}_{k-1}]+
  \mathbf{S}_k^{(-)}[\mathbf{e}_{k}]) \cdot {\alpha}\;,
\end{equation}
\begin{equation}
  \label{eq:seis02}
  \mathbf{U}_1[\mathbf{r}]_k=(\mathbf{S}_k^{(+)}[\mathbf{r}_{k-1}]+
  \mathbf{S}_k^{(-)}[\mathbf{r}_{k}]) \cdot {\beta}\;,
\end{equation}
\begin{equation}
  \label{eq:seis03}
  \mathbf{P}_2[\mathbf{e}]_k=(\mathbf{S}_k^{(+)}[\mathbf{e}_{k-1}]+
  \mathbf{S}_k^{(-)}[\mathbf{e}_{k}]) \cdot {\gamma}\;,
\end{equation}
\begin{equation}
  \label{eq:seis04}
  \mathbf{U}_2[\mathbf{r}]_k=(\mathbf{S}_k^{(+)}[\mathbf{r}_{k-1}]+
  \mathbf{S}_k^{(-)}[\mathbf{r}_{k}]) \cdot {\delta}\;,
\end{equation}
where the subscripts $1$ and $2$ represent the first and the
second stage. $\alpha$, $\beta$, $\gamma$, and $\delta$ are defined
numerically as follows:
\begin{eqnarray*}
\alpha &=& -1.586134342, \\ 
\beta  &=& -0.052980118, \\ 
\gamma &=& 0.882911076, \\
\delta &=& 0.443506852.
\end{eqnarray*}
One can combine equations~\ref{eq:c}, \ref{eq:r}, 
\ref{eq:seis01}, and \ref{eq:seis02} to finish the first stage, and 
repeatedly process the result by using equations~\ref{eq:c}, \ref{eq:r},
\ref{eq:seis03}, and \ref{eq:seis04}. The scale normalization factors
correspond to the CDF 9/7 biorthogonal wavelet transform
\cite[]{Daubechies98}. Scaling and coefficients
are as follows:
\begin{equation}
  \label{eq:cdf05} \mathbf{e}=\mathbf{e} \cdot K\;,
\end{equation}
\begin{equation}
  \label{eq:cdf06}
  \mathbf{o}=\mathbf{o} \cdot (1/K)\;,
\end{equation} 
where $K$= 1.230174105.
 
We used the high-order version of OC-seislet transform to process the
synthetic and field data examples used in this paper.

\subsection{Simple example}

Figure~\ref{fig:slice} shows a 2-D slice out of the benchmark French
model \cite[]{French74}. We created a 2-D prestack dataset
(Figure~\ref{fig:data,dinput}a) by Kirchhoff modeling. Three sections
in Figure~\ref{fig:data,dinput}a show the time slice at time position
0.6 s (top section), common-offset section at offset position of 0.2
km (bottom-left section), and common-midpoint gather at midpoint
position of 1.0 km (bottom-right section). The reflector with a round
dome and corners creates complicated reflection events along both
midpoint and offset axes. The inflection points of the reflector leads
to traveltime triplications at some
offsets. Figure~\ref{fig:data,dinput}b shows a preprocessed data cube
in the $F$-$K$-offset domain after the log-stretched NMO correction
and a double Fourier transform along the stretched time and midpoint
axes. We apply the OC-seislet transform described above along the
offset axis in Figure~\ref{fig:data,dinput}b. Thus, the offset axis
becomes the scale axis. The cube of the transform coefficients is
shown in Figure~\ref{fig:dfourier,dtran}b and should be compared with
the corresponding Fourier transform along the offset direction in
Figure~\ref{fig:dfourier,dtran}a. The OC-seislet transform
coefficients get concentrated at small scales, which enables an
effective compression. In contrast, the Fourier transform develops
large coefficients at coarser scales but has small residual
coefficients at fine scales. Figure~\ref{fig:compare} shows a
comparison between the decay of coefficients (sorted from large to
small) between the Fourier transform and the OC-seislet transform. A
significantly faster decay of the OC-seislet coefficients is evident.

\inputdir{french}
   \plot{slice}{width=0.75\textwidth}{2-D slice out of the benchmark
                 French model {\cite[]{French74}}.}

   \multiplot{2}{data,dinput}{width=0.7\textwidth}{2-D synthetic
                  prestack data in $t$-$x$-offset
                  domain (a) and in $F$-$K$-offset domain (b).}

   \multiplot{2}{dfourier,dtran}{width=0.7\textwidth}{Fourier
                  transform (a) and OC-seislet transform (b) of the
                  input data from Figure~\ref{fig:data,dinput}b
                  along the offset axis.}

    \plot{compare}{width=0.7\textwidth}{Transform coefficients sorted
                  from large to small, normalized, and plotted on a
                  decibel scale. Solid line: OC-seislet
                  transform. Dashed line: Fourier transform.}

\subsection{Iterative soft-thresholding}

The proposed OC-seislet transform uses physical offset continuation to
compress the reflection data after NMO and log-stretch transform of
the time coordinate, followed by double Fourier transforms of the
stretched time axis and midpoint axis. If seismic traces in the
midpoint direction are missing, the Fourier transform may produce
artifacts (spatial leakage) along the midpoint-wavenumber axis
\cite[]{Zwartjes07}. Additionally, the missing seismic traces in
the offset direction affect the continuity of predicted data. Once
the OC-seislet transform is applied for predicting and compressing,
the artifacts, discontinuity information, and random noise spread over
different scales while the predictable reflection information gets
compressed to large coefficients at small scales. A simple
thresholding method can easily remove the small coefficients of
artifacts. Finally, applying inverse OC-seislet transform, inverse
FFTs both in time and midpoint axes, inverse log-stretch, and inverse
NMO reconstructs the data while attenuating random noise, reducing
artifacts along the midpoint axis, reconstructing continuity in the
offset axis, and recovering main structural features without using any
assumptions about structural continuity in the midpoint-offset-time
domain. The key steps are shown in Figure~\ref{fig:denoise,jpocs}a.

The idea of sparse transforms has been thoroughly explored in the
literature, with application to Fourier and curvelet
transforms. \cite{Liu04} proposed a Fourier-based minimum weighted
norm interpolation (MWNI) algorithm with iterative inversion to
perform multidimensional reconstruction of seismic
wavefields. \cite{Xu05} suggested an iterative Fourier-based matching
pursuit (antileakage Fourier transform) for seismic data
regularization. \cite{Abma06} used a Fourier-based method with
iterative thresholding to solve seismic data interpolation
problems. \cite{Herrman08} presented a recovery method that exploits
the curvelet frame. Missing data interpolation is a particular case of
data regularization, where the input data are already given on a
regular grid, and one needs to reconstruct only the missing values in
empty bins \cite[]{Fomel01a}. For input data with nonuniform
spatial sampling, one can bin the data to a regular grid first and
then use the proposed method for filling empty bins. It is also
possible to generalize the method for combining irregular data
interpolation with a sparse transform \cite[]{Zwartjes07}. The
thresholding iteration helps a sparse transform to recover the missing
information. We adopt a similar thresholding strategy (fixed
thresholding value for each iteration) with the OC-seislet
transform. The key steps of the algorithm are illustrated
schematically in Figure~\ref{fig:denoise,jpocs}b.

\inputdir{XFig}

   \multiplot{2}{denoise,jpocs}{width=0.9\textwidth}{Schematic
              illustration of the thresholding
              workflow. Denoising with a simple soft-thresholding
              (a) and missing data interpolation with iterative
              soft-thresholding (b).}

 \section{Synthetic data tests}

 We continue to use the synthetic model from Figure~\ref{fig:slice} to
 test the proposed method. Figure~\ref{fig:noise} is the data with
 normally-distributed random noise added. Figure~\ref{fig:input,tran}a
 shows the datacube in the $F$-$K$-offset domain after the
 log-stretched NMO correction and a double Fourier transform along the
 stretched time axis and midpoint axis. We run the OC-seislet
 transform in parallel on individual frequency
 slices. Figure~\ref{fig:input,tran}b shows the OC-seislet
 coefficients. All reflection information is concentrated in a small
 scale range. However, since random noise cannot be predicted well by
 the offset-continuation operator, it spreads throughout the whole
 transform domain. The inverse Fourier transform both in time and
 midpoint directions and the inverse log-stretch return the OC-seislet
 coefficients to the time-midpoint-scale domain. The coefficients at
 the zero scale represent stacking along the offset direction, which
 is equivalent to the DMO stack (Figure~\ref{fig:ithr,inv-tran}b).

 We use a soft-thresholding method to separate reflection and random
 noise in the OC-seislet domain. Figure~\ref{fig:ithr,inv-tran}a
 displays the result after the inverse OC-seislet transform. Compared
 with Figure~\ref{fig:input,tran}a, data in the $F$-$K$-offset domain
 contain only useful information. Figure~\ref{fig:nidwt,inver}b shows
 the denoising result in the $t$-$x$-offset domain after the double
 inverse Fourier transform, the inverse log-stretch and the inverse
 NMO. All characteristics of reflection and diffraction events are
 preserved well. For comparison, we used PWD-seislet transform and
 soft-thresholding method with the same threshold values to process
 the noisy data (Figure~\ref{fig:noise}). The result is shown in
 Figure~\ref{fig:nidwt,inver}a. Because PWD-seislet transform is based
 on the local dip information, a mixture of different dips from the
 triplications makes it difficult to process the data in individual
 common-midpoint gathers. The PWD-seislet transform compresses all
 information along the local events slopes in each common-midpoint
 gather. It separates reflection signal and noise but smears the
 crossing events, especially at the far offset. The corresponding
 signal-to-noise ratios for denoised results with PWD-seislet
 transform and OC-seislet transform are 24.95 dB and 41.45 dB,
 respectively. The differences (Figure~\ref{fig:diff1,diff2}) between
 noisy data (Figure~\ref{fig:noise}) and denoised results with
 PWD-seislet transform (Figure~\ref{fig:nidwt,inver}a) and OC-seislet
 transform (Figure~\ref{fig:nidwt,inver}b) further
 illustrate the effectiveness of the OC-seislet
 transform.

\inputdir{french}

   \plot{noise}{width=0.7\textwidth}{2-D noisy data in
                $t$-$x$-offset domain.}

   \multiplot{2}{input,tran}{width=0.7\textwidth}{Noisy data in
                  $F$-$K$-offset domain (a) and OC-seislet
                  coefficients in $F$-$K$-scale domain (b).}

   \multiplot{2}{ithr,inv-tran}{width=0.7\textwidth}{Thresholded data
                 in $F$-$K$-offset domain (a) and OC-seislet
                 coefficients in $t$-$x$-scale domain (b).}

   \multiplot{2}{nidwt,inver}{width=0.7\textwidth}{Denoised result by
		  different methods. PWD-seislet transform
		  (a) and OC-seislet transform (b). (Compare with
		  Figure~\ref{fig:data,dinput}.)}

   \multiplot{2}{diff1,diff2}{width=0.7\textwidth}{Difference
		  between noisy data (Figure~\ref{fig:noise}) and
		  denoised results with different methods
		  (Figure~\ref{fig:nidwt,inver}). PWD-seislet
		  transform (a) and OC-seislet transform (b).}

For a data regularization test, we remove 80\% of randomly selected
traces (Figure~\ref{fig:czero,cinput}a) from the ideal data
(Figures~\ref{fig:data,dinput}a). The complex dip information makes it
extremely difficult to interpolate the data in individual
common-offset gathers. The dataset is also non-stationary in the
offset direction. Therefore, a simple offset interpolation scheme
would also fail. Figure~\ref{fig:czero,cinput}b shows the data after
NMO correction, log-stretch transform, and double Fourier
transforms. The missing traces introduce spatial artifacts in
midpoint-wavenumber axis and discontinuities along the offset
direction. After the OC-seislet transform, the reflection information
can be predicted and compressed. Meanwhile, the artifacts spread over
the whole transform domain (Figure~\ref{fig:ctran,cithr}a). The simple
soft-thresholding algorithm (i.e., the iterative strategy with
only one iteration) removes most of the artifacts, and the inverse
OC-seislet transform reconstructs the major reflection
information according to the offset-continuation prediction
(Figure~\ref{fig:ctran,cithr}b).

\inputdir{frenchint}
   \multiplot{2}{czero,cinput}{width=0.7\columnwidth}{Synthetic data
                 with 80\% traces removed (a) and missing data in
                 $F$-$K$-offset domain (b).}

   \multiplot{2}{ctran,cithr}{width=0.7\columnwidth}{OC-seislet
                 coefficients in $F$-$K$-scale domain (a) and
                 data of iteration 1 after soft-thresholding in
                 $F$-$K$-offset domain (b).}

   \multiplot*{2}{four3pocs,cpocs}{width=0.7\columnwidth}{
                 Interpolated results using iterative thresholding
                 with different sparse transforms. 3-D Fourier
                 transform (a) and OC-seislet transform (b). (Compare
                 with Figure~\ref{fig:data,dinput})}

One can also employ an iterative soft-thresholding strategy to
implement missing data interpolation. This method recovers missing
traces as long as seismic data are sparse enough in the transform
domain. To demonstrate the superior sparseness of the OC-seislet
coefficients, we compare the proposed method with the 3-D Fourier
transform. Figure~\ref{fig:four3pocs,cpocs}a displays the interpolated
result after a 3-D Fourier interpolation using iterative thresholding
\cite[]{Abma06}. The Fourier transform cannot provide enough
sparseness of coefficients for complex reflections and, therefore,
fails in recovering all missing traces. The OC-seislet transform is
based on a physical prediction, and provides a much sparser domain for
both reflections and diffractions. Iterative thresholding succeeds in
interpolating the missing traces (Figure~\ref{fig:four3pocs,cpocs}b).

 \section{Field data tests} 

 We use a historic marine dataset from the Gulf of Mexico
 \cite[]{Claerbout08a} to evaluate the proposed method for both noise
 attenuation and missing data interpolation problems. The input data
 are shown in Figure~\ref{fig:bei,bei-tran}a. Near- and far- offset
 information is completely missing. After log-stretched NMO and
 Fourier transforms along time and midpoint axes convert the original
 data from time-midpoint-offset domain to frequency-wavenumber-offset
 domain, the OC-seislet transform compresses the predictable
 information according to the physical connection between different
 offsets. Unpredictable artifacts and reflections display different
 tendencies: artifacts disperse throughout the whole transform domain
 while reflections are compressed to a small scale range
 (Figure~\ref{fig:bei,bei-tran}b). If we choose the significant
 coefficients using soft-thresholding, the inverse processing steps
 (including the inverse OC-seislet transform, the inverse FFTs both in
 time and midpoint axes, and the inverse log-stretch NMO) effectively
 remove incoherent artifacts from the data
 (Figure~\ref{fig:bei-denoise,bei-diff}a). The difference between
 input data (Figure~\ref{fig:bei,bei-tran}a) and denoised result
 (Figure~\ref{fig:bei-denoise,bei-diff}a) is shown in
 Figure~\ref{fig:bei-denoise,bei-diff}b. Most coherent events
 including reflections and diffractions are well protected by the
 proposed method thanks to the ability of the offset-continuation
 operator to predict structurally complex wavefields. However, some
 energy of the events is partly missing. To achieve a better result,
 one may need to use a more accurate NMO velocity. Thus, denoising is
 a naturally defined operation in the OC-seislet domain. The
 soft-thresholding techniques eliminate only part of spatial aliasing
 because of sparse offset sampling. Additional offset interpolation
 can further remove spatial aliasing.

\inputdir{bei}
  \multiplot{2}{bei,bei-tran}{width=0.8\textwidth}{Marine data in
               $t$-$x$-offset domain (a) and OC-seislet coefficients
               in $F$-$K$-scale domain (b).}

  \multiplot{2}{bei-denoise,bei-diff}{width=0.8\textwidth}{Denoised
                result in $t$-$x$-offset domain (a) and difference
                between input data (Figure~\ref{fig:bei,bei-tran}a)
                and denoised result
                (Figure~\ref{fig:bei-denoise,bei-diff}a) (b).}

For a data regularization test, we removed 70\% of randomly selected
traces from the input data (Figure~\ref{fig:bei-zero}). After applying
the OC-seislet transform and iterative soft-thresholding, spatial
artifacts and offset discontinuity in the transform domain coming from
missing traces get attenuated. The interpolated result is shown in
Figure~\ref{fig:bfour3pocs,bei-pocs}b. Missing traces have been
interpolated well even where diffractions are present. For comparison,
we applied 3-D Fourier transform with iterative soft-thresholding, as
proposed by \cite{Abma06}. The same parameters of soft-thresholding
for those in the OC-seislet transform were used.  The final result is
shown in Figure~\ref{fig:bfour3pocs,bei-pocs}a. The Fourier transform
fails to recover the missing information as accurately as the
OC-seislet transform. Fundamentally, the Fourier coefficients are
simply not sparse enough for this dataset. Figure~\ref{fig:fdmos,dmos}
shows the comparison after DMO stacking. The result from 3-D Fourier
transform displays obvious gaps which are caused by inperfect
interpolation (Figure~\ref{fig:fdmos,dmos}a), while the OC-seislet
transform and iterative soft-thresholding strategy provide a
reasonably accurate result (Figure~\ref{fig:fdmos,dmos}b).

\inputdir{fault}
  \plot{bei-zero}{width=0.8\textwidth}{Marine data with 70\% traces removed.}

  \multiplot{2}{bfour3pocs,bei-pocs}{width=0.8\textwidth}{Interpolated
               results using iterative thresholding with different
               transforms. 3-D Fourier transform (a) and OC-seislet
               transform (b).}

  \multiplot{2}{fdmos,dmos}{width=0.7\textwidth}{Stacking
                of interpolated results corresponding
                to Figure~\ref{fig:bfour3pocs,bei-pocs}. 3-D Fourier
                transform (a) and OC-seislet transform (b).}
		     
 \section{Conclusions}

 We have introduced the OC-seislet transform, a new domain for
 analyzing prestack reflection data in offset, midpoint, and time
 coordinates. Thanks to its physical basis, the new transform is
 able to characterize and compress structurally complex seismic data
 better than either the 3-D Fourier transform or the PWD-seislet
 transform. The offset-continuation operator serves as a
 transformational bridge between different offsets, which allows the
 OC-seislet transform to compress all predictable reflections and
 diffractions into small scales. In the new transform domain, random
 noise spreads over different scales while the predictable reflection
 seismic events get compressed to large coefficients at small
 scales. Therefore, a thresholding approach is successful in
 separating seismic signal and noise. Reconstructing missing seismic
 data is a more difficult task, where iterative thresholding is
 required to help the proposed OC-seislet transform recover the
 missing seismic traces. Tests using both synthetic and field data
 demonstrate that the proposed transform can succeed in handling
 structurally complex data even in the presence of strong random
 noise.

\section{Acknowledgments}
We thank BGP Americas for a partial financial support of this
work. We thank Tamas Nemeth, Mauricio Sacchi, Sandra
Tegtmeier-Last, and two anonymous reviewers for their constructive
comments and suggestions. This publication was authorized by the
Director, Bureau of Economic Geology, The University of Texas at
Austin.

\append{Review of differential offset continuation}

In this appendix, we review the theory of differential offset
continuation from \cite{Fomel03b,GEO68-02-07180732}. The partial
differential equation for offset continuation (differential azimuth
moveout) takes the form
\begin{equation}
\left( \mathbf{h}^T\,\mathbf{P}_{xx}\,\mathbf{h} - 
  h^2\,{\partial^2 P \over \partial h^2} \right) \, = \, 
h\,t_n \, {\partial^2 P \over {\partial t_n \,
\partial h}} \;,
\label{eqn:OCequation3} 
\end{equation}
where $P(\mathbf{x},\mathbf{h},t_n)$ is the seismic data in the
midpoint-offset-time domain, $t_n$ is the time coordinate after the
normal moveout (NMO) correction, $\mathbf{h}^T$ denotes the transpose
of $\mathbf{h}$, and $\mathbf{P}_{xx}$ is the tensor of the
second-order midpoint derivatives.

A particularly efficient implementation of offset continuation results
from a log-stretch transform of the time coordinate
\cite[]{Bolondi82}, followed by a Fourier transform of the stretched
time axis. After these transforms, the offset-continuation equation
takes the form
\begin{equation}
\label{eqn:OC}
\left( \mathbf{h}^T\,\Tilde{\mathbf{P}}_{xx}\,\mathbf{h} - 
  h^2\,\frac{\partial^2 \Tilde{P}}{\partial h^2} \right) \, = \, 
i\,\Omega\,h\,\frac{\partial \Tilde{P}}{\partial h} \;,
\end{equation}
where $\Omega$ is the dimensionless frequency corresponding to the
stretched time coordinate and $\Tilde{P}
(\mathbf{x},\mathbf{h},\Omega)$ is the transformed data. As in other
frequency-space methods, equation~\ref{eqn:OC} can be applied
independently and in parallel on different frequency slices.
\par
In the frequency-wavenumber domain, the extrapolation operator is
defined by solving an initial-value problem for
equation~\ref{eqn:OC}. The analytical solution takes the form
\begin{equation}
  \Tilde{\Tilde{P}}(\mathbf{k},\mathbf{h}_2,\Omega) = 
  \Tilde{\Tilde{P}}(\mathbf{k},\mathbf{h}_1,\Omega)\,
  \frac{Z_{\lambda}(\mathbf{k} \cdot \mathbf{h}_2)}{Z_{\lambda}(\mathbf{k} 
  \cdot \mathbf{h}_1)}\;,
\label{eqn:OKOC}
\end{equation}
where $\Tilde{\Tilde{P}}(\mathbf{k},\mathbf{h},\Omega)$ 
is the double-Fourier-transformed
data, $\lambda = (1 + i \Omega)/2$, $Z_\lambda$ is the special
function defined as
\begin{equation}
Z_{\lambda}(x) = \Gamma(1-\lambda)\,\left(x \over 2\right)^{\lambda}\,
J_{-\lambda}(x)=
{}_0F_1\left(;1-\lambda;-\frac{x^2}{4}\right)\;,
\label{eqn:z}
\end{equation}
$\Gamma$ is the gamma function, $J_{-\lambda}$ is the Bessel function,
and ${}_0F_1$ is the confluent hypergeometric limit function \cite[]{ab}. The
wavenumber $\mathbf{k}$ in equation~\ref{eqn:OKOC} corresponds to the
midpoint $\mathbf{x}$ in the original data domain. In high-frequency
asymptotics, the offset-continuation operator takes the form
\begin{equation}
  \Tilde{\Tilde{P}}(\mathbf{k},\mathbf{h}_2,\Omega) 
  \approx \Tilde{\Tilde{P}}(\mathbf{k},\mathbf{h}_1,\Omega)\,
\frac
{F(\frac{2 \mathbf{k} \cdot \mathbf{h}_2}{\Omega})\,
  \exp{\left[i\Omega\,\psi\left(\frac{2 \mathbf{k} \cdot \mathbf{h}_2}{\Omega}\right)
    \right]}}
{F(\frac{2 \mathbf{k} \cdot \mathbf{h}_1}{\Omega})\,
 \exp{\left[i\Omega\,\psi\left(\frac{2 \mathbf{k} \cdot \mathbf{h}_1}{\Omega}\right)\right]}}\;,
\label{eqn:AOKOC}
\end{equation}
where
\begin{equation}
F(\epsilon)=\sqrt{{1+\sqrt{1+\epsilon^2}} \over
{2\,\sqrt{1+\epsilon^2}}}\,
\exp\left({1-\sqrt{1+\epsilon^2}} \over 2\right)\;,
\label{eqn:F}
\end{equation}
and
\begin{equation}
\psi(\epsilon)={1 \over 2}\,\left(1 - \sqrt{1+\epsilon^2} +
\ln\left({1 + \sqrt{1+\epsilon^2}} \over 2\right)\right)\;.
\label{eqn:psi}
\end{equation}
The phase function $\psi$ defined in equation~\ref{eqn:psi}
corresponds to the analogous term in the exact-log DMO and AMO
\cite[]{GEO55-05-05950607,GEO61-03-08150820,Biondi02}.

\bibliographystyle{seg}
\bibliography{SEG,SEP2,paper}


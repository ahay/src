% You can define macros here

\author{Yujin Liu and William W. Symes}
\title{Extended modeling in Madagascar}{Madagascar course at TRIP}


\section{Introduction}

This guide demonstrates how to implement extended modeling in Madagascar.


\section{Theory}

Acoustic constant density (ACD) Equation:
\begin{equation}
    \nabla^2 U - \frac{1}{v^2} \frac{\partial^2 U}{\partial t^2} = f(t)
\label{eqn:1}
\end{equation}
Here $\nabla$ is Laplacian operator, $f(t)$ is the source wavelet, $v$ is velocity, $U$ is a scalar wavefield.

Extended acoustic constant density (EACD) Equation:
\begin{equation}
    \nabla^2 U - \frac{1}{v^2} \circ \frac{\partial^2 U}{\partial t^2} = f(t)
\label{eqn:2}
\end{equation}
Notes: scalar $\frac{1}{v^2}$ in equation ~\ref{eqn:1} became an operator \cite[]{symes_geopro_2008,krebs_fast_2009,duquet_3d_1999}.


\section{Implementation}
In this part, I will give the implementations in detail.

Discrete form of equation ~\ref{eqn:1} is:

\begin{equation}
U^{n+1}_{i,j} = \left[ \Delta U^{n}_{i,j} -f(t) \right] v^2_{i,j} \Delta t^2 + 2 U^{n}_{i,j} - U^{n-1}_{i,j} \;,
\end{equation}
where $n-1$, $n$ and $n+1$ represent various time steps, ${i,j}$ represent vertical and horizontal space grid respectively.

Discrete form of equation ~\ref{eqn:2} is:
\begin{equation}
U^{n+1}_{i,j} = \sum_k \left[ \Delta U^{n}_{i,j-k} -f(t) \right] v^2_{i,j-k,k} \Delta t^2 + 2 U^{n}_{i,j} - U^{n-1}_{i,j} \;,
\end{equation}
where $n-1$, $n$ and $n+1$ represent various time steps, ${k}$ represent offset grid.

\section{Results}

Model info: constant velocity (v=1.5km/s), nx=151, dx=0.01, nz=151, dz=0.01, fpeak=10hz, nh = 31, oh=-15, dh=dx

\inputdir{flat}
\plot{time200}{width=0.45\textwidth}{Wavefiled at time 0.2s, left is ordinary modeling and right is extended modeling.}
\sideplot{time600}{width=0.45\textwidth}{Wavefiled at time 0.6s, left is ordinary modeling and right is extended modeling.}

\multiplot{2}{time200,time600}{width=0.45\textwidth}{Wavefiled at (a) time 0.2s, (b) time 0.6s (Left is ordinary modeling, right is extended modeling).}


\begin{figure}
    \includegraphics[width=0.45\textwidth]{flat/Fig/time800} 
    \caption{Wavefiled at time 0.8s, left is ordinary modeling and right is extended modeling.}
\label{Fig:noise}
\end{figure}

From figures~\ref{fig:time200}-~\ref{fig:time600}, we can see that they are same when assuming no energy at non-zero offset of extended model.

\section{Conclusions}


This simple document only shows we can use Madagascar as an useful tool and the way how to use it in our research.

Give you the key, open the door by yourself.

\bibliographystyle{seg}
\bibliography{demobib}

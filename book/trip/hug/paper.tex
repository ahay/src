%% This file is automatically generated. Do not edit!
\title{Hug your Data: range stability of seismic modeling}
\author{William. W. Symes \thanks{The Rice Inversion Project,
Department of Computational and Applied Mathematics, Rice University,
Houston TX 77251-1892 USA, email {\tt symes@caam.rice.edu}.}}

\maketitle
\parskip 12pt


\begin{abstract}
Wave equation migration velocity analysis (WEMVA) and WEMVA-like
approaches to Full Waveform Inversion (FWI) avoid the classic
cycle-skipping pitfall of FWI by literal data fit, using extended
models with non-physical degrees of freedom. This algorithm
design poses the obvious question: is it really possible to fit data
with wrong velocities? A theoretical answer is available in the WEMVA,
that is, linearized modeling, context. This paper uses the shot-record
extension to illustrate the how a ray-theoretic criterion (sharing
receiver rays) predicts correctly when data derived from one velocity
model may be accurately fit witn another.
\end{abstract} 

\section{Introduction}
\section{Theory}
\section{Numerical Examples}

\plot{csq24}{width=15cm}{Marmousi velocity model

\maketitle

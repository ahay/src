Custom java program to load trace headers \\
/home/karl/src/geomload1/GsiObsLogLine/InterpText.java \\
 \\
import java.io.*; \\
import java.util.ArrayList; \\
 \\
/** \\
 * @author karl Schleicher 281 701 9032 \\
 */ \\
public class InterpText \{ \\
    ArrayList<Double$>$ xin = new ArrayList<Double$>$(); \\
    ArrayList<Double$>$ yin = new ArrayList<Double$>$(); \\
    InterpText(String filename) throws Exception \{  \\
            this(filename, 1, 2); \\
    \} \\
    InterpText(String filename, int field1, int field2) throws Exception \{ \\
        File file; \\
        BufferedReader reader;       \\
        try \{ \\
           file = new File(filename); \\
           reader = new BufferedReader(new FileReader(file)); \\
        \} catch(Exception ex) \{ \\
           System.out.println(" exception onfile reader for "  \\
	+ filename); \\
            ex.printStackTrace(); \\
            throw new IOException(); \\
        \} \\
        try \{ \\
            String line = null; \\
            String[] tokens = null; \\
            while ((line=reader.readLine()) != null)\{ \\
                double x; \\
                double y; \\
                // trim removes the leading spaces that might look           \\
                // like extra field the \\s+ s regular expression  \\
 	     //to match one or more spaces \\
                tokens = (line.trim()).split("\\s+"); \\
 \\
                if(field1$>$0 \& \& (field1<tokens.length+1) \& \& \\
                   field2$>$0 \& \& (field2<tokens.length+1)) \{ \\
                    try \{ \\
                       x=Double.valueOf(tokens[field1-1]. \\
	               trim()).doubleValue(); \\
                    \} catch (NumberFormatException nfe) \{ \\
                        System.out.println("NumberFormatException on x: " + \\
                                            nfe.getMessage()); \\
                        throw new IOException(); \\
                    \} \\
                    xin.add(x); \\
                    try \{ \\
                        y=Double.valueOf(tokens[field2-1].trim()).doubleValue(); \\
                    \} catch (NumberFormatException nfe) \{ \\
                        System.out.println("NumberFormatException on y: " + \\
                                            nfe.getMessage()); \\
                        throw new IOException(); \\
                    \} \\
                    yin.add(y); \\
                \} \\
                \} \\
            \} \\
        \} catch(Exception ex) \{ \\
            System.out.println(" exception trying to setup file reader for " + filename); \\
            ex.printStackTrace(); \\
            throw new IOException(); \\
        \} \\
	/* \\
          System.out.println("xin    yin"); \\
          for (int i=0; i<xin.size(); i++)\{ \\
            System.out.println(xin.get(i) + \\
                              "    " + yin.get(i)); \\
          \} \\
	*/ \\
   \} \\
    public double getnum(StreamTokenizer stok) throws IOException\{ \\
        double num; \\
        if($ stok.ttype != StreamTokenizer.TT\_EOL \& \& \\
            stok.ttype == StreamTokenizer.TT\_NUMBER$ )\{ \\
            num=stok.nval; \\
        \} else \{ \\
          System.out.println("oh no.  this is not a number!"); \\
          num=0.0; \\
        \} \\
        stok.nextToken(); \\
        return num; \\
    \} \\
    public double linearInterp(double x)\{ \\
        // assume the xin array always increases.  should put in  \\
        // constructor \\
	if (x<=xin.get(0))\{ \\
            return yin.get(0); \\
        \} \\
 \\
        for (int i=0; i<xin.size()-1; i++)\{ \\
            if(x<=xin.get(i+1))\{ \\
                 return yin.get(i)+ \\
                          (x-xin.get(i)) \\
			    *(yin.get(i+1)-yin.get(i))/(xin.get(i+1)-xin.get(i)); \\
            \} \\
        \} \\
        return yin.get(yin.size()-1); \\
    \} \\
    public static void main(String[] args) throws Exception \{ \\
        String spnElevFileName="/home/karl/data/alaska/31-81/spnElev.txt"; \\
        InterpText spnElev = new InterpText(spnElevFileName); \\
         \\
        String recnoSpnFileName="/home/karl/data/alaska/31-81/recnoSpn.txt"; \\
        InterpText recnoSpn = new InterpText(recnoSpnFileName); \\
 \\
       $ File file= new File("/home/karl/jobs/alaska/31-\\
81/load/hdrfile");$ \\
       $ BufferedReader reader = new BufferedReader\\
(new FileReader(file));$ \\
        String line = null; \\
        String[] tokens = null; \\
	// System.out.println("start while file="+ file); \\
 \\
        while ((line=reader.readLine()) != null)\{ \\
	    // System.out.println("in while"); \\
            tokens = (line.trim()).split("\\s+"); \\
            int tracl=Integer.valueOf(tokens[0].trim()); \\
            int fldr =Integer.valueOf(tokens[1].trim()); \\
            int tracf=Integer.valueOf(tokens[2].trim()); \\
 \\
            int ep    =-999999; \\
            int sx    =-999999999; \\
            int sy    =-999999999; \\
            int gx    =-999999999; \\
            int gy    =-999999999; \\
            int cdp   =-999999; \\
            int sutracf=-999999; \\
            int offset=-999999; \\
            double shotSpn=-999999999; \\
            double gSpn=-999999999; \\
            int sdepth=-999; \\
            int selev=-999; \\
            int gelev=-999; \\
            int sstat=-999; \\
            int gstat=-999; \\
            int tstat=-999; \\
            int shot=(int)Math.round(recnoSpn.linearInterp(fldr)); \\
            if(shot$>$ 0 \& \& tracf<97)\{ \\
                ep=shot; \\
                $// make trace 1 negative offset so it sorts to the left \\
                // of a shot record.  On alaska data trace 1 is at a smaller \\
                // surface location than the shot, so: \\
                // offset=(rec_station-shotstation)*station_interval \\
                double roffset=((double)tracf-((96.0+1.0)/2.0))*110.0; \\
                offset=(int)Math.round(roffset); \\
                double r_sx=(double)ep*440.0; \\
                sx=(int)Math.round(r_sx); \\
                sy=0; \\
                gx=(int)Math.round(r_sx+roffset); \\
                gy=0; \\
                cdp=(int)Math.round((((double)sx+(double)gx)/2.0- \\
                         55.0/2.0)/55.0)-752+101; \\
                sutracf=(int)Math.round((gx/440.0+55.0/440.0)*4.0); \\
                shotSpn=((double)sx)/440.0; \\
                gSpn=((double)gx)/440.0; \\
                // elevation at the shot \\
                selev=(int)Math.round(spnElev.linearInterp(shotSpn)); \\
                //elevation at the receiver \\
                gelev=(int)Math.round(spnElev.linearInterp(gSpn)); \\
                if(shotSpn<123)sdepth=(int)Math.round(82.5); \\
                else           sdepth=(int)Math.round(67.5); \\
                sstat=(int)Math.round((-selev+sdepth)*1000.0/10000.0); \\
                gstat=(int)Math.round(-gelev*1000.0/10000.0); \\
                tstat=(int)Math.round((-gelev-selev+sdepth) \\
                                                     *1000.0/10000.0);$ \}\\
            System.out.println(              //tracl  +" "+ \\
                               ep     +" "+ \\
                               sx     +" "+ \\
                               sy     +" "+ \\
                               gx     +" "+ \\
                               gy     +" "+ \\
                               cdp    +" "+ \\
                               sutracf+" "+ \\
                               offset +" "+ \\
                               selev  +" "+ \\
                               gelev  +" "+ \\
                               sstat  +" "+ \\
                               gstat  +" "+ \\
                               tstat  +" "+ \\
                               /* \\
                               shotSpn+" "+ \\
                               gSpn   +" "+ \\
                                * \\
                                */ \\
                               " " \\
                               ); \\
\}  \\  
\}  \\  
\} 
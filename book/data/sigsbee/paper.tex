\title{Sigsbee Models}
\author{Trevor Irons}
\maketitle

\lstset{language=python,numbers=left,numberstyle=\tiny,showstringspaces=false}

\noindent
\textbf	{Data Type:} \emph{2D model and acoustic finite difference synthetic data set with constant density}\\
\textbf	{Source:} \emph{SMAART consortium comprised of BHPBilliton Petroleum, BP, and the ChevronTexaco Exploration and Production Technology Company}\\
\textbf {Location:} \emph{http://www.delphi.tudelft.nl/SMAART/sigsbee2a.htm}\\
\textbf	{Format:} \emph{SEGY} \\
\textbf{
	Date of origin:} \emph{Data were publically released between September 2001 and November 2002.}\\ 

\section{Introduction}
The Subsalt Multiples Attenuation and Reduction Technology Joint Venture (SMAART JV) publicly released several data sets between September 2001 and November 2002.  These synthetic data model the geologic setting found in on the Sigsbee escarpment in the deep water Gulf of Mexico.  Additional information may be found at: \emph{http://www.delphi.tudelft.nl/SMAART/}.  The data sets remain the property of SMAART and are used under the agreement found at: \emph{http://www.delphi.tudelft.nl/SMAART/DataReleaseAgreement.pdf}.  

The file \emph{sigsbee/FILES} lists all files contained in the \emph{sigsbee} repository of Madagascar.  These files may all be downloaded to local machines using ftp protocols.  \emph{Files} is reproduced in \emph{table~\ref{tbl:FILES}} for reference.  
\tabl{FILES}{\emph{FILES} containes a list of all available files in the Madagascar repository} 
{
\tiny
\lstinputlisting[frame=single]{FILES}
\normalsize
}

\section{Sigsbee 2A}
\emph{Sigsbee2A} models the geologic setting found in on the Sigsbee escarpment in the deep water Gulf of Mexico. The model exhibits illumination problems due to the complex salt shape with rugose salt top found in this area. The data set was calculated with an absorbing free surface condition and a weaker than normal water bottom reflection, i.e. the data do not contain free surface multiples and less than normal internal multiples.  A number of normal and thrust faults separate sedimentary blocks. The syncline segments of the salt top focus reflection energy from the salt bottom and the sub salt reflections and produce non-hyperbolic arrival travel time curves.

\subsection{Sigsbee 2A velocity models}
\textbf{Madagascar} can easily display the velocity model used by \emph{Sigsbee2a}.  While \textbf{Madagascar} functions may be called from the terminal command line \textbf{SCons} scripts offer more elegant way to process data.  A \textbf{SCons} script, \emph{SConstruct}, is found within the \emph{RSF/book/data/sigsbee/model2A/} directory of the \emph{RSF book repository} and is presented in \emph{table ~\ref{tbl:2ASConstruct}}. 

\tabl{2ASConstruct}{Contents of \emph{model2A/SConstruct} script.}
{
\tiny
\lstinputlisting[frame=single]{model2A/SConstruct}
\normalsize
}
This script can be ran by entering \texttt{\$ scons view} at the terminal command line within the \emph{sigsbee/model2A} directory.  This set of rules produces images of the stratigraphic and smoothed migrated velocity models as shown in \emph{figure~\ref{fig:vmig2A} and ~\ref{fig:vstr2A}.}  

\inputdir{model2A}
\multiplot{2}{vmig2A,vstr2A}{width=.45\textwidth}{Sigsbee 2A velocity models.}
%\plot{vstr2A}{width=\textwidth}{Sigsbee 2A Stratigraphic velocity.}

A plot of the reflection coefficients are shown in figure \emph{figure~\ref{fig:reflectionCoefficients}}.  
\inputdir{model2A}
\plot{reflectionCoefficients}{width=\textwidth}{Sigsbee model 2A reflection coefficients}

\subsection{Sigsbee 2A shot records}
The \emph{sigsbee 2A} shot records may be displayed using the script found at \emph{data2A/SConstruct} whose contents are displayed in table\emph{~\ref{tbl:data2ASConstruct}.}  This script generates the \textbf{Madagascar} formatted data file \emph{shots.rsf} in addition to shot snapshots.
  
\tabl{data2ASConstruct}{Contents of \emph{data2A/SConstruct} script.}
{
\tiny
\lstinputlisting[frame=single]{data2A/SConstruct}
\normalsize
}

\textbf{Madagascar} contains the function \emph{sfin} that displays header information about each \emph{.rsf} file that it generates.  The output of the command \texttt{sfin shots.rsf} is shown in table\emph{~\ref{tbl:inShot2A}.}  RSF format requires data to be arranged along three axis: time, offset, and shot number.  These three axis are called 1-3 respectively.  Each axis is further categorized by the parameters n,o,and d; where n is the number of datums o is the initial data offset and d is the sampling rate delta.  

\tabl{inShot2A}{Display of function \textbf{sfin} performed on the file \emph{shots.rsf}.}
{
\tiny
\lstinputlisting[frame=single]{data2A/inShot.rsf}
\normalsize
}
Additionally a plot of the wavefield is produced by the \emph{SConstruct} script and is shown in \emph{figure ~\ref{fig:shot}.}

\inputdir{data2A}
\plot{shot}{width=\textwidth}{Snapshot of shot performed on \emph{sigsbee 2A}.}
%%%%%%%%%%%%%%%%%%%%%%%%%%%%%%%%%%%%%%
%%   Sigsbee 2B
%%%%%%%%%%%%%%%%%%%%%%%%%%%%%%%%%%%%%%
\section{Sigsbee 2B}
\subsection{Sigsbee 2B velocity models}
The \emph{Sigsbee 2B} model uses the same structural model as Sigsbee2A but the velocity contrast at the water bottom has been increased to a normal level thus generating significant internal and FS multiples. These data sets are released in October 2002. The Sigsbee2B data set is featured in paper SP3.8 "Observations from the Sigsbee2B synthetic data set" at the 2002 SEG meeting in Salt Lake City.  

\emph{Table~\ref{tbl:model2BSConstruct}} shows the contents of the \emph{SConstruct} script found in the directory \emph{sigsbee/model2A}.  This file is similar to the \emph{SConstruct} file found in the subsection Sigsbee 2A. 

\inputdir{model2B}
\tabl{model2BSConstruct}{Contents of \emph{model2B/SConstruct} script.}
{
\tiny
\lstinputlisting[frame=single]{model2B/SConstruct}
\normalsize
}
This script may be ran similarly by typing \texttt{scons view} at the command line.  The resulting plots are shown in \emph{figures~\ref{fig:vmig2B}, \ref{fig:vstr2B} and ~\ref{fig:reflectionCoefficientsB}.}
\inputdir{model2B}
\multiplot{2}{vmig2B,vstr2B}{width=.45\textwidth}{Sigsbee 2B Velocity Models}
\plot{reflectionCoefficientsB}{width=\textwidth}{Sigsbee 2B reflection coefficients}

\subsection{Sigsbee 2B Shot Records}
The \emph{Sigsbee 2B} library contains two sets of shot data, \emph{nfs} and \emph{fs}.  These shots were modeled with free and non free surface boundary conditions.  

\subsubsection{Free surface model}  
A \emph{SConsctuct} script found at \textit{sigsbee/data2B/fs/} is presented in table~\ref{tbl:fsSConstruct}.  This script reads the \emph{segy} source file and converts it to Madagascar's \emph{RSF} format, \emph{shotFs2B.rsf}.  The contents of this file are displayed in table~\ref{tbl:inShotFs2B}.  The familiar \emph{n,o,} and \emph{d} paramaters are displayed there.  

\tabl{fsSConstruct}{Contents of \emph{data2B/fs/SConstruct} script.}
{
\tiny
\lstinputlisting[frame=single]{fs2B/SConstruct}
\normalsize
}

\tabl{inShotFs2B}{Output of \textbf{sfin} function performed on file \emph{shotFs2B.rsf}}
{
\tiny
\lstinputlisting[frame=single]{fs2B/inShotFs2B.rsf}
\normalsize
}
Shot number 70 is plotted in figure~\ref{fig:shotFs2B}.   

\inputdir{fs2B}
\plot{shotFs2B}{width=\textwidth}{Shot 70 performed on \emph{sigsbee 2B FS} model.}

\subsubsection{No free surface model}
A \emph{SConsctuct} script found at \textit{sigsbee/data2B/fs/} is presented in table~\ref{tbl:nfsSConstruct}.  This script translates the \emph{segy} source data file and converts it into \emph{rsf} format.  The header of the \emph{shotFs2B} file is shown in table~\ref{tbl:inShotNfs2B}. 

\tabl{nfsSConstruct}{Contents of \emph{data2B/nfs/SConstruct} script.}
{
\tiny
\lstinputlisting[frame=single]{nfs2B/SConstruct}
\normalsize
}
\tabl{inShotNfs2B}{Output of \textbf{sfin} on \emph{shotNfs2B.rsf}.}
{
\tiny
\lstinputlisting[frame=single]{nfs2B/inShotNfs2B.rsf}
}

\inputdir{nfs2B}
\plot{shotNfs2B}{width=\textwidth}{Shot 70 performed in \emph{Sigsbee 2B NFS} model.}

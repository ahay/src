\title{Marmousi model}
\author{}
\lstset{language=python,numbers=left,numberstyle=\tiny,showstringspaces=false}

\maketitle
\noindent
\textbf {Data Type:} \emph{Synthetic 2D acoustic model}\\
\textbf {Source:} \emph{Institut Fran\c{c}ais du P\'{e}trole}\\
\textbf {Location:} \emph{http://www.ifp.fr/IFP/en/aa.htm}\\
\textbf {Format:} \emph{Native, origionally Sierra Geophysical Format} \\
\textbf{Date of origin:} \emph{1988}\\

\section{Introduction}
The Marmousi model was created in 1988 by the Institut Fran\c{c}ais du P\'{e}trole (IFP) in 1988.  The geometry of this model is based on a profile through the North Quenguela trough in the Cuanza basin Versteeg. The geometry and velocity model were created to produce a complex seismic data which requiere advanced processing techniques to obtain a correct earth image. The Marmousi dataset was used for the workshop on practical aspects of seismic data inversion at the 52nd EAEG meeting in 1990.

Since its inception in 1990 Marmousi has come be a sort of industry standard and almost classic dataset. The Madagascar repository contains the Marmousi files shown in table \ref{tbl:FILES}.

\tabl{FILES}{A list of all files contained in the Marmousi repository}
{
\tiny
\lstinputlisting[frame=single]{FILES}
\normalsize
}

\section{Model}
The Marmousi model contains 158 horizontally layered horizons.  A series of normal faults and resulting tilted blocks complicates the model towards its center.  The model sits under approximately 32 m of water and is 9.2 km in length and 3 km in depth.  

The velocity model found in the Madagascar repository, \emph{marmvel.hh} can easily be displayed.  This grid contains 751 data points in the Z direction and 2301 data points in the x direction.  Table \ref{tbl:modelHeader} displays the proper header configuration.  

\tabl{modelHeader}{Header information for Marmousi velocity models}
{
\begin{tabular}{|llllll|}
        \hline        
    n1=751    &     d1=4   &        o1=0  &        label= Depth & unit1=m &  \\ 
    n2=2301   &     d2=4   &        o2=0  &        label2=X     & unit2=m &  \\
	\hline
\end{tabular}
}

The script found at \emph{marmousi/model/SConstruct} was written to obtain the Marmousi model datasets, append the headers as necessary and display the  data.  This file is presented in table \ref{tbl:velSConstruct}.
 
\tabl{velSConstruct}{\emph{SConstruct} script generating the Marmousi velocity model images}
{
\tiny
\lstinputlisting[frame=single]{model/SConstruct}
\normalsize
}

\inputdir{model}
\plot{marmvel}{width=\textwidth}{Velocity model}

%\section{Shot Records}

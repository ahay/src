% $Header: /cvsroot/latex-beamer/latex-beamer/solutions/generic-talks/generic-ornate-15min-45min.en.tex,v 1.4 2004/10/07 20:53:08 Titan Exp $

% This file is a solution template for:

% - Giving a talk on some subject.
% - The talk is between 15min and 45min long.
% - Style is ornate.

% Copyright 2004 by Till Tantau <tantau@users.sourceforge.net>.
%
% In principle, this file can be redistributed and/or modified under
% the terms of the GNU Public License, version 2.
%
% However, this file is supposed to be a template to be modified
% for your own needs. For this reason, if you use this file as a
% template and not specifically distribute it as part of a another
% package/program, I grant the extra permission to freely copy and
% modify this file as you see fit and even to delete this copyright
% notice. 
\newcommand{\liubang}{\fontsize{6pt}{\baselineskip}\selectfont}
\newcommand{\qibang}{\fontsize{7pt}{\baselineskip}\selectfont}
\newcommand{\babang}{\fontsize{8pt}{\baselineskip}\selectfont}
\newcommand{\jiubang}{\fontsize{9pt}{\baselineskip}\selectfont}
\newcommand{\shibang}{\fontsize{10pt}{\baselineskip}\selectfont}
\newcommand{\shiwubang}{\fontsize{15pt}{\baselineskip}\selectfont}
\newcommand{\shijiubang}{\fontsize{19pt}{\baselineskip}\selectfont}
\newcommand{\ershierbang}{\fontsize{22pt}{\baselineskip}\selectfont}

\title[Tutorial of Madagascar]{Seismic field data processing example} 

\subtitle{\textcolor{yellow}{---\;\emph{Towards real world}\;---}} % (optional)

\author[Y. Liu]{Yang Liu}
% - Use the \inst{?} command only if the authors have different
%   affiliation.

\institute[BEG, UT at Austin] % (optional, but mostly needed)
{
  Bureau of Economic Geology \\
  Jackson School of Geosciences \\
  The University of Texas at Austin \\
  yangliu1979@gmail.com
}
% - Use the \inst command only if there are several affiliations.
% - Keep it simple, no one is interested in your street address.

\date[Madagascar school 2010] % (optional)
{July 24, 2010}
%{\today}

%\subject{Talks}
% This is only inserted into the PDF information catalog. Can be left
% out. 

% If you have a file called "university-logo-filename.xxx", where xxx
% is a graphic format that can be processed by latex or pdflatex,
% resp., then you can add a logo as follows:

% \pgfdeclareimage[height=0.5cm]{university-logo}{university-logo-filename}
% \logo{\pgfuseimage{university-logo}}

% Delete this, if you do not want the table of contents to pop up at
% the beginning of each subsection:
\AtBeginSection[]
{
  \begin{frame}<beamer>
    \frametitle{Outline}
    \tableofcontents[currentsection,currentsubsection]
  \end{frame}
}

\AtBeginSubsection[]
{
  \begin{frame}<beamer>
    \frametitle{Outline}
    \tableofcontents[currentsection,currentsubsection]
  \end{frame}
}

% If you wish to uncover everything in a step-wise fashion, uncomment
% the following command: 

%\beamerdefaultoverlayspecification{<+->}

%\begin{document}
\begin{frame}
  \titlepage
  \begin{pgfpicture}{0cm}{0cm}{128mm}{2mm}
%\pgfputat{\pgfxy(11.5,1)}{\pgfbox[right,base]{\pgfimage[height=3cm]{../lum/figures/Fig/UTlogo}}} 
  \end{pgfpicture}

\end{frame}

\begin{frame}
  \frametitle{Outline}
  \tableofcontents
  % You might wish to add the option [pausesections]
\end{frame}

% Since this a solution template for a generic talk, very little can
% be said about how it should be structured. However, the talk length
% of between 15min and 45min and the theme suggest that you stick to
% the following rules:  

% - Exactly two or three sections (other than the summary).
% - At *most* three subsections per section.
% - Talk about 30s to 2min per frame. So there should be between about
%   15 and 30 frames, all told.

%  You can create overlays\dots
%  \begin{itemize}
%  \item using the \texttt{pause} command:
%    \begin{itemize}
%    \item
%      First item.
%      \pause
%    \item    
%      Second item.
%    \end{itemize}
%  \item
%    using overlay specifications:
%    \begin{itemize}
%    \item<3->
%      First item.
%    \item<4->
%      Second item.
%    \end{itemize}
%  \item
%    using the general \texttt{uncover} command:
%    \begin{itemize}
%      \uncover<5->{\item
%        First item.}
%      \uncover<6->{\item
%        Second item.}
%    \end{itemize}
%  \end{itemize}
\section{Assignment}

\begin{frame}
  \frametitle{Assignment}
    \begin{block}{}
Field seismic data are always complicated,
a tutorial is given for setting up a
simple processing workflow.
    \end{block}
\end{frame}

\begin{frame}
  \frametitle{Tutorial workflow}

  \begin{itemize}
  \item Conversion to RSF format
  \item Initial data checking
  \item Initial signal analysis and quality control
     \begin{itemize}
       \item First break mute
       \item Subsampling with anti-aliasing filter
       \item Ground-roll attenuation with time-frequency analysis tool
     \end{itemize}
   \item Initial velocity analysis
   \item Brute stacking
   \item Toy prestack time migration
  \end{itemize}
\end{frame}

\section{Simple workflow}

\subsection{Conversion to RSF format}

\addtocounter{framenumber}{-1}
\begin{frame}[fragile]
  \frametitle{\jiubang{Convert to RSF format}}
\babang{
\begin{verbatim}
from rsf.proj import * 

#################################################################
tgz = '2D_Land_data_2ms.tgz'
Fetch(tgz,
      server='http://www.freeusp.org',
      top='RaceCarWebsite/TechTransfer/Tutorials/Processing_2D',
      dir='Data')

files = map(lambda x: 'Line_001.'+x,Split('TXT SPS RPS XPS sgy'))
Flow(files,tgz,
     'gunzip -c $SOURCE | tar -xvf -',stdin=0,stdout=-1)
#################################################################

Flow('line tline','Line_001.sgy','segyread tfile=${TARGETS[1]}')
\end{verbatim}
}
\end{frame}

\addtocounter{framenumber}{-1}
\begin{frame}[fragile]
  \frametitle{\jiubang{Display some shots}}
\qibang{
\begin{columns}
\column{2.2in}
\begin{block}{}
\begin{verbatim}
Result('first','line',
       '''
       window n2=1000 |
       agc rect1=250 rect2=100 |
       grey title="First 1000 traces"
       ''')
\end{verbatim}
\end{block}

\column{2.5in}
\framebox{\includegraphics[width=2.5in]{data/Fig/first}}
\end{columns}
}
\end{frame}

\addtocounter{framenumber}{-1}
\begin{frame}[fragile]
  \frametitle{\jiubang{True geometry}}
\liubang{
\begin{verbatim}
lines = {'S':251,'R':782}
color = {'S':4, 'R':2}
for case in 'SR':
    Flow(case+'.asc','Line_001.%cPS' % case,'''awk 'NR > 20 {print $8, " ", $9}' ''')
    Flow(case,case+'.asc',
         '''
         echo in=$SOURCE data_format=ascii_float n1=2 n2=%d | dd form=native 
         ''' % lines[case],stdin=0)
    Plot(case,
         '''
         scale dscale=0.001 | dd type=complex |
         graph symbol=* title=%c plotcol=%d
         min1=684 max1=705 min2=3837 max2=3842
         ''' % (case,color[case]))
\end{verbatim}
}
\end{frame}

\addtocounter{framenumber}{-1}
\begin{frame}[fragile]
  \frametitle{\jiubang{True geometry}}
\liubang{
\begin{verbatim}
Plot('s118','S',
     '''
     window n2=1 f2=117 | scale dscale=0.001 | dd type=complex |
     graph symbol=O wanttitle=n plotcol=3 symbolsz=4 plotfat=10
     min1=684 max1=705 min2=3837 max2=3842
     ''')
Result('SRO','R S s118','Overlay')
\end{verbatim}
}
\end{frame}

\begin{frame}
  \frametitle{\jiubang{Shots and receivers position}}
    \begin{center}
    \includegraphics[width=0.8\textwidth]{data/Fig/SRO}
    \end{center}
\end{frame}

\addtocounter{framenumber}{-1}
\begin{frame}[fragile]
  \frametitle{\jiubang{Arrange receiver coordinates}}
\liubang{
\begin{verbatim}
# Arrange receiver coordinates 
shots = []
for shot in range(lines['S']):
    line = 'line%d' % shot
    Flow(line,'R','window f2=%d n2=282' % (2*shot))
    shots.append(line)
Flow('rece',shots,'rcat axis=3 ${SOURCES[1:%d]}' % len(shots))
\end{verbatim}
}
\end{frame}

\addtocounter{framenumber}{-1}
\begin{frame}[fragile]
  \frametitle{\jiubang{Plot stacking diagram}}
\liubang{
\begin{verbatim}
Plot('rec','rece',
     '''
     window n1=1 j2=5 j3=2 | scale dscale=0.001 | 
     rtoc | math output="input+I*x2" |
     graph symbol=* plotcol=%d title="Stacking Diagram"
     label1=Distance unit1=km label2="Shot Number"
     min1=684 max1=705
     ''' % color['R'])
Plot('sou','S',
     '''
     window n1=1 j2=2 | scale dscale=0.001 | 
     rtoc | math output="input+I*x1" |
     graph symbol=* plotcol=%d wanttitle=n wantaxis=n min1=684 max1=705
     ''' % color['S'])
Result('diagram','rec sou','Overlay')
\end{verbatim}
}
\end{frame}

\begin{frame}
  \frametitle{\jiubang{Stacking diagram}}
    \begin{center}
    \includegraphics[width=0.8\textwidth]{data/Fig/diagram}
    \end{center}
\end{frame}

\addtocounter{framenumber}{-1}
\begin{frame}[fragile]
  \frametitle{\jiubang{Plot shots using true geometry}}
\liubang{
\begin{verbatim}
# Display one shot using true geometry
Flow('sour','S','spray axis=2 n=282')
Flow('sx','sour','window n1=1 | scale dscale=0.001')
Flow('sy','sour','window n1=1 f1=1 | scale dscale=0.001')
Flow('rx','rece','window n1=1 | scale dscale=0.001')
Flow('ry','rece','window n1=1 f1=1 | scale dscale=0.001')

Flow('offset','sx sy rx ry',
     '''
     math SX=${SOURCES[0]} SY=${SOURCES[1]}
     RX=${SOURCES[2]} RY=${SOURCES[3]}
     output="sqrt((RX-SX)^2+(RY-SY)^2)"
     ''')
\end{verbatim}
}
\end{frame}

\addtocounter{framenumber}{-1}
\begin{frame}[fragile]
  \frametitle{\jiubang{Plot shots using true geometry}}
\liubang{
\begin{verbatim}
Flow('shot118','lines','window n2=1 f2=117')
Flow('offset118','offset','window n2=1 f2=117')
Flow('boff1','offset118','window n1=141 | math output="-input"')
Flow('boff2','offset118','window f1=141')
Flow('boff118','boff1 boff2','cat axis=1 ${SOURCES[1]}')
Result('shot118','shot118 boff118',
       '''
       agc rect1=50 rect2=50 |
       wiggle xpos=${SOURCES[1]} transp=y yreverse=y poly=y
       wherexlabel=t wheretitle=b title="Shot 118"
       ''')
\end{verbatim}
}
\end{frame}

\begin{frame}
  \frametitle{\jiubang{Plot shot 118}}
    \begin{center}
    \includegraphics[width=0.8\textwidth]{data/Fig/shot118}
    \end{center}
\end{frame}

\subsection{Initial data checking}

\addtocounter{framenumber}{-1}
\begin{frame}[fragile]
  \frametitle{\jiubang{Map to regular shot gathers}}
\babang{
\begin{verbatim}
Flow('lines','line',
     '''
     intbin yk=cdpt | window f3=2 |
     put
     label2=Source d2=0.05  o2=688  unit2=km
     label3=Offset d3=0.025 o3=-3.5 unit3=km
     label1=Time unit1=s
     ''')

Result('lines',
       '''
       byte gainpanel=each |
       grey3 frame1=500 frame2=100 frame3=120 flat=n
       title="Raw Data"
       ''')
\end{verbatim}
}
\end{frame}

\begin{frame}
  \frametitle{\jiubang{2-D shot gathers (suppose regular coordinates)}}
    \begin{center}
    \includegraphics[width=0.8\textwidth]{data/Fig/lines}
    \end{center}
\end{frame}

\addtocounter{framenumber}{-1}
\begin{frame}[fragile]
  \frametitle{\jiubang{Convert shots to CMPs}}
\babang{
\begin{verbatim}
Flow('rcmps mask','lines',
     '''
     transp memsize=1000 plane=23 |
     shot2cmp mask=${TARGETS[1]} half=n |
     put o2=-1.75 label2="Half-offset"
     ''')
Result('rcmps',
       '''
       byte gainpanel=each |
       grey3 frame1=500 frame2=35 frame3=643 flat=n
       point1=0.8 point2=0.4 title="CMPs"
       ''')
\end{verbatim}
}
\end{frame}

\begin{frame}
  \frametitle{\jiubang{Display CMPs}}
    \begin{center}
    \includegraphics[width=0.8\textwidth]{data/Fig/rcmps}
    \end{center}
\end{frame}

\addtocounter{framenumber}{-1}
\begin{frame}[fragile]
  \frametitle{\jiubang{Raw stacking}}
\babang{
\begin{verbatim}
Flow('fold','mask','dd type=float | stack axis=1 norm=n')
Flow('rstack','rcmps','stack')
Result('rstack','fold rstack',
       '''
       spray axis=1 n=1501 d=0.002 o=0 label=Time unit=s |
       add scale=1,1000 ${SOURCES[1]} |
       grey color=j title="Raw Stack (with Fold)"
       ''')
\end{verbatim}
}
\end{frame}

\begin{frame}
  \frametitle{\jiubang{Raw stacking}}
    \begin{center}
    \includegraphics[width=0.8\textwidth]{data/Fig/rstack}
    \end{center}
\end{frame}

\subsection{Initial signal analysis and quality control}

\addtocounter{framenumber}{-1}
\begin{frame}[fragile]
  \frametitle{\jiubang{First break mute}}
\babang{
\begin{verbatim}
# Select 4 shots every tenth sequential shot
Flow('inpmute','lines',
     '''
     window f2=198 j2=10 n2=4 |
     transp plane=23
     ''')
Result('inpmute',
       '''
       put n2=1128 n3=1 |
       agc rect1=50 rect2=20 | grey wanttitle=n
       ''')
\end{verbatim}
}
\end{frame}

\begin{frame}
  \frametitle{\jiubang{Input}}
    \begin{center}
    \includegraphics[width=0.8\textwidth]{data/Fig/inpmute}
    \end{center}
\end{frame}

\addtocounter{framenumber}{-1}
\begin{frame}[fragile]
  \frametitle{\jiubang{First break mute}}
\babang{
\begin{verbatim}
# Select muting parameter for background noise
Flow('outmute','inpmute',
     '''
     mutter t0=0.1 v0=5.2
     ''')
Result('outmute',
       '''
       put n2=1128 n3=1 |
       agc rect1=50 rect2=20 | grey wanttitle=n
       ''')
\end{verbatim}
}
\end{frame}

\begin{frame}
  \frametitle{\jiubang{Output}}
    \begin{center}
    \includegraphics[width=0.8\textwidth]{data/Fig/outmute}
    \end{center}
\end{frame}

\addtocounter{framenumber}{-1}
\begin{frame}[fragile]
  \frametitle{\jiubang{Apply muting on all shots}}
\babang{
\begin{verbatim}
# First muting for all shots
Flow('mutes','lines',
     '''
     transp memsize=1000 plane=23 |
     mutter t0=0.1 v0=5.2 | 
     transp memsize=1000 plane=23
     ''' )
\end{verbatim}}
\liubang{
\begin{verbatim}
[yourcomputer directory]$ sfin lines.rsf
lines.rsf:
    in="$RSFDATA/.../.../lines.rsf@"
    esize=4 type=float form=native 
    n1=1501        d1=0.002       o1=0          label1="Time" unit1="s" 
    n2=251         d2=0.05        o2=688        label2="Source" unit2="km" 
    n3=282         d3=0.025       o3=-3.5       label3="Offset" unit3="km" 
	106243782 elements 424975128 bytes
\end{verbatim}}
\end{frame}

\addtocounter{framenumber}{-1}
\begin{frame}[fragile]
  \frametitle{\jiubang{Subsampling -- input}}
\babang{
\begin{verbatim}
# Shot 198
Flow('shot198','mutes','window n2=1 f2=198')
Plot('shot198','grey title="Shot 198" labelfat=4 titlefat=4')

# Spectra
Flow('spec198','shot198','spectra2')
Plot('spec198',
     '''
     grey color=j yreverse=n title="Spectra 198" bias=0.08
     label1=Frequency unit1=Hz label2=Wavenumber unit2=1/km
     labelfat=4 titlefat=4
     ''')
Result('input198','shot198 spec198','SideBySideAniso')
\end{verbatim}
}
\end{frame}

\begin{frame}
  \frametitle{\jiubang{Input}}
    \begin{center}
    \includegraphics[width=0.9\textwidth]{data/Fig/input198}
    \end{center}
\end{frame}

\addtocounter{framenumber}{-1}
\begin{frame}[fragile]
  \frametitle{\jiubang{Subsampling -- anti-alias filter design}}
\babang{
\begin{verbatim}
# Test anti-aliasing filter
Flow('spike',None,'spike n1=1501 d1=0.002 o1=0 k1=750 mag=1 nsp=1')
Flow('bandp','spike','bandpass flo=3 fhi=125 nphi=8 ')
Result('sbandp','bandp',
       '''
       spectra |
       graph title="Transfer function"
       labelsz=4. plotfat=10 grid=y
       ''')
\end{verbatim}
}
\end{frame}

\begin{frame}
  \frametitle{\jiubang{Filter}}
    \begin{center}
    \includegraphics[width=0.7\textwidth]{data/Fig/sbandp}
    \end{center}
\end{frame}

\addtocounter{framenumber}{-1}
\begin{frame}[fragile]
  \frametitle{\jiubang{Subsampling -- output}}
\babang{
\begin{verbatim}
# Subsampling all shots to 4ms
Flow('subsample','mutes',
     'bandpass flo=3 fhi=125 nphi=8 | window j1=2')
Flow('subshot198','subsample','window n2=1 f2=198')
Plot('subshot198','grey title="Subsampled 198" labelfat=4 titlefat=4')
# Spectra
Flow('subspec198','subshot198','spectra2')
Plot('subspec198',
     '''
     grey color=j yreverse=n title="Spectra 198" bias=0.08
     label1=Frequency unit1=Hz label2=Wavenumber unit2=1/km
     labelfat=4 titlefat=4
     ''')
Result('output198','subshot198 subspec198','SideBySideAniso')
\end{verbatim}
}
\end{frame}

\begin{frame}
  \frametitle{\jiubang{Output}}
    \begin{center}
    \includegraphics[width=0.9\textwidth]{data/Fig/output198}
    \end{center}
\end{frame}

\addtocounter{framenumber}{-1}
\begin{frame}[fragile]
  \frametitle{\jiubang{Calculate time-frequency spectra}}
\babang{
\begin{verbatim}
Flow('ltft198','subshot198',
     '''
     ltft rect=20 verb=n nw=50 dw=2 niter=50
     ''')
Result('ltft198',
       '''
       math output="abs(input)" | real |
       byte allpos=y gainpanel=100 pclip=99 |
       grey3 color=j  frame1=120 frame2=7 frame3=71 label1=Time flat=y 
       unit1=s label3=Offset label2="\F5 f \F-1" unit3=km
       screenht=10 screenratio=0.7 parallel2=n format2=%3.1f
       point1=0.8 point2=0.3 wanttitle=n labelfat=4 font=2 titlefat=4
       ''')
\end{verbatim}
}
\end{frame}

\begin{frame}
  \frametitle{\jiubang{Local time-frequency (LTF) spectra}}
    \begin{center}
    \includegraphics[width=0.9\textwidth]{data/Fig/ltft198}
    \end{center}
\end{frame}

\addtocounter{framenumber}{-1}
\begin{frame}[fragile]
  \frametitle{\jiubang{Thresholding}}
\qibang{
\begin{verbatim}
Flow('thr198','ltft198',
     '''
     transp plane=23 memsize=1000 |
     threshold2 pclip=25 verb=y |
     transp plane=23 memsize=1000
     ''')
Result('thr198',
       '''
       math output="abs(input)" | real |
       byte allpos=y gainpanel=100 pclip=99 |
       grey3 color=j  frame1=120 frame2=7 frame3=71 label1=Time flat=y 
       unit1=s label3=Offset label2="\F5 f \F-1" unit3=km
       screenht=10 screenratio=0.7 parallel2=n format2=%3.1f
       point1=0.8 point2=0.3 wanttitle=n labelfat=4 font=2 titlefat=4
       ''')
\end{verbatim}
}
\end{frame}

\begin{frame}
  \frametitle{\jiubang{Thresholded LTF spectra}}
    \begin{center}
    \includegraphics[width=0.9\textwidth]{data/Fig/thr198}
    \end{center}
\end{frame}

\addtocounter{framenumber}{-1}
\begin{frame}[fragile]
  \frametitle{\jiubang{Signal and noise separation}}
\babang{
\begin{verbatim}
Flow('noise198','thr198','ltft inv=y | mutter t0=-0.5 v0=0.7')
Plot('noise198',
     'grey title="Ground-roll 198" unit2=km labelfat=4 titlefat=4')

Flow('signal198','subshot198 noise198','add scale=1,-1 ${SOURCES[1]}')
Plot('signal198',
     'grey title="Ground-roll removal" labelfat=4 titlefat=4')
Result('sn198','signal198 noise198','SideBySideAniso')
\end{verbatim}
}
\end{frame}

\begin{frame}
  \frametitle{\jiubang{Signal and noise separation}}
    \begin{center}
    \includegraphics[width=0.9\textwidth]{data/Fig/sn198}
    \end{center}
\end{frame}

\addtocounter{framenumber}{-1}
\begin{frame}[fragile]
  \frametitle{\jiubang{Process the whole dataset}}
\babang{
\begin{verbatim}
Flow('ltfts','subsample',
     '''
     ltft rect=20 verb=y nw=50 dw=2 niter=50
     ''',split=[3,282],reduce="cat axis=4")
Flow('thresholds','ltfts',
     '''
     transp plane=24 memsize=1000 | threshold2 pclip=25 verb=y  |
     transp plane=24 memsize=1000
     ''',split=[3,251])
Flow('noises','thresholds',
     '''
     ltft inv=y | transp plane=23 memsize=1000 | 
     mutter t0=-0.5 v0=0.7 | transp plane=23 memsize=1000
     ''')
Flow('signals','subsample noises','add scale=1,-1 ${SOURCES[1]}')
\end{verbatim}
}
\end{frame}

\subsection{Initial velocity analysis}

\addtocounter{framenumber}{-1}
\begin{frame}[fragile]
  \frametitle{\jiubang{Convert shots to CMPs}}
\babang{
\begin{verbatim}
Flow('cmps','signals',
     '''
     transp memsize=1000 plane=23 |
     mutter v0=3. |
     shot2cmp half=n | put o2=-1.75 label2="Half-offset"
     ''')
Result('cmps',
       '''
       byte gainpanel=each | 
       grey3 frame1=500 frame2=36 frame3=642 flat=n
       title="CMP gathers" point1=0.7 label2=Offset label3=Midpoint
       ''')
\end{verbatim}
}
\end{frame}

\begin{frame}
  \frametitle{\jiubang{New CMPs}}
    \begin{center}
    \includegraphics[width=0.8\textwidth]{data/Fig/cmps}
    \end{center}
\end{frame}

\addtocounter{framenumber}{-1}
\begin{frame}[fragile]
  \frametitle{\jiubang{Velocity scan parameters}}
\jiubang{
\begin{verbatim}
# Set up velocity scan parameters
v0 = 2.15
dv = 0.025
nv = 100
\end{verbatim}
}
\end{frame}

\addtocounter{framenumber}{-1}
\begin{frame}[fragile]
  \frametitle{\jiubang{Velocity scan for all CMPs}}
\babang{
\begin{verbatim}
# Velocity scanning for all CMP gathers
Flow('scn','cmps',
     '''
     vscan semblance=y v0=%g nv=%d dv=%g half=y str=0 |
     mutter v0=0.9 t0=-4.5 inner=y
     ''' % (v0,nv,dv),split=[3,1285])
Flow('vel','scn','pick rect1=15 rect2=25 gate=100 an=10 | window')
Result('vel',
       '''
       grey title="NMO Velocity" label1="Time" label2="Lateral"
       color=j scalebar=y allpos=y bias=2.1 %g barlabel="Velocity"
       barreverse=y o2num=1 d2num=1 n2tic=3 labelfat=4 font=2 titlefat=4
       ''' % (v0+0.5*nv*dv))
\end{verbatim}
}
\end{frame}

\begin{frame}
  \frametitle{\jiubang{Display NMO velocity}}
    \begin{center}
    \includegraphics[width=0.8\textwidth]{data/Fig/vel}
    \end{center}
\end{frame}

\subsection{Brute stacking}

\addtocounter{framenumber}{-1}
\begin{frame}[fragile]
  \frametitle{\jiubang{Normal moveout (NMO)}}
\babang{
\begin{verbatim}
Flow('nmo','cmps vel',
     '''
     nmo velocity=${SOURCES[1]} half=y
     ''')
Result('nmo',
       '''
       byte gainpanel=e |
       grey3 frame1=500 frame2=36 frame3=642 flat=n
       title="NMOed Data" point1=0.7
       label2=Offset label3=Midpoint
       ''')
\end{verbatim}
}
\end{frame}

\begin{frame}
  \frametitle{\jiubang{Display NMOed CMPs}}
    \begin{center}
    \includegraphics[width=0.8\textwidth]{data/Fig/nmo}
    \end{center}
\end{frame}

\addtocounter{framenumber}{-1}
\begin{frame}[fragile]
  \frametitle{\jiubang{Brute stack}}
\babang{
\begin{verbatim}
# Brute stacking
Flow('bstack','nmo','stack')
Result('bstack',
       '''
       agc rect1=50 |
       grey title="Brute stacking" labelfat=4 font=2 titlefat=4
       ''')
\end{verbatim}
}
\end{frame}

\begin{frame}
  \frametitle{\jiubang{Display brute stacking result}}
    \begin{center}
    \includegraphics[width=0.8\textwidth]{data/Fig/bstack}
    \end{center}
\end{frame}

  \subsection{Toy migration}

\addtocounter{framenumber}{-1}
\begin{frame}[fragile]
  \frametitle{\jiubang{Toy prestack Kirchhoff time migration}}
\babang{
\begin{verbatim}
# Prestack Kirchhoff time migration
Flow('tcmps','cmps','transp memsize=1000 plane=23')
Flow('pstm','tcmps vel',
     '''
     mig2 vel=${SOURCES[1]} apt=5 antialias=1
     ''',split=[3,71,[0]],reduce='add')
Result('pstm',
       '''
       grey title="Prestack kirchhoff time migration"
       labelfat=4 font=2 titlefat=4
       ''')

End()
\end{verbatim}
}
\end{frame}

\begin{frame}
  \frametitle{\jiubang{Display migration result}}
    \begin{center}
    \includegraphics[width=0.8\textwidth]{data/Fig/pstm}
    \end{center}
\end{frame}

\begin{frame}
  \frametitle{To be continued ...}
    \begin{center}
     Modify/add your own modules, improve the final result ...
    \end{center}
\end{frame}

\addtocounter{framenumber}{-1}
\begin{frame}[fragile]
  \frametitle{Software}
    \begin{center}
%  \includegraphics[height=0.24\textwidth]{../freqlet/figures/Fig/Madagascar}\\
   \shijiubang{http://www.ahay.org}
       \end{center}
\babang{
\begin{verbatim}
     cd $RSFSRC/book/rsf/usp/data/
     pscons lock
     cd $RSFSRC/book/rsf/usp/
     scons school10.read
\end{verbatim}}

       \begin{block}{}
    \begin{center}
     \shijiubang{Thanks for attention and have a fun!}
    \end{center}
   \end{block}
\end{frame}

  % The following outlook is optional.
  %\vskip0pt plus.5fill
  %\begin{itemize}
  %\item
  %  Outlook
  %  \begin{itemize}
  %  \item
  %    Something you haven't solved.
  %  \item
  %    Something else you haven't solved.
  %  \end{itemize}
  %\end{itemize}


%\end{document}



\renewcommand{\thefootnote}{\fnsymbol{footnote}} 

\title{Introducing RSF, a computational platform for geophysical data processing and reproducible numerical experiments}

\author{Sergey Fomel\footnotemark[1], University of Texas at Austin, Paul Sava, Colorado School of Mines, and Felix Herrmann, University of British Columbia}

\maketitle

This talk is the first public announcement of RSF (from
\emph{regularly sampled format}), an open-source software package
developed in collaboration by a group of geophysicists, petroleum
engineers, and computational scientists. It is also an open invitation
for other scientists to join the project as users or developers.

The four main features of RSF are:

\begin{enumerate}

\item RSF is a \emph{new} package. It started in 2003 and was
developed entirely from scratch. Being a new package, it follows
modern software engineering practices such as module encapsulation and
test-driven development. A rapid development of a project of this
scope (more than 300 main programs and more than 3000 tests) would not
be possible without standing on the shoulders of giants and learning
from the 30 years of previous experience in open packages such as
SEPlib and Seismic Unix. We have borrowed and reimplemented
functionality and ideas from these packages.

\item RSF is a \emph{test-driven} package. Test-driven development is
not only an agile software programming practice but also a way of
bringing scientific foundation to geophysical research that involves
numerical experiments. Bringing reproducibility and peer review, the
backbone of any real science, to the field of computational geophysics
is the main motivation for RSF development. The package consists of
two levels: low-level main programs (typically developed in the C
programming language and working as data filters) and high-level
processing flows (described with the help of the Python programming
language) that combine main programs and completely document data
processing histories for testing and reproducibility. Experience
shows that high-level programming is easily mastered even by beginning
students that have no previous programming experience.

\item RSF is an \emph{open-source} package. It is distributed under
the standard GPL open-source license, which places no restriction on
the usage and modification of the code. Access to modifying the source
repository is not controlled by one organization but shared equally
among different developers. This enables an open collaboration among
different groups spread all over the world, in the true spirit of the
open source movement.

\item RSF uses a \emph{simple, flexible, and universal} data format
that can handle very large datasets but is not tied specifically to
seismic data or data of any other particular kind. This ``regularly
sampled'' format is borrowed from the traditional SEPlib and is also
related to the DDS format developed by Amoco and BP. A universal data
format allows us to share general-purpose data processing tools with
scientists from other disciplines such as petroleum engineers working
on large-scale reservoir simulations.

\end{enumerate}
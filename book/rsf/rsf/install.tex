\title{RSF Installation}
\author{Sergey Fomel}
\lefthead{Fomel}
\righthead{RSF installation}

%\begin{document}

\maketitle

RSF is currently developed on a Linux platform. It should be possible to
install and use it on other POSIX-compliant platforms. Installation on
non-standard platforms is prone to problems, which will get fixed as RSF
gets more developed and tested.

\section{Prerequisites}

Installing RSF from source requires the following:
\begin{enumerate}
\item CVS (concurrent versions system). CVS is a software system for managing
  distributed software projects. CVS comes pre-installed on RedHat Linux. It
  is available at \url{http://www.cvshome.org/}.
\item C compiler. ANSI-compliant compiler such as GCC should work. GCC is
  available at \url{http://gcc.gnu.org/}. GCC is pre-installed on all Linux
  machines.
\item Python interpreter. Python is an interpretable programming language. It
  is used by RSF in installation scripts and project management scripts.
  Python comes pre-installed on RedHat Linux. It is available at \\
  \url{http://www.python.org/}.
\item SCons (Software Construction). SCons is a Python-based tool for project
  management and software construction. It is a modern replacement for the
  Unix ``make'' utility. RSF uses SCons for both software construction and
  processing flow management. The current version of SCons is available at  
  \url{http://www.scons.org/}.
\end{enumerate}

To display and manipulate the figures generated by RSF, you may also need the
vplot-manipulation programs from SEPlib. Refer to the SEPlib documentation at
\\ \url{http://sepwww.stanford.edu/software/seplib/} for installation
instructions. You only need programs specified under \texttt{vplot/filters} in
the SEPlib source tree. RSF will eventually provide analogous functionality.

\section{Downloading}

Download the latest version of RSF by running \\
\texttt{cvs~-d~:pserver:anonymous@begpc132.beg.utexas.edu:/home/cvs~co~RSF} \\
Alternatively, you can first set the environmental variable \texttt{CVSROOT}
to  \\ \texttt{:pserver:anonymous@begpc132.beg.utexas.edu:/home/cvs} \\
and then run the \texttt{cvs co RSF} command. This command should install the
RSF directory tree on your computer.

\section{Installation}

\subsection{Environmental variables}

Set the \texttt{RSFROOT} environmental variable to the directory where
you want RSF installed. To use RSF effectively, you may also want to add
\texttt{\$RSFROOT/bin} to your \texttt{PATH} environemntal variable, set
\texttt{PYTHONPATH} to \texttt{\$RSFROOT/lib}, add set \texttt{DATAPATH} to
the directory for storing temporary data files. Example configuration for
\texttt{csh} and \texttt{tcsh}:
\begin{verbatim}
setenv RSFROOT /usr/local/rsf
set path = ($path $RSFROOT/bin)
setenv PYTHONPATH $RSFROOT/lib
setenv DATAPATH /var/tmp/
\end{verbatim}
Notice the slash at the end of the DATAPATH variable.

\subsection{Software construction}

Change to the \texttt{RSF} directory and run \texttt{scons install}. If
everything works, this should compile and install all the necessary files and
programs under \texttt{RSFROOT}.

Alternatively, you can run the following three commands in succession: 
\begin{verbatim}
scons config
scons
scons install
\end{verbatim}

The first command automatically detects the software construction environment.
After running \texttt{scons config}, you can check \texttt{config.py} file for
the construction variables set by SCons. Run \texttt{scons -h} to get a full
list of customizable construction variables.

\subsection{Bugs}

Please report all problems encountered during software construction to \\
\texttt{<sergey.fomel@beg.utexas.edu>}.

%\end{document}

%%% Local Variables: 
%%% mode: latex
%%% TeX-master: t
%%% End: 

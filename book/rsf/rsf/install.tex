\title{RSF Installation}
\email{sergey.fomel@beg.utexas.edu}
\author{Sergey Fomel}
\lefthead{Fomel}
\righthead{RSF installation}

%\begin{document}

\maketitle

\begin{abstract}
Madagascar has been installed and is periodically tested on
\begin{itemize}
\item Different Linux distributions (Fedora, RedHat, SuSE, Debian, Ubuntu)
\item FreeBSD and OpenBSD
\item Solaris
\item MacOS X
\item Windows under the \href{http://www.cygwin.com/}{Cygwin} environment and 
Microsoft's \href{http://www.microsoft.com/technet/interopmigration/unix/sfu/default.mspx}{Services for UNIX} environment.
\end{itemize}
\end{abstract}

\section{Prerequisites}
\pdfbookmark[1]{Prerequisites}{prq}

\begin{enumerate}
\item C compiler. ANSI-compliant compiler such as
  \href{http://gcc.gnu.org/}{GCC} should work. GCC usually comes pre-installed
  on Linux machines.
\item Python interpreter. \href{http://www.python.org/}{Python} is an
  interpretable programming language. It is used by RSF in installation
  scripts and project management scripts.  Python comes pre-installed on
  some versions of Linux.
\end{enumerate}

\section{Installation}

\subsection{Environmental variables}

\begin{enumerate}
\item  Set the \texttt{RSFROOT} environmental variable to the directory where you want RSF installed.
\item Set the \texttt{PYTHONPATH} environmental variable to include \texttt{\$RSFROOT/lib}. 
\item Add \texttt{\$RSFROOT/bin} to your \texttt{PATH} environmental variable.
\item Set \texttt{DATAPATH} to the directory for temporary data files. 
\end{enumerate}

Example configuration for csh and tcsh:
\begin{verbatim}
setenv RSFROOT /usr/local/rsf
if ($?PYTHONPATH) then
   setenv PYTHONPATH ${PYTHONPATH}:$RSFROOT/lib
else
   setenv PYTHONPATH $RSFROOT/lib
endif
set path = ($path $RSFROOT/bin)
setenv DATAPATH /var/tmp/
\end{verbatim}

Example configuration for \texttt{bash}:
\begin{verbatim}
export RSFROOT=/usr/local/rsf
if [ -n "$PYTHONPATH" ]; then
   export PYTHONPATH=${PYTHONPATH}:$RSFROOT/lib
else
   export PYTHONPATH=$RSFROOT/lib
fi
export PATH=$PATH:$RSFROOT/bin
export DATAPATH=/var/tmp/
\end{verbatim}

Notice the slash at the end of the \texttt{DATAPATH} variable.

\subsection{Software construction}

\begin{enumerate}

\item Configuration.

Change to the top RSF source directory and run
\begin{verbatim}
./configure
\end{verbatim}

You can examine the \texttt{config.py} file that this command
generates.  Additional options are available. You can obtain a full
list of customizable variables by running \texttt{scons -h}. For
example, to install C++ and Fortran-90 API bindings in addition to the
basic package, run
\begin{verbatim}
scons API=c++,fortran-90 config
\end{verbatim}

\item Building and installing the package.

Run \texttt{make install} or the following two commands in succession: 
\begin{verbatim}
make
make install
\end{verbatim}

If you need ``root'' privileges for installing under \texttt{\$RSFROOT}, you may need to run
\begin{verbatim}
make
su
make install
\end{verbatim}
or
\begin{verbatim}
make
sudo make install
\end{verbatim}

\item Cleaning.

To clean all intermediate files generated by SCons, run
\begin{verbatim}
make clean
\end{verbatim}

To clean all intermediate files and all installed files, run
\begin{verbatim}
make distclean
\end{verbatim}

\end{enumerate}

\section{Other installations}

There are two other packages that might be useful in conjunction with RSF:

\subsection{RSF reproducible figures}

Using Subversion, run
\begin{verbatim}
svn co http://egl.beg.utexas.edu/svn/rsffigs $RSFROOT/figs
\end{verbatim}
This installs a wide collection of more than 3,000 reproducible
figures. It may take a long time to download and some space on disk.
The figures are preserved with the purpose to do regression testing
whenever the software or the environment change.

\subsection{\LaTeX\ package}

\href{http://segtex.sourceforge.net}{SEGTeX} is a LaTeX package for geophysical publications. 
It can be used with madagascar for writing reproducible papers.

\section{Troubleshooting}

\subsection{Troubles with compilers}
If the configuration part ends with the message like
\begin{verbatim}
checking if cc works ... failed
\end{verbatim}
the problem may be that your compiler is in unusual place. By default,
\texttt{scons} does not inherit your environmental variables including
\texttt{PATH}. Try
\begin{verbatim}
scons CC=/full/path/to/cc config
\end{verbatim}
or
\begin{verbatim}
scons CC=`which cc` config
\end{verbatim}

On Windows under SFU, use the \texttt{gcc} compiler
\begin{verbatim}
scons CC=/opt/gcc.3.3/bin/gcc config
\end{verbatim}

\subsection{Troubles with cygwin}
If, configuring on \texttt{cygwin}, you get a message about RPC
libraries, install the \texttt{sunrpc} package located under
\texttt{Libs} in the \texttt{Setup} program.

%\end{document}

%%% Local Variables: 
%%% mode: latex
%%% TeX-master: t
%%% End: 

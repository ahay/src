\title{Introduction to RSF}
\author{Sergey Fomel}
\lefthead{Fomel}
\righthead{Introduction to RSF}

\maketitle

\section{What is RSF?}

RSF is a software package for digital data processing and seismic imaging,
which consists of
\begin{enumerate}
\item A collection of main programs. Most programs act as filters on input
  data and can be chained in a Unix pipeline. For example:
\begin{verbatim}
< data.rsf sfwindow n1=100 | sfbandpass fhi=60 > data2.rsf
\end{verbatim}
This approach follows the Unix philosophy, as formulated by Doug McIlroy,
the inventor of Unix pipes \cite[]{salus}:
\begin{enumerate}
\item Write programs that do one thing and do it well. 
\item Write programs to work together. 
\item Write programs to handle text streams, because that is a universal
  interface.
\end{enumerate}

Running a command (such as \texttt{sfwindow}) without parameters or the
necessary input and output files shows a brief documentation, explaining the
program purpose and parameters.  Alternatively, brief documentation is
provided by \texttt{sfdoc} program. Main program documentation in HTML format
is available on the web at \url{http://begpc132.beg.utexas.edu/RSF/}.

RSF uses \emph{Regularly Sampled File} format for data files, which is similar
to the format used in the SEPlib library developed at Stanford Exploration
Project (SEP). The file format describes regularly sampled hypercubes. Up to 9
dimensions are supported. In accordance with the Unix philosophy, each RSF
file (such as \texttt{data.rsf}) is a simple readable text. It contains a
pointer (\texttt{in=} parameter) to the location of the binary data. 

RSF adopts Vplot file format, also developed at SEP, for generated graphics
files. RSF provides programs for conversion to and from other popular formats
such as SEG-Y and SU.

\item An API (application programmer's interface) for programmers writing
  their own software to manipulate RSF files. The main software language of
  the RSF package is C. Interfaces to other languages (C++, Fortran-77,
  Fortran-90, Python) are also provided.
  
\item A project management system. The system uses and extends \texttt{SCons},
  an open-source software construction
  package\footnote{\url{http://www.scons.org/}}, to document and maintain data
  processing flows. Documented projects become computational recipes that can
  be easily exchanged among RSF users.
\item A collection of reproducible documents, organized in living books. Each
  reproducible book contains a collection of RSF recipes (\texttt{SConstruct}
  files) used to generate the book illustrations. The recipes cover a variety
  of data processing and imaging tasks described in the books.
\end{enumerate}

%The RSF file format and the main program interface are described in \emph{RSF
%  user manual}. The RSF API is described in \emph{RSF hacker manual}. The
%project management system is described in \emph{RSF project management
%  manual}.

\section{Why RSF?}

The RSF mission is to provide
\begin{enumerate}
\item a convenient environment
\item a convenient technology transfer tool
\end{enumerate}
for researchers working with digital processing of data and images. The
package is intended to be available in an open-source form to allow effective
collaboration of a wide community of developers. The technology developed
using the RSF project management system is transferred in the form of recorded
processing histories which become ``computational recipes'' to be verified,
exchanged, and modified by users of the system.

\section{Copyright notice}

The RSF package is released in an open-source form under the standard GNU GPL
license. In simple words, this license puts no restrictions on the use of the
software (including copying, modifying, selling, etc.) However, there are
restrictions on the software redistribution intended to prevent the package
from loosing its open-source status. Users are encourages to submit their
modifications back to the original package to the benefit of the whole user
community.

\section{Alternatives} 

In the present form, the RSF package, while being completely written from
scratch, borrows from the design of SEPlib, a publicly available software
package, maintained by Bob Clapp at the Stanford Exploration Project
\cite[]{Claerbout.sep.70.413,Dellinger.sep.73.461,Nichols.sep.82.257,Biondi.sep.92.343,Clapp.sep.102.bob1}\footnote{SEPlib
  is available at \url{http://sepwww.stanford.edu/software/seplib/}.}.
Generations of SEP students and researchers contributed to SEPlib. Most
important contributions came from Rob Clayton, Jon Claerbout, Dave Hale, Stew
Levin, Rick Ottolini, Joe Dellinger, Steve Cole, Dave Nichols, Martin
Karrenbach, Biondo Biondi, and Bob Clapp.

RSF also borrows ideas from Seismic Unix (SU), a package maintained by John
Stockwell at the Center for Wave Phenomenon at the Colorado School of Mines
\cite[]{TLE16-07-10451049,su}\footnote{SU is available at
  \url{http://timna.mines.edu/cwpcodes/}.}. Main contributors to SU
included Einar Kjartansson, Shuki Ronen, Jack Cohen, Chris Liner, Dave Hale,
and John Stockwell.

Another option for a seismic processing system is Free USP. USP is a
processing package developed by Amoco and released by BP \footnote{Free USP is
  available at \url{http://www.freeusp.org/}.}.

None of these three packages can be qualified as free software according to
the Free Software
Foundation\footnote{\url{http://www.fsf.org/philosophy/free-sw.html}} or as
open-source software according to the Open Source
Initiative\footnote{\url{http://www.opensource.org/docs/definition.php}}.
However, they are available for free in the source form under certain
conditions. The RSF package is both free and open-source. 

\bibliographystyle{segnat}
\bibliography{intro,SEG}

%\end{document}


%%% Local Variables: 
%%% mode: latex
%%% TeX-master: t
%%% TeX-master: t
%%% End: 

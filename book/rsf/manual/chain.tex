\section{Chaining linear operators (chain.c)}\label{sec:chain}
Calculates products of operators.



\subsection{{sf\_chain}}\label{sec:sf_chain}
Chains two operators $L_1$ and $L_2$: 
\begin{gather*}
	d = (L_2L_1)m.
\intertext{Its adjoint is}
	m = (L_2L_1)^*d=L_1^*L_2^*d.
\end{gather*}

\subsubsection*{Call}
\begin{verbatim}sf_chain (oper1,oper2, adj,add, nm,nd,nt, mod,dat,tmp);\end{verbatim}

\subsubsection*{Definition}
\begin{verbatim}
void sf_chain( sf_operator oper1     /* outer operator */, 
               sf_operator oper2     /* inner operator */, 
               bool adj              /* adjoint flag */, 
               bool add              /* addition flag */, 
               int nm                /* model size */, 
               int nd                /* data size */, 
               int nt                /* intermediate size */, 
               /*@out@*/ float* mod  /* [nm] model */, 
               /*@out@*/ float* dat  /* [nd] data */, 
               float* tmp            /* [nt] intermediate */) 
/*< Chains two operators, computing oper1{oper2{mod}} 
  or its adjoint. The tmp array is used for temporary storage. >*/
{
   ...
}
\end{verbatim}

\subsubsection*{Input parameters}
\begin{desclist}{\tt }{\quad}[\tt oper2]
   \setlength\itemsep{0pt}
   \item[oper1] outer operator, $L_1$ (\texttt{sf\_operator}). 
   \item[oper2] inner operator, $L_2$ (\texttt{sf\_operator}). 
   \item[adj] adjoint flag (\texttt{bool}). If \texttt{true}, then the adjoint is computed, i.e.~$m\leftarrow (L_2L_1)^*d$ or $m\leftarrow m+(L_2L_1)^*d$. 
   \item[add] addition flag (\texttt{bool}). If \texttt{true}, then $d\leftarrow d+(L_2L_1)m$ or $m\leftarrow m+(L_2L_1)^*d$ is computed.  
   \item[nm]    size of the model \texttt{mod} (\texttt{int}). 
   \item[nd]    size of the data \texttt{dat} (\texttt{int}). 
   \item[nt]    size of the intermediate result \texttt{tmp}  (\texttt{int}). 
   \item[mod]   the model, $m$ (\texttt{float*}).
   \item[dat]   the data, $d$ (\texttt{float*}).
   \item[tmp]   intermediate result (\texttt{float*}).
\end{desclist}




\subsection{{sf\_cchain}}
The same as \hyperref[sec:sf_chain]{\texttt{sf\_chain}} but for complex data.

\subsubsection*{Call}
\begin{verbatim}sf_cchain (oper1,oper2, adj,add, nm,nd,nt, mod, dat, tmp);\end{verbatim}

\subsubsection*{Definition}
\begin{verbatim}
void sf_cchain( sf_coperator oper1         /* outer operator */, 
                sf_coperator oper2         /* inner operator */, 
                bool adj                   /* adjoint flag */, 
                bool add                   /* addition flag */, 
                int nm                     /* model size */, 
                int nd                     /* data size */, 
                int nt                     /* intermediate size */, 
                /*@out@*/ sf_complex* mod  /* [nm] model */, 
                /*@out@*/ sf_complex* dat  /* [nd] data */, 
                sf_complex* tmp            /* [nt] intermediate */) 
/*< Chains two complex operators, computing oper1{oper2{mod}} 
    or its adjoint. The tmp array is used for temporary storage. >*/
{
   ...    
}
\end{verbatim}

\subsubsection*{Input parameters}
\begin{desclist}{\tt }{\quad}[\tt oper2]
   \setlength\itemsep{0pt}
   \item[oper1] outer operator, $L_1$ (\texttt{sf\_coperator}). 
   \item[oper2] inner operator, $L_2$ (\texttt{sf\_coperator}). 
   \item[adj] adjoint flag (\texttt{bool}). If \texttt{true}, then the adjoint is computed, i.e.~$m\leftarrow (L_2L_1)^*d$ or $m\leftarrow m+(L_2L_1)^*d$. 
   \item[add] addition flag (\texttt{bool}). If \texttt{true}, then $d\leftarrow d+(L_2L_1)m$ or $m\leftarrow m+(L_2L_1)^*d$ is computed.  
   \item[nm]    size of the model \texttt{mod} (\texttt{int}). 
   \item[nd]    size of the data \texttt{dat} (\texttt{int}). 
   \item[nt]    size of the intermediate result \texttt{tmp}  (\texttt{int}). 
   \item[mod]   the model, $m$ (\texttt{sf\_complex*}).
   \item[dat]   the data, $d$ (\texttt{sf\_complex*}).
   \item[tmp]   intermediate result (\texttt{sf\_complex*}).
   \item[tmp]   the intermediate storage (\texttt{sf\_complex*}).
\end{desclist}




\subsection{{sf\_array}}
For two operators $L_1$ and $L_2$, it calculates: 
\begin{gather*}
	d = Lm,
\intertext{or its adjoint}
	m = L^*d,
\intertext{where}
	L = \begin{bmatrix}L_1\\L_2\end{bmatrix}\quad\textrm{and}\quad 
	d = \begin{bmatrix}d_1\\d_2\end{bmatrix}
\end{gather*}

\subsubsection*{Call}
\begin{verbatim}sf_array (oper1,oper2, adj,add, nm,nd1,nd2, mod,dat1,dat2);\end{verbatim}

\subsubsection*{Definition}
\begin{verbatim}
void sf_array( sf_operator oper1     /* top operator */, 
               sf_operator oper2     /* bottom operator */, 
               bool adj              /* adjoint flag */, 
               bool add              /* addition flag */, 
               int nm                /* model size */, 
               int nd1               /* top data size */, 
               int nd2               /* bottom data size */, 
               /*@out@*/ float* mod  /* [nm] model */, 
               /*@out@*/ float* dat1 /* [nd1] top data */, 
               /*@out@*/ float* dat2 /* [nd2] bottom data */) 
/*< Constructs an array of two operators, 
    computing {oper1{mod},oper2{mod}} or its adjoint. >*/
{
   ...
}
\end{verbatim}

\subsubsection*{Input parameters}
\begin{desclist}{\tt }{\quad}[\tt oper2]
   \setlength\itemsep{0pt}
   \item[oper1] top operator, $L_1$ (\texttt{sf\_operator}). 
   \item[oper2] bottom operator, $L_2$ (\texttt{sf\_operator}). 
   \item[adj]   adjoint flag (\texttt{bool}). If \texttt{true}, then the adjoint is computed, i.e.~$m\leftarrow L^*d$ or $m\leftarrow m + L^*d$. 
   \item[add] addition flag (\texttt{bool}). If \texttt{true}, then $d\leftarrow d + Lm$ or $m\leftarrow m + L^*d$ is computed.  
   \item[nm]    size of the model, \texttt{mod} (\texttt{int}). 
   \item[nd1]   size of the top data, \texttt{dat1} \texttt{dat1} (\texttt{int}). 
   \item[nd2]   size of the bottom data, \texttt{dat2} (\texttt{int}). 
   \item[mod]   the model, $m$ (\texttt{float*}).
   \item[dat1]  the top data, $d_1$ (\texttt{float*}).
   \item[dat2]  the bottom data, $d_2$ (\texttt{float*}).
\end{desclist}




\subsection{{sf\_normal}}
Applies a normal operator (self-adjoint) to the model, i.e.~it calculates
\begin{gather*}
	d = LL^*m.
\end{gather*}


\subsubsection*{Call}
\begin{verbatim}sf_normal (oper, add, nm,nd, mod,dat,tmp);\end{verbatim}

\subsubsection*{Definition}
\begin{verbatim}
void sf_normal (sf_operator oper /* operator */, 
                bool add         /* addition flag */, 
                int nm           /* model size */, 
                int nd           /* data size */, 
                float *mod       /* [nd] model */, 
                float *dat       /* [nd] data */, 
                float *tmp       /* [nm] intermediate */)
/*< Applies a normal operator (self-adjoint) >*/
{
   ...
}
\end{verbatim}

\subsubsection*{Input parameters}
\begin{desclist}{\tt }{\quad}[\tt oper]
   \setlength\itemsep{0pt}
   \item[oper]  the operator, $L$ (\texttt{sf\_operator}). 
   \item[add]   addition flag (\texttt{bool}). If \texttt{true}, then $d\leftarrow d+LL^*m$.  
   \item[nm]    size of the model, \texttt{mod} (\texttt{int}). 
   \item[nd]    size of the data, \texttt{dat} (\texttt{int}). 
   \item[mod]   the model, $m$ (\texttt{float*}).
   \item[dat]   the data, $d$ (\texttt{float*}).
   \item[tmp]   the intermediate result (\texttt{float*}).
\end{desclist}




\subsection{{sf\_chain3}}
Chains three operators $L_1$, $L_2$ and $L_3$: 
\begin{gather*}
	d = (L_3L_2L_1)m.
\intertext{Its adjoint is}
	m = (L_3L_2L_1)^*d=L_1^*L_2^*L_3^*d.
\end{gather*}

\subsubsection*{Call}
\begin{verbatim}
sf_chain3 (oper1,oper2,oper3, adj,add, nm,nt1,nt2,nd, mod,dat,tmp1,tmp2);
\end{verbatim}

\subsubsection*{Definition}
\begin{verbatim}
void sf_chain3 (sf_operator oper1 /* outer operator */, 
                sf_operator oper2 /* middle operator */, 
                sf_operator oper3 /* inner operator */, 
                bool adj          /* adjoint flag */, 
                bool add          /* addition flag */, 
                int nm            /* model size */, 
                int nt1           /* inner intermediate size */, 
                int nt2           /* outer intermediate size */, 
                int nd            /* data size */, 
                float* mod        /* [nm] model */, 
                float* dat        /* [nd] data */, 
                float* tmp1       /* [nt1] inner intermediate */, 
                float* tmp2       /* [nt2] outer intermediate */)
/*< Chains three operators, computing oper1{oper2{poer3{{mod}}} or its adjoint.
  The tmp1 and tmp2 arrays are used for temporary storage. >*/
{
   ...    
}
\end{verbatim}

\subsubsection*{Input parameters}
\begin{desclist}{\tt }{\quad}[\tt oper3]
   \setlength\itemsep{0pt}
   \item[oper1] outer operator (\texttt{sf\_operator}). 
   \item[oper2] middle operator (\texttt{sf\_operator}). 
   \item[oper3] inner operator (\texttt{sf\_operator}). 
   \item[adj]   adjoint flag (\texttt{bool}). If \texttt{true}, then the adjoint is computed, i.e.~$m\leftarrow L_1^*L_2^*L_3^*d$ or $m\leftarrow m+L_1^*L_2^*L_3^*d$. 
   \item[add]   addition flag (\texttt{bool}). If \texttt{true}, then $d\leftarrow d+L_3L_2L_1m$ or $m\leftarrow m+L_1^*L_2^*L_3^*d$.  
   \item[nm]    size of the model, \texttt{mod} (\texttt{int}). 
   \item[nt1]   inner intermediate size (\texttt{int}). 
   \item[nt2]   outer intermediate size (\texttt{int}). 
   \item[ny]    size of the data, \texttt{dat} (\texttt{int}). 
   \item[mod]   the model, $x$ (\texttt{float*}).
   \item[dat]   the data, $d$ (\texttt{float*}).
   \item[tmp1]  the inner intermediate result (\texttt{float*}).
   \item[tmp2]  the outer intermediate result (\texttt{float*}).
\end{desclist}







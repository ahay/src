\section{Priority queue (heap sorting) (pqueue.c)}




\subsection{{sf\_pqueue\_init}}
Initializes the heap with the maximum size given in the input.

\subsubsection*{Call}
\begin{verbatim}sf_pqueue_init (n);\end{verbatim}

\subsubsection*{Definition}
\begin{verbatim}
void sf_pqueue_init (int n)
/*< Initialize heap with the maximum size >*/
{
   ...
}
\end{verbatim}

\subsubsection*{Input parameters}
\begin{desclist}{\tt }{\quad}[\tt ]
   \setlength\itemsep{0pt}
   \item[n] maximum size of the heap (\texttt{int}).  
\end{desclist}




\subsection{{sf\_pqueue\_start}}
Sets the starting values for the queue.

\subsubsection*{Call}
\begin{verbatim}sf_pqueue_start ();\end{verbatim}

\subsubsection*{Definition}
\begin{verbatim}
void sf_pqueue_start (void)
/*< Set starting values >*/
{
   ...
}
\end{verbatim}




\subsection{{sf\_pqueue\_close}}
Frees the space allocated by \texttt{sf\_pqueue\_init}.

\subsubsection*{Call}
\begin{verbatim}sf_pqueue_close();\end{verbatim}

\subsubsection*{Definition}
\begin{verbatim}
void sf_pqueue_close (void)
/*< Free the allocated storage >*/
{
   ...
}
\end{verbatim}




\subsection{{sf\_pqueue\_insert}}
Inserts an element in the queue. The smallest element goes first.

\subsubsection*{Call}
\begin{verbatim}sf_pqueue_insert (v);\end{verbatim}

\subsubsection*{Definition}
\begin{verbatim}
void sf_pqueue_insert (float* v)
/*< Insert an element (smallest first) >*/
{
   ...
}
\end{verbatim}

\subsubsection*{Input parameters}
\begin{desclist}{\tt }{\quad}[\tt ]
   \setlength\itemsep{0pt}
   \item[v] element to be inserted, smallest first (\texttt{float*}).  
\end{desclist}




\subsection{{sf\_pqueue\_insert2}}
Inserts an element in the queue. The largest element goes first.

\subsubsection*{Call}
\begin{verbatim}sf_pqueue_insert2 (v);\end{verbatim}

\subsubsection*{Definition}
\begin{verbatim}
void sf_pqueue_insert2 (float* v)
/*< Insert an element (largest first) >*/
{
   ...
}
\end{verbatim}

\subsubsection*{Input parameters}
\begin{desclist}{\tt }{\quad}[\tt ]
   \setlength\itemsep{0pt}
   \item[v] element to be inserted, largest first (\texttt{float*}).  
\end{desclist}




\subsection{{sf\_pqueue\_extract}}
Extracts the smallest element from the list.

\subsubsection*{Call}
\begin{verbatim}v = sf_pqueue_extract();\end{verbatim}

\subsubsection*{Definition}
\begin{verbatim}
float* sf_pqueue_extract (void)
/*< Extract the smallest element >*/
{
    unsigned int c;
    int n;
   ...
    return v;
}
\end{verbatim}

\subsubsection*{Output}
\begin{desclist}{\tt }{\quad}[\tt ]
   \setlength\itemsep{0pt}
   \item[v] the extracted smallest element (\texttt{float*}).  
\end{desclist}




\subsection{{sf\_pqueue\_extract2}}
Extracts the largest element from the list.

\subsubsection*{Call}
\begin{verbatim}v = sf_pqueue_extract2();\end{verbatim}

\subsubsection*{Definition}
\begin{verbatim}
float* sf_pqueue_extract2 (void)
/*< Extract the largest element >*/
{
   ...
}
\end{verbatim}

\subsubsection*{Output}
\begin{desclist}{\tt }{\quad}[\tt ]
   \setlength\itemsep{0pt}
   \item[v] the extracted largest element (\texttt{float*}).  
\end{desclist}





\subsection{{sf\_pqueue\_update}}
Updates the heap.

\subsubsection*{Call}
\begin{verbatim}sf_pqueue_update (v);\end{verbatim}

\subsubsection*{Definition}
\begin{verbatim}
void sf_pqueue_update (float **v)
/*< restore the heap: the value has been altered >*/ 
{
   ... 
}
\end{verbatim}

\subsubsection*{Input parameters}
\begin{desclist}{\tt }{\quad}[\tt ]
   \setlength\itemsep{0pt}
   \item[v] elements to be inserted, largest first (\texttt{float**}).  
\end{desclist}






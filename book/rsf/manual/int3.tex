\section{3-D interpolation (int3.c)}
\index{interpolation!3D}




\subsection{{sf\_int3\_init}}\label{sec:sf_int3_init}
Initializes the required variables and allocates the required space for 3D interpolation.

\subsubsection*{Call}
\begin{verbatim}
sf_int3_init (coord, o1,o2,o3, d1,d2,d3, 
              n1,n2,n3, interp, nf_in, nd_in);
\end{verbatim}

\subsubsection*{Definition}
\begin{verbatim}
void  sf_int3_init (float** coord          /* coordinates [nd][3] */, 
                    float o1, float o2, float o3,
                    float d1, float d2, float d3,
                    int   n1, int   n2,   int n3 /* axes */, 
                    sf_interpolator interp /* interpolation function */, 
                    int nf_in              /* interpolator length */, 
                    int nd_in              /* number of data points */)
/*< initialize >*/
{
   ...
}
\end{verbatim}

\subsubsection*{Input parameters}
\begin{desclist}{\tt }{\quad}[\tt interp]
   \setlength\itemsep{0pt}
   \item[coord]  coordinates (\texttt{float**}).  
   \item[o1]     origin of the first axis (\texttt{float}).  
   \item[o2]     origin of the second axis (\texttt{float}).  
   \item[o3]     origin of the third axis (\texttt{float}).  
   \item[d1]     sampling of the first axis (\texttt{float}).  
   \item[d2]     sampling of the second axis (\texttt{float}).  
   \item[d3]     sampling of the third axis (\texttt{float}).  
   \item[n1]     length of the first axis (\texttt{float}).  
   \item[n2]     length of the second axis (\texttt{float}).  
   \item[n3]     length of the third axis (\texttt{float}).  
   \item[interp] interpolation function (\texttt{sf\_interpolator}).  
   \item[nf\_in] interpolator length (\texttt{int}).  
   \item[nd\_in] number of data points (\texttt{int}).  
\end{desclist}




\subsection{{sf\_int3\_lop}}
Applies the linear operator for 3D interpolation.

\subsubsection*{Call}
\begin{verbatim}sf_int3_lop (adj, add, nm, ny, mm, dd);\end{verbatim}

\subsubsection*{Definition}
\begin{verbatim}
void  sf_int3_lop (bool adj, bool add, int nm, int ny, float* mm, float* dd)
/*< linear operator >*/
{ 
   ...
}
\end{verbatim}

\subsubsection*{Input parameters}
\begin{desclist}{\tt }{\quad}[\tt add]
   \setlength\itemsep{0pt}
   \item[adj] a parameter to determine whether the output is \texttt{x} or \texttt{ord} (\texttt{bool}).
   \item[add] a parameter to determine whether the input needs to be zeroed (\texttt{bool}).
   \item[nm]  size of \texttt{x} (\texttt{int}).
   \item[ny]  size of \texttt{ord} (\texttt{int}).
   \item[x]   output or operator, depending on whether \texttt{adj} is true or false (\texttt{float*}).
   \item[ord] output or operator, depending on whether \texttt{adj} is true or false (\texttt{float*}).
\end{desclist}




\subsection{{sf\_int3\_close}}
Frees the space allocated for 3D interpolation by \hyperref[sec:sf_int3_init]{\texttt{sf\_int3\_init}}.

\subsubsection*{Call}
\begin{verbatim}int3_close ();\end{verbatim}

\subsubsection*{Definition}
\begin{verbatim}
void int3_close (void)
/*< free allocated storage >*/
{
   ...
}
\end{verbatim}


\section{Simbol Table for parameters (simtab.c)}




\subsection{{sf\_simtab\_init}}\label{sec:sf_simtab_init}
Creates a table to store the parameters input either from command line or a file. It takes the required size (type \texttt{int}) of the table as input. The output is a pointer to the allocated table and it is of the defined data type \texttt{sf\_simtab}.

\subsubsection*{call}
\begin{verbatim}table = sf_simtab_init(size);\end{verbatim}

\subsubsection*{Definition}
\begin{verbatim}
sf_simtab sf_simtab_init(int size)
/*< Create simbol table. >*/
{
   ...
}
\end{verbatim}

\subsubsection*{Input parameters}
\begin{desclist}{\tt }{\quad}[\tt table]
   \setlength\itemsep{0pt}
   \item[size] size of the table to be allocated (\texttt{int}).
\end{desclist}

\subsubsection*{Output}
\begin{desclist}{\tt }{\quad}[\tt table]
   \setlength\itemsep{0pt}
   \item[table] a pointer of type \texttt{sf\_simtab} pointing to the allocated block of memory for the symbol table.
\end{desclist}




\subsection{{sf\_simtab\_close}}\label{sec:sf_simtab_close}
Frees the allocated space for the table.

\subsubsection*{Call}
\begin{verbatim}sf_simtab_close(table);\end{verbatim}

\subsubsection*{Definition}
\begin{verbatim}
void sf_simtab_close(sf_simtab table)
/*< Free allocated memory >*/
{
   ...
}
\end{verbatim}

\subsubsection*{Input parameters}
\begin{desclist}{\tt }{\quad}[\tt table]
   \setlength\itemsep{0pt}
   \item[table] the table whose allocated memory has to be deleted. Must be of type \texttt{sf\_simtab}.
\end{desclist}




\subsection{{sf\_simtab\_enter}}\label{sec:sf_simtab_enter}
Enters a value in the table, which was created by \texttt{sf\_simtab\_init}. In the input it must be told which table to enter the value in, this is the first input argument and is of type \texttt{sf\_simtab}. The second and the third arguments are the pointers of \texttt{const char*} type. The first one points to \texttt{key}, which would be the name of the argument from command line or file. Second argument is the pointer to the value to be input.

\subsubsection*{Call}
\begin{verbatim}sf_simtab_enter(table, key, val);\end{verbatim}

\subsubsection*{Definition}
\begin{verbatim}
void sf_simtab_enter(sf_simtab table, const char *key, const char* val)
/*< Add an entry key=val to the table >*/
{
    ...
}
\end{verbatim}

\subsubsection*{Input parameters}
\begin{desclist}{\tt }{\quad}[\tt table]
   \setlength\itemsep{0pt}
   \item[table] the table in which the the key value is to be stored. Must be of type \texttt{sf\_simtab}.
   \item[key]   pointer to the name of the key value to be input (\texttt{const char*}).
   \item[val]   pointer to the key value to be input (\texttt{const char*}).
\end{desclist}




\subsection{{sf\_simtab\_get}}\label{sec:sf_simtab_get}
Extracts the value of the input key from the symbol table. It is used in other functions such as \texttt{sf\_simtab\_getint}.

\subsubsection*{Call}
\begin{verbatim}val = sf_simtab_get(table, key); \end{verbatim}

\subsubsection*{Input parameters}
\begin{desclist}{\tt }{\quad}[\tt table]
   \setlength\itemsep{0pt}
   \item[table] the table from which the vale has to be extracted. Must be of type \texttt{sf\_simtab}.
   \item[key]   the name of the entry which has to be extracted (\texttt{const char*}).
\end{desclist}

\subsubsection*{Output}
\begin{desclist}{\tt }{\quad}[\tt NULL]
   \setlength\itemsep{0pt}
   \item[val] pointer of type \texttt{char} to the desired key value stored in the table. This is the output in case there is a match between the required key and a key in the table. If there is no match between the required key and the key stored in the table, then \texttt{NULL} is returned.
\end{desclist}




\subsection{{sf\_simtab\_getint}}\label{sec:sf_simtab_getint}
Extracts an integer from the table. If the extraction is successful returns a boolean true, otherwise returns a false.

\subsubsection*{Call}
\begin{verbatim}success = sf_simtab_getint (table, key, par);\end{verbatim}

\subsubsection*{Definition}
\begin{verbatim}
bool sf_simtab_getint (sf_simtab table, const char* key,/*@out@*/ int* par)
/*< extract an int parameter from the table >*/
{
   ...
}
\end{verbatim}

\subsubsection*{Input parameters}
\begin{desclist}{\tt }{\quad}[\tt table]
   \setlength\itemsep{0pt}
   \item[table] the table from which the vale has to be extracted. Must be of type \texttt{sf\_simtab}.
   \item[key]   the name of the entry which has to be extracted (\texttt{const char*}).
   \item[par]   pointer to the integer variable where the extracted value is to be copied.
\end{desclist}

\subsubsection*{Output}
\begin{desclist}{\tt }{\quad}[\tt success]
   \setlength\itemsep{0pt}
   \item[success] a boolean value. It is \texttt{true}, if the extraction was successful and \texttt{false} otherwise.
\end{desclist}




\subsection{{sf\_simtab\_getlargeint}}\label{sec:sf_simtab_getlargeint}
Extracts a large integer from the table. If the extraction is successful, it returns a boolean true, otherwise a false. 

\subsubsection*{Call}
\begin{verbatim}success = sf_simtab_getlargeint (table, key, par);\end{verbatim}

\subsubsection*{Definition}
\begin{verbatim}
bool sf_simtab_getlargeint (sf_simtab table, const char* key,/*@out@*/ off_t* pa
r)
/*< extract a sf_largeint parameter from the table >*/
{
   ...
}
\end{verbatim}

\subsubsection*{Input parameters}
\begin{desclist}{\tt }{\quad}[\tt table]
   \setlength\itemsep{0pt}
   \item[table] the table from which the vale has to be extracted. Must be of type \texttt{sf\_simtab}.
   \item[key]   the name of the entry which has to be extracted (\texttt{const char*}).
   \item[par]   pointer to the large integer variable where the extracted value is to be copied.
\end{desclist}

\subsubsection*{Output}
\begin{desclist}{\tt }{\quad}[\tt success]
   \setlength\itemsep{0pt}
   \item[success] a boolean value. It is \texttt{true}, if the extraction was successful and \texttt{false} otherwise.
\end{desclist}




\subsection{{sf\_simtab\_getfloat}}\label{sec:sf_simtab_getfloat}
Extracts a float value from the table. If the extraction is successful, it returns a boolean true, otherwise a false. 

\subsubsection*{Call}
\begin{verbatim}success = sf_simtab_getfloat (table, key, par);\end{verbatim}

\subsubsection*{Definition}
\begin{verbatim}
bool sf_simtab_getfloat (sf_simtab table, const char* key,/*@out@*/ float* par)
/*< extract a float parameter from the table >*/
{
   ...
}
\end{verbatim}

\subsubsection*{Input parameters}
\begin{desclist}{\tt }{\quad}[\tt table]
   \setlength\itemsep{0pt}
   \item[table] the table from which the vale has to be extracted. Must be of type \texttt{sf\_simtab}.
   \item[key]   the name of the entry which has to be extracted (\texttt{const char*}).
   \item[par]   pointer to the float type value variable where the extracted value is to be copied.
\end{desclist}

\subsubsection*{Output}
\begin{desclist}{\tt }{\quad}[\tt success]
   \setlength\itemsep{0pt}
   \item[success] a boolean value. It is \texttt{true}, if the extraction was successful and \texttt{false} otherwise successfully.
\end{desclist}




\subsection{{sf\_simtab\_getdouble}}\label{sec:sf_simtab_getdouble}
Extracts a double type value from the table. If the extraction is successful, it returns a boolean true, otherwise a false. 

\subsubsection*{Call}
\begin{verbatim}success = sf_simtab_getdouble (table, key, par);\end{verbatim}

\subsubsection*{Definition}
\begin{verbatim}
bool sf_simtab_getdouble (sf_simtab table, const char* key,/*@out@*/ double* par
)
/*< extract a double parameter from the table >*/
{
   ...
}
\end{verbatim}

\subsubsection*{Input parameters}
\begin{desclist}{\tt }{\quad}[\tt table]
   \setlength\itemsep{0pt}
   \item[table] the table from which the vale has to be extracted. Must be of type \texttt{sf\_simtab}.
   \item[key]   the name of the entry which has to be extracted (\texttt{const char*}).
   \item[par]   pointer to the double type value variable where the extracted value is to be copied.
\end{desclist}

\subsubsection*{Output}
\begin{desclist}{\tt }{\quad}[\tt success]
   \setlength\itemsep{0pt}
   \item[success] a boolean value. It is \texttt{true}, if the extraction was successful and \texttt{false} otherwise.
\end{desclist}




\subsection{{sf\_simtab\_getfloats}}\label{sec:sf_simtab_getfloats}
Extracts an array of float values from the table. If the extraction is successful, it returns a boolean \texttt{true}, otherwise a \texttt{false}. 

\subsubsection*{Call}
\begin{verbatim} success = sf_simtab_getfloats (table, key, par, n);\end{verbatim}

\subsubsection*{Definition}
\begin{verbatim}
bool sf_simtab_getfloats (sf_simtab table, const char* key,
                          /*@out@*/ float* par,size_t n)
/*< extract a float array parameter from the table >*/
{
   ... 
}
\end{verbatim}

\subsubsection*{Input parameters}
\begin{desclist}{\tt }{\quad}[\tt table]
   \setlength\itemsep{0pt}
   \item[table] the table from which the vale has to be extracted. Must be of type \texttt{sf\_simtab}.
   \item[key]   the name of the \texttt{float} array which has to be extracted (\texttt{const char*}).
   \item[par]   pointer to the array of \texttt{float} type value variable where the extracted value id to be copied.
   \item[n]     size of the array to be extracted (\texttt{size\_t}).
\end{desclist}

\subsubsection*{Output}
\begin{desclist}{\tt }{\quad}[\tt success]
   \setlength\itemsep{0pt}
   \item[success] a boolean value. It is \texttt{true}, if the extraction was successful and \texttt{false} otherwise.
\end{desclist}




\subsection{{sf\_simtab\_getstring}}\label{sec:sf_simtab_getstring}
Extracts a string pointed by the input key from the symbol table. If the value is \texttt{NULL} it will return \texttt{NULL}, otherwise it will allocate a new block of memory of char type and copy the memory block from the table to the new block and return a pointer to the newly allocated block of memory. 

\subsubsection*{Call}
\begin{verbatim}string = sf_simtab_getstring (table, key);\end{verbatim}

\subsubsection*{Definition}
\begin{verbatim}
char* sf_simtab_getstring (sf_simtab table, const char* key) 
/*< extract a string parameter from the table >*/
{
   ...
}
\end{verbatim}

\subsubsection*{Input parameters}
\begin{desclist}{\tt }{\quad}[\tt table]
   \setlength\itemsep{0pt}
   \item[table] the table from which the string has to be extracted. Must be of type \texttt{sf\_simtab}.
   \item[key]   the name of the string which has to be extracted (\texttt{const char*}).
\end{desclist}

\subsubsection*{Output}
\begin{desclist}{\tt }{\quad}[\tt success]
   \setlength\itemsep{0pt}
   \item[string] a pointer to allocated block of memory containing a string of characters. 
\end{desclist}




\subsection{{sf\_simtab\_getbool}}\label{sec:sf_simtab_getbool}
Extracts a boolean value from the table. If the extraction is successful, it returns a boolean \texttt{true}, otherwise a \texttt{false}.

\subsubsection*{Call}
\begin{verbatim}success = sf_simtab_getbool (table, key, par);\end{verbatim}

\subsubsection*{Definition}
\begin{verbatim}
bool sf_simtab_getbool (sf_simtab table, const char* key,/*@out@*/ bool *par)
/*< extract a bool parameter from the table >*/
{
   ...
}
\end{verbatim}

\subsubsection*{Input parameters}
\begin{desclist}{\tt }{\quad}[\tt table]
   \setlength\itemsep{0pt}
   \item[table] the table from which the value has to be extracted. Must be of type \texttt{sf\_simtab}.
   \item[key]   the name of the entry which has to be extracted (\texttt{const char*}).
   \item[par]   pointer to the bool variable where the extracted value is to be copied.
\end{desclist}

\subsubsection*{Output}
\begin{desclist}{\tt }{\quad}[\tt success]
   \setlength\itemsep{0pt}
   \item[success] a boolean value. It is \texttt{true}, if the extraction was successful and \texttt{false} otherwise.
\end{desclist}




\subsection{{sf\_simtab\_getbools}}\label{sec:sf_simtab_getbools}
Extracts an array of boolean values from the table. If the extraction is successful, it returns a boolean \texttt{true}, otherwise a \texttt{false}. 

\subsubsection*{Call}
\begin{verbatim}success = sf_simtab_getbools (table, key, par, n);\end{verbatim}

\subsubsection*{Definition}
\begin{verbatim}
sf_simtab_getbools (sf_simtab table, const char* key,/*@out@*/bool *par,size_t n)
/*< extract a bool array parameter from the table >*/
{
   ...
}
\end{verbatim}

\subsubsection*{Input parameters}
\begin{desclist}{\tt }{\quad}[\tt table]
   \setlength\itemsep{0pt}
   \item[table] the table from which the vale has to be extracted. Must be of type \texttt{sf\_simtab}.
   \item[key]   the name of the boolean array which has to be extracted. Must be of XXXXXXXXX  pointer to the array of \texttt{bool} type value variable where the extracted value is to be copied. 
   \item[n]     size of the array to be extracted (\texttt{size\_t}).
\end{desclist}

\subsubsection*{Output}
\begin{desclist}{\tt }{\quad}[\tt success]
   \setlength\itemsep{0pt}
   \item[success] a boolean value. It is \texttt{true}, if the extraction was successful and \texttt{false} otherwise.
\end{desclist}




\subsection{{sf\_simtab\_getints}}\label{sec:sf_simtab_getints}
Extracts an array of integer values from the table. If the extraction is successful, it returns a boolean \texttt{true}, otherwise a \texttt{false}. 

\subsubsection*{Call}
\begin{verbatim}success = sf_simtab_getints (table, key, par, n);\end{verbatim}

\subsubsection*{Definition}
\begin{verbatim}
bool sf_simtab_getints (sf_simtab table, const char* key,
                        /*@out@*/ int *par,size_t n)
/*< extract an int array parameter from the table >*/
{    
   ...
}
\end{verbatim}

\subsubsection*{Input parameters}
\begin{desclist}{\tt }{\quad}[\tt table]
   \setlength\itemsep{0pt}
   \item[table] the table from which the vale has to be extracted. Must be of type \texttt{sf\_simtab}.
   \item[key]   the name of the integer array which has to be extracted (\texttt{const char*}).
   \item[par]   pointer to the array of integer type value variable where the extracted  value id to be copied.
   \item[n]     size of the array to be extracted. Must be of \texttt{size\_t}.
\end{desclist}

\subsubsection*{Output}
\begin{desclist}{\tt }{\quad}[\tt success]
   \setlength\itemsep{0pt}
   \item[success] a boolean value. It is \texttt{true}, if the extraction was successful and \texttt{false} otherwise.
\end{desclist}




\subsection{{sf\_simtab\_getstrings}}\label{sec:sf_simtab_getstrings}
Extracts an array of strings from the table. is successful, it returns a boolean \texttt{true}, otherwise a \texttt{false}.

\subsubsection*{Call}
\begin{verbatim}success = sf_simtab_getstrings (table, key, par, n);\end{verbatim}

\subsubsection*{Definition}
\begin{verbatim}
bool sf_simtab_getstrings (sf_simtab table, const char* key,
                           /*@out@*/ char **par,size_t n)
/*< extract a string array parameter from the table >*/
{    
   ...
}
\end{verbatim}

\subsubsection*{Input parameters}
\begin{desclist}{\tt }{\quad}[\tt table]
   \setlength\itemsep{0pt}
   \item[table] the table from which the vale has to be extracted. Must be of type \texttt{sf\_simtab}.
   \item[key]   the name of the string array which has to be extracted (\texttt{const char*}).
   \item[par]   pointer to the pointer to array of integer type value variable where the extracted value is to be copied.
   \item[n]     size of the array to be extracted. Must be of \texttt{size\_t}.
\end{desclist}

\subsubsection*{Output}
\begin{desclist}{\tt }{\quad}[\tt success]
   \setlength\itemsep{0pt}
   \item[success] a boolean value. It is \texttt{true}, if the extraction was successful and \texttt{false} otherwise.
\end{desclist}




\subsection{{sf\_simtab\_put}}\label{sec:sf_simtab_put}
Writes a new key together with its value to the symbol table. The new entry must be in the form \texttt{key=val} and must be of the \texttt{const char*} type, that is, this function must be given a pointer to \texttt{key=val}. Since the type of the pointer is \texttt{const char*} this can be a direct input from the command line and in that case the pointer will be \texttt{acgv[n]} where \texttt{n} specifies the position in the command line.  

\subsubsection*{Call}
\begin{verbatim}sf_simtab_put (table, keyval);\end{verbatim}

\subsubsection*{Definition}
\begin{verbatim}
void sf_simtab_put (sf_simtab table, const char *keyval) 
/*< put a key=val string to the table >*/
{
   ...
}
\end{verbatim}

\subsubsection*{Input parameters}
\begin{desclist}{\tt }{\quad}[\tt keyval]
   \setlength\itemsep{0pt}
   \item[table] the table in which the value has to be entered. Must be of type \texttt{sf\_simtab}. 
   \item[keyval] pointer to \texttt{key=val} which is to be entered.
\end{desclist}




\subsection{{sf\_simtab\_input}}\label{sec:sf_simtab_input}
Inputs a table from one file and copies it into another and also adds the new entry into the internal table using \hyperref[sec:sf_simtab_put]{\texttt{sf\_simtab\_put}}.

\subsubsection*{Call}
\begin{verbatim}sf_simtab_input ( table, fp, out);\end{verbatim}

\subsubsection*{Definition}
\begin{verbatim}
void sf_simtab_input (sf_simtab table, FILE* fp, FILE* out) 
/*< extract parameters from a file >*/
{
   ...
}
\end{verbatim}

\subsubsection*{Input parameters}
\begin{desclist}{\tt }{\quad}[\tt table]
   \setlength\itemsep{0pt}
   \item[table] the table in which the value has to be entered. Must be of type \texttt{sf\_simtab}. 
   \item[fp]    pointer to the file from which the parameter is to be read. It must be of type\texttt{FILE*}.
   \item[out]   pointer to the file in which the parameter is to be written. It must be of type FILE*.
\end{desclist}




\subsection{{sf\_simtab\_output}}\label{sec:sf_simtab_output}
Reads the parameters from the internal table and writes them to a file.

\subsubsection*{Call}
\begin{verbatim}sf_simtab_output ( table, fp);\end{verbatim}

\subsubsection*{Definition}
\begin{verbatim}
void sf_simtab_output (sf_simtab table, FILE* fp) 
/*< output parameters to a file >*/
{
   ...
}
\end{verbatim}

\subsubsection*{Input parameters}
\begin{desclist}{\tt }{\quad}[\tt table]
   \setlength\itemsep{0pt}
   \item[table] the table in which the value has to be entered. Must be of type \texttt{sf\_simtab}. 
   \item[fp]    pointer to the file in which the parameter is to be written. It must be of type \texttt{FILE*}.
\end{desclist}





\author{Boris Delaunay}
%%%%%%%%%%%%%%%%%%%%
\title{Homework 6}

\begin{abstract}
  This homework has three computational parts. 
  \begin{enumerate}
  \item Spatial interpolation contest using rainfall data from Switzerland.
  \item Stochastic simulation of natural patterns.
  \item Revisiting missing data interpolation for ocean floor topography.
  \end{enumerate}
\end{abstract}

\section{Spatial interpolation contest}
\inputdir{rain}

In 1997, the European Communities organized a Spatial Interpolation
Comparison. Many different organizations participated with the results
published in a special issue of the \emph{Journal of Geographic
Information and Decision Analysis} \cite[]{dubois} and a separate
report \cite[]{rain}.

\sideplot{elev}{width=\textwidth}{Digital elevation map of Switzerland.}

The comparison used a dataset from rainfall measurements in
Switzerland on the 8th of May 1986, the day of the Chernobyl disaster.
Figure~\ref{fig:elev} shows the data area: the Digital Elevation Model
of Switzerland with superimposed country's borders.  A total of 467
rainfall measurements were taken that day. A randomly selected subset
of 100 measurements was used as the input data the 1997 Spatial
Interpolation Comparison in order to interpolate other measurements
using different techniques and to compare the results with the known
data. Figure~\ref{fig:raindata} shows the spatial locations of the
selected data samples and the full dataset.

\plot{raindata}{width=\textwidth}{Left: locations of weather stations used as input data in the spatial interpolation contest.
Right: all weather stations locations.}

In this assignment, you will try different techniques of spatial data
interpolation and will participate in the interpolation contest.

\subsection{Delaunay triangulation}

The first technique we are going to try is Delaunay triangulation with
linear interpolation of rainfall values inside each triangle. The
result is shown in Figure~\ref{fig:trian}. Does it succeed in hiding
the acquisition footprint? Figure~\ref{fig:trian-pred} provides a
comparison between interpolated and known data values. It also
indicates the value of the correlation coefficient.

\multiplot{2}{trian,trian-pred}{width=0.45\textwidth}{(a) Rainfall data
interpolated using Delaunay triangulation. (b) Correlation between
interpolated and true data values.}

\subsection{Laplacian regularization}

An alternative technique is a solution of the regularized
least-squares optimization problem
\begin{equation}
\label{eq:laplace}
\min\left( \|\mathbf{L}\,\mathbf{m} - \mathbf{d}\|^2 + \epsilon^2 \|\mathbf{R}\,\mathbf{m}\|^2\right)\;,
\end{equation}
where $\mathbf{d}$ is irregular data, $\mathbf{m}$ is model estimated
on a regular grid, $\mathbf{L}$ is forward interpolation from the
regular grid to irregular locations, $\epsilon$ is a scaling
parameter, and $\mathbf{R}$ is the regularization operator related to
the inverse of the assumed model covariance. In this experiment,
$\mathbf{R}$ is the finite-difference approximation of the Laplacian operator.

\plot{laplace}{width=\textwidth}{Rainfall data
interpolated using regularization with the Laplacian filter.}

Figure~\ref{fig:laplace} shows the interpolation result after 10 and
1,000 iterations. Even 1,000 iterations are not enough to converge to
an acceptable solution, as is evident from the correlation analysis in Figure~\ref{fig:laplace1000-pred}.

\sideplot{laplace1000-pred}{width=\textwidth}{Correlation between
  interpolated and true data values for Laplacian regularization with 1,000 iterations.}

\subsection{Shaping regularization}

The next approach is shaping regularization: an iterative solution of the inverse problem
\begin{equation}
\widehat{\mathbf{m}} = 
  \left(\mathbf{L}^T\,\mathbf{L} + \mathbf{S}^{-1} -
    \mathbf{I}\right)^{-1}\,\mathbf{L}^T\,\mathbf{d}
  = \left[\mathbf{I} + 
    \mathbf{S}\,\left(\mathbf{L}^T\,\mathbf{L} - \mathbf{I}\right)\right]^{-1}\,
  \mathbf{S}\,\mathbf{L}^T\,\mathbf{d}\;,
  \label{eqn:shape}  
\end{equation}
where $\mathbf{S}$ is the shaping operator, which, in this experiment,
is taken as a two-dimensional triangle smoothing.

\plot{shape}{width=\textwidth}{Rainfall data
interpolated using shaping regularization with a triangle filter.}

Figure~\ref{fig:laplace} shows the interpolation result after 10 and
the maximum number of iterations. The correlation analysis with the
ground-truth data is shown in Figure~\ref{fig:shape1000-pred}.

\sideplot{shape1000-pred}{width=\textwidth}{Correlation between
  interpolated and true data values for Shaping regularization.}

\subsection{Your task}

\begin{enumerate}
\item Change directory to \verb#geo391/hw6/rain#
\item Run 
\begin{verbatim}
scons view
\end{verbatim}
to reproduce the figures on your screen.
\item Modify the \texttt{SConstruct} file to find the number of iterations required by the Laplacian regularization to converge.
\item What can you conclude about the three methods used in this comparison?
\item Participate in the Spatial Interpolation Contest. Find and
implement a method that would provide a better interpolation of the
missing values than either of the methods we tried so far. You can change any of the parameters in the existing methods or 
write your own program but you can use only the 100 original data points as input.
\end{enumerate}

\lstset{language=python,numbers=left,numberstyle=\tiny,showstringspaces=false}
\lstinputlisting[frame=single]{rain/SConstruct}

\section{Natural patterns}
\inputdir{pattern}

In this section we will extract multidimensional spatial patterns from
natural images using the method of \cite{textures}. Four examples,
shown in Figures~\ref{fig:horizon}-\ref{fig:your}, contain:
\begin{enumerate}
\item Seismic horizon slice.
\item A slice from a CT-scan of a rock sample.
\item A remote-sensing satellite image.
\item Your own data (to be replaced by you).
\end{enumerate}
In each of the cases, we follow the same workflow:
\begin{enumerate}
\item Remove a linear trend from the data.
\item Estimate a multi-dimensional prediction-error filter (PEF) on a helix.
\item Apply the inverse of the estimated PEF to random normally-distributed numbers
  to create a random spatial texture, which shares the covariance with the input. 
\end{enumerate}

\plot{horizon}{width=0.8\textwidth}{Pattern extraction from a seismic time horizon.}
\plot{square}{width=0.8\textwidth}{Pattern extraction from a CT-scan of a rock sample.}
\plot{sat}{width=0.8\textwidth}{Pattern extraction from a satellite image.}
\plot{your}{width=0.8\textwidth}{Pattern extraction from your own data.}

\lstset{language=python,numbers=left,numberstyle=\tiny,showstringspaces=false}
\lstinputlisting[frame=single]{pattern/SConstruct}

Your task:
\begin{enumerate}
\item Change directory to \verb#geo391/hw6/pattern#
\item Run 
\begin{verbatim}
scons view
\end{verbatim}
to reproduce the figures on your screen.
\item Modify the \texttt{SConstruct} file to replace Figure~\ref{fig:your} with the figure containing your own data.
\item Why does the method fail in extracting some of the patterns? Did it succeed in extracting patterns from your data?
\end{enumerate}

\section{Revisiting ocean floor data}
\inputdir{seab}

In this section, we return to the problem of interpolating ocean floor
data, discussed in Homework 5. Program~\texttt{interpolate.c}
implements two alternative methods: regularized inversion (similar to
equation~\ref{eq:laplace} but using a multi-dimensional PEF for
$\mathbf{R}$) and preconditioning, which uses the inverse of
$\mathbf{R}$ (recursive deconvolution or polynomial division on a
helix) for model reparameterization.

\plot{data}{width=0.8\textwidth}{Left: input data. Right: mask for known data values.}

The input data and a mask for known values are shown in
Figure~\ref{fig:data}. Results from the two methods are shown in Figure~\ref{fig:seabeam}.

\plot{seabeam}{width=0.8\textwidth}{Left: missing data interpolation using regularization by convolution with a prediction-error filter. 
Right: missing data interpolation using model reparameterization by deconvolution (polynomial division) with a prediction-error filter.}

\lstset{language=c,numbers=left,numberstyle=\tiny,showstringspaces=false}
\lstinputlisting[frame=single,firstline=19]{seab/interpolate.c}

\lstset{language=python,numbers=left,numberstyle=\tiny,showstringspaces=false}
\lstinputlisting[frame=single]{seab/SConstruct}

Your task:
\begin{enumerate}
\item Change directory to \verb#geo391/hw6/seab#
\item Run 
\begin{verbatim}
scons view
\end{verbatim}
to reproduce the figures on your screen.
\item Modify the \texttt{SConstruct} file to implement the following tasks
\begin{enumerate}
\item Plot the PEF-based texture simulation of the ocean floor topography similar to the previous section.
\item Find the number of iterations required for both methods to achieve similar results.
\item Generate multiple interpolation realizations using the method of Homework~5.
\end{enumerate}
\item \textbf{EXTRA CREDIT:} To provide a more quantitative comparison, modify the
\texttt{interpolate.c} program to output a measure of convergence
(such as the least-squares model misfit) as a function of the number
of iterations. Generate figures comparing convergence with and without
preconditioning.

You can study the interfaces to the \texttt{sf\_solver} and
\texttt{sf\_solver\_prec} programs\footnote{See
\url{http://rsf.svn.sourceforge.net/viewvc/rsf/trunk/filt/lib/bigsolver.c?view=markup}}
to find appropriate parameters.
\item Include your results in the paper.
\end{enumerate}

\section{Completing the assignment}

\begin{enumerate}
\item Change directory to \verb#geo391/hw6#.
\item Edit the file \texttt{paper.tex} in your favorite editor and change the
  first line to have your name instead of Delaunay's.
\item Run
\begin{verbatim}
sftour scons lock
sftour scons -c
\end{verbatim}
and
\begin{verbatim}
scons pdf
\end{verbatim}
\item Submit your result (file \texttt{paper.pdf}) by printing it out
  or by e-mail.
\end{enumerate}

\bibliographystyle{seg}
\bibliography{pattern}



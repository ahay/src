\author{Isaac Newton}
%%%%%%%%%%%%%%%%%%%%%%
\title{Homework 4}

\begin{abstract}
In this lab, we will use zero-offset and
shot-record migration to create 
images in complex velocity media.
%
Every seismic experiment (i.e. shot) can be used to create
an image. 
Every shot can be decomposed in frequency components. 
How many shots do we need to create a good image?
How many frequencies do we need to create a good image?
%
How do we build a velocity model?
What is the impact of the velocity model on the migrated image?
%
These and other questions are addressed in this lab.
\end{abstract}

\definecolor{frame}{rgb}{0.905,0.905,1.0}
\lstset{language=Python,backgroundcolor=\color{frame},showstringspaces=false}

\section{Introduction}

Start by running
\begin{verbatim}
> cd ~/geo391
> svn update
\end{verbatim}

% ------------------------------------------------------------
\section{Sigsbee 2A migration}
% ------------------------------------------------------------
In this exercise, you will modify parameters controlling 
shot-record wavefield extrapolation migration.
Everything is set-up for migration. Your task is to
identify relevant parameters, change them to generate 
new figures, include them in this document and discuss
the importance/effect of your various changes.

\inputdir{sigsbee}
\plot{slo}{width=6.0in}{Sigsbee 2A velocity}

\begin{enumerate}
%% 
\item Change directory 

\begin{verbatim}
> cd ~/geo391/hw4/sigsbee
\end{verbatim}
  \item Run
\begin{verbatim}
> scons view
\end{verbatim}
to generate figures and display them on your screen.  

%%
\item
Open the \texttt{SConstruct} file and 
find the dictionary named \texttt{par}.
Parameters 
\texttt{ns}, 
\texttt{js}, and
\texttt{fs} control sampling on the shot axis,
and parameters 
\texttt{nw}, 
\texttt{jw}, and 
\texttt{fw} control sampling on the 
frequency axis (``n''=number, ``j''=jump, ``f''=first).

%%
\item
Locate the loop over migration configuration.
A local dictionary  named \texttt{loc} allows you to 
make local changes to some parameters without affecting
the others.
Two examples are included in this document
(Figures~\ref{fig:img0} and \ref{fig:img1}).
You will include and discuss more figures.

\plot{img0}{width=6.0in}
{Monochromatic image for $1$ shot 
($ns=1$, $fs=250$, $nw=1$, $ow=2$).}

\plot{img1}{width=6.0in}{ 
Monochromatic image for $3$ shots 
($ns=3$, $fs=50$, $js=150$, $nw=1$, $ow=5$).}

%%
\item
Keep the number of frequencies fixed 
(e.g. $nw=1$, $ow=2$),
and modify the number of shots used in migration
(change $ns$, $js$, $os$).
How many shots do you need in order to see the main 
features of the structure?
Can you identify the structure?
Include a new figure and discuss your observations.

%% 
 % \plot ...
 %%


%%
\item
Keep the number of shots fixed (e.g. $ns=1$) 
and change the migration frequency (change $ow$).
The migrated images will change accordingly and
the subsurface illumination will change.
Modify the location of the shot (change $fs$)
and observe illumination patterns, especially around
the salt body.
How does illumination change with frequency?
Include figures for images at different frequencies
and discuss your observations.

%% 
 % \plot ...
 %%

%%
\item
Keep the number of shots fixed ($ns=1$) 
and change the number of frequencies 
(change $nw$, $jw$, $ow$).
How does the image change with increased frequency band?
Include figures for shots at different locations in the 
image and discuss your observations.

%% 
 % \plot ...
 %%

%%
\item
Increase the number of shots 
(change $ns$, $js$, $os$)
and frequencies 
(change $nw$, $jw$, $ow$).
until you are satisfied with the quality of the migrated image.
How did you decide when to stop?
What parameters did you use?
Include one or more new figures and discuss your 
observations and choice of parameters.

%% 
 % \plot ...
 %%

%%
\item
Using your optimal choice of parameters for shots
and frequencies from the preceding question, 
re-migrate the Sigsbee 2A data using a smooth velocity model.
Generate new migration rules,
experiment with different smoothing parameters and 
discuss how accurate does the velocity model need to be
in order to obtain a good image.
Where is the impact of the velocity model accuracy largest?
How did you decide how much to smooth? Why?
Include one or more new figures and discuss your 
observations and choice of parameters.

%% 
 % \plot ...
 %%


%%
\item
This is a free-form question.
Experiment with migration parameters on your own.
Include and discuss another migration configuration
of your choice. Discuss your new figure and observations.

%% 
 % \plot ...
 %%

%%    
\item After you are done, run

\begin{verbatim}
> scons lock
> scons -c
\end{verbatim} 
  
\end{enumerate}


% ------------------------------------------------------------
\section{Blake Outer Ridge migration}
% ------------------------------------------------------------
In this exercise, you will experiment with zero-offset 
migration and simple velocity analysis.
The \texttt{SConstruct} file is configured with all the rules
used to create a velocity model from semblance scans and
depth-migrate the data.

\inputdir{blake}
\plot{noff}{width=6.0in}{Blake Outer Ridge near-offset data.}

\plot{vscan}{width=6.0in}{Velocity scan.}

\plot{picks}{width=6.0in}{Stacking velocity.}

\plot{velz}{width=6.0in}{Blake Outer Ridge interval velocity model.}

\plot{velz}{width=6.0in}{Blake Outer Ridge interval velocity model.}

\plot{img}{width=6.0in}{Blake Outer Ridge migration of 
near-offset data.}

\begin{enumerate}
%% 
\item Change directory 

\begin{verbatim}
> cd ~/geo391/hw4/blake
\end{verbatim}
  \item Run
\begin{verbatim}
> scons view
\end{verbatim}
to generate figures and display them on your screen.  

%%
\item
Review all rules in the \texttt{SConstruct} used to generate
Figures~\ref{fig:velz} and \ref{fig:img}.
\begin{itemize}
\item Identify the rules for velocity scans.
\item Identify the rules for conversion from stacking
to interval velocity.
\item Identify the rules for conversion of interval velocity
from time to depth.
\item Identify the rules for datuming and depth migration.
\end{itemize}

%%
\item
Add rules to perform another migration with constant velocity.
Create a new Result and compare it with the image
obtained by migration with a laterally variable velocity.
How do the two images compare?
Include one or more new figures and discuss your 
observations.

%% 
 % \plot ...
 %%

%%
\item
In the original example, the velocity analysis
is performed in the CMP-domain every $10^{th}$ gather.
Add rules to recompute velocity scans at every CMP
(also adjust the "rect1= rect2=" parameters for optimal 
smoothing). 
Create another figure with the image obtained using
your new velocity model and compare it with the image
produced by the original \texttt{SConstruct}.
How do the two images compare?
Include one or more new figures and discuss your 
observations.

%% 
 % \plot ...
 %%

%%    
\item After you are done, run

\begin{verbatim}
> scons lock
> scons -c
\end{verbatim} 
  
\end{enumerate}


% ------------------------------------------------------------
\section{Wrap-up}
% ------------------------------------------------------------

\begin{enumerate}  

%%
\item 
Edit the file
\verb#~/geo391/hw4/paper.tex# in your favorite editor and 
change the first line to have your name. Run
\begin{verbatim}
> scons pdf
\end{verbatim}
and submit your result (file \texttt{paper.pdf}) on paper 
or by e-mail.

This homework is due in class on November $29^{th}$, $2005$.

\end{enumerate}

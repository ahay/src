

\published{IEEE Transactions on Geoscience and Remote Sensin, 63, 1-9, Art no. 5902809, (2025)}

\title{Fast Streaming Local Time-frequency Transform for Nonstationary Seismic Data Processing}

\renewcommand{\thefootnote}{\fnsymbol{footnote}}

\ms{TGRS-2024-05672}

\address{
\footnotemark[1] College of Geo-exploration Science and Technology,\\
Jilin University, Changchun, China }

\author{Jiawei Chen\footnotemark[1], Yang Liu\footnotemark[1], You Tian\footnotemark[1], and Peihong Xie\footnotemark[1]}

\footer{TGRS-2024-05672}
\lefthead{Chen et al.}
\righthead{SLTFT}

\maketitle

\begin{abstract}
	Time-frequency analysis serves as a useful approach to solve
	different complex problems in seismic data processing.  From a
	practical standpoint, the majority of time-frequency transform
	techniques frequently grapple with the trade-off between time
	and frequency localization adaptability, flexibility in
	sampling time and frequency, and the pursuit of computational
	efficiency.  To address this, we tailor the streaming
	computation to implement a fast time-frequency transform,
	namely the streaming local time-frequency transform (SLTFT),
	which can significantly decrease the computational cost of
	adaptive time-frequency analysis. We add a localization scalar
	to the proceeding streaming algorithm to circumvent the need
	for taper functions, which provides rapid forward and inverse
	transforms and applicability in various scenarios.  We
	demonstrate the adaptive time-frequency characteristics of the
	proposed method, which offers a nonstationary time-frequency
	representation with variable time-frequency
	localization. Numerical tests indicate that the proposed SLTFT
	is a more balanced method compared to previous time-frequency
	adaptive transforms. It proves suitable for a range of
	practical applications in nonstationary seismic data
	processing, including ground-roll attenuation, inverse-Q
	filtering, and multicomponent data registration.
\end{abstract}

\section{Introduction}

Spectral estimation and corresponding time-frequency analysis for
nonstationary signals is a cornerstone in geophysical data
analysis \cite[]{Tary2014}, and has been widely applied in various
geophysical data processing
tasks \cite[]{Tan2022,Yao2022,Bai2022,Mohammadigheymasi2022,chen2021}. The
short-time Fourier transform (STFT) is a fundamental technique for
time-frequency analysis \cite[]{Allen77,liu2016stft,lu2013stft}. This
method involves applying a series of short tapers to the input signal
and subsequently performing a Fourier transform on each tapered
segment. The spectrum provides localized time-frequency distributions
for each data point and displays a detailed view of the frequency
characteristics over time. The continuous wavelet transform
(CWT) \cite[]{Chakraborty95} is another widely used alternative for
time-frequency analysis. CWT replaces the length-fixed taper in STFT
with mother wavelets that can be shifted in time and stretched by a
scale, thus yielding a variable and concentrated time-frequency map of
the input time series. The
S-transform \cite[]{Stockwell96,liu2019st,Wu2023} combines the
properties of both STFT and CWT, and can be viewed as a middle ground
between these two methods. It employs a time-shifted and squeezed
Gaussian taper, which is analogous to the mother wavelet used by CWT,
while keeps a uniform frequency sampling and retains the original
signal phase \cite{stockwell2007}. Recently developed methods, such as
matching pursuit (MP) \cite{mallat1993}, basis pursuit
(BP) \cite[]{chen2001}, Wigner-Vile
distribution \cite[]{Boashash1987}, empirical mode decomposition
(EMD) \cite[]{huang1998emd,flandrin2004emd,Zhang2020}, the
synchrosqueezing transform \cite[]{daubechies2011}, sparse
time-frequency transform \cite[]{gholami2012sparse,yang2021sparse} and
the deep learning (DL)-based
method \cite[]{liu2023NN,Qian2022,Yang2022} have been designed to
mitigate issues such as spectral smearing and leakage. These methods
aim to achieve superior time-frequency localization, which enhances
the precision and clarity of signal analysis. Although these methods
can be powerful tools for spectra estimation, their fixed frequency
sampling and stationary time-frequency localization limits their
applications in nonstationary seismic data processing. Additionally,
they often suffers in high computational costs, which make them less
suitable for complex tasks such as massive seismic data processing.

Local attributes method employs regularized nonstationary regression
to decompose input data into a number of nonstationary
components \cite[]{Fomel07a}, and is proved an effective tool to
measure nonstationary seismic data characteristics and has been
successfully applied to seismic data stacking \cite[]{Liu11b}, noise
attenuation \cite[]{Zheng2022,Liu11b}, image
registration \cite[]{Fomel09b} and phase
detection \cite[]{Fomel10}. \cite{Liu11} applied shaping regularized
nonstationary regression for computing nonstationary local frequency
attributes. They used the Gaussian smoothing operator to control the
time-frequency localization and avoids the window strategy used in
those conventional methods. \cite{Liu13} expanded the local frequency
attributes \cite[]{Liu11} by designing an invertible nonstationary
time-frequency decomposition, which is effective in practical
processing tasks. \cite{chen2021} further developed the nonstationary
local time-frequency transform method by applying smoothness
constraints in all physical dimensions, which ensures a higher
resolution and antinoise ability of the LTF decomposition
method. Although the local attributes and LTF analysis methods prove
to be effective in seismic data analysis and
processing \cite[]{Liu11,Liu13,chen2021}, their high computational
cost brought by iterative inversion framework limits their practical
applications. To improve calculation efficiency, \cite{Geng24}
employed the streaming algorithm \cite[]{Fomel16} to estimate local
seismic attributes. Instead of using the shaping
regularization, \cite{Geng24} incorporated streaming computation with
local attributes to obtain time-frequency map. However, the direct
application of the streaming algorithm leads to a global frequency map
rather than a local one, so \cite{Geng24} implemented a window
strategy to ensure localization.


The streaming method, first proposed by Fomel and
Claerbout \cite[]{Fomel16,Fomel24}, is an efficient non-iterative
approach to compute nonstationary prediction-error filters
(PEFs). \cite{Geng24} proposed an efficient streaming seismic
attributes method by employing the streaming method to solve the
nonstationary regression problem in local frequency
attributes \cite[]{Liu11}. In this paper, we expand the streaming
seismic attributes method \cite[]{Geng24} by improving the streaming
calculation and designing an invertible streaming local time-frequency
transform (SLTFT). We tailor the streaming computation by adding a
localization scalar to control the localization of the time-frequency
spectrum, which avoids the window strategy and provides an adaptively
localized time-frequency map. The proposed SLTFT is computational
efficient with its forward and inverse transform requiring only
elementary algebraic operations and no iterations. The benchmark
signal example indicate that the proposed SLTFT can provide more
flexible time-frequency resolution than the classical S-transform and
STFT. Field data tests including ground-roll attenuation, inverse-Q
filtering and multicomponent data registration indicate that the
proposed SLTFT provides an effective tool for analyzing and processing
nonstationary seismic data.

\section{Theory}
\subsection{Streaming Local Time-frequency Transform}
The discrete Fourier transform is a vital tool for signal analysis and
processing. Given a casual discrete signal $s[n]$ with a fixed length
of $N$, its Fourier series can be derived by the inner product of the
signal with a family of sines and cosines
\begin{equation}
	\label{eq:dft}
	C[k] = \frac{1}{N}\left< s[n],\psi_k[n] \right> = \frac{1}{N}\sum_{n=0}^{N-1} s[n]e^{-\text{j}2\pi k\Delta f (n/N)},
\end{equation}
\noindent where $\Delta f$ is the frequency interval, $k$ and $n$ are the
discrete frequency index and time index, respectively. $C[k]$ are the
Fourier coefficients, $\psi_k[n] = \exp\left[-\text{j}2\pi k\Delta f
(n/N)\right]$ are the complex-value bases, and $[\cdot]$ stands for
the index of a discrete sequence. The signal can be expressed by the
inverse form of equation \ref{eq:dft}, which is written as
\begin{equation}
	\label{eq:idft} s[n] = \left< C[k] ,\psi^*_k[n] \right>
	= \sum_{k=0}^{N-1} C[k]e^{\text{j}2\pi k\Delta f (n/N)},
\end{equation}
where $\psi^*_k[n]$ is the complex conjugate of $\psi_k[n]$.  We can
rewrite the equation \ref{eq:idft} in a prediction-error regression
form as\\
\begin{equation}
	\label{eq:pef}
    e[n] = s[n] - \sum_{k=0}^{N-1} C[k] \psi^*_k[n],
\end{equation}
\noindent where $e[n]$ is the prediction error. Then we can apply the
nonstationary regression \cite[]{Fomel09} to the Fourier series
regression in equation \ref{eq:pef}, which allows the coefficients
$C[k]$ to vary over time coordinate $n$. The error turns
into \cite[]{Fomel09}
\begin{equation}
	\label{eq:npef} e[n] = s[n] - \sum_{k=0}^{N-1}
    C[k,n] \psi^*_k[n].
\end{equation}
The nonstationary coefficients $C[k,n]$ can be obtained by solving the
least-squares minimization problem\cite[]{Liu13}:

\begin{equation}
	\label{eq:lsdft}
	\min_{C}\,\left\Vert s[n]-\sum_{k=0}^{N-1} C[k,n] \psi_k^{*}[n]\right\Vert_2^2.
\end{equation}

The nonstationary regression makes the minimization of
equation \ref{eq:lsdft} become ill-posed, and a reasonable solution is
to include additional constraints. Classical regularization methods,
such as Tikhonov regularization \cite[]{Tikhonov63} and shaping
regularization \cite[]{Fomel07b,Liu13,chen2021}, can be used to solve
the ill-posed problem. The streaming computation is an efficient
algorithm which enables a fast way to solve the nonstationary
regression \cite[]{Fomel16,Geng24,Fomel24}. It employs a streaming
regularization, where the new coefficients are assumed to be close to
the previous ones
\begin{equation}
    \label{eq:approx}
    C[k,n] \approx C[k,n-1].
\end{equation}
In matrix notation, these conditions can be combined into an
overdetermined linear system \cite[]{Fomel16,Fomel24}:

\begin{equation}
		\begin{aligned}
			\label{eq:streaming}
            \left[
			\begin{array}{c}
				s[n]\\
				\lambda C[0,n-1]\\\lambda C[1,n-1]\\
				\cdots \\
				\lambda C[k,n-1]\\
			\end{array}
			\right] 
			\approx&\left[
			\begin{array}{cccc}
				\psi^{*}_0[n]  & \psi^{*}_1[n] & \cdots & \psi^{*}_k[n]\\
				\lambda & 0 & \cdots & 0\\
				0 & \lambda & \cdots & 0\\
				\vdots   & \vdots  & \ddots  & \vdots \\
				0 & 0 & \cdots & \lambda\\
			\end{array}
			\right]\\
			&\times \left[
			C[0,n], C[1,n], \cdots, C[k,n]
			\right]^T,
		\end{aligned}
	\end{equation}
where $\lambda$ is the parameter that controls the deviation of
$C[k,n]$ from $C[k,n-1]$. To simplify the notation, one can rewrite
equation \ref{eq:streaming} in a shortened block-matrix form as

\begin{eqnarray}
	\label{eq:block}
	\left[
	\begin{array}{c}
		\mathbf{\Psi}[n]\\
		\lambda {\mathbf{I}}
	\end{array}
	\right]
	\mathbf{C}[n]
	\approx
	\left[
	\begin{array}{c}
		s[n]\\
		\lambda \mathbf{C}[n-1]
	\end{array}
	\right],
\end{eqnarray}
\noindent where $\mathbf{I}$ is an identity matrix and
\begin{equation}
	\begin{aligned}
		\label{eq:def}
		\mathbf{\Psi}[n]&=
		\left[
		\begin{array}{cccc}
			\psi^{*}_0[n], & \psi^{*}_1[n], & \cdots, & \psi^{*}_k[n]
		\end{array}
		\right], \\
		\mathbf{C}[n]&=
		\left[
		\begin{array}{cccc}
			C[0,n],&C[1,n],& \cdots, & C[k,n]
		\end{array}
		\right]^T.
	\end{aligned}
\end{equation}

However, the streaming regularization in equation \ref{eq:approx}
offers an equal approximation for all previous time samples, which
means $C[k,0]$ influences $C[k,n]$ as much as $C[k,n-1]$ does. This
leads to a global frequency spectrum (similar to the discrete Fourier
transform) rather than a local one. Hence, \cite{Geng24} uses the
taper strategy and performs streaming computations repeatedly to
obtain the local frequency attributes. Although streaming algorithm
can speed up the progress, it has a limited efficiency due to a large
amount of repetitive calculations caused by the window functions. In
this study, we introduced a localization scalar $\varepsilon$ to limit
the smoothing radius and avoid the repeated computations brought by
taper functions. The modified inverse problem is expressed as follows
\begin{eqnarray}
	\label{eq:block1}
	\left[
	\begin{array}{c}
		\mathbf{\Psi}[n]\\
		\lambda {\mathbf{I}}
	\end{array}
	\right]
	\mathbf{C}[n]
	\approx
	\left[
	\begin{array}{c}
		s[n]\\
		\lambda\varepsilon^2\mathbf{C}[n-1]
	\end{array}
	\right].
\end{eqnarray}

The localization scalarin $\varepsilon$ is defined in
$\left(0,1\right]$ and provides a decaying and localized smoothing
constraint that
\begin{equation}
    \label{eq:approx1}
    \mathbf{C}[n] \approx \varepsilon^2\mathbf{C}[n-1]\approx \varepsilon^4\mathbf{C}[n-2].
\end{equation}
The influence of the previous coefficients on the current coefficients
decreases exponentially with the distance between them, which yields a
localized frequency spectrum while avoiding the taper strategy. We use
the least-squares algorithm to solve equation \ref{eq:block1}. The
minimization problem is
\begin{equation}
	\label{eq:ls1}
	\min_{\mathbf{C}}\,\left\Vert s[n]-\mathbf{\Psi}[n] \mathbf{C}[n] \right\Vert_2^2+ 
	\lambda^2 \left\Vert \mathbf{C}[n]-\varepsilon^2\mathbf{C}[n-1]\right\Vert_2^2\;,
\end{equation}
\noindent and the least-squares solution of equation \ref{eq:ls1}
is \cite[]{Fomel16,Fomel24}
\begin{equation}
	\begin{aligned}
		\label{eq:solution0}
		\mathbf{C}[n]= &\left(\mathbf{\Psi}^T[n]\mathbf{\Psi}[n]+\lambda^2{\mathbf{I}}\right)^{-1} \\
		& \times \left(s[n]\mathbf{\Psi}^T[n]+\lambda^2\varepsilon^2\mathbf{C}[n-1] \right).
	\end{aligned}
\end{equation}
\noindent The Sherman-Morrison formula \cite[]{sherman1950} is used to
directly calculate the inverse matrix as follows
\begin{equation}
	\label{eq:SM}
	\left(\mathbf{\Psi}^T[n]\mathbf{\Psi}[n]+\lambda^2{\mathbf{I}}\right)^{-1}=
	\frac{1}{\lambda^2}\left({\mathbf{I}}-\frac{\mathbf{\Psi}^T[n]\mathbf{\Psi}[n]}{{\lambda^2+\mathbf{\Psi}}[n]\mathbf{\Psi}^T[n]} \right) .
\end{equation}
\noindent After substituting equation \ref{eq:SM} into equation
\ref{eq:solution0}, the coefficients $\mathbf{C}[n]$ can be obtained by
\begin{equation}
	\label{eq:solution1}
	\begin{aligned}
		\mathbf{C}[n] &=\frac{1}{\lambda^2}
		\left(
		{\mathbf{I}}-
		\frac{\mathbf{\Psi}^T[n]\mathbf{\Psi}[n]}
		{{\lambda^2+ \mathbf{\Psi}}[n]\mathbf{\Psi}^T[n]} 
		\right)
		\\&\hspace{0.26\hsize}\times\left(
		s[n]\mathbf{\Psi}^T[n]+\lambda^2\varepsilon^2\mathbf{C}[n-1] 
		\right) \\
		&=\varepsilon^2 \mathbf{C}[n-1] + 
		\frac{s[n]-\varepsilon^2\mathbf{\Psi}[n]\mathbf{C}[n-1]}
		{\lambda^2+\mathbf{\Psi}[n]\mathbf{\Psi}^T[n]}
		\mathbf{\Psi}^T[n].
	\end{aligned}
\end{equation}

Equation \ref{eq:solution1} shows that the coefficients
$\mathbf{C}[n]$ at $n$ is calculated by the data point $s[n]$ and the
previous coefficients $\mathbf{C}[n-1]$, but the frequency information
of the data points after $n$ is not included.

This could lead to a small time-shift in the time-frequency domain. To
avoid the time-shift effect, we obtain the center-localized spectrum
by implementing the streaming computation forward and backward along
the time direction and adding the results
together. Fig.\ref{fig:streamingLTFTa,streamingLTFTb} illustrates the
main processes of the proposed SLTFT and the streaming local
attribute \cite[]{Geng24}. The proposed method avoids repeatedly
windowing the data, and can obtain the result streamingly all at
once. Analogous to LTF decomposition \cite[]{Liu13}, the absolute
value of $|\mathbf{C}[n]|$ represents the localized time-frequency
distribution of $s[n]$, and equation \ref{eq:solution1} can be simply
inverted to reconstruct the original data $s[n]$ from the coefficients
$\mathbf{C}[n]$ \cite[]{Fomel16,Fomel24}:
\begin{equation}
	\label{eq:invert}
	\begin{aligned}
		s[n] =& \left( \frac{\lambda^2}{\mathbf{\Psi}[n]\mathbf{\Psi}^T[n]}+1\right)\mathbf{\Psi}[n]\mathbf{C}[n]\\
		&-\frac{\lambda^2\varepsilon^2}{\mathbf{\Psi}[n]\mathbf{\Psi}^T[n]}\mathbf{\Psi}[n]\mathbf{C}[n-1].
	\end{aligned}
\end{equation}

The inversion using equation \ref{eq:invert} is suffer from the
trade-off between accuracy and
efficiency \cite[]{Fomel16,Fomel24}. Another way to reconstruct the
original data is to directly apply the
equation \ref{eq:idft}. Additionally, we use an amplitude recovery
factor $\alpha$ defined by
\begin{equation}
	\label{eq:amp}
	\alpha = \frac{s[n]}{\mathbf{\Psi}[n]\mathbf{C}[n]},
\end{equation}
to ensure the amplitude unchanged. Then the amplitude-preserving
coefficients $\hat{\mathbf{C}}[n]$ can be obtained by
\begin{equation}
	\label{eq:acoeffs}
	\hat{\mathbf{C}}[n] = \alpha\mathbf{C}[n]=\frac{s[n]}{\mathbf{\Psi}[n]\mathbf{C}[n]}\mathbf{C}[n],
\end{equation}
and the original signal $s[n]$ can be precisely reconstructed by
\begin{equation}
	\label{eq:idft1}
	s[n] = \mathbf{\Psi}[n]\hat{\mathbf{C}}[n].
\end{equation}

We utilize a benchmark chirp signal (see Fig.\ref{fig:cchirps,sltft,sltft1}a)
to further illustrate the role of the localization
scalar. Fig.\ref{fig:cchirps,sltft,sltft1}b presents the time-frequency map
derived from the streaming local attribute \cite[]{Geng24}, which
fails in providing a localized time-frequency map if the taper
function is removed to reduce the computational costs. However, the
proposed SLTFT without window functions can provide a reasonable
result (see Fig.\ref{fig:cchirps,sltft,sltft1}c).

Compared to the streaming local attribute with the taper function, the
proposed SLTFT by updating the coefficients according to
equation \ref{eq:solution1} requires only elementary algebraic
operations, which effectively reduces computational cost without
iteration. According to equation \ref{eq:def}, $\mathbf{C}[n]$ is a
$k\times 1$ vector and the size of $\mathbf{\Psi}[n]$ is $1\times k$
for any $n$, thus the computational complexities of
$\varepsilon^2\mathbf{\Psi}[n]\mathbf{C}[n-1]\mathbf{\Psi}^T[n]$ and
$s[n]\mathbf{\Psi}^T[n]$ are both $O(k)$. Meanwhile,
\begin{equation}
	\label{eq:constant}
	\begin{aligned}
		\mathbf{\Psi}[n]\mathbf{\Psi}^T[n] &=
		\left[\begin{array}{c}
			\psi^{*}_0[n], \\ \psi^{*}_1[n], \\\cdots, \\ \psi^{*}_k[n]
		\end{array}\right]^T
		\left[\begin{array}{c}
			{\psi_0[n]}, \\ {\psi_1[n]}, \\ \cdots, \\ {\psi_k[n]}
		\end{array}\right] \\
		&=\sum_{i=0}^{k}{e^{\text{j}2\pi i (n/N)}e^{-\text{j}2\pi i (n/N)}}=k, 
	\end{aligned}
\end{equation}
\noindent which is a constant for every $n$ and is computed only once.
According to equations \ref{eq:solution1} and \ref{eq:acoeffs}, the
computational complexity of the proposed SLTFT method is $O(N\cdot
k)$. Table \ref{tb:comparison} compares the complexities of different
approaches, which shows the proposed method has the lowest requirement
for computational resources. Fig.\ref{fig:time} further shows the CPU
time of the different methods. We select the fixed frequency sample of
500 and the fixed window length of 100 (for those who need a taper
function). The number of iteration is set to 50 in the LTF
decomposition. All these records are obtained by taking the average of
5 measurements. Fig.\ref{fig:time} is visually in line with the
theoretical complexity shown in table \ref{tb:comparison}. It is clear
that the proposed method offers a fast transform almost equivalent to
the STFT and is much more efficient than the LTF decomposition and the
streaming local attributes method. Moreover, it combines the
advantages of flexible frequency sampling and the adaptability of time
and frequency localization, which are not achievable with the
STFT. This enables fast local time-frequency analysis and processing,
especially for large-scale seismic data, e.g., passive seismic
data \cite[]{Geng24}.

\inputdir{.}
\multiplot{2}{streamingLTFTa,streamingLTFTb}{width=0.85\hsize}{Schematic illustration of (a) the proposed SLTFT and (b) the streaming local attributes method.}

\begin{table}
	\centering
	\caption{Comparison of Theoretical Computational Costs among Different Methods.
	}
	\label{tb:comparison}
    \resizebox{\textwidth}{!}{
	\begin{threeparttable}
		\begin{tabular}[htbp]{cccccc}
			\toprule
			\multirow{2}*{Algorithm}
			& \multicolumn{3}{c}{
				\multirow{2}*{Parameters}}
			& \multirow{2}*{\makecell{Theoretical\\complexity}}
			& \multirow{2}*{Invertible} \\ \\
			\cline{1-6}
			\multirow{3}*{STFT} 
			& \multicolumn{3}{c}{\multirow{6}{*}{
					\makecell{Data size of $N$,\\window length $m$,\\frequency samples $k$}}} 
			& \multirow{3}*{$O(N\cdot m\cdot \log k)$} &
			\multirow{3}*{Yes} 
			\\ \\ \\
			\cline{1-1}\cline{5-6}
			\multirow{3}*{\makecell{streaming\\local\\attributes}}
			& \multicolumn{3}{c}{\multirow{3}*{}}
			& \multirow{3}*{$O(N\cdot m\cdot k)$} &
			\multirow{3}*{No} 
			\\ \\ \\
			\cline{1-6}
			\multirow{3}*{\makecell{the LTF\\decomposition}}
			& \multicolumn{3}{c}{\multirow{3}*{
					\makecell{Data size of $N$,\\ iteration numbers $l$,\\frequency samples $k$}}} 
			& \multirow{3}*{$O(N\cdot l\cdot k)$} &
			\multirow{3}*{Yes} 
			\\ \\ \\
			\cline{1-6}
			\multirow{3}*{\makecell{SLTFT}}
			& \multicolumn{3}{c}{\multirow{3}*{
					\makecell{Data size of $N$,\\frequency samples $k$}}} 
			& \multirow{3}*{$O(N\cdot k)$} &
			\multirow{3}*{Yes} 
			\\ \\ \\
			\bottomrule
		\end{tabular}

	\end{threeparttable}
    }
\end{table}


\inputdir{chirps}
\multiplot{3}{cchirps,sltft,sltft1}{width=0.45\hsize}{Schematic illustration
of (a) the synthetic chirp signal and its time-frequency map obtained
by (b) the streaming local attributes without window and (c) the
proposed SLTFT.}

\inputdir{.}
\plot{time}{width=0.6\hsize}{The CPU time comparison among the time-frequency
analysis methods. Orange line: STFT; blue dash line: SLTFT; red dot
line: streaming attributes; green dash-dot line: LTF
decomposition. The convergence speed affects the CPU time of the LTF
decomposition, resulting in a non-smooth curve.}


\subsection{Adaptive Time-frequency Localization of SLTFT} 
\inputdir{adaptive}

The key different between S-transform and STFT is that S-transform
employs a frequency-varying Gaussian taper, which provides
$t\textrm{-}f$ spectrum with variable $t\textrm{-}f$
localization. Considering that the localization scalar $\varepsilon$
controls the $t\textrm{-}f$ localization of SLTFT, we design the new
localization scalar varying over frequency
\begin{equation}
	\varepsilon = \varepsilon(f),
\end{equation}
\noindent where $f$ is the frequency. Increment of the localization scalar
enhances the time localization of the spectrum, albeit at the cost of
the reduced frequency localization, and vice versa. This inverse
relationship allows us to adjust the balance between time and
frequency localization according to specific requirements.

We use two synthetic nonstationary models to to illustrate the
time-frequency characterization of the proposed method. The first
model (refered to as signal 1) comprises two hyperbolic signals
$s_1(t)$, $s_2(t)$ and Gaussian noise, where
\begin{empheq}[left={\empheqlbrace}]{align}
	s_1(t) &= \cos\left[{
		2\pi\left( {10t+\frac{3.75}{0.77-t}} \right)
	}\right], \label{eq:s1}\\
	s_2(t) &= \cos\left[{
		2\pi\left( {-1.5t+\frac{1.25}{0.78-t}} \right)
	}\right]. \label{eq:s2}
\end{empheq}

The signal 1 with a time interval of 0.5 ms contains 2000 samples
between each sample. We windowed the periodic signals by a
length-fixed cosine taper. The first signal $s_1(t)$ has a time
duration ranging from 0.1 to 0.7 seconds, while the second signal
$s_2(t)$ spans from 0.25 to 0.75 seconds. The S-transform provides a
variable $t\text{-}f$ localization (see Fig.\ref{fig:st}) that allows
the spectrum to exhibit enhanced frequency resolution at lower
frequencies and superior time localization at higher
frequencies. Fig.\ref{fig:stft} shows the spectrum calculated from the
STFT with fixed window length. It is noted that the spectrum exhibits
improved frequency localization at lower frequencies. However, this
enhancement comes at the expense of reduced time localization around
higher frequencies. Therefore, we can apply the frequency-varying
localization scalers (see Fig.\ref{fig:epss}) to the SLTFT, and the
result (see Fig.\ref{fig:st,st1,stft,stft1,sltft,sltft1,epss}e)
provides a reasonable time-frequency spectrum similar to that of the
S-transform.

We selected two different hyperbolic signals $s_3(t)$, $s_4(t)$ in
replace those in the second model (refered to as signal 2) to further
test the SLTFT, where
\begin{empheq}[left={\empheqlbrace}]{align}
	s_3(t) &= \cos\left[{
		2\pi\left( {690t-\frac{3.75}{0.77-t}} \right)
	}\right], \label{eq:s3}\\
	s_4(t) &= \cos\left[{
		2\pi\left( {700t-\frac{1.25}{0.78-t}} \right)
	}\right]. \label{eq:s4}
\end{empheq}
In this case, the S-transform encounters difficulties in producing an
accurate spectrum shown in Fig.\ref{fig:st1}, which displays
significant aliasing since its high time resolution leads to
diminished frequency localization at higher frequencies. This
observation shows that the variable time-frequency localization
inherent in the S-transform introduces a trade-off between the
representation of low and high frequency components. For comparison,
the STFT with constant window length is hard to achieve a balance
between time and frequency localization. As illustrated in
Fig.\ref{fig:stft1}, the spectrum obtained by the SLTFT with
constant localization scaler exhibits better frequency localization
(around higher frequencies), but comes at the expense of diminished
time localization (at lower frequencies). Fig.\ref{fig:epss}
shows the frequency-varying localization scalers applied in this case,
which has a trend opposite to that of signal 1. The flexible
frequency-varying parameters lead to a resonable spectrum obtained by
the proposed SLTFT (see Fig.\ref{fig:sltft1}).

\multiplot{7}{st,st1,stft,stft1,sltft,sltft1,epss}{width=0.45\hsize}{The
time-frequency map of synthetic signal 1 obtained by (a) S-transform,
(c) STFT with fixed window length and (e) SLTFT with frequency-varying
localization scalers. The time-frequency map of synthetic signal 2
obtained by (b) S-transform, (d) STFT with fixed window length and (f)
SLTFT with frequency-varying localization scalers. (g) The
frequency-varying localization scalers of SLTFT used in
Fig.\ref{fig:st,st1,stft,stft1,sltft,sltft1,epss}e (solid line) and
Fig.\ref{fig:sltft1} (dash line).}

The above examples show the adaptive and flexible frequency sampling
of SLTFT in signal analysis, which provides an adjustable
time-frequency representation. Next, we use several field datasets to
show its performance in seismic data processing tasks.

\section{Applications}
\subsection{Fast Ground-roll Attenuation in Time-frequency Domain}
\inputdir{wz}
Ground roll or surface wave is a common type of interference wave in
land seismic surveys, which is distinguished by high amplitudes and
low frequencies. Time-frequency algorithms are effective for
ground-roll attenuation \cite[]{elboth2010}.

We first use an open-source OZ-25 dataset, which is a typical dataset
suitable for testing the ground-roll noise attenuation performance and
has been widely used in the geophysics
community \cite[]{yilmaz1987,yarham2006,chen2015,Tao20}. Fig.\ref{fig:field}
shows the raw dataset in common-shot domain with 81 traces and 2000
samples per trace. The temporal and spatial sampling intervals are
0.002 s and 0.05 km, respectively.  We use the proposed SLTFT to
generate the $t\text{-}f\text{-}x$ spectrum coefficients (see
Fig.\ref{fig:field,sltft,cmask}b). The noise energy is primarily concentrated
within a triangular zone surrounding the near offset in low frequency
band. We borrow a similar strategy from \cite{Liu13} and design a
filter mask to attenuate the noise energy cluster localized in both
frequency and space (see Fig.\ref{fig:field,sltft,cmask}c). Then we use the
inverse SLTFT to bring back the separated signal (see
Fig.\ref{fig:isltft}). The denoised dataset shows that the strong
interference noise is well-attenuated and the reflection signals are
preserved. We use bandpass filtering (see Fig.\ref{fig:bp}) and LTF
decomposition (see Fig.\ref{fig:iltft}) to compare the denoising
performance.
Fig.\ref{fig:bp_wig1,iltft_wig1,isltft_wig1,bp_wig2,iltft_wig2,isltft_wig2}
shows zoomed-in sections from the separated signals in
Fig.\ref{fig:bp,bp_err,iltft,iltft_err,isltft,isltft_err}.
Fig.\ref{fig:bp_wig1}
and \ref{fig:bp_wig2} show that the bandpass filtering fails to
attenuate the overlapped noise. Compared to the denoised results of
LTF decomposition in Fig.\ref{fig:iltft_wig1}
and \ref{fig:iltft_wig2}, the proposed SLTFT method brought more
satisfactory results (see Fig.\ref{fig:isltft_wig1}
and \ref{fig:isltft_wig2}) with the strong interference noise is
well-attenuated.

\multiplot{3}{field,sltft,cmask}{width=0.45\hsize}{(a) Raw OZ-25 dataset,
(b) the $t\textrm{-}f\textrm{-}x$ spectrum obtained by the SLTFT
($\varepsilon=0.99$) and (c) the filter mask in
$t\textrm{-}f\textrm{-}x$ domain.}
	
\multiplot{6}{bp,bp_err,iltft,iltft_err,isltft,isltft_err}{width=0.3\hsize}{Estimated
signals and separated noise using (a) and (b) high-pass filter with
$f_{\text{hi}}$=20 Hz, (c) and (d) LTF decomposition, (e) and (f) the
proposed SLTFT.}

\multiplot{6}{bp_wig1,iltft_wig1,isltft_wig1,bp_wig2,iltft_wig2,isltft_wig2}{width=0.3\hsize}{Magnified sections of the denoised profile in
Fig.\ref{fig:bp,bp_err,iltft,iltft_err,isltft,isltft_err}. (a), (b)
and (c) correspond to the sections outlined by dashed rectangles in
Figs. \ref{fig:bp}, \ref{fig:iltft}, and \ref{fig:isltft},
respectively.  (d), (e) and (f) correspond to the sections outlined by
dashed rectangles in Figs. \ref{fig:bp}, \ref{fig:iltft},
and \ref{fig:isltft}, respectively.}

\inputdir{dune}

Next, we use another widely-used field dataset, the Saudi Arabia Dune
dataset, to further test the performance of the proposed method. The
field dataset contains strong ground-roll noise with hyperbolic
moveout (see Fig.\ref{fig:dat,sltft,cmask}a) and is popular as a benchmark
dataset for ground-roll noise
attenuation \cite[]{Fomel02,Zheng2022,kaur2020,Yang2024}. We use a
mask filter in the $t\textrm{-}f\textrm{-}x$ domain (see
Fig.\ref{fig:dat,sltft,cmask}c) to suppress the ground-roll noise. The
estimated signal by using the high-pass filter with 20-Hz cutoff
frequency contains more low-level ground-roll noise (see
Fig.\ref{fig:bpsign}). The denoised dataset obtained by the
proposed SLTFT (see Fig.\ref{fig:dat,sltft,cmask}b) shows that the proposed
method achieved the separation goal that the underlying reflection
events clearly appear in the estimated section, and the ground-roll
noise is well suppressed.

Although the LTF decomposition can also produce a reasonable result
(see Fig.\ref{fig:iltft} and \ref{fig:ltftsign}), its
computational cost is way too much. The proposed method demonstrates a
significant reduction in time cost, for example, the computational
time by the SLTFT is 7.77 s for the OZ-25 dataset and 1.90 s for the
Dune dataset when compared with 62.37 s and 16.69 s needed from the
LTF decomposition.

\multiplot{3}{dat,sltft,cmask}{width=0.45\hsize}{(a) Raw Saudi Arabia
Dune dataset, (b) the $t\textrm{-}f\textrm{-}x$ spectrum obtained by
the SLTFT ($\varepsilon=0.99$) and the filter mask in
$t\textrm{-}f\textrm{-}x$ domain.}

\multiplot{6}{bpsign,bpnoiz,ltftsign,ltftnoiz,sltftsign,sltftnoiz}{width=0.3\hsize}{Estimated signals and separated noise using (a) and (b) high-pass
filter with $f_{\text{hi}}$=24 Hz, (c) and (d) LTF decomposition, (e)
and (f) the proposed SLTFT.}

\subsection{Fast Inverse-Q Filtering in Time-frequency Domain}
\inputdir{inverseq}

Local Time-frequency map can also used for time-varying Q-factor
estimation and inverse-Q filtering. \cite{wang2020} developed the
local centroid frequency shift (LCFS) method for time-varying
Q-estimation in time-frequency domain. The SLTFT method can provide
accurate time-frequency spectrum for Q-factor estimation with high
efficiency. We selected a 2D poststack seismic profile to perform
numerical test (see Fig.\ref{fig:powdata}). The seismic
section has a time length of 4.5 s with a time sampling of 4 ms and
contains 247 traces in total. We calculated the
$t\textrm{-}f\textrm{-}x$ spectrum of the seismic section (see
Fig.\ref{fig:sltft0}) by using the SLTFT
($\varepsilon=0.985$). The spectrum shows that the frequency range is
getting narrow as the seismic wave propagates. The local centroid
frequency (LCF) can be calculated by using the
$t\textrm{-}f\textrm{-}x$ map \cite[]{Liu13,wang2020}, as is shown in
Fig.\ref{fig:ltft_cf}.  We used the LCFS method in
time-frequency domain to estimate the time-varying equivalent
Q-factors for each
trace. Fig.\ref{fig:st_eqvq}-\ref{fig:sltft_eqvq}
show the calculated Q-factors by using the S transform, the LTF
decomposition and the SLTFT, respectively. The Q-factors obtained by
the LTF decomposition and the SLTFT are reasonably distributed
according to the LCF information (see Fig.\ref{fig:ltft_cf}),
while the S-transform fails in providing an unbiased Q-factor map. The
estimated Q-factors are used to perform inverse-Q filtering, and
Fig.\ref{fig:stresult} - \ref{fig:sltftresult} show
the enhanced seismic sections corresponded to
Fig.\ref{fig:st_eqvq}-\ref{fig:sltft_eqvq},
respectively. Because of its biased estimation of Q-factors, the
S-transform exhibits limitations in effectively enhancing the
resolution of deep reflection signals (around 3 $\sim$ 4 s) (see
Fig.\ref{fig:stresult}) when compared to the results of the
LTF decomposition and the SLTFT (see Fig.\ref{fig:ltftresult}
and \ref{fig:sltftresult}), where the attenuation of the
reflections is compensated well, and the original structural
characteristics are reasonably preserved. Meanwhile, the proposed
method effectively reduces more computational costs (see
Table. \ref{tb:comparison1}) than the LTF decomposition. Additionally,
it achieves storage cost reduction through flexible frequency
down-sampling. Specifically, in the SLTFT, we utilize half the
frequency samples compared to the S-transform, while maintaining a
superior enhancing profile.

\multiplot{3}{powdata,sltft0,ltft_cf}{width=0.45\hsize}{(a) 2D poststack
seismic section, (b) the corresponding $t\textrm{-}f\textrm{-}x$ cube
and (c) local centroid frequency.}

\multiplot{3}{st_eqvq,ltft_eqvq,sltft_eqvq}{width=0.45\hsize}{The estimated
equivalent Q-factors by using (a) S-transform, (b) LTF decomposition
and (c) SLTFT ($\varepsilon=0.985$).}

\multiplot{3}{stresult,ltftresult,sltftresult}{width=0.45\hsize}{Enhanced
seismic sections by using (a) S-transform, (b) LTF decomposition and
(c) SLTFT ($\varepsilon=0.985$).}

\subsection{Fast Multicomponent Data Registration}
\inputdir{vecta}

Multicomponent seismic data registration is an important step before
quantitative seismic data interpretation and joint amplitude versus
offset (AVO) analysis \cite[]{lu2015,gao2018}. The time-frequency
transform is suitable for nonstationary registration of multicomponent
images in frequency domain \cite[]{Liu13,chen2021}. The main idea is
to match the spectra of compressional (PP) and shear (SS) reflections
to that of Ricker wavelets, making the balanced PP and SS images share
a similar spectral content. Then one can squeeze the SS image and make
PP and SS images display in the same coordinate
system. Fig.\ref{fig:pp} and \ref{fig:ss} shows the PP and
SS images from a nine-component land survey, respectively. The
interleaved image (where the PP and SS traces are interleaved one by
one) of the raw PP and PS data shows the obviously discontinuous
reflectors (see Fig.\ref{fig:before}). We use the LTF
decomposition as a reference comparison with the proposed SLTFT
method. Fig.\ref{fig:after} and \ref{fig:after1} shows the
interleaved images after the registration by using the LTF
decompostion and the SLTFT ($\varepsilon=0.98$), respectively. Both
the LTF decomposition and the SLTFT method create the results with
spatially coherent registration and high resolution, especially at the
locations of the rectangle boxes. In terms of computational cost, the
proposed method provides a more efficient time-frequency
representation (see Table. \ref{tb:comparison1}), which makes it more
suitable in high-dimensional field data processing tasks.

\multiplot{2}{pp,ss}{width=0.45\hsize}{(a) PP and (b) SS images from a
multicomponent land survey.}

\multiplot{2}{before,after}{width=0.45\hsize}{The interleaved image of
(a) raw data, after registration by using (b) LTF decomposition.}

\inputdir{vecta1}
\plot{after1}{width=0.45\hsize}{The interleaved image after registration
by using SLTFT.}

\begin{table*}
	\centering
	\caption{Comparison of Time Consumptions.$^1$}
	\label{tb:comparison1}
    \resizebox{\textwidth}{!}{
	\begin{threeparttable}
		\begin{tabular}[htbp]{lcccc}
			\toprule
			\multirow{3}*{\makecell{Data size}} & \multicolumn{2}{c}{Ground-roll attenuation}& Inverse-Q filtering& Data registration\\
			& OZ-25 field data & Saudi Arabia Dune field data & Poststack data & Poststack data$^2$\\
			& 2000 $\times$ 81 & 500 $\times$ 96 & 1126 $\times$ 247 & 1024 $\times$ 471\\
			Frequency samples	& 400 & 400 & 250 & 500\\
			\midrule
			LTF decomposition  & 6.237 $\times$ 10$^1$ s & 1.669 $\times$ 10$^1$ s & 5.112 $\times$ 10$^1$ s & 1.792 $\times$ 10$^3$ s \\
			SLTFT & 7.772 $\times$ 10$^0$ s & 1.903 $\times$ 10$^0$ s & 2.051 $\times$ 10$^0$ s & 0.936 $\times$ 10$^3$ s \\
			\bottomrule
		\end{tabular}
		\begin{tablenotes}
			\item[1] {All these records are obtained by taking the average of 5 measurements. The computational platform equips with 2.1 GHz i7-12700 CPU and 32 GB of RAM.}
			\item[2] {Includes Ricker wavelet inversion and other time costs.}
		\end{tablenotes}
	\end{threeparttable}
    }
\end{table*}


\section{Discussion}
We have seen how the proposed SLTFT is applied in ground-roll
attenuation, inverse-Q filtering, and multicomponent data
registration. As metioned before, the localization scalar
$\varepsilon$ is a key parameter in the SLTFT. It controls the balance
between time and frequency resolution. Overall, it is similar to the
window length in STFT. A small $\varepsilon$ value means a rapid
decaying smoothing radius, which leads to higher time resolution and
lower frequency resolution. In contrast, a large $\varepsilon$ value
means a slow decaying smoothing radius, which leads to higher
frequency resolution and lower time resolution. The $\varepsilon$
value should be set according to the specific requirements of the
seismic data processing task. For example, in the ground-roll
attenuation and inverse-Q filtering tasks, a larger $\varepsilon$
value can be used to achieve better frequency resolution. The
frequency varying scalar can also be set using the same principle,
according to the desired frequency-varying resolution.


\section{Conclusion} 

We have proposed an efficient approach to compute and apply adaptive
time-frequency transform. The proposed SLTFT method allows us to
better control the balance among time-frequency localizations, inverse
transform, accuracy and computation cost. The revised streaming
algorithm by adding an extra control parameter guarantee the adaptive
time-frequency localization. Instead of the iterative strategy, the
analytical solution for the nonstationary Fourier series estimation
has the low computational complexity even when dealing with
large-scale seismic data. Meanwhile, the custom frequency points in
the least-squares problem makes the proposed method more flexible. The
SLTFT provides a convenient time-frequency analysis domain with the
adjustable resolution fitting for different kinds of nonstationary
signals. Traditional seismic data processing tasks such as ground-roll
attenuation, inverse-Q filtering, and multicomponent data registration
become well defined in the SLTFT domain and allow for efficient and
effective algorithms. Other possible applications of the proposed
algorithm in seismic data analysis and processing may include seismic
attenuation analysis, low frequency shadow detection, channel
detection and so on.


\bibliographystyle{SEG}
\bibliography{paper}

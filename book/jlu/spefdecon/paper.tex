\published{Journal of Geophysics and Engineering, 18, 825-833, (2021)}

\title{Multichannel adaptive deconvolution based on streaming
  prediction-error filter}

\renewcommand{\thefootnote}{\fnsymbol{footnote}}
{}
             
\address{
College of Geo-exploration Science and Technology,\\
Jilin University \\
No.938 Xi minzhu street \\
Changchun, China, 130026}

\author{Qinghan Wang, Yang Liu, Cai Liu, and Zhisheng Zheng}

\maketitle

\begin{abstract}
  Deconvolution mainly improves the resolution of seismic data by
  compressing seismic wavelets, which is of great significance in
  high-resolution processing of seismic data. Prediction-error
  filtering/least-square inverse filtering is widely used in seismic
  deconvolution and usually assumes that seismic data is
  stationary. Affected by factors such as earth filtering, actual
  seismic wavelets are time- and space-varying. Adaptive
  prediction-error filters are designed to effectively characterize
  the nonstationarity of seismic data by using iterative methods,
  however, it leads to problems such as slow calculation speed and
  high memory cost when dealing with large-scale data. We have
  proposed an adaptive deconvolution method based on a streaming
  prediction-error filter. Instead of using slow iterations,
  mathematical underdetermined problems with the new local smoothness
  constraints are analytically solved to predict time-varying seismic
  wavelets. To avoid the discontinuity of deconvolution results along
  the space axis, both time and space constraints are used to
  implement multichannel adaptive deconvolution. Meanwhile, we define
  the parameter of the time-varying prediction step that keeps the
  relative amplitude relationship among different reflections. The new
  deconvolution improves the resolution along the time direction while
  reducing the computational costs by a streaming computation, which
  is suitable for handling nonstationary large-scale data. Synthetic
  model and filed data tests show that the proposed method can
  effectively improve the resolution of nonstationary seismic data,
  while maintaining the lateral continuity of seismic
  events. Furthermore, the relative amplitude relationship of
  different reflections is reasonably preserved.
\end{abstract}

\section{Introduction}
Thin interbedded reservoirs and subtle reservoirs with complex
lithology are becoming key areas of seismic exploration and
development. Conventional seismic data is difficult to accurately
characterize the thin reservoir, therefore, improvement of seismic
resolution is a persistent problem in seismic exploration. Two major
categories improving the vertical resolution of seismic data include
deconvolution \cite[]{ver08,ver12, Margrave11, Li13} and inverse Q
filtering \cite[]{Wang02, Wang06}, and those techniques can
effectively broaden the frequency bandwidth and improve the accuracy
of seismic data interpretation, especially in the identification of
thin reservoir. The inverse Q filtering method compensates the
attenuation of wave amplitude and fixes the phase distortion caused by
the absorption of subsurface media; however, the correction depends on
the accuracy of the quality factor Q.  Predictive deconvolution method
improves the resolution of seismic data by compressing seismic
wavelet, which depends on the assumptions of minimal phase wavelet and
whitening reflection coefficients. Thus, predictive deconvolution is
suitable for vibroseis seismic data \cite[]{Ristow75}.

Prediction-error filtering (PEF) or least-square inverse filtering has
been applied in seismic deconvolution for decades, and it has proved
its effectiveness for resolution improvement and multiple
elimination. The theory of predictive deconvolution was introduced by
\cite{Robinson57, Robinson67}.  \cite{Peacock69} proved the
effectiveness of predictive deconvolution for enhancing resolution and
suppressing periodic multiples.  To take full advantage of the spatial
characteristics of seismic data and suppress noise, several authors
developed multichannel predictive deconvolution \cite[]{Claerbout76,
  Porsani07, Li16}. The traditional deconvolution method is designed
under the assumption of stationary data and becomes less effective
because seismic data are nonstationary in nature.  \cite{Clarke68}
proposed a nonstationary deconvolution in time domain based on optimal
Wiener filtering. \cite{Wang69} gave the criteria for determining the
optimal length of the filtering window on the assumption of a
piecewise stationary. \cite{Griffiths77} proposed an adaptive
predictive deconvolution method that adaptively updates the filter
coefficients for each data point. \cite{Koehler85} proposed a
generalized mathematical theory of time-varying deconvolution and used
the conjugate gradient algorithm to calculate the filter coefficients.
\cite{Prasad80} compared three adaptive deconvolution methods and
demonstrated that all three methods perform better than traditional
predictive deconvolution when dealing with nonstationary data.
\cite{Liu11} obtained smoothly nonstationary PEF coefficients by
solving a global regularized least-squared problem, however, iterative
approach leads to slow computation speed and high memory cost.
\cite{Fomel16} proposed the concept of streaming computation, which
can adaptively update the filter coefficients without iteration, and
the properties of nonstationary representation and low computational
cost are useful for the single-channel deconvolution model and random
noise attenuation of seismic data \cite[]{Liu18}. To improve the
resolution effectively for nonstationary seismic data, in this paper,
we design a multichannel adaptive deconvolution method based on the
streaming prediction error filter in time-space domain. The time and
space constraints added to the objective function can guarantee the
continuity of deconvolution results in space direction, and the
relationship between the prediction step and wavelet frequency
reasonably improve the fidelity of the reflection coefficients in the
deconvolution result.

This paper is organized as follows. First, we introduce the streaming
computation for adaptive PEF. Then, we propose the improved streaming
PEF method that involves spatial constraints and time-varying
prediction step.  Finally, the synthetic data and real data are used
to demonstrate that the proposed method can be effective and efficient
in vertical resolution improvement of nonstationary seismic data.

\section{Theory}
According to the theory of predictive deconvolution, time-varying filter
coefficients are designed to predict nonstationary seismic data, which can
be expressed as a nonstationary autoregressive equation,
\begin{equation}
  \label{eq:1}
  r(t)=s(t)-\sum_{n=1}^N c_n(t)s(t-n+1-\alpha)=s(t)-\mathbf{S}^T\mathbf{C},
\end{equation}
where $\alpha$ is the prediction step, $s(t)$ and $r(t)$ are the original
seismic data and prediction error, respectively,
$\mathbf{S}^T=[s(t-\alpha),s(t-1-\alpha),\cdots,s(t-N+1-\alpha)]$ ,
which represents the causal translation of $s(t)$, all prediction
coefficients $\mathbf{C}=[c_1(t),c_2(t),\cdots,c_N(t)]$
are estimated in a time variant manner, and $N$ denotes the length of
the predictive filter. The filter coefficients are obtained by solving
the least-squares problem:
\begin{equation}
  \label{eq:2}
  \min_{\mathbf{C}}\parallel{s(t)-\mathbf{S}^T\mathbf{C}}\parallel_2^2,
\end{equation}
which is a classical underdetermined problem. The local smoothness
constraints with streaming computation \cite[]{Fomel16}
avoids the problem of high computational costs associated with iterative
approaches. One can assume that the filter coefficients corresponding to
two adjacent data stay close, and the autoregressive equation can be
expressed by an overdetermined equation as

\begin{equation}
\begin{aligned}
  \label{eq:3}
  \begin{bmatrix}
   s(t-\alpha) & s(t-1-\alpha) & \cdots & s(t-N+1-\alpha) \cr
    \epsilon_t  & 0             & \cdots & 0 \cr
    0           & \epsilon_t    & \cdots & 0 \cr
    \vdots      & \vdots        & \ddots & \vdots \cr
    0           & 0             & \cdots & \epsilon_t
  \end{bmatrix}
  \times
  \begin{bmatrix}
  c_1(t) \cr c_2(t) \cr  \vdots \cr c_N(t)
  \end{bmatrix}
  =
 \begin{bmatrix}
 s(t) \cr \epsilon_tc_1(t-1) \cr \epsilon_tc_2(t-1) \cr \vdots \cr
 \epsilon_tc_N(t-1)
 \end{bmatrix} 
\end{aligned}.
\end{equation}

Equation~\ref{eq:3} can be written in terms of a shortened block-matrix
notation

\begin{equation}
  \label{eq:4}
 \begin{bmatrix}
 \mathbf{S}^T \cr \epsilon_t\mathbf{I}
  \end{bmatrix}
  \mathbf{C}=
 \begin{bmatrix}
 s(t) \cr \epsilon_t\overline{\mathbf{C}}_t
 \end{bmatrix},
\end{equation}
where $\overline{\mathbf{C}}_t=[c_1(t-1),c_2(t-1),\cdots,c_N(t-1)]$, which
represents the previous filter coefficient on time axis, $\epsilon_t$ is
the constant scale parameter controlling the variability of two adjacent
filter coefficients in the time axis, and $\mathbf{I}$ is the identity
matrix. Consider a multichannel seismic data, an extra spatial constraint
can introduce to the smoothness of the time-varying filter coefficients
along the space axis. The new prediction filter can be expressed in the
form of a block matrix as

\begin{equation}
  \label{eq:5}  
 \begin{bmatrix}
 \mathbf{S}_m^T \cr \epsilon_t\mathbf{I} \cr \epsilon_x\mathbf{I}
 \end{bmatrix}
 \mathbf{C}=
 \begin{bmatrix}
 s_m(t) \cr \epsilon_t\overline{\mathbf{C}}_t \cr
   \epsilon_x\overline{\mathbf{C}}_x
  \end{bmatrix},
\end{equation}
where $s_m(t)$ and
$\mathbf{S}_m^T=[s_m(t-\alpha),s_m(t-1-\alpha),\cdots,s_m(t-N+1-\alpha)]$
represent the multichannel data with space index $m$ and the causal
translation of the $m$th seismic trace, $\epsilon_x$ is the scalar
regularization parameter in the space axis, and
$\mathbf{C}=[c_1^m(t),c_2^m(t),\cdots,c_N^m(t)]$ is the time-varying filter
coefficient with space-varying index $\emph{m}$. The previous filter
$\overline{\mathbf{C}}_t=[c_1^m(t-1),c_2^m(t-1),\cdots,c_N^m(t-1)]$ on the 
time axis and the previous filter on space axis
$\overline{\mathbf{C}}_x=[c_1^{m-1}(t),c_2^{m-1}(t),\cdots,c_N^{m-1}(t)]$ are
similar to the current filter
$\mathbf{C}=[c_1^m(t),c_2^m(t),\cdots,c_N^m(t)]$.

Assuming that adjacent filter coefficients are similar, the current filter
coefficients at a certain point can be constrained by the adjacent filter
coefficients in both time and space directions, and the regularization
constraint terms are
$\epsilon_t^2\parallel{\mathbf{C}-\overline{\mathbf{C}}_t}\parallel_2^2$ and
$\epsilon_x^2\parallel{\mathbf{C}-\overline{\mathbf{C}}_x}\parallel_2^2$.
$\epsilon_t$ and $\epsilon_x$ are weights for regularization constraint
terms along time and space directions, respectively, which control the
similarity of the adjacent filter coefficients. In this case, the
underdetermined problem is transformed into an overdetermined problem.
The filter coefficients are calculated by solving the regularized
autoregression problem
\begin{equation}
  \label{eq:6}
  \min_{\mathbf{C}}\parallel{s_m(t)-\mathbf{S}_m^T\mathbf{C}}\parallel_2^2+
  \epsilon_t^2\parallel{\mathbf{C}-\overline{\mathbf{C}}_t}\parallel_2^2+
  \epsilon_x^2\parallel{\mathbf{C}-\overline{\mathbf{C}}_x}\parallel_2^2,
\end{equation}
the least-squares solution of equation~\ref{eq:6} is
\begin{equation}
  \label{eq:7}
  \mathbf{C}=(\mathbf{S}_m\mathbf{S}_m^T+\epsilon_t^2\mathbf{I}+ 
  \epsilon_x^2\mathbf{I})^{-1}(s_m(t)\mathbf{S}_m+\epsilon_t^2
  \overline{\mathbf{C}}_t+\epsilon_x^2\overline{\mathbf{C}}_x).
\end{equation}

The Sherman-Morrison formula in linear algebra \cite[]{Hager89} is able to
directly transform the inverse matrix in equation~\ref{eq:7} without
iterations:
\begin{equation}
  \label{eq:8}
  (\mathbf{S}_m\mathbf{S}_m^T+\epsilon_t^2\mathbf{I}+\epsilon_x^2
  \mathbf{I})^{-1}=\frac{1}{\epsilon_t^2+\epsilon_x^2}(\mathbf{I}-
  \frac{\mathbf{S}_m\mathbf{S}_m^T}{\epsilon_t^2+\epsilon_x^2+
  \mathbf{S}_m^T\mathbf{S}_m}),
\end{equation}
where $\mathbf{S}_m$ is a column vector and $\mathbf{S}_m^T$ is the
transpose of $\mathbf{S}_m$. Substituting equation~\ref{eq:8} into
~\ref{eq:7}, the streaming PEF coefficients can be calculated as
\begin{equation}
  \label{eq:9}
  \mathbf{C}=\overline{\mathbf{C}}+(\frac{s_m(t)-\mathbf{S}_m^T\overline{
  \mathbf{C}}}{\epsilon^2+\mathbf{S}_m^T\mathbf{S}_m})\mathbf{S}_m,
\end{equation}
where
\begin{equation}
  \label{eq:10}
  \begin{cases}
    \epsilon^2=\epsilon_t^2+\epsilon_x^2 \\
    \overline{\mathbf{C}}=\frac{\epsilon_t^2\overline{\mathbf{C}}_t+
      \epsilon_x^2\overline{\mathbf{C}}_x}{\epsilon^2}
  \end{cases}.
\end{equation}

Equation~\ref{eq:9} shows that the adaptive coefficients get updated by
adding a scaled version of the data, and the scale is proportional to the
streaming prediction error. Updating the filter coefficients requires
only elementary algebraic operation without iteration.

According to the definition of prediction error (equation~\ref{eq:1}) and
prediction coefficients (equation~\ref{eq:9}), the deconvolution result of
streaming PEF can be expressed as
\begin{equation}
  \label{eq:11}
  r_m(t)=s_m(t)-\mathbf{S}_m^T\mathbf{C}=\frac{\epsilon^2(s_m(t)-\mathbf
  {S}_m^T\overline{\mathbf{C}})}{\epsilon^2+\mathbf{S}_m^T\mathbf{S}_m}.
\end{equation}

In this paper, we select the minimum-phase wavelet as the source wavelet
to verify the effectiveness of the proposed deconvolution method. The
minimum-phase wavelet can be expressed as
\begin{equation}
  \label{eq:12}
  w(t)=e^{-(2\pi f_mt/25)^2}sin(2\pi f_mt),
\end{equation}
where $f_m$ is the dominant frequency of the wavelet.

The minimum-phase wavelet (figure~\ref{fig:wave,refl,in,spef1}a) and the
sequence of reflection coefficients (figure~\ref{fig:wave,refl,in,spef1}b)
generate a simple 1D convolution model (figure~\ref{fig:wave,refl,in,spef1}c)
, where the wavelet frequencies corresponding to every two reflection
coefficients with opposite amplitude are selected to be 45 Hz, 35 Hz, 25 Hz,
and 20 Hz, respectively. Figure~\ref{fig:wave,refl,in,spef1}d is the result
by using the streaming PEF deconvolution ($N=10$, $\epsilon_t=0.2$, and
$\alpha=1$). The streaming PEF deconvolution method effectively improve the
time resolution, however, the relative amplitude relationship among
different reflections has been changed, which occur more in predictive
deconvolution methods. Notice that the amplitude distortion is related to
the dominant frequency of the wavelet: the lower dominant frequency is,
the worse amplitude fidelity is shown because the peak amplitude of
minimum-phase wavelets is hard to happen in the first sample point.

\inputdir{simple}
\multiplot{4}{wave,refl,in,spef1}{width=0.47\columnwidth}{Analysis of
              amplitude distortion for streaming PEFs. Minimum-phase
              wavelet (a), the reflectivity (b), the nonstationary 
              synthetic seismic trace (c), and the deconvolution result
              by using streaming PEF with constant prediction step (d).}

The red line in figure~\ref{fig:gapdif} represents the theoretical curve of
sample number between the start point and peak-amplitude point of
minimum-phase wavelets, which is the function of the dominant frequency.
We select an empirical equation to fit the curve as follows:
\begin{equation}
  \label{eq:13}
  gap=Round(\frac{b}{f\times\Delta t}),
\end{equation}
where $Round$ represents a rounding function, $gap$ is the sample number
between the start point and the peak-amplitude point, $\Delta t$ is the
time interval, and $b$ is a constant. One can determine parameter $b$ from
any point in the curve, e.g., $b$ is selected to be 0.232 in
figure~\ref{fig:gapdif}, where the dominant frequency is 1 Hz, the sample
interval is 1 ms, and the $gap$ is 232. The blue dotted line calculated
by equation~\ref{eq:12} reasonably fits the red theoretical curve.
Therefore, we further design the streaming PEF deconvolution method with
the time-varying prediction step to preserve the relative amplitude
relationship of different reflection coefficients. The adaptive prediction
step is selected to be the parameter $gap$, which makes the prediction error
after deconvolution close to the maximum amplitude value of the original
wavelet. The new streaming deconvolution residual is rewritten as follow:
\begin{equation}
  \label{eq:14}
  r(t)=s(t)-\sum_{n=1}^N c_n(t)s(t-n+1-\alpha (t)),
\end{equation}
where $\alpha(t)$ is the time-varying prediction step, which is determined
by a time-varying $gap$ value
\begin{equation}
  \label{eq:15}
  \alpha(t)=gap(t)=Round(\frac{b}{f(t)\times\Delta t}),
\end{equation}
where $f(t)$ is the time-varying local frequency obtained by the shaping
regularization \cite[]{Fomel07a}. The instantaneous frequency calculated
instantaneously at each signal point sometimes appears noisy and contains
physically unreasonable negative frequency, so \cite{Fomel07a} defined
the local frequency by using the shaping regularization \cite[]{Fomel07b} to
constrain the linear inversion problem. When calculating the local frequency,
the continuity and smoothness of the local frequency can be controlled only
by adjusting a smooth radius parameter, and the local frequency shows better
physical meaning than the instantaneous frequency.

\inputdir{gap-fre}
\plot{gapdif}{width=0.5\columnwidth}{Parameter $gap$ curve of minimum phase
      wavelet with different frequencies.}

Figure~\ref{fig:lfe,vvlag,dif}a shows the local frequency calculated from
the synthetic trace (figure~\ref{fig:wave,refl,in,spef1}c), and the 
time-varying prediction steps (figure~\ref{fig:lfe,vvlag,dif}b) are obtained 
according to equation~\ref{eq:14}, where the parameter $b$ is 0.232 and the 
time interval is 1 ms. Figure~\ref{fig:lfe,vvlag,dif}c shows the result 
processed by the streaming PEF deconvolution with the time-varying 
predictionstep. Compared with original data (dotted line in
figure~\ref{fig:lfe,vvlag,dif}c), the proposed method keeps the relative
amplitude relationship in the deconvolution result
(solid line in figure~\ref{fig:lfe,vvlag,dif}c) at the cost of a lower
resolution improvement.

\inputdir{simple}
\multiplot{3}{lfe,vvlag,dif}{width=0.47\columnwidth}{The deconvolution
              result by using the streaming PEF with time-varying
              prediction steps. The local frequency (a), time-varying
              prediction step (b), the deconvolution result with the
              proposed method (solid line), which is compared with the
              original trace (dotted line) (c).}
              
The proposed method mainly includes four parameters: the filter length
($N$), time constraint factor ($\epsilon_t$), spatial constraint factor
($\epsilon_x$) and constant $b$. According to the experiment, a reasonable
deconvolution results can be obtained when $N\le10$. The constraint factors
$\epsilon_t$ and $\epsilon_x$ are the key parameters for the proposed method.
The denominator in equation~\ref{eq:9} suggests that $\epsilon_t^2$
and $\epsilon_x^2$ should have the same order of the magnitude as
$\mathbf{S}_m^T\mathbf{S}_m$. Too small a constraint factor would make the
deconvolution results unstable, and too large a constraint factor would lead
to the filter coefficients not being updated effectively. When the filter 
coefficients are constrained only in the time direction, it is only necessary
to set the spatial constraint factor to zero ($\epsilon_x=0$). In theory,
the $b$ value is the ratio of the peak-amplitude time to the period
of the wavelet. For minimum-phase wavelets, $b$ is typically less than 0.25.

Like the traditional predictive deconvolution, the relative amplitude
relationship of the results that are generated by streaming PEF
deconvolution with constant prediction step is not consistent with the
original data, so we introduce the time-varying prediction step and derive
its empirical formula. After adding the time-varying prediction step, the
amplitude of the deconvolution result of the synthetic model
(figure~\ref{fig:lfe,vvlag,dif}) tends to be consistent with the true
amplitude of the original data. However, the actual seismic data is complex
due to the earth absorption attenuation and other factors, so the amplitude
of deconvolution results is not true amplitude, but its relative amplitude
relationship is closer to the original data.

Since the proposed method can adaptively update the filter coefficients
without iteration and characterize the nonstationary properties of the data,
both the storage and computational cost of the filters are less than those of
the adaptive predictive filtering methods based on the iterative algorithms.
Table 1 compares the storage and computational cost of the different methods.

The proposed filter is constrained in both the time and space directions
while the filter is still one-dimensional, that is, the multichannel adaptive
deconvolution technology is based on the streaming PEF with one-dimensional
structure and two-dimensional constraints. To ensure the stability of the
calculation, the first boundary trace only uses the time constraint
condition to compute the streaming PEF coefficients, and the constraints in
both space and time directions are added to the other trace. An extension
of the proposed method to 3D is straightforward and provides a fast adaptive
multichannel deconvolution implementation for high-dimensional seismic data. 

\tabl{table}{Rough cost comparison among the different PEF estimation
methods. $N_a$ is the the filter size, $N_t$ is the data length in the time
direction, $N_x$ is the data length in the space direction, $N_{iter}$ is
the number of iterations.}
{
\begin{center}
 \begin{tabular}{|l|l|l|}
  \hline
   Method & Storage & Cost \\
  \hline
   Stationary PEF & $O(N_a)$ & $O(N_a^2N_t)$ \\
  \hline
   1D nonstationary PEF with iterative algorithm & $O(N_aN_t)$ &
   $O(N_aN_tN_{iter})$ \\
  \hline
   1D streaming PEF & $O(N_a)$ & $O(N_aN_t)$ \\
  \hline
   2D streaming PEF & $O(N_aN_t)$ & $O(N_aN_tN_x)$ \\
  \hline
 \end{tabular}
\end{center}
}

\section{Numerical examples}

\subsection{1D attenuation model}
We start with a 1D synthetic example with the quality factor $Q$
attenuation according to the modified Kolsky model
\cite[]{Wang04,Wang08}. In this model, the phase velocity is defined
as
\begin{equation}
  \label{eq:16}
  \frac{1}{v(\omega)}=\frac{1}{v_r}(1-\frac{1}{\pi Q_r}\ln{\left\vert
  \frac{\omega}{\omega_h} \right\vert})\approx\frac{1}{v_r}
  {\left\vert \frac{\omega}{\omega_h} \right\vert}^{-\gamma},
\end{equation}
where $\gamma=\frac{1}{\pi Q_r}$, $Q_r$ and $v_r$ are the quality factor and
phase velocity at a reference frequency $\omega_r$ (the dominant frequency
in genernal), $\omega_h$ is the tuning frequency. We generate a time-varying
trace (figure~\ref{fig:refl,in}b), where the dominant frequency of the
unattenuated minimum-phase wavelet is 40 Hz, the time interval is 1 ms, and
the Q value is 30. Figure~\ref{fig:refl,in}a shows the actual reflectivity.
For comparison, we use the traditional predictive deconvolution to squeeze
all wavelets (figure~\ref{fig:tpef,wtpef,spef0,wspef0}a), the filter length
$N$ is ten and the prewhitening factor is 0.0001. The traditional method
produces a reasonable result at the high-frequency locations; however, the
predictive deconvolution still loses part of the original amplitudes. We
design the streaming PEF deconvolution with a constant prediction step
($N=3$, $\epsilon_t=1.5$, and $\alpha=1$) to further handle the variability
of wavelet (figure~\ref{fig:tpef,wtpef,spef0,wspef0}c). The streaming PEF
residual visually shows a result similar to the traditional method, however,
a close-up comparison between the traditional deconvolution
(figure~\ref{fig:tpef,wtpef,spef0,wspef0}b) and the proposed deconvolution
(figure~\ref{fig:tpef,wtpef,spef0,wspef0}d) shows an obvious resolution
difference, which proves better nonstationary characteristics of the
streaming PEF.

Next, we improve the adaptive deconvolution result by involving the
time-varying prediction step, and the result is shown in
figure~\ref{fig:lfe,vlag0,nodif,zsdif}.
Figures~\ref{fig:lfe,vlag0,nodif,zsdif}a and
~\ref{fig:lfe,vlag0,nodif,zsdif}b show the decay of local frequency
and the time-varying prediction step by using equation~\ref{eq:14}
where $b=0.06$, respectively. Figure~\ref{fig:lfe,vlag0,nodif,zsdif}c
shows that the proposed method keeps the relative amplitude
relationship without auto gain correction (AGC) and the time
resolution is reasonably enhanced.
Figure~\ref{fig:lfe,vlag0,nodif,zsdif}d shows amplitude spectrum of
the synthetic data before and after deconvolution, where the grey line
is the original synthetic data and the black line is the deconvolution
result. It can be seen from figure 6d that the amplitude spectrum
broadens after the deconvolution.

\inputdir{atten-model}
\multiplot{2}{refl,in}{width=0.47\columnwidth}{A synthetic seismic
              trace example. The reflectivity (a), synthetic trace with Q
              attenuation (b).}
\multiplot{4}{tpef,wtpef,spef0,wspef0}{width=0.47\columnwidth}{Deconvolution
              results by using different methods. Traditional predictive
              deconvolution (a), local display of (a) (b), streaming PEF
              deconvolution (c), local display of figure (c) (d).}
\multiplot{4}{lfe,vlag0,nodif,zsdif}{width=0.47\columnwidth}{Deconvolution
              by using streaming PEF with time-varying prediction steps.
              Local frequency (a), time-varying prediction step (b), the
              deconvolution result with the proposed method (solid line),
              which is compared with the original trace (dotted line) (c),
              amplitude spectrum (The grey line is the original data,
              and the black line is the deconvolution result) (d).}

\subsection{2D wedge model}

The second example is shown in
figure~\ref{fig:wedge,wseis2,tpef,apef,spef0,spef1}a. We use a 2D
benchmark wedge model to prove the necessity of the spatial constraint
for the streaming PEF deconvolution. The velocity of the wedge in the
model is 10 kft/s, and the velocity of the upper and lower media is 20
kft/s, therefore, the wave impedance corresponding to the top and
bottom interfaces of the wedge are reversed. The minimum-phase wavelet
with the dominant frequency of 30 Hz is selected to create the
synthetic data (figure~\ref{fig:wedge,wseis2,tpef,apef,spef0,spef1}b),
where the wavelet of the top and bottom interfaces appears
interference started from the 45th trace. The synthetic data are
firstly processed using the traditional predictive deconvolution
method (the filter length is 3) and the regularizednon-stationary
autoregressive (RNA) method (the filter length is 3) based on the
iterative algorithm \cite[]{Liu11}, and the deconvolution results are
shown in figures~\ref{fig:wedge,wseis2,tpef,apef,spef0,spef1}c and
~\ref{fig:wedge,wseis2,tpef,apef,spef0,spef1}d, respectively. Due to
the model is stationary data, both methods can effectively improve the
resolution and distinguish the top and bottom interfaces of the wedge
model, but the traditional predictive deconvolution method is not
suitable for processing nonstationary data (see
figure~\ref{fig:tpef,wtpef,spef0,wspef0}) and iterative RNA
deconvolution produces high computational cost. Then we design a
streaming PEF with 3 (time) coefficients, the prediction step
$\alpha=1$, and the time constraint factor $\epsilon_t=0.2$ for each
sample to further verify the effectiveness of the spatial constraint.
Figures~\ref{fig:wedge,wseis2,tpef,apef,spef0,spef1}e and
~\ref{fig:wedge,wseis2,tpef,apef,spef0,spef1}f show the streaming PEF
deconvolution results without spatial constraint ($\epsilon_x=0$) and
with spatial constraint ($\epsilon_x=0.5$), respectively. Both
single-channel and multichannel deconvolution improve the vertical
resolution, however, the result without spatial constraint appears
with unstable fluctuation and spatial discontinuity, especially at
rectangle location in
figure~\ref{fig:wedge,wseis2,tpef,apef,spef0,spef1}e. The spatial
constraint can effectively reduce the fluctuation and enhance the
structural continuity of deconvolution result. Meanwhile, the
computation time of the traditional method, iterative method,
single-channel, and multichannel streaming PEF deconvolution method is
0.011 s, 0.220 s, 0.011 s, and 0.012 s, respectively.

\inputdir{wedge}
\multiplot{6}{wedge,wseis2,tpef,apef,spef0,spef1}{width=0.47\columnwidth}{
              Wedge model. Wedge velocity model (a), synthetic data (b),
              the result of traditional predictive deconvolution (c), the
              result of iterative deconvolution (d), the result of adaptive
              single-channel deconvolution without spatial constraint (e),
              the result of adaptive multichannel deconvolution with
              patial constraint (f).}

\subsection{Field example}

For the field data test, we use a 2D poststack section with time interval of
2 ms. The input is shown in figure~\ref{fig:data,tpef,apef,vlag-spef}a. For
comparison, we apply the traditional predictive deconvolution (the filter
length is 6) and the iterative method (the filter length is 6) to enhance
the time resolution, as shown in figures~\ref{fig:data,tpef,apef,vlag-spef}b
and ~\ref{fig:data,tpef,apef,vlag-spef}c.
Figure~\ref{fig:data,tpef,apef,vlag-spef}d shows a processing result using
the proposed streaming PEF deconvolution method. The streaming PEF parameters
are 6 ($N$), 0.032 ($b$), 25000 ($\epsilon_t$), and 10000 ($\epsilon_x$).
The computation time of traditional, iterative method, and streaming
PEF deconvolution methods are 0.024 s, 18.767 s, and 1.018 s, respectively,
however, the traditional deconvolution method cannot enhance the time
resolution at all time because of nonstationary of the field data. The
proposed deconvolution and iterative methods can improve the vertical
resolution at different times, so both methods are more suitable for
processing nonstationary data. Moreover, compared with the traditional
and iterative methods, the proposed method can better keep the
continuity of events. Furthermore, we select a part of the data near to the
reservoir layer from 3-3.5 s to calculate the average amplitude spectrum of
the data before and after deconvolution, as shown in
figure~\ref{fig:zspec0,zspec1,zspec2,zspec3}.
Figure~\ref{fig:zspec0,zspec1,zspec2,zspec3} confirms that the average
amplitude spectrum of the seismic section after being processed by the
streaming PEF deconvolution is broader than that of the traditional
deconvolution result and slightly narrower than that of the iterative
deconvolution result in the effective frequency range. However, according to
the computation time of the different deconvolution methods, the
computational efficiency of the proposed method is significantly improved
compared with the iterative method. It further verifies the effectiveness
and high efficiency of the streaming PEF deconvolution method in processing
nonstationary seismic data.

\inputdir{rdata}
\multiplot{4}{data,tpef,apef,vlag-spef}{width=0.47\columnwidth}{
              Deconvolution results by using different methods.
              Poststack field data (a), traditional predictive 
              deconvolution (b), iterative deconvolution (c), streaming PEF
              deconvolution (d).}
\multiplot{4}{zspec0,zspec1,zspec2,zspec3}{width=0.47\columnwidth}{
              Comparison of the average amplitude spectrum of the
              deconvolution results and original field data. Original field
              data (a), traditional predictive deconvolution (b), iterative
              deconvolution (c), streaming PEF deconvolution (d).}

\section{Conclusion}

We have improved the theory of streaming PEF with temporal and spatial
constraints and proposed a multichannel adaptive deconvolution based
on streaming PEF in the time-space domain. Our approach uses a
time-varying prediction step to guarantees the reasonable amplitude
relationship of deconvolution results. The proposed method updates the
filter coefficients simultaneously when each new data point arrives,
which effectively represents the nonstationary characteristics of the
seismic data. The efficient computational feature of streaming
computation is useful for efficient adaptive deconvolution in
high-dimensional data processing. Synthetic examples and field data
test confirm that the proposed deconvolution method is successful in
enhancing the temporal resolution of seismic data while preserving the
relative amplitude relationship and structural continuity.

\section{Acknowledgments}

This work is supported by Nation Natural Science Foundation of China
(grant nos. 41974134 and 41774127).

\bibliographystyle{seg}
\bibliography{paper}



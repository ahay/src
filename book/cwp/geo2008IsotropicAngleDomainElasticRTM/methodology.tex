% ------------------------------------------------------------
\section{Wavefield imaging}
% ------------------------------------------------------------

Seismic imaging is based on numerical solutions to wave equations,
which can be classified into ray-based (integral) solutions and
wavefield-based (differential) solutions. Kirchhoff migration is a
typical ray-based imaging procedure which is computationally efficient
but often fails in areas of complex geology, such as sub-salt, because
the wavefield is severely distorted by lateral velocity variations
leading to complex multipathing. Wavefield imaging works better for
complex geology, but is more expensive than Kirchhoff
migration. Depending on computational time constraints and available
resources, different levels of approximation are applied to accelerate
imaging, i.e. one-way vs. two-way, acoustic vs. elastic, isotropic
vs. anisotropic, etc.

Despite the complexity of various types of wavefield migration
algorithms, any wavefield imaging method can be separated into two
parts: wavefield reconstruction followed by the application of an
imaging condition.  For prestack depth migration, source and receiver
wavefields have to be reconstructed at all locations in the
subsurface.  The wavefield reconstruction can be carried out
\geosout{in both} \geouline{using extrapolation in either} depth
\geosout{and} \geouline{or} time \geosout{domains}, and with different
modeling approaches, such as finite-difference\geouline{s}
\cite[]{GEO51-01-00540066,alford:834}, finite-element\geouline{s}
\cite[]{GEO41-01-01450151}, \geouline{or} spectral method\geouline{s}
\cite[]{seriani:1561,seriani:1285,dai:427} \geosout{,etc}. After
reconstructing wavefields with the recorded data as boundary
conditions into the subsurface, an imaging condition must be applied
at all locations in the subsurface in order to obtain a seismic
image. The simplest types of imaging conditions are based on
cross-correlation or deconvolution of the reconstructed wavefields
\cite[]{GEO36-03-04670481}. These imaging conditions can be
implemented in the time or frequency domain depending on the domain
in which wavefields have been reconstructed. Here, we concentrate on
reverse-time migration with wavefield reconstruction and imaging
condition implemented in the time domain.

\subsection{Reverse-time migration}

Reverse-time migration reconstructs the source wavefield forward in
time and the receiver wavefield backward in time. It then applies an
imaging condition to extract reflectivity information out of the
reconstructed wavefields.  The advantages of reverse-time migration
over other depth migration techniques are that the extrapolation in
time does not involve evanescent energy, and no dip limitations exist
for the imaged structures
~\cite[]{mcmechan:613,mcmechan:413,whitmore:382,GEO48-11-15141524}. Although
conceptually simple, reverse-time migration has not been used
extensively in practice due to its high computational cost. However,
the algorithm is becoming more and more attractive to the industry
because of its robustness in imaging complex geology, e.g. sub-salt
\cite[]{jones:2140,boechat:2427}.

\cite{mcmechan:613,mcmechan:413}, \cite{whitmore:382} and
\cite{GEO48-11-15141524} first used reverse-time migration for
poststack or zero-offset data. The procedure underlying poststack
reverse-time migration is the following: first, reverse the recorded
data in time; second, use these reversed data as sources along the
recording surface to propagate the wavefields in the subsurface;
third, extract the image at zero time, e.g. apply an imaging
condition. The principle of poststack reverse-time migration is that
the subsurface reflectors work as exploding reflectors and that the
wave equation used to propagate data can be applied either forward or
backward in time by simply reversing the time axis~\cite[]{levin:581}.

\cite{chang:67} apply reverse-time migration to prestack
data. Prestack reverse-time migration reconstructs source and receiver
wavefields. The source wavefield is reconstructed forward in time, and
the receiver wavefield is reconstructed backward in
time. \cite{chang:67,chang:597} use a so called excitation-time
imaging condition, where images are formed by extracting the receiver
wavefield at the time taken by a wave to travel from the source to the
image point. This imaging condition is a special case of the
cross-correlation imaging condition of \cite{GEO36-03-04670481}.

\subsection{Elastic imaging vs. acoustic imaging }

Multicomponent elastic data are often recorded in land or marine
(ocean-bottom) seismic experiments. However, as mentioned earlier,
elastic vector wavefields are not usually processed by specifically
designed imaging procedures, but rather by extensions of techniques
used for scalar wavefields. Thus, seismic data processing does not
take full advantage of the information contained by elastic
wavefields. In other words, it does not fully unravel reflections from
complex geology or correctly preserve imaging amplitudes and estimate
model parameters, etc.

\geosout{Elastic wave propagation in a homogeneous and arbitrarily
  anisotropic medium is characterized by the wave equation} 
\par
\geosout{
\parbox{\textwidth}{
\[
\rho\dtwo{u_i}{t} - c_{ijkl}\mtwo{u_k}{x_j}{x_l}=f_i \;,
\]
}}
\geosout{
where $u_i$ is the component $i$ of the displacement vector $\uu$,
$\rho$ is the density, $c_{ijkl}$ is the stiffness tensor and $f_i$
are the Cartesian components of the vector source $\mathbf{f}\lp
t\rp$.}

\geouline{Elastic wave propagation in a infinite homogeneous isotropic
medium is characterized by the wave equation}
\cite[]{akirichards.2002}
%
\beq\label{eqn:EWE} 
\rho\dtwo{\uu}{t}
  =\ff 
  +\lp\lambda+2\mu \rp\nabla\lp \DIV{\uu} \rp
  -\mu\CURL{\CURL{\uu}} 
\eeq
\geouline{where $\uu$ is the vector displacement wavefield, $t$ is
  time, $\rho$ is the density, $\ff$ is the body source force,
  $\lambda$ and $\mu$ are the Lam\'{e} moduli.}  This wave equation
assumes a slowly varying stiffness tensor over the imaging
space. \geosout{The coupling of displacements for different directions
  in the wave equation \ren{EWE} makes it difficult to derive a
  dispersion relation that can be used to extrapolate wavefields in
  depth. This makes it natural to extrapolate elastic wavefields in
  time.} \geouline{For isotropic media, one can process the elastic
  data either by separating wave-modes and migrating each mode using
  methods based on acoustic wave theory, or by migrating the whole
  elastic data set based on the elastic wave \req{EWE}. The elastic
  wavefield extrapolation using \req{EWE} is usually performed in time
  by Kirchhoff migration or reverse-time migration.}  \geosout{There
  are two main options for elastic imaging in the time domain:
  Kirchhoff migration and reverse-time migration.}

%\subsubsection{Elastic vs. acoustic Kirchhoff migration}
Acoustic Kirchhoff migration is based on diffraction summation, which
accumulates the data along diffraction curves in the data space and
maps them onto the image space. For multicomponent elastic data,
\cite{kuo:1223} discuss Kirchhoff migration for shot-record
data. Here, identified PP and PS reflections can be migrated by
computing source and receiver traveltimes using P-wave velocity for
the source rays, and P- and S-wave velocities for the receiver
rays. \cite{hokstad:861} performs multicomponent anisotropic
Kirchhoff migration for multi-shot, multi-receiver experiments, where
pure-mode and converted mode images are obtained by \geosout{downward
continuation of} \geouline{redatuming} visco-elastic vector wavefields and application of
\geouline{a} survey-sinking imaging condition to the reconstructed vector
wavefields. The wavefield separation is effectively done by the
Kirchhoff integral which handles both P- and S-waves, although this
technique fails in areas of complex geology where ray theory breaks
down.

%\subsubsection{Elastic vs. acoustic reverse-time migration}
Elastic reverse-time migration \geosout{have} \geouline{has} the same two
components as acoustic reverse-time migration: reconstruction of
source and receiver wavefield and application of an imaging condition.
The source and receiver wavefields are reconstructed by forward and
backward propagation in time with various modeling approaches. For
acoustic reverse-time migration, wavefield reconstruction is done with
the acoustic wave-equation using the recorded scalar data as boundary
condition.  In contrast, for elastic reverse-time migration, wavefield
reconstruction is done with the elastic wave-equation using the
recorded vector data as boundary condition.

Since pure-mode and converted-mode reflections are mixed on all
components of recorded data, images produced with reconstructed
elastic wavefields are characterized by crosstalk due to the
interference of various wave modes. In order to obtain images with
clear physical meanings, most imaging conditions separate wave modes
\geosout{,because they are mixed in all components of the data}. There
are two potential approaches to separate wavefields and image elastic
seismic wavefields.  The first option is to separate P and S modes on
the acquisition surface from the recorded elastic wavefields. This
procedure involves either approximations for the propagation path and
polarization direction of the recorded data, or reconstruction of the
seismic wavefields in the vicinity of the acquisition surface by a
numerical solution of the elastic wave equation, followed by wavefield
separation of scalar and vector potentials using Helmholtz
decomposition \cite[]{etgen:972,GEO62-02-05980613}.  An alternative
data decomposition using P and S potentials \geosout{is to}
reconstruct\geouline{s} wavefields in the subsurface using the elastic
wave equation, then decompose\geouline{s} the wavefields in\geouline{to} P-
and S-wave modes. \geouline{This is followed by forward extrapolation of}
the separated wavefields back to the surface using the acoustic wave
equation with the appropriate propagation velocity for the various
wave modes \cite[]{sun.elastic.rtm} by conventional procedures used
for scalar wavefields.

The second option is to extrapolate wavefields in the subsurface using
a numerical solution to the elastic wave equation and then apply an
imaging condition that extracts reflectivity information from the
source and receiver wavefields. \geosout{In case} \geouline{In the case
  where} extrapolation is implemented by finite-difference methods
\cite[]{chang:67,chang:597}, this procedure is known as elastic
reverse-time migration, and is conceptually similar to acoustic
reverse-time migration \cite[]{GEO48-11-15141524}, which is more
frequently used in seismic imaging.

Many imaging conditions can be used for reverse-time migration.
Elastic imaging condition\geouline{s} are more complex than\geosout{an}
acoustic imaging condition\geouline{s} because both source and receiver
wavefields are vector fields.  Different elastic imaging conditions
have been proposed for extracting reflectivity information from
reconstructed elastic wavefields. \geosout{For example,
}\cite{GEO63-05-16851695} use elastic reverse-time migration with
Lam\'e potential methods. \cite{chang:67} use the excitation-time
imaging condition which extracts reflectivity information from
extrapolated wavefields at traveltimes from the source to image
positions computed by ray tracing, etc. Ultimately, these imaging
conditions represent special cases of a more general type of imaging
condition that involves time cross-correlation or deconvolution of
source and receiver wavefields at every location in the subsurface.



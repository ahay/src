% ------------------------------------------------------------
\section{Conventional elastic imaging conditions}
% ------------------------------------------------------------

For vector elastic wavefields, the cross-correlation imaging condition
needs to be implemented on all components of the displacement
field. The problem with this type of imaging condition is that the
source and receiver wavefields contain a mix of P- and S-wave modes
which cross-correlate independently, thus hampering interpretation of
migrated images. An alternative to this type of imaging \geosout{is to}
perform\geouline{s} wavefield separation of scalar and vector potentials
after wavefield reconstruction in the imaging volume, but prior to the
imaging condition and then cross-correlate pure modes from the source
and receiver wavefields, as suggested by \cite{GEO55-07-09140919} and
illustrated by \cite{Cunha.thesis}.

% ------------------------------------------------------------
\subsection{Imaging with scalar wavefields}

As mentioned earlier, assuming single scattering in the Earth (Born
approximation), a conventional imaging procedure consists of two
components: wavefield extrapolation and imaging. Wavefield
extrapolation is used to reconstruct in the imaging volume the seismic
wavefield using the recorded data on the acquisition surface as \geouline{a}
boundary condition, and imaging is used to extract reflectivity
information from the extrapolated source and receiver wavefields.

Assuming scalar recorded data, wavefield extrapolation using a scalar
wave equation reconstructs scalar source and receiver wavefields,
$\US{}\ofxt$ and $\UR{}\ofxt$, at every location \geouline{$\mathbf{x}$}
in the subsurface. Using the extrapolated scalar wavefields, a
conventional imaging condition \cite[]{Claerbout.iei} can be
implemented as cross-correlation at zero-lag time:
%
\beq \label{eqn:CIC}
\IM{}\ofx = \int \US{}\ofxt \UR{}\ofxt dt \;.
\eeq
%
Here, $\IM{}\ofx$ denotes a scalar image obtained from scalar
wavefields $\US{}\ofxt$ and $\UR{}\ofxt$, $\xx=\{x,y,z\}$ represent
Cartesian space coordinates, and $t$ represents time.


% ------------------------------------------------------------
\subsection{Imaging with vector displacements}

Assuming vector recorded data, wavefield extrapolation using a vector
wave equation reconstructs source and receiver wavefields $\uu_s\ofxt$
and $\uu_r\ofxt$ at every location \geouline{$\mathbf{x}$} in the
subsurface. Here, $\uu_s$ and $\uu_r$ represent displacement fields
reconstructed from data recorded by multicomponent geophones at the
surface boundary. Using the vector extrapolated wavefields
${\uu}_s=\{{u_s}_x,{u_s}_y,{u_s}_z\}$ and
${\uu}_r=\{{u_r}_x,{u_r}_y,{u_r}_z\}$, an imaging condition can be
formulated as a straightforward extension of \req{CIC} by
cross-correlating all combinations of components of the source and
receiver wavefields. Such an imaging condition for vector
displacements can be formulated mathematically as
%
\beq \label{eqn:EICij}
\IM{ij}\ofx = \int \US{i}\ofxt \UR{j}\ofxt dt \;,
\eeq
%
where the quantities $u_i$ and $u_j$ stand for the Cartesian
components ${x,y,z}$ of the vector source and receiver wavefields,
$\uu\ofxt$. For example, $\IM{zz}\ofx$ represents the image component
produced by cross-correlating of the $z$ components of the source and
receiver wavefields, \geouline{and} $\IM{zx}\ofx$ represents the image component
produced by cross-correlating of the $z$ component of the source
wavefield with the $x$ component of the receiver wavefield, etc. In
general, an image produced with this procedure has nine components at
every location in space.

The main drawback of applying this type of imaging condition is that
the wavefield used for imaging contains a combination of P- and S-wave
modes. Those wavefield vectors interfere with one-another in the
imaging condition, since the P and S components are not separated in
the extrapolated wavefields. The crosstalk between various components
of the wavefield creates artifacts and makes it difficult to interpret
the images in terms of pure wave modes, e.g. PP or PS
reflections. This situation is similar to the case of imaging with
acoustic data contaminated by multiples or other types of coherent
noise which are mapped in the subsurface using \geouline{an} incorrect
velocity.

% ------------------------------------------------------------
\subsection{Imaging with scalar and vector potentials}

An alternative to the elastic imaging condition from \req{EICij} is to
separate the extrapolated wavefield into P and S potentials after
extrapolation and image using cross-correlations of the vector and
scalar potentials \cite[]{GEO55-07-09140919}. Separation of scalar and
vector potentials can be achieved by Helmholtz decomposition, which is
applicable to any vector field $\uu\ofxt$:
%
\beq \label{eqn:helmholtz}
\uu = \GRAD{\SPOT} + \CURL{\VPOT} \;,
\eeq
%
where $\SPOT\ofxt$ represents the scalar potential of the wavefield
$\uu\ofxt$ and $\VPOT\ofxt$ represents the vector potential of the
wavefield $\uu\ofxt$, and $\DIV{\VPOT}=0$. For isotropic elastic
wavefields, \req{helmholtz} is not used directly in practice, but the
scalar and vector components are obtained indirectly by the
application of the divergence ($\DIV{}$) and curl ($\CURL{}$)
operators to the extrapolated elastic wavefield $\uu\ofxt$:
%
\beqa \label{eqn:PS}
        P        &=& \DIV {\uu} =  \LAPL{\SPOT} \;, \\
\mathbf S	 &=& \CURL{\uu} = -\LAPL{\VPOT} \;.
\eeqa
%
For isotropic elastic fields far from the source, quantities $P$ and
$\textbf S$ describe compressional and transverse components of the
wavefield, respectively \cite[]{akirichards.2002}.  \geouline{In 2D, the
  quantity $\textbf S$ corresponds to SV waves that are polarized in
  the propagation plane.}

Using the separated scalar and vector components, we can formulate an
imaging condition that combines various incident and reflected wave
modes. The imaging condition for vector potentials can be formulated
mathematically as
%
\beq \label{eqn:PICij}
\IM{ij}\ofx = \int \MS{i}\ofxt \MR{j}\ofxt dt \;,
\eeq
%
where the quantities $\alpha_i$ and $\alpha_j$ stand for the various
wave modes $\alpha=\{P, S\}$ of the vector source and receiver
wavefields $\uu\ofxt$. For example, $\IM{PP}\ofx$ represents the image
component produced by cross-correlating of the $P$ wave mode of the
source and receiver wavefields, and $\IM{PS}\ofx$ represents the image
component produced by cross-correlating of the $P$ wave mode of the
source wavefield with the $S$ wave-mode of the receiver wavefield,
etc. \geouline{In isotropic media}, an image produced with this procedure
has four \geouline{independent} components at every location in space,
similar to the image produced by the cross-correlation of the
various Cartesian components of the vector displacements. However, in
this case, the images correspond to various combinations of incident P
or S and reflected P- or S-waves, thus having clear physical meaning
and being easier to interpret for physical properties.


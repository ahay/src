
\section{WEAK ANISOTROPY APPROXIMATION for VTI media}
%%%%%%%%%%%%%%%%%%%%%%%%%%%%%%%%%%%%%
In transversely isotropic media, velocities of seismic waves depend
on the direction of propagation measured from the symmetry
axis. \cite{GEO51-10-19541966} introduced a notation for VTI
media by replacing the elastic stiffness coefficients
%parameterization of this dependence, replacing the general notation of
%elastic anisotropy in terms of stiffness coefficients
%$C_{\alpha\beta}$ 
with the $P$- and $S$-wave velocities along the symmetry
axis and three dimensionless anisotropic parameters.  As shown by Tsvankin
\cite[]{tsvan1996}, the $P$-wave seismic signatures in 
%vertically transverse isotropic 
VTI media can be conveniently expressed in terms of
Thomsen's parameters $\epsilon$ and $\delta$.
%, and $\gamma$. 
Deviations of these parameters from zero characterize the relative
strength of anisotropy. For small values of these parameters,the
weak-anisotropy approximation \cite[]{GEO51-10-19541966,tsvantom}
reduces to simple linearization.  \par The squared group velocity
$V_g^2$ of $P$-waves in weakly anisotropic VTI media can be expressed
as a function of the group angle $\psi$ measured from the vertical
symmetry axis as follows:
\begin{equation}
   V_g^2(\psi) = V_z^2 \, \left(1 + 2 \, \delta \, \sin^2{\psi}\,\cos^2{\psi} 
                                  + 2 \, \epsilon\,\sin^4{\psi} \right)\;,
\label{eqn:vg}
\end{equation}
where $V_z = V_g(0)$ is the $P$-wave vertical velocity, and $\delta$ and
$\epsilon$ are Thomsen's dimensionless anisotropic parameters, which
are assumed to be small quantities:
\begin{equation}
   | \epsilon | \ll 1, \quad | \delta | \ll 1.
\label{eqn:epsdel} 
\end{equation}
Both parameters are equal to zero in isotropic media.
%Their
%connection with the stiffness coefficients has the following
%expressions \cite{GEO51.10.19541966}:
%\begin{eqnarray}
%\delta & = & {{(C_{13} + C_{44})^2 - (C_{33} - C_{44})^2} \over
%{2\,C_{33}\,(C_{33} - C_{44})}}\;,
%\label{eqn:delta} \\
%\epsilon & = & {{C_{11} - C_{33}} \over {2\,C_{33}}}\;.
%\label{eqn:epsilon} 
%\end{eqnarray}
\par
Equation (\ref{eqn:vg}) is accurate up to the second-order terms in
$\epsilon$ and $\delta$. We retain this level of accuracy throughout
the paper. As follows from equation
(\ref{eqn:vg}), the velocity $V_x$ corresponding to ray
propagation in the horizontal direction is
\begin{equation}
V_x^2 = V_g^2(\pi/2) = V_z^2\,(1 + 2\,\epsilon)\;.
\label{eqn:vx}
\end{equation}
Equation (\ref{eqn:vx}) is actually exact, valid
for any strength of anisotropy.  Another important quantity is the
normal-moveout (NMO) velocity, $V_n$, that determines the small-offset 
$P$-wave reflection moveout in homogeneous VTI media above a
horizontal reflector. Its exact expression is \cite[]{GEO51-10-19541966}
\begin{equation}
V_n^2 = V_z^2\,(1 + 2\,\delta)\;.
\label{eqn:vn}
\end{equation}
If $\delta = 0$ as, for example, in the ANNIE model proposed by
\cite{annie}, the normal-moveout velocity 
is equal to the vertical velocity.
%\par
%One example of a physical anisotropic model is ANNIE, proposed by
%Schoenberg, Muir, and Sayers \shortcite{annie} to describe anisotropy
%of shales. According to this model, the elasticity tensor (stiffness
%matrix) in transversely isotropic shales is represented by the three
%parameter approximation
%\begin{equation}
%C = \left[\begin{array}{cccccc}
%\lambda + 2\,\mu_H & \lambda & \lambda & 0 & 0 & 0 \\
%\lambda & \lambda + 2\,\mu_H & \lambda & 0 & 0 & 0 \\
%\lambda & \lambda & \lambda + 2\,\mu   & 0 & 0 & 0 \\
%0 & 0 & 0 & \mu & 0 & 0 \\
%0 & 0 & 0 & 0 & \mu & 0 \\
%0 & 0 & 0 & 0 & 0 & \mu_H \end{array}\right]\;,
%\label{eqn:sms}
%\end{equation}
%where $\lambda$, $\mu$, and $\mu_H$ are density-normalized elastic
%parameters. Formula (\ref{eqn:delta}) shows that Thomsen's parameter
%$\delta$ in this case is equal to zero, which corresponds to the known
%fact that the normal-moveout velocity for shales is approximately
%equal to the vertical velocity. The parameter $\epsilon$ in this case
%is defined by the equation
%\begin{equation}
%\epsilon = {{\mu_H - \mu} \over {\lambda + 2\,\mu}}\;.
%\label{eqn:smsepsilon}
%\end{equation}

\inputdir{Sage}

\par
It is convenient to rewrite equation (\ref{eqn:vg}) in the form 
\begin{equation}
V_g^2(\psi) = V_z^2\,\left(1 + 2\,\delta\,\sin^2{\psi} + 
2\,\eta\,\sin^4{\psi}\right)\;,
\label{eqn:vgeta}
\end{equation}
where 
\begin{equation}
   \eta = \epsilon - \delta \;.
\label{eqn:eta}
\end{equation}
Equation~(\ref{eqn:eta}) is the weak-anisotropy approximation for the
%The parameter $\eta$ is equivalent under the weak anisotropy assumption
%to the 
{\em anellipticity}$\,$ coefficient $\eta$ introduced by
\cite{aktsvan}. For the elliptic anisotropy, $\epsilon = \delta$ and
$\eta = 0$. To see why the group-velocity function becomes elliptic in
this case, note that for small $\delta$
\begin{equation}
\left. \frac{1}{V_g^2(\psi)} \, \right|_{\eta=0} = 
       \frac{1}{V_z^2\left(1 + 2\,\delta\,\sin^2{\psi} \right) } 
\approx 
\frac{\cos^2{\psi}}{V_z^2} + \frac{(1 - 2\,\delta)\,\sin^2{\psi}}{V_z^2}
\approx
\frac{\cos^2{\psi}}{V_z^2} + \frac{\sin^2{\psi}}{V_n^2} \;.
\label{eqn:vgeta1}
\end{equation}
Seismic data often indicate that $\epsilon > \delta$, so the anellipticity
coefficient $\eta$ is usually positive.
%In practical cases of VTI media, $\epsilon$ is often greater than
%$\delta$, so the anelliptic parameter $\eta$ is positive.

\par
An equivalent form of equation (\ref{eqn:vg}) can be obtained in terms of
the three characteristic velocities $V_z$, $V_x$, and $V_n$:
\begin{equation}
V_g^2(\psi) = V_z^2\,\cos^2{\psi} + 
\left(V_n^2 - V_x^2\right)\,\sin^2{\psi}\,\cos^2{\psi} + 
V_x^2\,\sin^2{\psi}\;.
\label{eqn:vgvs}
\end{equation}
From equation (\ref{eqn:vgvs}), in the linear
approximation the anelliptic behavior of velocity is controlled by
the difference between the normal moveout and horizontal velocities
or, equivalently, by the difference between anisotropic 
coefficients $\epsilon$ and $\delta$.
\par
We illustrate different types of the group velocities (wavefronts)
in Figure \ref{fig:nmofro}.
%anisotropy in Figure \ref{fig:nmofro}, which shows the wavefronts 
%for different values of the anisotropic parameters. 
The wavefront, circular in the isotropic case (Figure
\ref{fig:nmofro}a), becomes elliptical when $\epsilon=\delta \neq 0$
(Figure \ref{fig:nmofro}b). In the ANNIE model, the vertical and NMO
velocities are equal (Figure \ref{fig:nmofro}c). If $\epsilon > 0$ and
$\delta < 0$, the three characteristic velocities satisfy the
inequality $V_x > V_z > V_n$ (Figure \ref{fig:nmofro}d).

\plot{nmofro}{width=6in,height=3in}{Wavefronts in 
isotropic medium, $\epsilon=\delta=0$ (a),
elliptically anisotropic medium, $\epsilon=\delta=0.2$ (b),
ANNIE model, $\epsilon=0.2$, $\delta=0$ (c), and
anisotropic medium with $\epsilon=0.2$, $\delta=-0.2$ (d).
Solid curves represent the wavefronts. 
%??? -- Make them thicker as was before.
%??? -- Replace notation on plots with a, b, c, and d.
Dashed lines correspond to isotropic wavefronts for the vertical and
horizontal velocities. 
%??? -- Make them really dashed, not dash-dotted. 
%??? -- No need to do it for Vnmo because it is not clear what they show.
%Top left: isotropic case
%$(\epsilon=\delta=0)$; top right: elliptic case
%$(\epsilon=\delta=0.2)$; bottom left: ANNIE model $(\epsilon=0.2,
%\delta=0)$; bottom right: strongly anelliptic case $(\epsilon=0.2,
%\delta=-0.2)$.
}


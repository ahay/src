

\section{Wave-mode separation for 2D TI media}
%how does separation for VTI work
\subsection{Wave-mode separation for symmetry planes of VTI media}
\cite{GEO55-07-09140919} separate {\it quasi-}P 
and {\it quasi-}SV modes in 2D VTI media by projecting the wavefields
onto the directions in which P and S modes are polarized.  For
example, in the wavenumber domain, one can project the wavefields onto
the P-wave polarization vectors $\UU_P$ to obtain {\it quasi-}P ({\it
q}P) waves:
\beq\label{AniDivK}
\widetilde{{\it q}P}=i\, \UU_P(\kk) \cdot \WWK 
=i\, U_x\,\WK_x+i\, U_z\,\WK_z\, ,                
\eeq
where $\widetilde{{\it q}P}$ is the P-wave mode in the wavenumber
domain, $\kk=\{k_x,k_z\}$ is the wavenumber vector, $\WWK$ is the
elastic wavefield in the wavenumber domain, and $\UU_P(\kk)$ is the
P-wave polarization vector as a function of the wavenumber $\kk$.

%Christoffel equation

% what is different about TTI and VTI
The polarization vectors $\UU(\kk)$ of plane waves for VTI media in
the symmetry planes can be found by solving the Christoffel equation
~\cite[]{akirichards.2002,Tsvankin}:
\beq\label{3dChristoffel}\lb {\bf G} - \rho V^2 {\bf I} \rb \UU = 0 \, ,
\eeq
where {\textbf G} is the Christoffel matrix with
$G_{ij}=c_{ijkl}n_jn_l$, in which $c_{ijkl}$ is the stiffness tensor.
The vector $\mathbf n=\frac{\kk}{\left|\kk\right|}$ is the unit
vector orthogonal to the plane wavefront, with $n_j$ and $n_l$ being
the components in the $j$ and $l$ directions,
$i,j,k,l=1,2,3$. The eigenvalues $V$ of this system correspond to the
phase velocities of different wave-modes and are dependent on the
plane wave propagation direction $\mathbf k$.

For plane waves in the vertical symmetry plane of a TTI medium, since
{\it q}P and {\it q}SV modes are decoupled from the SH-mode and
polarized in the symmetry planes, one can set $n_y=0$ and obtain
\def\c11{c_{11}}
\def\c55{c_{55}}
\def\c13{c_{13}}
\def\c33{c_{33}}
\beq\label{VtiChristoffel}
\lb 
 \mtrx{ G_{11}-\rho V^2 &  G_{12}\\
        G_{12}          &  G_{22} -\rho V^2 }
\rb
\lb\mtrx{ U_x\\U_z} \rb
=0 \, ,
\eeq
where
\begin{eqnarray}
G_{11}&=&c_{11} n_x^2 +c_{55} n_z^2 \, ,\\
G_{12}&=&\lp c_{13}+c_{55}\rp n_xn_z\, ,\\
G_{22}&=&c_{55} n_x^2 +c_{33} n_z^2\, .
\end{eqnarray}
\rEq{VtiChristoffel} allows one to compute the polarization vectors
$\UU_P=\{U_x,U_z\}$ and
$\UU_{SV}=\{-U_z,U_x\}$ (the eigenvectors of the matrix
{\textbf G}) given the stiffness tensor at every location of the
medium.

\rEq{AniDivK} represents the separation process for the 
P-mode in 2D homogeneous VTI media. To separate wave-modes for
heterogeneous models, one needs to use different polarization vectors at every
location of the model~\cite[]{yan:WB19}, because the polarization
vectors change spatially with medium parameters.  In the space domain,
an expression equivalent to \req{AniDivK} at each grid point is
\beq\label{AniDivX}
{\it q}P=\nabla_a\cdot \WW     = L_x[W_x] + L_z[W_z] \, ,
\eeq
where $L\lb\,\cdot\,\rb$ indicates spatial filtering, and $L_x$ and $L_z$
are the filters to separate P waves representing the inverse Fourier
transforms of $i\, U_x$ and $i\, U_z$, respectively. The terms $L_x$
and $L_z$ define the ``pseudo-derivative operators'' in the $x$ and $z$
directions for a VTI medium, respectively, and they change according to the material
parameters, $V_{P0}$, $V_{S0}$ ($V_{P0}$ and $V_{S0}$ are the P and S
velocities along the symmetry axis, respectively), $\epsilon$, and
$\delta$~\cite[]{thomsen:1954}.

\subsection{Wave-mode separation for symmetry planes of TTI media}
My separation algorithm for TTI models is similar to the approach
used for VTI models.  The main difference is that for VTI media, the
wavefields consist of P- and SV-modes, and \reqs{AniDivK}
and \ren{AniDivX} can be used for separation in all vertical planes of
a VTI medium. However, for TTI media, this separation only works in
the plane containing the dip of the reflector, where P- and SV-waves are
polarized, while other vertical planes contain SH-waves as well.


To obtain the polarization vectors for P and S modes in the symmetry
planes of TTI media, one needs to solve for the Christoffel \req{VtiChristoffel}
with 
\begin{eqnarray}
G_{11}&=&c_{11} n_x^2 +2c_{15}n_xn_z+c_{55} n_z^2 \, ,\\
G_{12}&=&c_{15} n_x^2+\lp c_{13}+c_{55}\rp n_xn_z+c_{35} n_z^2\, ,\\
G_{22}&=&c_{55} n_x^2 +2c_{35}n_xn_z+c_{33} n_z^2\, .
\end{eqnarray}
Here, since the symmetry axis of the TTI medium does not align with
the vertical axis $k_z$, the TTI Christoffel matrix is different from
its VTI equivalent. The stiffness tensor is determined by the
parameters $V_{P0}$, $V_{S0}$ , $\epsilon$, $\delta$, and the tilt angle
$\nu$.

In anisotropic media, $\UU_P$ generally deviates from the wave vector
direction $\kk=\frac{\omega}{V}\nn$, where $\omega$ is the angular
frequency, $V$ is the phase vector. \rFgs{VTIpolar} and \subrfn{TTIpolar}
show the P-mode polarization in the wavenumber domain for a VTI medium
and a TTI medium with a 30$^\circ$ tilt angle, respectively. The
polarization vectors for the VTI medium deviate from radial
directions, which represent the isotropic polarization vectors
$\kk$. The polarization vectors of the TTI medium are rotated
30$^\circ$ about the origin from the vectors of the VTI medium.


\rFgs{dK_notaper_VTI} and \subrfn{dK_notaper_TTI} show the components
of the P-wave polarization of a VTI medium and a TTI medium with a
30{$^\circ$} tilt angle, respectively. \rFg{dK_notaper_rot_TTI} shows
that the polarization vectors in \rfg{dK_notaper_TTI} rotated to the
symmetry axis and its orthogonal direction of the TTI
medium. Comparing \rFgs{dK_notaper_VTI} and \subrfn{dK_notaper_rot_TTI},
we see that within the circle of radius $\pi$~radians, the components of this
TTI medium are rotated 30{$^\circ$} from those of the VTI
medium. However, note that the $z$ and $x$ components of the
polarization vectors for the VTI medium (\rFg{dK_notaper_VTI}) are
symmetric with respect to the $x$ and $z$ axes, respectively; in
contrast, the vectors of the TTI medium (\rFg{dK_notaper_rot_TTI}) are
not symmetric because of the non-alignment of the TTI symmetry with
the Cartesian coordinates.


\def\sk#1{\sin\lp #1 k\rp}
\def\done#1#2{\frac{\partial#1}{\partial#2}}
To maintain continuity at the negative and positive Nyquist
wavenumbers for Fourier transform to obtain space-domain filters,
i.e. at $k_x,k_z=\pm\pi$~radians, one needs to apply tapers to the vector
components. For VTI media, a taper corresponding to the
function~\cite[]{yan:WB19}
\beq\label{sintaper}
f(k)= -\frac{8\sk{}}{5k} + \frac{2\sk{2}}{5k} -\frac{8\sk{3}}{105k}
+ \frac{\sk{4}}{140k}
\eeq
can be applied to the $x$ and $z$ components of the polarization
vectors (\rFg{dK_notaper_VTI}), where $k$ represent the components
$k_x$ and $k_z$ of the vector $\kk$.
This taper ensures that $U_x$ and $U_z$ 
are zero at $k_z=\pm\pi$~radians and $k_x=\pm\pi$~radians, respectively.  
The components $U_x$ and $U_z$ are continuous in the
$z$ and $x$ directions across the Nyquist wave numbers, respectively,
due to the symmetry of the VTI media.  
Moreover, the application of this taper transforms polarization vector 
components to 8$^{th}$ order derivatives.
If the components of the isotropic polarization vectors $\kk$ are
tapered by the function in \req{sintaper} and then transformed to the
space domain, one obtains the conventional 8$^{th}$ order finite
difference derivative operators $\done{}{x}$ and
$\done{}{z}$~\cite[]{yan:WB19}. Therefore, the VTI separators reduce
to conventional derivatives---the components of the divergence and
curl operators---when the medium is isotropic.

For TTI media, due to the asymmetry of the Fourier domain derivatives
(\rFg{dK_notaper_TTI}), one needs to apply a rotational symmetric
taper to the polarization vector components to obtain continuity
across Nyquist wavenumbers. A simple Gaussian taper
%
\beq\label{gaussiantaper}
g(\kk)=C \, exp\lb-\frac{\left|\kk\right|^2}{2\sigma^2}\rb
\eeq
%
can be used, where C is a normalizing constant.  When one chooses a
standard deviation of $\sigma=1$ radian, the magnitude of this taper
at $\left|\kk\right|=\pi$~radians is about 0.7\% of the peak value,
and therefore the TTI components can be safely assumed to be
continuous across the Nyquist wavenumbers.
Tapering the polarization vector components
in \rFg{kdomain.notaper} with the function 
in \req{gaussiantaper}, one obtains the plots
in \rFg{kdomain.tapered}.  The panels
in \rFg{kdomain.tapered}, which exhibits circular continuity
across the Nyquist wavenumbers, transform to the space-domain separators 
in \rFg{xdomain.tapered}.  The space-domain
filters for TTI media is rotated from the VTI filters, also by the
tilt angle $\nu$.
%With the application of
%this taper, even for isotropic media, the operators constructed this
%way are 2D stencils instead of the conventional 1D finite difference
%operators.

The value of $\sigma$ determines the size of the operators in the
space domain and also affects the frequency content of the separated
wave-modes. For example, \rfg{gauss.sigma} shows the component $U_z$
and operator $L_z$ for $\sigma$ values of $0.25$, $1.00$, and $1.25$
radians.  A larger value of $\sigma$ results in more concentrated
operators in the space domain and better preserved frequency of the
separated wave-modes. However, one needs to ensure that the function
$g(\kk)$ at $\left|\kk\right|=\pi$~radians is small enough to assume
continuity of the value function across Nyquist wavenumbers. When one
chooses $\sigma=1$ radian, the TTI components can be safely assumed to
be continuous across the Nyquist wavenumbers. %discuss different sigma

For heterogeneous models, I can pre-compute the polarization vectors
at each grid point as a function of the $V_{P0}/ V_{S0}$ ratio, the
Thomsen parameters $\epsilon$ and $\delta$, and tilt angle $\nu$. I
then transform the tapered polarization vector components to the space
domain to obtain the spatially-varying separators $L_x$ and $L_z$. The
separators for the entire model are stored and used to separate P- and
S-modes from reconstructed elastic wavefields at different time
steps. Thus, wavefield separation in TI media can be achieved simply
by non-stationary filtering with spatially varying operators.  I
assume that the medium parameters vary slowly in space and that they
are locally homogeneous. For complex media, the localized operators
behave similarly to the long finite difference operators used for
finite difference modeling at locations where medium parameters change
rapidly.


%%%%%%%%%%%%%%%%%%%%%%%%%%%%%%%%%%%%%%%%%%%%%%%%%%%%%%%%%%%%%%
\inputdir{Matlab}
\multiplot{2}{VTIpolar,TTIpolar}{width=.48\textwidth}
{The polarization vectors of P-mode as a function of normalized
wavenumbers $k_x$ and $k_z$ ranging from $-\pi$~radians to $+\pi$~radians, for (a) a
VTI model with $V_{P0}=3.0$~km/s, $V_{S0}=1.5$~km/s, $\epsilon=0.25$ and
$\delta=-0.29$, and for (b) a TTI model with the same model parameters
as (a) and a symmetry axis tilt $\nu=30^\circ$. The vectors in (b) are
rotated 30$^\circ$ with respect to the vectors in (a) around $k_x=0$
and $k_z=0$.}


\inputdir{operator}
\multiplot{1}{dK-notaper-VTI,dK-notaper-TTI,dK-notaper-rot-TTI}
{width=.7\textwidth} {The $z$ and $x$ components of the polarization
vectors for P-mode in the Fourier domain for (a) a VTI medium with
$\epsilon=0.25$ and $\delta=-0.29$, and for (b) a TTI medium with
$\epsilon=0.25$, $\delta=-0.29$, and $\nu=30^\circ$. Panel (c)
represents the projection of the polarization vectors shown in (b)
onto the tilt axis and its orthogonal direction.
\label{fig:kdomain.notaper} }

\multiplot{1}{dK-VTI,dK-TTI,dK-rot-TTI}{width=.7\textwidth}
{The wavenumber-domain vectors
in \rFg{dK-notaper-VTI,dK-notaper-TTI,dK-notaper-rot-TTI} are tapered
by the function in \req{gaussiantaper} to avoid Nyquist
discontinuity. Panel (a) corresponds to \rFg{dK-notaper-VTI}, panel
(b) corresponds to \rfg{dK-notaper-TTI}, and panel (c) corresponds
to \rFg{dK-notaper-rot-TTI}.
\label{fig:kdomain.tapered}}

\multiplot{1}{dX-VTI,dX-TTI,dX-rot-TTI}{width=.7\textwidth}
{The space-domain wave-mode separators for the medium shown
in \rFg{VTIpolar,TTIpolar}. They are the Fourier transformation of the
polarization vectors shown in \rFg{kdomain.tapered}. Panel (a)
corresponds to \rFg{dK-VTI}, panel (b) corresponds to \rfg{dK-TTI},
and panel (c) corresponds to \rFg{dK-rot-TTI}. The zoomed views show
$24\times24$ samples out of the original $64\times64$ samples around the center of
the filters.
\label{fig:xdomain.tapered}}


\multiplot{1}{dzKX-sig0-TTI,dzKX-sig1-TTI,dzKX-sig2-TTI}
{width=.7\textwidth} {Panels (a)--(c) correspond to component $U_z$
(left) and operator $L_z$ (right) for $\sigma$ values of $0.25$,
$1.00$, and $1.25$ radians in \req{gaussiantaper}, respectively.  A larger
value of $\sigma$ results in more spread components in the wavenumber
domain and more concentrated operators in the space domain.
\label{fig:gauss.sigma}}



\section{Wave-mode separation for 3D TI media}
In order to separate all three modes---P, SV, and SH---in a 3D TI
medium, one needs to construct 3D separators. \cite{dellinger.thesis}
shows that P-waves can be separated from two shear modes by a
straightforward extension of the 2D algorithm. Indeed, for 3D TI
media, one can always obtain the P-mode by constructing P-wave separators
represented by the polarization vector $\UU_P=\{U_x, U_y, U_z \}$
and then projecting the 3D elastic wavefields onto the vector $\UU_P$. The
P-wave polarization vector with components $\{U_x, U_y, U_z\}$ 
is obtained by solving the 3D
Christoffel matrix~\cite[]{akirichards.2002,Tsvankin}:
\beq\label{VtiChristoffel3d.ch3}
\lb 
 \mtrx{ G_{11}-\rho V^2 &  G_{12}           & G_{13}\\
        G_{12}          &  G_{22} -\rho V^2 & G_{23} \\
        G_{13}          & G_{23}            & G_{33} -\rho V^2
 }
\rb
\lb\mtrx{ U_x\\U_y\\U_z} \rb
=0 \, .
\eeq
The notations in this equation have the same definitions as in \req{3dChristoffel}.
For TTI media, the matrix ${\mathbf G}$ has the elements
\begin{eqnarray}
G_{11}&=&c_{11}n_x^2+c_{66}n_y^2+c_{55}n_z^2 
     +2 c_{16}n_xn_y+2 c_{15}n_xn_z+2c_{56}n_yn_z \, ,\\
G_{22}&=&c_{66}n_x^2+c_{22}n_y^2+c_{44}n_z^2 
     +2 c_{26}n_xn_y+ (c_{45}+c_{46})n_xn_z+2c_{24}n_yn_z \, ,\\
G_{33}&=&c_{55}n_x^2+c_{44}n_y^2+c_{33}n_z^2 
     +2 c_{45}n_xn_y+2 c_{35}n_xn_z+2c_{34}n_yn_z \, ,\\
G_{12}&=&c_{16}n_x^2+c_{26}n_y^2+c_{45}n_z^2 
     + (c_{12}+c_{66})n_xn_y+ (c_{14}+c_{56})n_xn_z+(c_{25}+c_{46})n_yn_z \, ,\nonumber\\ \\
G_{13}&=&c_{15}n_x^2+c_{46}n_y^2+c_{35}n_z^2 
     + (c_{14}+c_{56})n_xn_y+ (c_{13}+c_{55})n_xn_z+(c_{36}+c_{45})n_yn_z \, ,\nonumber\\\\
G_{23}&=&c_{56}n_x^2+c_{24}n_y^2+c_{34}n_z^2 
     + (c_{25}+c_{46})n_xn_y+ (c_{36}+c_{45})n_xn_z+(c_{23}+c_{44})n_yn_z \, .\nonumber\\
\end{eqnarray}

When constructing shear mode separators, one faces an additional
complication: SV- and SH-waves have the same velocity along the
symmetry axis of a 3D TI medium, and this singularity prevents one
from obtaining polarization vectors for shear modes in this
particular direction by solving the Christoffel
equation~\cite[]{Tsvankin}. In 3D TI media, the polarization of the
shear modes around the singular directions are non-linear and cannot
be characterized by a plane-wave solution. Consequently, constructing
3D global separators for fast and slow shear modes is difficult.

% how does S-wave-mode separaton work
To mitigate the effects of the shear wave-mode singularity, I use
the mutual orthogonality among the P, SV, and SH modes depicted
in \rFg{polar3d}. In this figure, vector
$\nn=\{\sin\nu\cos\alpha,\sin\nu\sin\alpha,\cos\nu\}$ represents the
symmetry axis of a TTI medium, with $\nu$ and $\alpha$ being the tilt
and azimuth of the symmetry axis, respectively. The wave vector $\kk$
characterizes the propagation direction of a plane wave. Vectors
${\mathbf P}$, ${\mathbf {SV}}$, and ${\mathbf {SH}}$ symbolize the
compressional, and fast and slow shear polarization directions, respectively. 
For TI media, plane waves propagate in symmetry planes, and the symmetry axis
$\nn$ and any wave vector $\kk$ form a symmetry plane. For a plane
wave propagating in the direction $\kk$, the P-wave is polarized in
this symmetry plane and deviates from the vector $\kk$; the SV- and
SH-waves are polarized perpendicular to the P-mode, in and out of the
symmetry plane, respectively.

Using this mutual orthogonality among all three modes, I first
obtain the SH-wave polarization vector $\UU_{SH}$ by cross multiplying
vectors $\nn$ and $\kk$, which ensures that the SH mode is
polarized orthogonal to symmetry planes:
\bea\label{ShPolar}
\UU_{SH} 
&=&\nn\times\kk \nonumber \\
&=&\{ k_z n_y-k_y  n_z, \nonumber \\
&&~\,   k_x n_z- k_z n_x,  \nonumber \\
&&~\,   k_y n_x - k_x n_y \} \, .
\eea

Then I calculate the SV polarization vector $\UU_{SV}$ by 
cross multiplying polarization vectors P and SH modes, which ensures
the orthogonality between SV and P modes and SV and SH modes:
\bea\label{SvPolar}
\UU_{SV}
&=&\UU_{P}\times\UU_{SH} \, , \nonumber  \\
&=&\{     
     k_y n_x U_y - k_x n_y U_y+k_z n_x U_z - k_x n_z U_z,  \nonumber \\
&&~\,  k_z n_y U_z - k_y n_z U_z+k_x n_y U_x - k_y n_x U_x,  \nonumber \\
&&~\,  k_x n_z U_x - k_z n_x U_x+k_y n_z U_y - k_z n_y U_y
\} \, .
\eea
Here, the magnitude of the P-wave polarization vectors for a certain
wavenumber $|\kk|$ is a constant:
\beq
\left| U_{P} \right| = \sqrt{U_x^2+U_y^2+U_z^2}=c \, .
\eeq
This ensures that for a certain wavenumber, P-waves obtained by
projecting the elastic wavefields onto the polarization vectors are
uniformly scaled. For comparison, the magnitudes of all three modes
are respectively
\bea
\left| U_{P} \right| &=&c \, , \\
\left| U_{SV}\right| &=&c\sin\phi \, , \\
\left| U_{SH}\right| &=&c\sin\phi  \, ,
\eea
where $\phi$ is the polar angle of the propagating plane wave, i.e.,
the angle between vectors $\kk$ and $\nn$.
\rFg{polar3dP,polar3dS2,polar3dS1} shows the polarization
vectors of P-, SH-, and SV-modes computed
using \reqs{VtiChristoffel3d.ch3}, \ren{ShPolar}, and \ren{SvPolar},
respectively. The P-wave polarization vectors in \rfg{polar3dP}
all have the same magnitude, but the SV and SH polarization vectors
in \rfgs{polar3dS1} and \subrfn{polar3dS2} vary in magnitude. In the
symmetry axis direction, they become zero.  The zero amplitude of the
shear modes in the symmetry axis direction is not an abrupt but a
continuous change over nearby propagation angles. Using separators
represented by solutions to \req{VtiChristoffel3d.ch3} and
expressions \ren{ShPolar} and \ren{SvPolar} to filter the wavefields,
I obtain separated shear modes that are scaled differently than the
P-mode. For a certain wavenumber, the shear modes are scaled by
$\sin\phi$, with $\phi$ being the polar angle, which increases from
zero in the symmetry axis to unity in the orthogonal propagation
directions. Therefore, the separated SV- and SH-waves have zero
amplitude in the symmetry axis direction, and the amplitudes of the
shear modes are just kinematically correct.

The components of the polarization vectors for P-, SV-, and SH-waves
can be transformed back to the space domain to construct spatial
filters for 3D heterogeneous TI media.  For example, \rfg{filters3d}
illustrates nine spatial filters transformed from the Cartesian
components of the polarization vectors shown
in \rfg{polar3dP,polar3dS2,polar3dS1}.  All these filters can be
spatially varying when the medium is heterogeneous.  Therefore, in
principle, wave-mode separation in 3D would perform well even for
models that have complex structures and arbitrary tilts and azimuths
of TI symmetry.

\inputdir{XFig}
\plot{polar3d}
{width=\textwidth}{A schematic showing the elastic wave-modes
polarization in a 3D TI medium. The three parallel planes represent
the isotropy planes of the medium. The vector $\nn$ represents the
symmetry axis, which is orthogonal to the isotropy plane. The vector
$\kk$ is the propagation direction of a plane wave. The wave-modes P, SV,
and SH are polarized in the direction ${\mathbf P}$, ${\mathbf {SV}}$,
and ${\mathbf {SH}}$, respectively. The three modes are polarized orthogonal
to each other.}



\inputdir{Matlab}
\multiplot{2}{polar3dP,polar3dS2,polar3dS1}{width=.45\textwidth}
{The wave-mode polarization for P-, SH-, and SV-mode for a VTI medium
with parameters $V_{P0}=4.95$~km/s, $V_{S0}=2.48$~km/s,
$\epsilon=0.4$, and $\delta=0.1$. The P-mode polarization is computed
using the 3D Christoffel equation, and SV and SH polarizations are
computed using \rEqs{SvPolar} and~\ren{ShPolar}. Note that the SV- and
SH-wave polarization vectors have zero amplitude in the vertical
direction.}

\inputdir{vti3}
\plot{filters3d}{width=\textwidth}{The separation 
filters $L_x$, $L_y$, and $L_z$ for the P, SV, and SH modes for a VTI
medium. The corresponding wavenumber-domain polarization vectors are
shown in \rFg{polar3dP,polar3dS2,polar3dS1}. Note that the filter
$L_z$ for the SH mode is blank because the $z$ component of the
polarization vector is zero.  The zoomed views show $24\times24$ samples out
of the original $64\times64$ samples around the center of the filters. }

\section{Introduction}

%%
 % write about imaging in complex areas
%%

In regions characterized by complex subsurface structure,
wave-equation depth migration is a powerful tool for accurately
imaging the earth's interior. The quality of the final image greatly
depends on the quality of the velocity model and on the quality of the
technique used for wavefield reconstruction in the subsurface
\cite[]{GEO66-05-16221640}.

%%
 % the need for imaging in the angle domain
%%

However, structural imaging is not the only objective of wave-equation
imaging. It is often desirable to construct images depicting
reflectivity as a function of reflection angles. Such images not only
highlight the subsurface illumination patterns, but could potentially
be used for image postprocessing for amplitude variation with angle
analysis. Furthermore, angle domain images can be used for tomographic
velocity updates.

%% 
 % ray based vs. wavefield based
%%

Angle gathers can be produced either using ray methods
\cite[]{SEG-1998-1538,GEO68-01-02320254} or by using wavefield methods
\cite[]{GEO55-09-12231234,SEG-1997-1379,SEG-1999-08240827,SEG-2002-13601363,RickettSava.geo.img,SavaFomel.geo.ang,GEO69-05-12831298,Wu.directionalIllumination}. Gathers
constructed with these methods have similar characteristics since they
simply describe the reflectivity as a function of incidence angles at
the reflector. However, as indicated by \cite{GEO69-02-05620575}, even
in perfectly known but strongly refracting media angle gathers are
damaged by undersampling of data on the surface, regardless of the
method used for their construction. In this paper, we address the
problem of wavefield-based angle decomposition.

%% 
 % before vs. after migration
%%

Angle decomposition can be applied either before or after the
application of an imaging condition. The two classes of methods differ
by the objects used to study the angle-dependent illumination of
subsurface geology. The methods operating before the imaging condition
decompose the extrapolated wavefields from the source and receivers
\cite[]{GEO55-09-12231234,SEG-1997-1379,SEG-1999-08240827,Wu.directionalIllumination}.
This type of decomposition is costly since it operates on individual
wavefields characterized by complex multipathing.  In contrast, the
methods operating after the imaging condition decompose the images
themselves which are represented as a function of space and additional
parameters, typically refered to as {\it extensions}
\cite[]{RickettSava.geo.img,SavaFomel.geo.ang,SavaFomel.geo.tsic,SavaVasconcelos.gpr.eic}.
In the end, the various classes of methods lead to similar
representations of the angle-dependent reflectivity represented by the
so-called scattering matrix. The main differences lie in the
complexity of the decomposition and in the cost required to achieve
this result. In this paper, we focus on angle decomposition of
extended images.

%% 
 % CIGs vs. CIPs
%%

Conventionally, angle-domain imaging uses common-image-gathers (CIGs)
describing the reflectivity as a function of reflection angles and a
space axis, typically the depth axis. An alternative way of
constructing angle-dependent reflectivity is based on
common-image-point-gathers (CIP) selected at various positions in the
subsurface. As pointed out by \cite{SavaVasconcelos.gpr.eic}, CIPs are
advantageous because they sample the image at the most relevant
locations (along the main reflectors), they avoid computations at
locations that are not useful for further analysis (inside salt
bodies), they can have higher density at locations where the structure
is more complex and lower density in areas of poor illumination, and
they avoid the depth bias typical for gathers constructed as a
function of the depth axis. In this paper, we focus on angle
decomposition using extended CIPs.

%% 
 % wide-azimuth
%%

A recent development in wave-equation imaging is the use of
wide-azimuth data \cite[]{regone:2896,michell:2905,clarke:1128}.
Imaging with such data poses additional challenges for angle-domain
imaging, mainly arising from the larger data size and the
interpretation difficulty of data of higher dimensionality.  Several
techniques have been proposed for wide-azimuth angle decomposition,
including ray-based methods \cite[]{KorenEtAl.localAngle} and
wavefield methods using wavefield decomposition before imaging
\cite[]{ZhuWu.localImageMatrices,GPR52-06-05750591} or after imaging
\cite[]{SavaFomel.segab2.2005}. Here, we complete the set of
techniques available for angle gather construction by describing an
algorithm applicable to extended common-image-point-gathers.


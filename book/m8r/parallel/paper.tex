\chapter{Parallel processing in Madagascar}

Modern computers provide parallel computing facilities in the form of
either multiple processing units on one node or multiple distributed
nodes in a cluster. By taking advantage of parallel architectures,
many computational tasks can be significantly accelerated.

A particularly simple yet practically important case is \emph{data
  parallel} algorithms, where the input data can be split into
parallel chunks, processed in parallel by serial algorithms and then
accumulated back into one data stream. Madagascar provides a number of
convenient tools for running serial code into a data parallel fashion
without the need to resolve to low-level programming.

\section{Posix threads}

\section{OpenMP programming and sfomp}

OpenMP (from \emph{Open Multi-Processing}) is an application
programming interface (API) that simplifies parallel tasks on
shared-memory architectures (multiple processing units). OpenMP
appeared in 1997 (with specifications for Fortran) and is currently
supported by most modern C/C++ and Fortran compilers. GCC has provided
support to OpenMP since version 4.2, which appeared in 2007.

[Add references]

To use OpenMP in your program, you can use...

To simplify OpenMP-based processing for data-parallel tasks,
Madagascar provides \texttt{sfomp} utility. Suppose, for example, ...

\section{MPI programming and sfmpi}

\section{GPU programming}

\section{Parallel processing in SCons: pscons}

\section{Parallel processing on shared clusters: sfbatch} 


\chapter{Managing processing flows using SCons}

There are several options for doing data processing using Madagascar:
you can run programs on the command line, you can collect a sequence
of commands in a Shell script or a Python script. A superior
alternative, as this chapter explains, is to use a workflow management
tool, such as SCons.

\section{What is SCons?}

SCons stands for \emph{Sofware Construction} and is a tool designed
primarily for compiling software \cite[]{scons}, as a modern
replacement for the Unix \texttt{make} utility \cite[]{make}. The
SCons configuration files (\texttt{SConstruct} scripts) are written in
the Python programming language.

\subsection{Useful SCons options}

\section{Why use SCons?}

\section{Managing data processing flows using \texttt{rsf.proj}}

\subsection{Seismic Unix processing using  \texttt{rsf.suproj}}

\section{Creating documents using \texttt{rsf.tex}}

\section{Creating books and reports using \texttt{rsf.book}}

\bibliographystyle{seg}
\bibliography{scons}

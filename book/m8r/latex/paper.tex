\chapter{Writing reproducible papers, books, and reports}

Chapter~\ref{chapter:scons} describes some of the SCons tools used by
Madagascar users to write papers, books, and reports with reproducible
examples. In this chapter, we will take a closer look at an example paper.

\section{Overview of \LaTeX and SEG\TeX}

\LaTeX is a document markup language developed originally by Leslie
Lamport in early 1980s \cite[]{latex} as an extension of Donuld
Knuth's \TeX system \cite[]{tex}. \LaTeX is widely used in different
fields of science and engineering.

\LaTeX files are text files that can be edited in regular text
editors. Similarly to HTML and Markdown, \LaTeX is a descriptive
language: one can learn it by example following other \LaTeX
documents. As of 2021, the Madagascar package contains nearly 300
papers and book chapters written in \LaTeX. You can find the source of
this chapter in
\href{https://github.com/ahay/src/blob/master/book/m8r/latex/paper.tex}
     {\$RSFSRC/book/m8r/latex/paper.tex}
and attached in the appendix.

SEG\TeX is a collection of \LaTeX macros developed for geophysical
publications, such as \emph{Geophysics} articles, SEG (Society of
Exploration Geophysicists) Annual Meeting abstracts, etc. SEG\TeX is
an open-source project maintained by a group of volunteers. It
integrates well with Madagascar to support the practice of
\emph{reproducible research}, as formulated by Jon Claerbout:
attaching software and data to the results of computational
experiments to allow the reader to verify all computations reported in
the paper using their own computer
\cite[]{fomel2008guest}. Computational reproducibility is not the goal
in itself but the means by which previous research can be extended
further. It effectively turns all reported computational results into
benchmarks and challenges future authors to try performing analagous
computations with more efficiency and accuracy \cite[]{fomel2014reproducible}.

\LaTeX processing scripts included with Madagascar can turn a single
document into multiple outputs: a paper in a journal, a chapter in a
book, an expanded abstract for a conference, etc. It can also produce
multiple output formats, most importantly PDF and HTML.

\section{SCons processing}

To enable handling of \LaTeX documents, Madagascar provides special
SCons rules provided in \texttt{rsf.tex} package (for handling
individual papers) and \texttt{rsf.book} package (for handling books
and reports).

If a \LaTeX paper file has the name
\texttt{article.tex}, the \texttt{SConstruct} script for processing it
may look like
\lstset{language=python,showstringspaces=false,frame=single}
\begin{lstlisting}
  from rsf.tex import *

  Paper('article',use='amsmath',options='10pt')
\end{lstlisting}
After running
\begin{verbatim}
scons article.pdf
\end{verbatim}
the paper will be processed to generate a PDF document.

According to the Madagascar convention, the source LaTeX file
(\texttt{article.tex}) contains neither the usual preamble
\texttt{\\documentclass}, nor
\texttt{\\begin{document}}--\texttt{\\end{document}} commands. This is
done so that the same paper could be easily integrated into a
collection of papers in a report or as a chapter in a book. Running
\texttt{scons article.pdf} generates an intermediate file
\texttt{article.ltx}, which contains the following preamble:
\lstset{language=tex,showstringspaces=false,frame=single}
\begin{lstlisting}
  \documentclass[10pt]{geophysics}
  \usepackage{amsmath}
  \begin{document}
\end{lstlisting}
  and ends with
\begin{lstlisting}
  \end{document}
\end{lstlisting}
  
Several optional arguments can be supplied to the \texttt{Paper()} command. Some of them are:
\begin{description}
%\item[source=] specifies
\item[lclass=] specifies the LaTeX class. For example, \texttt{lclass=segabs} can be used for SEG abstracts.
%\item[scons=] specifies
\item[use=] specifies additional packages to include with \texttt{\\usepackage}. For example, \texttt{use=hyperref} can be used to add the \texttt{hyperref} package for hypertext links.
\item[include=] specifies additional commands to include in the preamble of the \LaTeX file.
\item[options=] specifies additional options for the \LaTeX class.
\end{description}
Some additional arguments control the placement of reproducible
figures and are described in the next section.

The \texttt{SConstruct} file for paper processing may end with
\begin{verbatim}
End()
\end{verbatim}
which is equivalent to setting \texttt{paper.tex} as the default
\LaTeX file. The name can be changed by giving \texttt{paper=} command
to \texttt{End()}. All other agruments to \texttt{End()} are
effectively interpreted as \verb#Paper('paper')#. The \texttt{End()}
command is not strictly necessary, but it can be convenient, because
it enables command-line commands like \texttt{scons pdf},
\texttt{scons read}, etc. to be applied to the default file without
the need to spell out the name of the paper file every time.

\subsection{Including reproducible results}

A typical directory structure for writing reproducible papers follows the structure of \texttt{\$RSFSRC/book} directory in the Madagascar source repository:
\begin{verbatim}
book1/
    paper1/
        SConstruct [from rsf.tex import *]
        paper.tex
        project1/
            SConstruct [from rsf.proj import *]
        project2/
            SConstruct [from rsf.proj import *]
\end{verbatim}

To include the results from the project directories into the paper,
run first \texttt{scons lock} in each project directory. This command
generates results figures (specified by \texttt{Result()} in the
project \texttt{SConstruct}) files and then copies them to a separate
location. The figures and all intermediate files can be then safely
removed from the current directories and data directories by running
\texttt{scons -c}. The safe directory for locked figures can be
defined by \texttt{RSFFIGS} environmental variable. By default, it is
set to \texttt{\$RSFROOT/share/madagascar/figs}.

To generate and lock all figures at once, you can run \texttt{sftour
  scons lock} from the paper directory. Analogously, all intermediate
files in project directories can be cleaned by running \texttt{sftour
  scons -c}. The \texttt{sftour} scripts simply visits all
subdirectories (whose names start with a lower-case letter) and
executes the specified command in each subdirectory.
  
Note that, when the paper is generated by \LaTeX in the paper
directory, the figures are picked from the locked directory and not
from subdirectories. What is the purpose of this arrangement? First,
it explicitly separates two activities: running computational
experiments to generate figures and writing a paper or a book
describing the results. At the end of the first activity, the figures
are locked not to be touched again. Second, storing figures separately
allows for testing and maintaining reproducibility over time. Papers
included in Madagascar store their reproducible figures in a separate
repository\footnote{\url{https://github.com/ahay/figs}}, which is used
for a systematic testing of reproducibility, implemented through
continuous integration \cite[]{duvall2007continuous}.

\subsection{Multiple outputs}

Madagascar's \LaTeX conventions allow the same source to be used for
multiple different outputs: expanded abstracts, journal papers, report
papers, thesis chapters, etc. It can also produce the output in
different formats.

\subsubsection{Multiple styles}

To control the document class and the document style, use options of the \texttt{Paper} or \texttt{End} command in the paper-level \texttt{SConstruct} file, such as  \texttt{lclass=} and \texttt{use=}.

For example,
\lstset{language=python,showstringspaces=false,frame=single}
\begin{lstlisting}
  Paper('article',lclass='segabs')
\end{lstlisting}
produces an SEG expanded abstract, while
\lstset{language=python,showstringspaces=false,frame=single}
\begin{lstlisting}
  Paper('article',lclass='geophysics',options='manuscript',include='hyperref')
\end{lstlisting}
produces a Geophysics paper in the manuscript style using \textt{hyperref} package for hypertext links.

\LaTeX\ can also be used effectively for generating presentation slides. For example, the following will create slides using the \texttt{beamer} package and configuring it to use the \texttt{Madrid} theme:
\lstset{language=python,showstringspaces=false,frame=single}
\begin{lstlisting}
Paper('school10',lclass='beamer',
      include=r'''
      \usetheme{Madrid}
      ''')
\end{lstlisting}    

\subsubsection{Multiple formats}

PDF, HTML

\section{Including papers into books and reports}

\section{Acknowledgments}

The \LaTeX-processing scripts in Madagascar are inspired by a system developed
earlier at SEP (Stanford Exploration Project) for implementing Jon
Claerbout's vision of reproducible research \cite[]{sep}.

\bibliographystyle{seg}
\bibliography{latex}

% Add latex.bib !

%\appendix
%\section{Appendix: The source of this document}
%\verbatiminput{paper.tex}


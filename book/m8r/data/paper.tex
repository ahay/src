\chapter{Seismic data processing}

Although core Madagascar programs are applicable for analyzing any
kinds of data, there also programs specialized for seismic reflection
data processing. In this chapter, we will walk through processing one
particular dataset. Multiple additional examples are available under
\texttt{\$RSFSRC/book/data} and \texttt{\$RSFSRC/book/geo384s}. The
latter contains computational assignments from a seismic data
processing course at the University of Texas at Austin. An example
processing of a seismic land 3D dataset (Teapot Dome) is provided by
\cite{oren2018overview} and included in
\texttt{\$RSFSRC/book/data/teapotdome/canoren}.

\section{Poland dataset}

This dataset was made publicly available by Geofizyka Torun S.A. and
provided initially by Waldek Ogonowski for a tutorial on OpenUSP, a
seismic processing package from Amoco and
BP\footnote{\url{https://www.freeusp.org/RaceCarWebsite/TechTransfer/Tutorials/Processing_2D/Processing_2D.html}}.

The 2D reflection seismic survey was acquired on land using vibroseis
sources. The goal of a seismic processing project is to process the
dataset all the way to a seismic image.

\section{Analyzing acquisition geometry}

\bibliographystyle{seg}
\bibliography{data}


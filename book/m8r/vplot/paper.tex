\chapter{Vplot graphics}

The final results of a data analysis project are often represented by figures. Madagascar includes a versatile plotting library for generating scientific illustrations called Vplot. Vplot defines a data format for storing figures. Figures stored in that format can be displayed on the screen or transformed into other formats for inclusion in scientific papers, presentation slides, or other documents.

\section{Vplot history}

Vplot was developed in 1980s at the Stanford Exploration Project
\cite[]{{Cole.sep.60.349,Dellinger.sep.61.327}. At that time, there
  was no universal format for graphics files and no univeral libraries
  for representing graphics on hardware devices, such as monitors or
  printers. The purpose of the Vplot was to provide such a format so
  that the same plots could be easily transferred between devices.

  The man page for Vplot says
  \begin{quote}
    Vplot originally stood for `vector plot', but this name is now insufficient because  vplot
       supports not only vectors, but also filled areas (either raster patterns or hatched), text
       (Hershey fonts), and raster images  (including  grey-scale  dithered  ones  on  monochrome
       devices such as laser printers).
    \end{quote}

    Joe Dellinger was the main developer of Vplot, other major contributors at
    Stanford were Steve Cole and Dave Nichols, with additional
    contributions by Ray Abma, Stew Levin, David Lumley, and Hector
    Urdaneta. After Vplot was adopted by Madagascar, new filter
    programs (called ``pens'') for converting Vplot files to other
    formats were added.

\section{Vplot format}

\subsection{Python interface}

\subsection{sfpldb and sfplas}

\subsection{sfvplotdiff and reproducibility testing}

\section{Graphics programs}

\subsection{sfdots}

\subsection{sfgraph}

\subsection{sfgraph3}

\subsection{sfwiggle}

\subsection{sfbargraph}

\subsection{sfcontour}

\subsection{sfcontour3}

\subsection{sfthplot}

\subsection{sfgrey}

\subsection{sfgrey3}

\subsection{sfgrey4}

\subsection{sfbox}

\section{Pen programs}

\subsection{xtpen}

\subsection{oglpen}

\subsection{pspen}

\section{Vplot and interactivity}

\bibliographystyle{seg}
\bibliography{SEP2}
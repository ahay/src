
\lefthead{Claerbout}
\righthead{Reflection tomography}
\footer{SEP--79}

\title{Reflection tomography:  Kjartansson revisited}
%\keywords{velocity, modeling, deconvolution, amplitudes, tomography, slant stack }

\author{Jon F. Claerbout}
\maketitle

\begin{abstract}
	I begin experiments with Kjartansson's data set with the
	goal of constructing a near-but-below-surface velocity model.
	I test lateral extrapolation of data before tomography
	and propose an alternate definition of pilot trace.
	Traditionally, a pilot trace is defined as a stack trace.
	Here I define a pilot trace as
	the output of a three-dimensional prediction filter.
	This pilot trace is designed to be
	insensitive to velocity, dip, and shot statics.
	The crosscorrelation of this pilot trace with the observed trace
	is highly sensitive to geophone statics
	as well as deeper lateral velocity variations beneath the geophones.
	This crosscorrelation function of midpoint and offset
	is slant-stacked transforming offset to depth.
	With suitable normalization the result is a volume
	of coherency with axes of depth, midpoint, and slowness.
	The strange ``footprint'' of the 3-D prediction filter
	confuses the construction of a specific slowness model.
	A less confusing alternate is a Snell wave approach.
\end{abstract}


\section{INTRODUCTION}
\inputdir{kjar}
\cite{Kjartansson.sep.23}
showed that shallow absorption
or velocity anomalies can masquerade as gas-bearing bright spots.
(For a brief summary see \cite[]{Claerbout.blackwell.85}.)~
Kjartansson showed that the bright-spots masquerade is easily uncovered
by making a plot of energy in some time gate
as a function of midpoint and offset.
His plot shows streaking generally aligned
with the shot axis and with the geophone axis
whereas an amplitude streak resulting from a small bright spot should
be aligned on a constant midpoint.
More careful examination shows that
the alignments of Kjartansson's data
are not precisely along the shot and geophone axes,
but they are tilted slightly towards the midpoint axis.
For my recalculation of Kjartansson's figure, see
Figure~\ref{fig:kjamp}.
\plot{kjamp}{width=6in,height=1in}{
	Kjartansson's Grand Isle data,
	root-mean-square energy in the time interval 2.0-2.6 sec,
	after NMO with velocity 2.05.
	}

Kjartansson interprets the tilt as a measure of depth of burial
of the anomaly and he slant stacks to provide a map of
anomaly depth versus midpoint.
(Actually he does better than slant stack,
he attempts a planar decomposition).
In Figure~\ref{fig:kjamp} there is a pronounced streak at a
$45^\circ$ angle associated with a missing shot.
A missing shot is equivalent to perfect absorption directly under
the shot, so this $45^\circ$ streak corresponds to zero depth
under the shot.
Likewise, if we were to see a $45^\circ$ streak of opposite orientation,
it would correspond to zero depth under the geophone.
A vertical streak corresponds to absorption
at the reflector under the midpoint.
At any depth, there are two applicable angles, one tipping one way
(for shots) and one tipping the other (for geophones).

\par
Kjartansson does not analyze anomalies with regard to whether
they result from  absorption or focusing,
but those of us who have worked with migration for many years
tend to regard all amplitude variations as focusing phenomena because
focusing so easily gives amplitude variations.
Kjartansson shows one NMO corrected CMP gather
that clearly shows time shifts,
but when he assembles a plane of times in midpoint-offset space,
alignments of interesting events are not so clear as with amplitudes.
His slant stack of this plane is disappointing compared to his amplitudes.
(Those seismologists who believe that observational seismology
gives reliable timing measurements but unreliable amplitude measurements
will be surprised by Kjartansson's very convincing results.)
In summary, Kjartansson's work showed clearly the locations
of likely lateral-velocity variation anomalies, but provided
no direct path toward building a best fitting velocity model.

\par
A variety of factors led me to revisit this data and its analysis:
Two members of our group had developed wave-equation datuming code
and were about to struggle with the problem of estimating the velocity model.
Computers are much more powerful than they were when Kjartansson did his work.
Kjartansson's data (or a subset) were recently found and loaded on our machines.
My recent research direction,
data adaptive multidimensional filters (DAMF), encouraged me to believe
that I could readily accomplish potentially complicated corrections,
prediction of one shot profile from another, for example,
without need for an elaborate theory.
I suspected that DAMF might provide the missing components
needed to create a near surface (top 500 ms)
lateral velocity variation model
using only its distortion of deep reflections and so I set out to try.

%\par
%First we review the overall problem setting,
%and then turn to the more complicated matter of constructing
%pilot traces whose crosscorrelation with data contains
%essentially only geophone-neighborhood causes.

Figures~\ref{fig:kartom.0}-\ref{fig:kartom.2} show various
slant-stack like transformations of the logarithm of the amplitude
in Figure~\ref{fig:kjamp}.


\activeplot{kartom.0}{width=6in,height=2in}{\CAKEDIR}{
	Top shows the log amplitude of the energy in a
	time gate from 2.0-2.6s (after NMO with v=2.05).
	Bottom shows a slant stack ranging from common shot
	to common geophone.  Think of the vertical axis on
	the bottom plot as ranging along the ray path.
	The missing shot streak on the top panel becomes
	a clipped impulse function on the bottom at km=4.2 and time=-1.0.
	The missing shot streak truncates at near offset
	and at far offset. The near-offset truncation gives
	a near vertical streak through the clipped impulse.
	The wide-offset truncation gives a near $45^{\circ}$ streak
	through the pulse.
	Geophysically interesting are about a dozen ``blobs''
	on the bottom frame
	about 300-600ms from the earth surface either on
	the shot path or geophone path.
	}

\activeplot{extend}{width=6in,height=3in}{\CAKEDIR}{
	Top is the
	logarithm of amplitude as in Figure~\protect\ref{fig:kartom.0}
	except that it is padded.
	Bottom shows the padded amplitude after extension from
	an initial guess that the logarithm take on its mean value.
	Failure to fill in a short line segment from about 6 to 7km
	at about 2km offset results from the mask program assuming
	that only zero-valued data is missing.
	The extension uses a prediction filter predicting
	in the upward direction, so the upward side shows
	stronger short wavelengths than the downward side.
	}

\activeplot{kartom.2}{width=6in,height=3in}{\CAKEDIR}{
	Repetition of Figure~\protect\ref{fig:kartom.0} with
	extended data.
	}

To mimic Kjartansson's plot of amplitude versus midpoint and depth,
I folded and summed the bottoms panels
of each of
Figure~\ref{fig:kartom.0} and
Figure~\ref{fig:kartom.2} about $t=0$
thereby combining information from the upgoing and downgoing path.
The result is shown in Figure~\ref{fig:einarmodl}.
\activeplot{einarmodl}{width=6in,height=2in}{\CAKEDIR}{
	Tomographic distribution of energy with midpoint and traveltime depth.
	Top is for amplitudes before extension,
	bottom is after.
	}

\section{VALUE OF DATA EXTENSION}
The value of extending the data should not be judged
on results shown above.
First, the extension methodology is new and there was no time
to tune the parameters.
Another unfamiliar new problem is that logarithms have an additive constant
that I was unsure of how to optimize.
My motivation for the extension was caused by observing steady growth
of side boundary effects (midpoint axis) as the number of iterations
in the tomography problem increased.
Keep in mind the background that
this data set is $48\times 375=18,000$ points so assuming we seek
a model space of the same number of points,
the solver theoretically requires 18,000 iterations.
My experimental work was limited to  5-25 iterations.
A question in my mind was and is,
how close are these solutions to the final limit?
I hypothesize that truncation of a data set could severely
limit the rate of convergence during the first 5-25 iterations
and that extending the data would accelerate convergence.
Time did not allow adequate testing of this interesting hypothesis.
\par
I also saw that the solution
deteriorates in an interesting manner at high iterations.
Damping can be installed in the tomography,
but I had no time to experiment or think of the theory.
I wondered how the above problems would be affected by
extending the data, which seemed to help, but which
seems to have been less important than other variables,
particularly, the iteration count.
\par
An overarching doubt is whether the underlying absorption model
warrants any more effort, since I believe that focusing
is a more likely model.


\section{CODING THE TOMOGRAPHY }
I am using subroutine \texttt{xtomo()} \vpageref{/prog:xtomo}.
\gprogblock{xtomo}{xtomo.rt}
\par
The inversion problem in tomography was solved
by using the conjugate-gradient subroutine {\tt cgstep()}
described in PVI invoked by subroutine \texttt{tominv()} \vpageref{/prog:tominv}.
It allows for damping {\tt eps} but I did not test damping.
\gprogblock{tominv}{tominv.rt}

I omitted the listing of the main program
because it merely serves to connect
the listed subroutines to our file system.
%\gprogblock{Tomo}{Tomo.rs}



\section{SPECULATION ON PILOT TRACE AND VELOCITY ESTIMATION}

\par
Although I am enthusiastic
about the new potential I see in using DAMF
(data adaptive multidimensional filtering)
for determining a pilot trace,
DAMF is complicated and unfamiliar,
so we first review the sequence of steps
that follow the preparation of the pilot traces.
Kjartansson's pilot trace is simply a moved out CDP stack.
Kjartansson crosscorrelates pilots against
individual traces and then scans each crosscorrelation
for a maximum value, a process I will call picking.
It is the picked times that are plotted in the midpoint-offset plane.
The principle of superposition of seismic events
of various stepouts applies before picking, not after,
so it is not surprising that Kjartansson's attempted decomposition
of his already noisy time picks yields an unconvincing map of time anomalies.
I plan to attack this problem by simultaneously doing two things differently.

\par
First, my DAMF pilot traces
should be stable in the presence of velocity residual,
varying reflector image,
and shot statics but totally
independent of the single effect
of delay near and below the geophones.
(This makes them a good base from which to measure those delays.)~
Second, I do not pick the crosscorrelation until after
slant stacking (which transforms offset to depth).
Thus, I pick to collapse a volume of coherency
as a function of depth, midpoint, and slowness
to get a plane of slowness as a function of depth and midpoint.
\par
This plan raises an interesting question,
``What is the theoretical validity of the operation of
backprojecting a crosscorrelation function?''
The validity is certainly limited by some commutivity issues.
I plan to revisit this question after my codes are more complete.

\subsection{Prediction filter templates}
My new DAMF prediction filters are three dimensional.
The locations of filter coefficients in the
midpoint-offset space is shown in the mask below:
\begin{verbatim}
     .00     .00     .00    2.00     .00     .00     .00
    1.00     .00     .00     .00     .00     .00     .00
    1.00    1.00     .00     .00     .00     .00     .00
    1.00    1.00    1.00     .00     .00     .00     .00
    1.00    1.00    1.00    1.00     .00     .00     .00
    1.00    1.00    1.00    1.00    1.00     .00     .00
    1.00    1.00    1.00    1.00    1.00    1.00     .00
    1.00    1.00    1.00    1.00    1.00    1.00    1.00
    1.00    1.00    1.00    1.00    1.00    1.00    1.00
\end{verbatim}
\par
In the above mask, the vertical axis is midpoint and the horizontal is offset.
Where the mask contains a zero,
the filter coefficient is constrained to zero.
Where the mask contains ``1.00'',
the filter coefficient is free,
both in this filter plane and in planes above and below
which filter on the time axis.
Where the mask contains ``2.00''
the filter coefficient is constrained to be ``1.00''
in this plane and zero in other planes.
\par
Kjartansson's filter, by comparison, lies on one time slice.
It is a ``1.00'' on the predicted point
and $-1/47$ on each of the other traces at this midpoint.
\par
Figure~\ref{fig:xcorr} shows
the zero lag crosscorrelation of data with its prediction.
Naturally, the crosscorrelation is sensitive to amplitude
as well as timing, so the crosscorrelation at zero lag
is shown before and after normalization.
Naturally also, the crosscorrelation need not be positive,
but tends to be positive
(since we are correlating a signal with its prediction)
so I had a little trouble getting the brightness adjusted 
suitably for plotting.

\activeplot{xcorr}{width=6in,height=3.5in}{\CAKEDIR}{
	Zero lag crosscorrelation of data with its prediction.
	Top without normalization.
	Bottom with normalization.
	Amplitude normalization means the display
	is sensitive to time shifts.
	Press button for movie cube.
	}


\section{WITHOUT PILOT TRACES}
Because of my confusion in trying to imagine construction
of a specific velocity model from the odd-shaped footprint
of the prediction portion of the 3-D filter,
I offer a suggestion of another method.
\par
Recall that spectra and autocorrelations of independent signals add,
so we can think of observing a crosscorrelation at the surface
and then distributing it along a ray path.
\par
Suppose we forget fancy pilot traces
and cross correlate each surface trace with its neighbor.  
Then we could distribute (spray) each of these crosscorrelations
back along its ray from the surface to the reflector.
(You can see that subroutine \texttt{xtomo()} \vpageref{/prog:xtomo} was designed with this in mind.)
\par
Thus at all depths we get a crosscorrelation function
from which we can pick a lag $\Delta t$,
thus getting $\Delta t / \Delta g$
for all $(x,z)$.
Since we are principally concerned with a region fairly near
the surface, we can think of $g$ and $x$ as being much the same.
Integration over $g$ gives us the slowness model:
\begin{equation}
s(g,z) \quad = \quad
\int \ dg\ {\partial\over\partial z} \left(dt\over dg\right)
\label{eqn:corrtomo}
\end{equation}
\par
An alternate but similar approach is to start off
by transforming $s$ to $p_s$, i.e.~transform to Snell waves
\cite[]{Claerbout.blackwell.85}.
This completely smears out the shot statics
and has other interesting statistical attributes.
After moveout correction, in the presense of modest dips,
the upcoming waves should be quasi flat and the rays
quasi parallel.  Thus $\Delta t /\Delta g$
measured from the peak of the crosscorrelation of
adjoining geophones should be
approximately zero and departures from zero should
be sensitive to lateral velocity variations.
After tomographically projecting $\Delta t /\Delta g$
downward into the earth, application of equation~(\ref{eqn:corrtomo})
should produce slowness as a function of depth.
\par
Now we might ask, what are the similarity and difference
of the Snell-wave approach and the DAMF-prediction approach?
The Snell-wave approach seems clearest for constructing
the slowness model.
The DAMF-prediction approach should best be able to handle dip.
I conclude further experimental work is needed.

\section{FINAL SPECULATION}
This interesting and challenging data set deserves much further study.
I believe the supplemental code that is required is a prestack
finite difference upward and downward continuation.
Shots and geophones need to be upward and downward continued
through the top 600ms
with various experimental lateral velocity variations.
This might seem like a ``solved problem'' but
I see many difficulties.
It must be done rapidly and without excessive artifacts.
I think this means
it needs to be done {\it after} NMO,
again something which seems theoretically realistic,
but which we have little experience since early days of SEP.
\par
Given that we can master the technology of rapid upward
and downward continuation, suppose we estimate static shifts
by traditional delay of each trace from the CDP stack.
Then we can attribute half the delay to the shot path
and half to the geophone path.
To ascertain a better depth for the anomaly than the zero depth
we can make a Kjartansson-style amplitude tomograph.
Now we could downward continue shots only
treating the shot static as a velocity anomaly
to be distributed over the range indicated by the amplitude tomograph.
From this shot-downward continued data,
we could redo the amplitude tomograph,
presumably seeing a revised and improved model for geophones.
Hopefully an orderly bootstrapping procedure can be found.


\section{Acknowledgments}
I'd like to thank David Lumley for finding and reading
Kjartansson's data tape.


\bibliographystyle{seg}
\bibliography{SEP2}



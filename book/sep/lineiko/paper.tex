\lefthead{Fomel}
\righthead{Linearized Eikonal}
\footer{SEP--94}
%\shortnote
\published{SEP report, 94, 123-131 (1997)}
\title{Traveltime computation with \\ the linearized eikonal equation}

\email{sergey@sep.stanford.edu}
%\keywords{ eikonal, traveltime, imaging, velocity, tomography }
\author{Sergey Fomel}

\maketitle

\section{INTRODUCTION}

Traveltime computation is an important part of seismic imaging
algorithms. Conventional implementations of Kirchhoff migration
require precomputing traveltime tables or include traveltime
calculation in the innermost computational loop .  The cost of
traveltime computations is especially noticeable in the case of 3-D
prestack imaging where the input data size increases the level of
nesting in computational loops.
\par
The eikonal differential equation is the basic mathematical model,
describing the traveltime (eikonal) propagation in a given velocity
model.  Finite-difference solutions of the eikonal equation have been
recognized as one of the most efficient means of traveltime
computations
\cite[]{GEO55-05-05210526,GEO56-06-08120821,Popovici.sep.70.245}.  The
major advantages of this method in comparison with ray tracing
techniques include an ability to work on regular model grids, a
complete coverage of the receiver space, and a fair numerical
robustness. The most common implementations of the finite-difference
eikonal equation compute the \emph{first-arrival} traveltimes, though
frequency-dependent enhancements
\cite[]{SEG-1992-1315,Nichols.sepphd.81} can extend the method to
computing the most energetic arrivals. The major numerical complexity
of the finite-difference eikonal computations arises from the
fundamental non-linearity of the eikonal equation. The numerical
complexity is related not only to the direct cost of the computation,
but also to the accuracy and stability of finite-difference schemes.
\par
It is important to note that the current practice of seismic imaging
is not limited to a single migration. Moreover, it is repeated
migrations, with velocity analysis and refinement of the velocity
model at each step, that take most of the computational effort. When
the changes in the velocity model at each step are small compared to
the initial model, it is appropriate to linearize the eikonal equation
with respect to the slowness and traveltime perturbations.
Mathematically, the linearized eikonal equation corresponds precisely
to the linearization assumption, commonly used in traveltime
tomography.
\par
In this paper, I propose an algorithm of finite-difference traveltime
computations, based on an iterative linearization of the eikonal
equation. The algorithm takes advantage of an implicit
finite-difference scheme with superior stability and accuracy
properties.  I test the algorithm on a simple synthetic example and
discuss its possible applications in residual traveltime computation,
interpolation, and tomography.

\section{THE LINEARIZED EIKONAL EQUATION}

The eikonal equation, describing the traveltime propagation in an
isotropic medium, has the form
\begin{equation}
\left(\nabla \tau\right)^2 = n^2(x,y,z)\;,
\label{eqn:eikeqn}
\end{equation}
where $\tau (x,y,z)$ is the traveltime (eikonal) from the source to
the point with the coordinates $(x, y, z)$, and $n$ is the slowness at
that point (the velocity $v$ equals $1/n$.) In Appendix A, I review a
basic derivation of the eikonal and transport equations. To formulate
a well-posed initial-value problem on equation (\ref{eqn:eikeqn}), it is
sufficient to specify $\tau$ at some closed surface and to choose one
of the two branches of the solution (the wave going from or to the
source.)
\par
Equation (\ref{eqn:eikeqn}) is nonlinear. The nonlinearity is essential for
producing multiple branches of the solution. Multi-valued eikonal
solutions can include different types of waves (direct, reflected,
diffracted, head, etc.) as well as different branches of caustics.  To
linearize equation (\ref{eqn:eikeqn}), we need to assume that an initial
estimate $\tau_0$ of the eikonal $\tau$ is available. The traveltime
$\tau_0$ corresponds to some slowness $n_0$, which can be computed
from equation (\ref{eqn:eikeqn}) as
\begin{equation}
n_0 = \left|\nabla \tau_0\right|\;.
\label{eqn:n0}
\end{equation}
Let us denote the residual traveltime $\tau - \tau_0$ by $\tau_1$ and
the residual slowness $n - n_0$ by $n_1$. With these definitions, we
can rewrite equation (\ref{eqn:eikeqn}) in the form
\begin{equation}
\left(\nabla \tau_0 + \nabla \tau_1 \right)^2 = 
\left(\nabla \tau_0\right)^2 + 
2\,\nabla \tau_0 \cdot  \nabla \tau_1 +
\left(\nabla \tau_1\right)^2 =
(n_0 + n_1)^2  = n_0^2 + 2\,n_0\,n_1 + n_1^2\;,
\label{eqn:eik1}
\end{equation}
or, taking into account equality (\ref{eqn:n0}),
\begin{equation}
2\,\nabla \tau_0 \cdot  \nabla \tau_1 +
\left(\nabla \tau_1\right)^2 = 2\,n_0\,n_1 + n_1^2\;.
\label{eqn:eik2}
\end{equation} 
Neglecting the squared terms, we arrive at the equation
\begin{equation}
\nabla \tau_0 \cdot  \nabla \tau_1 = 
n_0\,n_1\;,
\label{eqn:lineik}
\end{equation} 
which is the linearized version of the eikonal equation (\ref{eqn:eikeqn}).
The accuracy of the linearization depends on the relative ratio of the
slowness perturbation $n_1$ and the true slowness model $n$. Though it
is difficult to give a quantitative estimate, the ratio of 10\% is
generally assumed to be a safe upper bound.
\par
The intimate connection of the linearized eikonal equation
and traveltime tomography is discussed in Appendix B.

\section{ALGORITHM}

Linearization of the eikonal equation suggests the following algorithm
of traveltime computation:
\begin{enumerate}
\item Start with an initial traveltime field $\tau_0$. The initial
traveltime may be the result of a previous computation or (for simple
models) the result of an approximate analytic evaluation.
\item Compute the finite-difference gradient $\nabla \tau_0$ and the
corresponding slowness model $n_0$ with equation
(\ref{eqn:n0}).
\item Compute the slowness perturbation $n_1$ as the difference
between the true slowness model $n$ and $n_0$. Exit the computation if
the perturbation is smaller than the desired accuracy.
\item Solve numerically equation (\ref{eqn:lineik}) for the traveltime
perturbation $\tau_1$.
\item Update the traveltime field $\tau_0$ by adding $\tau_1$ to it.
\item Repeat the loop.
\end{enumerate} 
\par
Equation (\ref{eqn:lineik}) can be solved numerically with a simple explicit
upwind finite-difference method.  

\begin{comment}
For a numerical test of the
algorithm, I chose to solve it by a less efficient but more robust
``brute-force'' implicit method, applying one of the generic linear
solvers. The gradient operator $\nabla$ was computed with centered
finite differences. The implicit method is unconditionally stable. Its
accuracy corresponds to the accuracy of the finite-difference gradient
approximation. I found it helpful to regularize the linear solver with
a smoothing preconditioner. The regularization assures that the
traveltime remains a smooth function of the spatial coordinates.
\par
An important feature of the suggested algorithm is that it does not
require an iterative solver to iterate until the full convergence. A
few iteration steps of the estimation process can be interlaced with
re-linearization in the main loop of the algorithm.
\par
Theoretically, a global convergence of the described procedure cannot
be guaranteed for all cases. However, I observed a stable convergence
in the preliminary numerical tests.
\end{comment}
  
\section{NUMERICAL TEST}
\inputdir{test}

For the first numerical test, I used a model with a smooth anomaly
inside a constant slowness background. The initial traveltime was
computed analytically, using the background slowness. The result of
the computation is shown in Figure \ref{fig:linear}. The computation
involved 5 re-linearization cycles.
% with 10 linear inversion iterations
%in each cycle.

\sideplot{linear}{width=\textwidth}{The traveltime
  contours for a smooth anomaly, computed by the linearized eikonal
  solver. The background slowness is 1~s/km. The maximum anomaly
  slowness is 2.25~s/km. The wave source is in on the top
  plane of the model.}

% The left plot shows a vertical slice. The right
%  plot shows a horizontal slice, taken at 2.5~km depth.}

\sideplot{mihai}{width=\textwidth}{The traveltime
  contours for a smooth anomaly, computed by the exact eikonal solver.
  The input and plotting parameters are the same as in the preceeding
  figure.}
\par
The result shows the expected behavior of the wavefronts. It agrees
with the result of a direct eikonal computation, shown in Figure
\ref{fig:mihai}. The direct computation was done with Mihai
Popovici's TTGES eikonal solver, which has outstanding efficiency and
stability properties. Obviously, more tests are required to evaluate
the comparative performance of the algorithm and the limits of its
practical applicability. The discussion section contains some
speculations about the perspective usage of the linearized algorithm.

\section{Discussion}

Although the first numerical experiments have been too incomplete for
drawing any solid conclusions, it is interesting to discuss the
possible applications of the linearized eikonal.

\begin{description}
\item[Multi-valued traveltimes] Conventional eikonal solvers usually
  force the choice of a particular branch of the multi-valued
  traveltime, most commonly the first-arrival branch. However, in some
  cases other branches may in fact be more useful for imaging or
  velocity estimation \cite[]{gray}. When the linearization assumption
  is correct, the linearized eikonal should follow the branch of the
  initial traveltime. This branch does not have to be the first
  arrival.  It can correspond to any other arrival, such as reflected
  waves or multiple reflections.
  
\item[Spherical Coordinates] Though the eikonal equation itself does
  not favor any particular direction, its solution for the case of a
  point source lands more naturally into a spherical coordinate
  system.  \cite{GEO56-06-08120821},
  \cite{Popovici.sep.70.245}, \cite{SEG-1994-1394}, and
  \cite{schneider} presented upwind finite-difference eikonal
  schemes based on a spherical computational grid. To use the
  linearized equation (\ref{eqn:lineik}) on such a grid, it is necessary to
  rewrite the gradient operator in the spherical coordinates, as
  follows:
  \[
  \nabla \tau = \left\{
    \frac{\partial \tau}{\partial r}\,,\;
    \frac{1}{r}\,\frac{\partial \tau}{\partial \theta}\,,\;
    \frac{1}{r \sin^2 \theta}\,\frac{\partial \tau}{\partial \phi}\right\}\;.
\].

\item[Interpolation] One of the most natural applications for the
  linearized eikonal is interpolation of traveltimes. Interpolating
  regularly gridded input (such as subsampled traveltime tables)
  reduces to \emph{masked} inversion of equation (\ref{eqn:lineik}).
  Interpolating irregular input (such as the result of a ray tracing
  procedure) reduces to \emph{regularized} inversion. In both cases, a
  simpler way of traveltime binning would be required to initiate the
  linearization.
  
\item[Tomography] Tomographic velocity estimation is possible when the
  input traveltime data corresponds to a collection of sources. In
  this case, we can reduce the linearized traveltime inversion to the
  system of equations
  \begin{equation}
    \label{eq:tomo}
    n_0^{(1)} \cdot \nabla \tau_1^{(1)} =
    n_0^{(2)} \cdot \nabla \tau_1^{(2)} =
    \cdots = s_1\;.
  \end{equation}
  Here $\tau_1^{(i)}$ stands for the traveltime from source $i$.
  Equations (\ref{eq:tomo}) are additionally constrained by the known
  values of the traveltime fields at the receiver locations.
  
\item[Amplitudes] The amplitude transport equation, briefly reviewed
  in Appendix A, has the form (\ref{eq:amp}). Introducing the
  logarithmic amplitude $J = - ln (A/A_0)$, where $A_0$ is the
  constant reference, we can rewrite this equation in the form
  \begin{equation}
    \label{eq:log}
    2 \nabla \tau \cdot \nabla J = \Delta \tau\;.
  \end{equation}
  The left-hand side of equation (\ref{eq:log}) has exactly the same
  form as the left-hand side part of the linearized eikonal equation
  (\ref{eqn:lineik}). This suggests reusing the traveltime computation
  scheme for amplitude calculations. The amplitude transport equation
  is linear. However, it explicitly depends on the traveltime.
  Therefore, the amplitude computation needs to be coupled with the
  eikonal solution.
  
\item[Anisotropy] In a recent paper, \cite{tariq} proposed a
  simple eikonal-type equation for seismic imaging in vertically
  transversally-isotropic media. Alkhalifah's equation should be
  suitable for linearization, either in the normal moveout velocity
  $V_{NMO}$ or in the dimensionless anisotropy parameter $\eta$. This
  untested opportunity looks promising because of the validity of the
  weak anisotropy assumption in many regions of the world.

\end{description}

\section{Conclusions}

I have presented a finite-difference method of traveltime
computations, based on the linearized eikonal equation. Preliminary
numerical experiments show that the method is as simple and robust as
can be expected from the theory. The required assumption is that a
reasonable estimate of the traveltime is available prior to
linearization. Such an estimate may result from the computation in a
different velocity model, with a different method (e.g., ray tracing),
or by an analytic evaluation.
\par
In the situations where the underlying assumption is valid, the
linearized approach may allow us
\begin{itemize}
\item to employ unconditionally stable implicit finite-difference
  schemes with an easy control of the numerical stability,
\item to parallelize the essential parts of the algorithm with minimum
  effort,
\item to compute branches of the multi-valued traveltime other than
  the first arrival,
\item to connect traveltime computations with tomographic velocity
  estimation,
\item to couple traveltime and amplitude computations.
\end{itemize}
Future research is necessary to confirm these expectations.

\section{ACKNOWLEDGMENTS}
I thank Biondo Biondi for interesting discussions on the traveltime
computation problem and its applications. The TTGES eikonal solver
that I used for testing was kindly made available to SEP by Mihai
Popovici of 3DGeo.

\bibliographystyle{seg}
\bibliography{SEG,SEP2,eiko}

\append{A SIMPLE derivation of the eikonal and transport equations}

In this Appendix, I remind the reader how the eikonal equation is
derived from the wave equation. The derivation is classic and can be
found in many popular textbooks. See, for example, \cite[]{cherv}.
\par
Starting from the wave equation,
\begin{equation}
  {\frac{\partial^2 P}{\partial x^2}} +
  {\frac{\partial^2 P}{\partial y^2}} +
  {\frac{\partial^2 P}{\partial z^2}} =
  {n^2 (x,y,z)\,\frac{\partial^2 P}{\partial t^2}}\;,
  \label{eq:wave}
\end{equation}
we introduce a trial solution of the form
\begin{equation}
  \label{eq:trial}
  P (x,y,t) = A (x,y,z) f (t - \tau (x,y,z))\;,
\end{equation}
where $\tau$ is the eikonal, and $A$ is the wave amplitude. The
waveform function $f$ is assumed to be a high frequency
(discontinuous) signal. Substituting solution (\ref{eq:trial}) into
equation (\ref{eq:wave}), we arrive at the constraint
\begin{equation}
  \label{eq:ray}
  \Delta A \, f - 2 \nabla A \cdot \nabla \tau f' -
  A \Delta \tau f '  + A \left(\nabla \tau\right)^2 f'' =
  n^2 A f''\;.
\end{equation}
Here $\Delta \equiv \nabla^2$ denotes the Laplacian operator.
Equation (\ref{eq:ray}) is as exact as the initial wave equation
(\ref{eq:wave}) and generally difficult to satisfy. However, we can
try to satisfy it asymptotically, considering each of the
high-frequency asymptotic components separately. The leading-order
component corresponds to the second derivative of the wavelet $f''$.
Isolating this component, we find that it is satisfied if and only if
the traveltime function $\tau (x,y,z)$ satisfies the eikonal equation
(\ref{eqn:eikeqn}).
\par
The next asymptotic order corresponds to the first derivative $f'$. It
leads to the \emph{amplitude transport equation}
\begin{equation}
  \label{eq:amp}
2 \nabla A \cdot \nabla \tau + A \Delta \tau = 0\;.
\end{equation}
The amplitude, defined by equation (\ref{eq:amp}), is often referred
to as the amplitude of the zero-order term in the ray series. A series
expansion of the function $f$ in high-frequency asymptotic components
produces recursive differential equations for the terms of higher
order. In practice, equation (\ref{eq:amp}) is sufficiently accurate
for describing the major amplitude trends in most of the cases. It
fails, however, in some special cases, such as caustics and diffraction.

\append{CONNECTION OF THE LINEARIZED EIKONAL EQUATION AND 
TRAVELTIME TOMOGRAPHY}

The eikonal equation (\ref{eqn:eikeqn}) can be rewritten in the form
\begin{equation}
\mathbf{n}  \cdot \nabla \tau = n \;,
\label{eqn:eikvec}
\end{equation} 
where $\mathbf{n}$ is the unit vector, pointing in the traveltime
gradient direction. The integral solution of equation (\ref{eqn:eikvec})
takes the form
\begin{equation}
\tau = \int_{\Gamma (\mathbf{n})} n dl\;,
\label{eqn:eikint}
\end{equation} 
which states that \emph{the traveltime $\tau$ can be computed by
integrating the slowness $n$ along the ray $\Gamma (\mathbf{n})$,
tangent at every point to the gradient direction $\mathbf{n}$}.
\par
Similarly, we can rewrite the linearized eikonal equation (\ref{eqn:lineik})
in the form
\begin{equation}
\mathbf{n_0}  \cdot \nabla \tau_1 = n_1 \;,
\label{eqn:linvec}
\end{equation}
where $\mathbf{n_0}$ is the unit vector, pointing in gradient
direction for the initial traveltime $\tau_0$.  The integral solution
of equation (\ref{eqn:linvec}) takes the form
\begin{equation}
\tau_1 = \int_{\Gamma (\mathbf{n_0})} n_1 dl\;,
\label{eqn:eikint}
\end{equation} 
which states that \emph{the traveltime perturbation $\tau_1$ can be
  computed by integrating the slowness perturbation $n_1$ along the
  ray $\Gamma (\mathbf{n_0})$, defined by the initial slowness model
  $n_0$}.  This is exactly the basic principle of traveltime
tomography.
\par
I have borrowed this proof from \cite{nsk}, who used linearization
of the eikonal equation as the theoretical basis for traveltime inversion.



% Preamble

\title{Iterative least-square inversion for amplitude balancing}
%\keywords{amplitudes, graphics, two-dimensional, modeling, least squares }
\email{arnaud@sep.Stanford.EDU, harlan@sep.Stanford.EDU}
\author{Arnaud Berlioux and William S. Harlan} 

\righthead{Amplitude balancing}
\lefthead{Berlioux \& Harlan}
\footer{SEP--89}
\maketitle

% End of Preamble

\begin{abstract}
	Variations in source strength and receiver amplitude can introduce
	a bias in the final AVO analysis of prestack seismic reflection data.
	In this paper we tackle the problem of the amplitude balancing of
	the seismic traces from a marine survey. We start with a 2-D energy
	map from which the global trend has been removed. In order to balance 
	this amplitude map, we first invert for the correction coefficients 
	using an iterative least-square algorithm. The coefficients are 
	calculated for each shot position along the survey line, each receiver 
	position in the recording cable, and each offset. Using these 
	coefficients, we then correct the original amplitude map for amplitude 
	variations in the shot, receiver, and offset directions.
\end{abstract}


\section{INTRODUCTION}
%%%%%%%%%%%%%%%%%%%%
\par
In 1994, Mobil provided SEP with a marine dataset on which we were to
perform an amplitude variation with offset (AVO)
analysis. However, \cite{Berlioux.sep.80.349} showed that the
amplitude of the traces in the survey present anomalies that need to
be preprocessed prior to the AVO analysis.

\par
Fluctuations of the source strength and the receiver amplitudes as well
as near-surface irregularities can create amplitude anomalies. It is
therefore often necessary to balance the amplitude of each trace in the survey.

\par
To do so we determine the source, receiver, and offset amplitude balancing
coefficients by using an iterative least-square algorithm. We then apply these
scaling factors to the original 2-D amplitude map to cancel the effect of the 
impulse response of the defective sources and receivers, and to compensate for 
irregularities of the sea bottom.

\inputdir{ampl}

\section{THE PROBLEM OF VARIATIONS OF AMPLITUDE DURING A SURVEY}
%%%%%%%%%%%%%%%%%%%%%%%%%%%%%%%%%%%%%%%%%%%%%%%%%%%%%%%%%%%%%%
\par
In a seismic survey, the amplitude recorded at each receiver for each shot
depends on the geology of the earth, and on the seismic source and receiver
impulse responses. Geophysicists and geologists are interested in the 
variations of the impulse response of the earth. The impulse responses
of the receivers and the source may vary, causing anomalous fluctuations
of the amplitude recorded during the experiment. It is therefore necessary 
to correct for such fluctuations, when observed, in order to restore the 
earth component, which is the valuable information.

\par
Figure \ref{fig:amplitude} shows an amplitude plot in source and offset
coordinates for a 2-D seismic survey provided by Mobil in 1994. For each 
shot and each offset position (i.e., for each trace of the survey)
the value of the amplitude has been calculated by taking the root mean
square of the trace amplitudes along the time axis. The global trend 
(low-frequency component of the earth) of the amplitude surface has been 
estimated by least-square fitting, and removed from the original surface
to leave a globally flat 2-D amplitude map \cite[]{Berlioux.sep.80.349}.

\plot{amplitude}{width=5.5in,height=2.25in}{2-D amplitude plot after 
	removal of the global trend and normalization by the
	root-mean-square value.}

\par
Figure \ref{fig:amplitude} shows the result after normalization by the 
root-mean-square value of the amplitude. In this figure the horizontal
stripes correspond to offsets where the hydrophone had an impulse response
that was weaker (darker stripe) or stronger (brighter stripe) than the
average response of the other receivers. Likewise, the vertical
stripes indicate where the source had an impulse response that varied from 
the average. Less obvious, though noticeable, are two other categories of 
stripes dipping to the left. One is quite visible at the bottom of 
the plot around the offset -0.5 km, dipping at approximately 10 degrees. 
An example of the second type of stripe, which dips 20 degrees to the left, 
is visible at the source position 16 km. Based on the amplitude plot in 
Figure \ref{fig:amplitude}, we have built a model of offset-, source-, midpoint-, 
and receiver-consistent stripes (Figure \ref{fig:model}). Comparing both figures 
we can identify the first category of dipping stripes as being 
midpoint-consistent, whereas the less steep stripes are receiver-consistent. 
The stripes in the receiver directions are broader than the others and 
therefore not as visible in the source and receiver coordinate system (Figure 
\ref{fig:sramplitude}). The stripes in the offset direction follow the descending 
diagonal in the transformed coordinate system.

\plot{model}{width=5.5in,height=2.25in}{Model of offset-, source-, 
	midpoint-, and receiver-consistent stripes, based on the 2-D amplitude 
	plot in Figure 1. The solid dipping lines correspond to the
	receiver-consistent stripes, and the dotted lines represent the
	midpoint-consistent stripes.}

\activeplot{sramplitude}{width=4.5in,height=7.5in}{}{Portion of the amplitude
	map in Figure 1 displayed in the source and receiver space. The 
	stripes along the descending diagonal follow the offset direction. 
	Midpoint and receiver stripes are less visible in this coordinate
	system.}

\par
\cite{Berlioux.sep.80.349} and \cite{Lumley.sep.84.125} proposed a method to estimate the source and 
offset correction coefficients in order to balance the amplitude of
each trace in the survey. This method, based on a simple amplitude
model, produces good results but does not take into account the
receiver-consistent stripes still visible after correction. Because
water has a substantially lower velocity than the underlying
sediments, the waves travel nearly vertically in the water. The
variations of the amplitude caused by the receiver can therefore be
associated with near-surface anomalies or irregular sea-bottom
topography that affects the receiver recording vertically above it. In
the next section, we use a more complex amplitude model, which allows
for these variations, and propose an iterative method to estimate the
coefficients in order to later correct the amplitude map and balance
each trace of the survey.


\section{AN ITERATIVE LEAST-SQUARE INVERSION SCHEME}
%%%%%%%%%%%%%%%%%%%%%%%%%%%%%%%%%%%%%%%%%%%%%%%%%%
\par
Our revised amplitude model is

\begin{equation}
	a^{total}_{s,h} \; = \; a_{s} \: a_{h} \: a_{y \, = \, s \, + \, h/2} 
		\: a_{r \, = \, s \, + \, h} \: a_{earth}
\end{equation}

\noindent
where $a_{earth}$ is the earth low-frequency component of the amplitude 
function; and $a_{s}$, $a_{h}$, $a_{y}$, and $a_{r}$ are the components of 
the amplitude caused by the source ($s$), the full offset ($h$), the midpoint 
($y$), and the receiver ($r$) variations, respectively.

\par
We now need to invert for the amplitude correction coefficients in order to 
remove the stripes in Figure \ref{fig:amplitude}. To do so, we use the following 
quadratic objective function

\begin{equation}
	\varphi \; = \; \; \parallel \, d \, \left( s, h \right) \: - \:
		a_{s} \: a_{h} \: a_{y} \: a_{r} \, \parallel^{2}
\end{equation}

\noindent
where we assume that the data $d$ can be modeled as the product of the source, 
offset, midpoint, and receiver. Normalization allows us to assume 
$a_{earth} \approx 1$.

\par
To estimate the coefficients $a_{s}$, $a_{h}$, $a_{y}$, and $a_{r}$ we choose 
the Gauss-Seidel algorithm which is an iterative least-square inversion 
scheme [see \cite{Stark.1970}]. We also assume that the coefficients
for which we are solving the objective function $\varphi$ are independent. 
Therefore, we can get an estimation of one type of coefficient ($a_{s}$, 
$a_{h}$, $a_{y}$, or $a_{r}$) while keeping the value of the other fixed.

\par
Under these assumptions, after minizing the objective function with respect 
to the source coefficients, we derive the following expression, giving the 
value of the coefficients at iteration $k$ as a function of the other 
coefficients at the preceding iteration:

\begin{equation}
	a_{s}^{(k)} \: = \: \frac{\sum_{h} \; d \; a_{h}^{(k \, - \, 1)} \:
		a_{y}^{(k \, - \, 1)} \: a_{r}^{(k \, - \, 1)}}
		{\sum_{h} \; \left[ a_{h}^{(k \, - \, 1)} \: 
		a_{y}^{(k \, - \, 1)} \: a_{r}^{(k \, - \, 1)} \right]^2}
\end{equation}

\noindent
We obtain a similar expression for the other three coefficients, where each
is expressed as a function of the data and all the other coefficients.

\par
Figures \ref{fig:source}, \ref{fig:offset}, \ref{fig:midpoint},
and \ref{fig:receiver} show the result of the inversion when the
algorithm has converged, which required 10 iterations. Comparing the
source and offset correction coefficient curves
(Figures \ref{fig:source} and \ref{fig:offset}) with those obtained
by \cite{Berlioux.sep.80.349}, we can see that the global shape of the
curves is the same. The curves in Figures \ref{fig:source} through
\ref{fig:receiver} show identical features: high-frequency variations of the 
coefficient value around a globally constant value.

\activeplot{source}{width=5in,height=1.5in}{}{Estimated source coefficients.}

\activeplot{offset}{width=5in,height=1.5in}{}{Estimated offset coefficients.}

\activeplot{midpoint}{width=5in,height=1.5in}{}{Estimated midpoint 
	coefficients.}

\activeplot{receiver}{width=5in,height=1.5in}{}{Estimated receiver 
	coefficients.}

\par
The next section shows how we use these estimated correction coefficients to 
cancel the stripes in the original 2-D amplitude map, and thus balance the 
traces in the survey.


\section{AMPLITUDE BALANCING}
%%%%%%%%%%%%%%%%%%%%%%%%%%%
\par
We use the coefficients calculated for the source, offset, and receiver
directions to remove the stripes from the plot in Figure \ref{fig:amplitude}.
Figure \ref{fig:sostripes} shows the 2-D synthetic amplitude map modeled using
the source and offset coefficient curves only. The similarity between the
stripes in the amplitude plots in Figures \ref{fig:amplitude} and \ref{fig:sostripes} 
is quite strong.

\activeplot{sostripes}{width=5.5in,height=2.25in}{}{Synthetic 2-D amplitude 
	map modeled using the source and offset coefficients.}

\par
We divide the original 2-D amplitude function by the source and offset 
correction coefficients to obtain the amplitude map in Figure \ref{fig:sobalcd}.
In this plot most of the source- and offset-related anomalous amplitude 
stripes seem to have disappeared, revealing the grey background (the earth 
impulse response and noise) and other stripes dipping to the left.

\activeplot{sobalcd}{width=5.5in,height=2.25in}{}{2-D amplitude map corrected
	for variations in the source and offset directions using the 
	coefficient curves in Figures 4 and 5.}

\par
Comparing Figure \ref{fig:sobalcd} and the modeled stripes in Figure \ref{fig:model}, 
we can associate the now more apparent dipping stripes with the receiver- and 
midpoint-consistent stripes. In Figure \ref{fig:sobalcd}, the fine stripes with 
a steep dip particularly visible at the bottom part of the plot can be
regarded as midpoint-consistent, whereas the less steep stripes are 
receiver-related.

\par
Figure \ref{fig:srsobalcd} represents a portion of the amplitude map in Figure 
\ref{fig:sobalcd} in the source and receiver coordinate system. The plot shows
a bright spot smeared along the offset direction (the descending diagonal). 
Two very broad darker stripes orthogonal to the receiver axis are visible 
around the receiver positions at 9.5 km and 12 km. In this coordinate system 
we can identify these two broad stripes as receiver-consistent. This 
observation is confirmed by Figure \ref{fig:receiver}, in which the receiver 
correction coefficient curve shows two local minima at the corresponding 
receiver location. The same stripes are also noticeable in the center of the 
plot in Figure \ref{fig:sobalcd}, though less obvious. They are also clearly 
visible in Figure \ref{fig:rvstripes}, which is a synthetic amplitude map modeled 
from the receiver coefficients only.

\activeplot{srsobalcd}{width=4.5in,height=7.5in}{}{Portion of the amplitude
	map in Figure 9 displayed in the source and receiver space.}

\activeplot{rvstripes}{width=5.5in,height=2.25in}{}{Synthetic 2-D amplitude 
	map modeled using the receiver coefficients.}

\par
Figure \ref{fig:balanced} shows the amplitude map corrected for variations
in the source, offset, and receiver directions. It is the result of the
division of the plot in Figure \ref{fig:sobalcd} by the estimated receiver
coefficients in Figure \ref{fig:receiver}. Figure \ref{fig:srbalanced} represents
the portion of the corrected amplitude map in Figure \ref{fig:balanced} in the
source and receiver coordinate system. After the correction has been applied,
the broad horizontal stripes around the receiver positions at 9.5 km and 
12 km have disappeared.

\activeplot{balanced}{width=5.5in,height=2.25in}{}{2-D amplitude map corrected
	for variations in the source, offset, and receiver directions using 
	the coefficient curves in Figures 4, 5, and 7.}

\activeplot{srbalanced}{width=4.5in,height=7.5in}{}{Portion of the corrected 
	amplitude map in Figure 12 displayed in the source and receiver space.}

\par
Although midpoint-consistent factors are not used to correct the data
amplitudes, we invert for the factors simultaneously to separate the
different components more fully.
The resulting amplitude map, corrected for anomalous variations in the source, 
offset, and receiver directions, shows the contribution of the earth and 
midpoint components to the amplitude of the recorded traces. Some variations 
may still be present, but not dominant.


\section{CONCLUSIONS}
%%%%%%%%%%%%%%%%%%%
\par
Using a more complex amplitude model than that described
by \cite{Berlioux.sep.80.349}, we have estimated the correction
coefficients to balance the trace amplitudes of the Mobil AVO
dataset. For this model, we have assumed that the coefficients we are
inverting for are independent. We have corrected the original 2-D
amplitude map with the source, offset, and receiver coefficients
determined by an iterative least-square inversion scheme. This new
method gives a more accurate result, taking into account variations of
the amplitude caused by the receivers.  The stripes in the source,
offset, and receiver directions have been successfully removed to
reveal the geological grey background of the amplitude plot. We
believe the same algorithm can be used for other data sets that
present a similar amplitude balancing problem.

%\newpage

%\bibliographystyle{sep}\bibliography{SEP,../arnaud/numeth}
\bibliographystyle{seg}
\bibliography{SEP2,numeth}

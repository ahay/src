\section{Conclusions}
We extend one-way wavefield extrapolation to 
Riemannian spaces which are described by
non-orthogonal curvilinear coordinate systems.
We choose semi-orthogonal Riemannian coordinates
that include, but are not limited to,
ray coordinate systems.
\par
We define an acoustic wave-equation for 
semi-orthogonal Riemannian coordinates, from which we derive
a one-way wavefield extrapolation equation.
We use ray coordinates initiated either from a point
source, or from an incident plane wave at the surface.
Many other types of coordinates are acceptable, as long
as they fulfill the semi-orthogonal condition of our
acoustic wave equation.
\par
Since wavefield propagation is mostly coincident with the
extrapolation direction, we can use inexpensive
$15^\circ$ finite-difference or mixed-domain 
extrapolators to achieve high-angle accuracy.
If the ray coordinate system overturns, our 
method can be used to image overturning waves 
with one-way wavefield extrapolation.
\par
Riemannian coordinates are better suited for
wavefield extrapolation, because they do not restrict 
wave propagation to the preferential vertical direction but allow 
numerical wave extrapolation to follow the direction of the natural 
wave  propagation. 
Coordinate system triplications pose challenges that can
be solved numerically, but which are better avoided.

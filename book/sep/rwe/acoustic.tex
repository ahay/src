\section{Acoustic wave-equation in 3-D Riemannian spaces}
The acoustic propagation of a monochromatic wave is
governed by the Helmholtz equation:
\beq \label{eqn:helm}
  \dd \W = - \frac{\ww^2}{\vv^2} \W \;,
\eeq
where $\ww$ is temporal frequency, 
$\vv$ is the spatially variable wave propagation velocity, and 
$\W$ represents a wavefield. 
\par
The Laplacian operator acting on a scalar function 
$\W$ in an arbitrary 
Riemannian space with coordinates 
$\qvec=\{\qone,\qtwo,\qthr\}$ takes the form
\beq \label{eqn:laplac}
  \Delta \W = \sum_{i=1}^{3}\frac{1}{\sqrt{|\mathbf{g}|}}\,
    \frac{\partial}{\partial \xi_i}\,
    \left(\sum_{j=1}^{3}\,g^{ij}\,\sqrt{|\mathbf{g}|}\,
      \frac{\partial \W}{\partial \xi_j}\right),
\eeq
where $g^{ij}$ is a component of the associated metric tensor, 
and $|\mathbf{g}|$ is its determinant \cite[]{tensor}.
The differential geometry of any coordinate system 
\sout{is fully represented by}
\uline{can be described with the help of}
the metric tensor $g^{ij}$.
\par
The expression simplifies if one of the coordinates,
e.g. the coordinate of one-way wave extrapolation $\qone$,
is orthogonal to the other coordinates $(\qtwo,\qthr)$. 
The metric tensor reduces to
\beq \label{eqn:metric}
  \left[g_{ij}\right] = \left[\begin{array}{ccc}
      \EE & \FF & 0      \\
      \FF & \GG & 0      \\
      0   &   0 & \AA^2
    \end{array}\right]\;,
\eeq
where $\EE$, $\FF$, $\GG$, and $\AA$ 
\sout{are differential forms that} 
can be found
from mapping Cartesian coordinates $\xvec=(\xone,\xtwo,\xthr)$ 
to general Riemannian coordinates
$\qvec=\{\qone,\qtwo,\qthr\}$, as follows:
\beqa
\label{eqn:emap} \EE   & = & \sum_k \done{x_k}{\qone} \done{x_k}{\qone}  \;, \nonumber \\
\label{eqn:fmap} \FF   & = & \sum_k \done{x_k}{\qone} \done{x_k}{\qtwo}  \;, \nonumber \\
\label{eqn:gmap} \GG   & = & \sum_k \done{x_k}{\qtwo} \done{x_k}{\qtwo}  \;, \nonumber \\
\label{eqn:gmap} \AA^2 & = & \sum_k \done{x_k}{\qthr} \done{x_k}{\qthr}  \;.
\eeqa
The associated metric tensor 
$\left[g^{ij}\right] = \left[g_{ij}\right]^{-1}$
has the matrix
\beq \label{eqn:ametric}
  \left[g^{ij}\right] =
  \left[\begin{array}{ccc}
      +\GG/J^2 & -\FF/J^2 & 0 \\
      -\FF/J^2 & +\EE/J^2 & 0 \\
             0 &        0 & 1/\AA^2
    \end{array}\right]\;,
\eeq
where $\JJ^2 = \EE\,\GG-\FF^2$. 
The metric determinant takes the form
\beq \label{eqn:det}
  |\mathbf{g}| = \AA^2\,\JJ^2\;.
\eeq
\par
Substituting \reqs{ametric} and \reqo{det} into
\reqo{laplac}, 
and making the notations
$\xi_1=\qx$, $\xi_2=\qy$, and $\xi_3=\qz$, 
with $\qz$ orthogonal to both $\qx$ and $\qy$,
we obtain the Helmholtz wave \req{helm}
for propagating waves in a {3-D} semi-orthogonal Riemannian space:
%%%%%%%%%%%%%%%%%%%%%%%%%%%%%%%%%%%%%%%%%%%%%%%%%%%%%%%%%%%%%%%%%%%
\beqa \label{eqn:weqrc.3d}
\frac{1}{\AA\,\JJ}
\lb
\eone{\lp \frac{\JJ}{\AA} \done{\W}{\qz}\rp}{\qz}
+
\eone{\lp 
\GG\frac{\AA}{\JJ} \done{\W}{\qx} -
\FF\frac{\AA}{\JJ} \done{\W}{\qy}
\rp}{\qx}
%%\right .
%%\nonumber \\
%%\left .
+
\eone{\lp 
\EE\frac{\AA}{\JJ} \done{\W}{\qy} -
\FF\frac{\AA}{\JJ} \done{\W}{\qx}
\rp}{\qy}
\rb
= - \frac{\ww^2}{\vv^2} \W \;.
\eeqa
%%%%%%%%%%%%%%%%%%%%%%%%%%%%%%%%%%%%%%%%%%%%%%%%%%%%%%%%%%%%%%%%%%%
In \req{weqrc.3d}, $\vv \lp \qx,\qy,\qz \rp$ 
is the wave propagation velocity mapped to Riemannian coordinates.
\par
For the special case of two dimensional spaces
($\FF=0$ and $\GG=1$),
the Helmholtz wave equation reduces to the simpler form:
\beq \label{eqn:weqrc.2d}
\frac{1}{\AA\JJ}
\lb \eone{\lp \frac{\JJ}{\AA} \done{\W}{\qz} \rp}{\qz} + 
    \eone{\lp \frac{\AA}{\JJ} \done{\W}{\qx} \rp}{\qx} \rb 
= - \frac{\ww^2}{\vv^2} \W \;,
\eeq
which corresponds to a curvilinear orthogonal coordinate system.
\par
Particular examples of coordinate systems for one-way wave propagation are:
\begin{description}
\item[Cartesian (propagation in depth):] 
  $x_1=\qx$, $x_2=\qy$, $x_3=\qz$,
  \begin{eqnarray*}
    \EE & = & \GG \quad = \quad \AA \quad = \quad \JJ \quad = \quad 1\;, \\
    \FF & = & 0\;.
  \end{eqnarray*}
\item[Cylindrical (propagation in radius):] 
  $x_1=\qz\,\cos{\qx}$,
  $x_2=\qz\,\sin{\qx}$,
  $x_3=\qy$,
  \begin{eqnarray*}
    \EE & = & \JJ \quad = \quad \qz^2\;, \\
    \GG & = & \AA \quad = \quad 1\;, \\
    \FF & = & 0\;.
  \end{eqnarray*}
\item[Spherical (propagation in radius):] 
  $x_1=\qz\,\sin{\qx}\,\cos{\qy}$,
  $x_2=\qz\,\sin{\qx}\,\sin{\qy}$,
  $x_3=\qz\,\cos{\qx}$,
  \begin{eqnarray*}
    \EE & = & \qz^2\;, \\
    \GG & = & \qz^2\,\sin^2{\qx}\;, \\
    \AA & = & 1\;, \\
    \JJ & = & \qz^2\,\sin{\qx}\;, \\
    \FF & = & 0\;.
  \end{eqnarray*}
\item[Ray family (propagation along rays):] $\qx$ and $\qy$ represent
  parameters defining a particular ray in the family (i.e. the ray take-off
  angles), $\JJ$ is the geometrical spreading factor, related to the
  cross-sectional area of the ray tube \cite[]{cerveny}. The coefficients 
  $\EE$, $\FF$, $\GG$, and $\JJ$ are easily computed by 
  finite-difference approximations
  with the Huygens wavefront tracing technique \cite[]{GEO66-03-08830889}. If
  the propagation parameter $\qz$ is taken to be time along the ray, then
  $\AA$ equals the propagation velocity~$\vv$.
\end{description}


\title{Maximum energy traveltimes calculated in the seismic frequency
  band}

\author{Dave Nichols}

\lefthead{Dave Nichols}
\righthead{Maximum energy traveltimes}
\footer{SEP--80}

\begin{abstract}
Prestack Kirchhoff migration using first arrival traveltimes has been
shown to fail in areas of complex structure. I propose a new method
for calculating traveltimes that estimates the traveltime of the
maximum energy arrival, rather than the first arrival. 
The method estimates a traveltime that is valid in the seismic
frequency band, not the usual high frequency approximation. Instead
of solving the eikonal equation for the traveltime, I solve the
Helmholtz equation to estimate the wavefield for a few frequencies. I
then perform a parametric fit to the wavefield to estimate a
traveltime, amplitude, and phase. 
The images created by using these parameters are shown to be superior
to those created by using first arrival traveltimes, or those created
using maximum amplitude traveltimes calculated by paraxial ray
tracing.
\end{abstract}

\section{Introduction}

Several authors have noted problems when first-arriving traveltimes
are used in prestack Kirchhoff migration. The method appears to fail
when complex velocity models are used. A good example of this is the
failure of first arrival traveltimes to image the Marmousi dataset
\cite{GEO58.04.564}. This dataset was created specifically to test
prestack velocity analysis and imaging algorithms \cite{marm-exp}.
When the true velocity model was released, it became clear that the
dataset could be imaged successfully with algorithms that used
recursive extrapolation of the full wavefield but it was not well
imaged by non-recursive Kirchhoff migration algorithms.There are two
possible reasons for this:
\begin{enumerate}
\item Most Kirchhoff algorithms use first arriving traveltimes to approximate
the full Green's function. The first arrivals may contain
little energy. Imaging using these traveltimes does not
coherently stack the most important parts of the wavefield.
\item The traveltimes are usually calculated in the high frequency limit.
If the medium is dispersive these traveltimes will not be a good
approximation to the traveltimes of the seismic wavefield.
\end{enumerate}

In this paper I propose a method that addresses both of these
limitations. As with most methods, the Green's functions are
approximated by a single event model. The model is parameterized by a
traveltime, amplitude and phase at each point. In my method the
traveltime chosen is the traveltime of the maximum energy arrival, not
the first arrival. This is the best single event approximation to the
full Green's function (in the $L_2$ norm). The traveltime is not
calculated from a solution to the eikonal equation. Instead the
parameters are estimated from solutions to the Helmholtz equations at
a few frequencies in the seismic frequency band. This ensures that the
traveltimes chosen are representative of traveltimes for waves in that
frequency band.

\section{BAND-LIMITED GREEN'S FUNCTIONS}

The most common model for a Green's function is one that parameterizes
the Green's function in terms of ``arrival time'' and ``amplitude''.
This parameterization has the advantage that these quantities can be
used directly in time domain Kirchhoff migration or modeling schemes.
The Green's function can be characterized by one or more ``events''
that arrive at each location. Parameterization by a traveltime implies
that the phase is a linear function of frequency for each event.

The justification for this event based model can be seen by looking at
any seismic section. The data is not a random mish-mash of
unrelated amplitudes, it has the appearance of a set of distinct
events arriving at different times. The fact that distinct events are
visible means that many frequencies are arriving at the same (or
nearly the same) time and are constructively combining to produce a
band-limited event. This implies that the phases of the different
frequencies {\em for each event} are not random; they must follow an
approximately linear trend as a function of frequency. 

A simple way of estimating a traveltime that is valid in the seismic
frequency band is to compute the full wavefield and then pick the
maximum energy arrival at each location. Indeed, this method has been
used for the imaging condition in shot-profile migration
\cite{GEO56.03.03780381}. Unfortunately this method is very expensive, if the
modeling is performed in the frequency domain, it requires a
finite-difference solution to the wave equation for every frequency in
the data. In contrast, a fast, explicit, eikonal solver
\cite{GEO56.06.08120821} requires only one finite-difference
calculation. The question that I attempt to answer is ``How few
frequencies can we compute and still recover the correct traveltime?''

\subsection{Single event models}

The simplest models are parameterized by one event. The simplicity of
this model is very appealing but it is only valid in very smooth
velocity fields. In a complex velocity field there are multiple paths
from the source to one subsurface location, and thus multiple events.
I will start by considering single event models and then progress to
multiple event models.

When there is only one un-dispersed arrival at each location, the
Green's function from source location, $\bf{s}$, to subsurface
location, $\bf{x}$, can be represented in terms of the amplitude,
$A({\bf s}, {\bf x} )$, traveltime, $\tau({\bf s},{\bf x})$, and
phase, $\phi_0(\bf{s},{\bf x})$, of that arrival. I assume here that
the arrival is an impulsive event with a constant phase shift. 

$$
G({\bf s},{\bf x},\omega ) = A({\bf s},{\bf x})\,e^{i \phi_0({\bf s},{\bf x}) }\,e^{i\omega \tau({\bf s},{\bf x})} \ .
$$

The total phase at any frequency is $\omega \tau({\bf s},{\bf x}) +
\phi_0({\bf s},{\bf x})$. The slope of the total phase gives the
traveltime and the intercept at zero frequency gives the constant
phase shift. If the Green's function fits this model then linear
interpolation of unwrapped phase will exactly recover the phase for
all frequencies. The amplitude is constant for all frequencies.

If, for the given velocity field, we know that there is only one
non-dispersed arrival, we need only extrapolate two frequencies. The
amplitude and unwrapped phase at each location provide enough
information to completely specify the Green's function at that
location.
From the amplitude, $A({\bf s},{\bf x},\omega)$, and unwrapped phase,
$\zeta({\bf s},{\bf x},\omega)$, at two frequencies $\omega_1$ and
$\omega_2$ we have:

The average amplitude,
$$
A({\bf s},{\bf x}) =  ( A({\bf s},{\bf x},\omega_1) + A({\bf s},{\bf x},\omega_2) )/2.
$$

The slope of the unwrapped phase,
$$
\tau({\bf s},{\bf x}) = ( \zeta({\bf s},{\bf x},\omega_2) - \zeta({\bf s},{\bf x},\omega_1) ) /( \omega_2 - \omega_1 ).
$$

The unwrapped phase intercept at zero frequency,
$$
 \phi_0({\bf s},{\bf x})  =  \zeta({\bf s},{\bf x},\omega_1) - \omega_1\tau({\bf s},{\bf x})
$$

If it can further be assumed that all the arrivals are zero phase then
only one frequency is needed, as we have the implied condition that
the phase at zero frequency is zero.
$$
A({\bf s},{\bf x}) =  A({\bf s},{\bf x},\omega_1)
$$
$$
\tau({\bf s},{\bf x}) =  \zeta({\bf s},{\bf x},\omega_1) /\omega_1 .
$$

Implicit in many asymptotic schemes is the assumption that total phase
is a linear function of frequency for all frequencies from zero to
very high frequencies. If this is not true the asymptotic methods may
give solutions that are inappropriate for the seismic bandwidth. In
contrast the simple interpolation of two frequencies in the seismic
frequency band only assumes that phase is a linear function over that
band.

\subsection{Multiple events}
\inputdir{XFig}

When there are multiple arrivals at a single location the phase is
no longer a linear, or even a smoothly varying, function of frequency.
Consider the case of two arrivals with amplitudes $A_1$, $A_2$ and
traveltimes $\tau_1$ and $\tau_2$. The wavefield is a linear superposition
of the two arrivals.
$$
P(\omega) = A_1 e^{i\omega\tau_1} + A_2 e^{i\omega\tau_2}
$$
However the phase of the combined wavefield is {\em not} a linear
superposition of the two phases.
\begin{eqnarray*}
P(\omega) & = & A_1 \cos(\omega\tau_1)+ A_2 \cos(\omega\tau_2) + i( A_1 \sin(\omega\tau_1) + A_2 \sin(\omega\tau_2)) \cr
\phi(\omega) & = & \arctan \left( \frac{ A_1 \sin(\omega\tau_1) + A_2 \sin(\omega\tau_2)}{  A_1 \cos(\omega\tau_1)+ A_2 \cos(\omega\tau_2) } \right)
\end{eqnarray*}
Figure~\ref{fig:twophase} shows the unwrapped phase for the combination of
two events, one of amplitude 1.0 at 40ms and one of amplitude 0.9 at
80ms. While the phase curve clearly has an overall linear trend it
would not be well approximated by linear interpolation. Indeed it
would be difficult to use any low order polynomial fit to a sparse
selection of points on the curve.

\plot{twophase}{height=3in}{Time domain plot and unwrapped phase curve
  for the superposition of two discrete events.}

\subsection{A non-linear problem}

To fit an {\it n}-event model to the calculated mono-frequency Green's functions, we
must find amplitudes, $A_i$, phases, $\phi_i$, and traveltimes, $\tau_i$,
such that the calculated Green's function, $P(\omega)$, is predicted correctly
for all modeled frequencies.
$$
 \sum_{i=1}^n A_i e^{i\phi_i}e^{ i \omega \tau_i } = P( \omega )
$$
An obvious way to solve this problem is to minimize a norm of the difference
between the predicted and observed data:
$$
\min_{A,\phi,\tau} ||  \sum_{i=1}^n A_i e^{i\phi_i}e^{ i \omega \tau_i } - P( \omega ) ||
$$ 
Any of a large number of norms could be chosen but the $L_2$ norm
is the most commonly used.  This is a non-linear problem and its
solution may be prone to problems associated with multiple minima, slow
convergence, etc.

A different approach to the problem can be taken by noticing that it
is the dual of a much more familiar problem. In geophysics, we often
have a sampled time series and we wish to estimate a sparse, spiky,
frequency spectrum.  In this case we have a sampled frequency series
and we wish to estimate a sparse, spiky, time-domain representation.
A large number of existing algorithms could be adapted for this task.
Many parametric spectral analysis methods are available, some have
been used in seismic signal deconvolution
\cite{Burg.sepphd.6,DEC00.00.00000482}, others have been used for high
resolution spectral analysis in other fields
\cite{orthog,MUSIC,pisarenk}. I chose to first try a very simple
method based on the Fourier transform. This scheme was so successful
that I did not experiment with the more complex methods. It is
possible that the more complex methods can accurately estimate the
traveltimes from Green's functions at fewer frequencies. This would be
important if the cost of calculating one frequency is large. 

\subsection{Event identification in the time domain }

One simple way to identify the events is to
inverse Fourier transform the data back to the time domain.
$$
P(t) = \int_{-\infty}^{\infty} P(\omega) e^{- i \omega t} d\omega
$$
Since the Green's function is not calculated for all frequencies,
this integral is replaced by a discrete form.
The Green's functions is calculated for the set of
frequencies, $\omega_k = k\, \delta\omega\ ;\ kl \leq k \leq kh $. The
integral then becomes the sum
$$
P(t) = \sum_{k=kl}^{kh} P(\omega_k)\, e^{ - i\, k\,\delta\omega\, t } \delta \omega \ .
$$
The discrete sampling in $\omega$ results in a replication in time:
$$
P(t + n\,2\pi/\delta\omega)\ =\ \sum_{k=kl}^{kh} P(\omega_k)\, e^{ - i\, k\,\delta\omega\, ( t + n\,2\pi/\delta\omega )}\ =\  \sum_{k=kl}^{kh} P(\omega_k)\, e^{ - i\, k\,\delta\omega\, t}e^{ -i kn 2\pi } \ =\ P(t) 
$$

When transformed back to the time domain the wavelet shapes and
traveltimes of the true wavefield are not lost. The wavefield is
merely replicated at regular intervals, it is aliased in time. If the
aliases do not overlap, and the approximate position of the true alias
is known then it can be uniquely retrieved.

\plot{aliases}{width=6in}{Aliasing in time caused by sampling in frequency.}

In order to track the correct alias I perform the calculation in a
polar coordinate frame. Once I have calculated the wavefield in the
frequency domain I can attempt to estimate the
traveltime/amplitude/phase that best fits the wavefield. If I know the
correct traveltime at one radius I can predict the position of the
correct alias of the wavefield at the next radius.  Given that
knowledge, I can pick the correct maximum-energy traveltime. By
extrapolating both the traveltime field, and the wavefield, outwards
from the origin I am able to overcome the problems caused by aliasing.
Figure~\ref{fig:aliases} shows the wavefield at one radius for a medium
with a circular velocity anomaly. The top left frame used all
sixty-four frequencies, the top right frame used thirty-two, the
bottom left used eight and the bottom right used four. It is clear
that if we know which is the true alias, it can be separated from the
others in all the plots except the one created using four frequencies.

In my algorithm a small number of frequencies (8-16) in the seismic
frequency band are extrapolated outwards from the source location
using a paraxial one-wave equation in polar coordinates
\cite{Nichols.sep.77.367}.  The traveltime and wavefield are both
known at the origin and they are extrapolated outwards to fill the
whole space. At each radius the wavefield is parameterized by a
traveltime/amplitude/phase triplet.  The traveltime is chosen to
correspond to the traveltime of the maximum energy event at each
location.

 The algorithm at each radius is
as follows.
\begin{enumerate}
\item Calculate the wavefield at the new radius for the sparsely
  sampled set of frequencies.
\item Choose a time window centered around the traveltime from previous
  radius.
\item Calculate a sampled time domain representation in the window by
  slow Fourier transform.
\item Pick the maximum energy sample.
\item Use a quadratic fit to find the traveltime of the local peak of
  the energy function.
\item Calculate the amplitude, and phase at this traveltime.
\end{enumerate}

The cost of this algorithm is surprisingly modest. In a constant
velocity model it costs about 8 times as much as an explicit
finite-difference solution to the eikonal equation.  In a complex
model the cost of the two algorithms is about the same. In the complex
velocity model, the explicit eikonal solver must use very small radial
steps to remain stable. The band-limited Green's function calculation
is based on a stable wavefield extrapolator, so the grid, and hence the
cost, is independent of model complexity.

The band-limited Green's functions have several desirable properties:
\begin{itemize}
\item They can be calculated in any slowness model, there is no
  smoothness constraint.
\item The solution is found at every point in the subsurface (no
  shadow zones).
\item The maximum energy arrival is found rather than the first
  arrival.
\item The solution is an estimate of the Green's function in the
  seismic frequency band not the solution at very high frequency.
\item Traveltime, amplitude and phase are calculated.
\end{itemize}
It also has some limitations:
\begin{itemize}
\item The traveltime field is discontinuous. This makes it harder to
  interpolate.
\item No explicit rays or takeoff angles are calculated. They must be
  inferred from the traveltime gradients.
\end{itemize}

\section{GREEN'S FUNCTIONS IN THE MARMOUSI MODEL}

Figure~\ref{fig:ttmaps} shows traveltime contours in the Marmousi velocity
model.  The top frame is a first arrival traveltime field calculated
by a finite-difference solution to the eikonal equation. The center
frame is a maximum amplitude traveltime field calculated by paraxial
ray-tracing, the bottom frame is the maximum energy traveltime field
calculated using my method. The lower two estimates are
 discontinuous but they are both a better fit to the significant
energy in this model. This is illustrated in figure~\ref{fig:all-tov-1.1},
each frame is one snapshot of the modeled impulse response with the
traveltime contour for that time superimposed. The top frame is shows
the traveltimes from a finite difference solution to the eikonal
equation. The middle frame shows the traveltimes from paraxial ray
tracing. The bottom frame shows the band-limited traveltime.

\activeplot{ttmaps}{height=8.2in,width=6in}{.}{Traveltimes maps for the
  Marmousi model. }
\activeplot{all-tov-1.1}{height=8.2in,width=6in}{.}{Traveltime contour for 1.1sec. 
superimposed on a wavefield snapshot at 1.1sec.}

Figure~\ref{fig:comp-amp} shows a comparison of the amplitude estimates
for the paraxial ray-tracing and the band-limited Green's functions.
The estimate made in the seismic frequency band is much smoother and
it does not show any of the instability associated with caustics that
can cause problems when using ray traced Green's functions.

\activeplot{comp-amp}{width=6in}{.}{Amplitude maps from paraxial ray tracing and band-limited traveltimes.}

\subsection{Imaging the Marmousi dataset}

In another paper in this report a full comparison is made between the
results of Kirchhoff migration using the Green's functions created
using my method, and migration images created using other methods
\cite{Audebert.sep.80.mig3d2}. In this paper I will only show two
images. The first, Figure~\ref{fig:marm-tp}, is the result of Kirchhoff
migration using just the traveltime and phase information from my
band-limited Green's functions. The second,
Figure~\ref{fig:mig-shot-prof}, is the result of full wavefield shot
profile migration using a correlation imaging condition. Both of
these methods produce structural images rather than reflectivity
estimates. The two images are very similar, Kirchhoff migration can
produce good images in a complex model as long as the Green's
functions are good estimates of the full wavefield.

It is also possible to create a reflectivity estimate using Kirchhoff
migration/inversion \cite{Lumley.sep.70.165}.  Figure~\ref{fig:refl}
shows a close-up of the reflectivity estimates in the target zone of
the Marmousi model. The top frame is a bandpass filtered version of
the true reflectivity, the center frame is the reflectivity estimate
created using paraxial ray tracing, the bottom frame is the estimate
created using band-limited Green's functions. The bottom frame has a
more continuous image and it is a better match to the true
reflectivity than the center frame.

\activeplot{marm-tp}{height=8.in,width=6in}{.}{Prestack Kirchhoff
  migration of the Marmousi dataset using band-limited Green's
  functions.}

\activeplot{mig-shot-prof}{height=8.in,width=6in}{.}{Shot profile migration
of the Marmousi dataset. Imaged using a correlation imaging condition.}

\activeplot{refl}{height=8.2in,width=5.5in}{.}{Close up of reflectivity
  estimates in the target zone.}

\section{CONCLUSIONS}

I have presented a new method for estimating Green's functions in
complex media. As in many traditional methods, the Green's functions
are parameterized by a traveltime/amplitude/phase triplet at every
point in the model. However these parameters are not calculated using
a high frequency approximation. Instead the solution to the Helmholtz
equation is computed at a few frequencies in the seismic frequency
band. The parameters are then chosen to represent a maximum energy
arrival that fits these solutions.

Prestack Kirchhoff migration images created using these ``band-limited
Green's functions'' are superior to those created using other
traveltime estimation methods. They are very close in quality to the
images created by full wavefield extrapolation methods.

\bibliographystyle{seg}
\bibliography{SEG,SEP2}


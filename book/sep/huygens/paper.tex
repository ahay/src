\lefthead{Sava \& Fomel}
\righthead{Huygens tracing}
\footer{SEP-95}
\published{SEP Report, 95, 101-113 (1997)}
\title{Huygens wavefront tracing: \\ A robust alternative to conventional
ray tracing}

\email{paul@sep.stanford.edu, sergey@sep.stanford.edu}
%\keywords{ray tracing, traveltimes, eikonal}

\author{Paul Sava and Sergey Fomel}

\maketitle

\begin{abstract}
We present a method of ray tracing that is based on a system of
differential equations equivalent to the eikonal equation, but formulated
in the ray coordinate system. We use a first-order discretization scheme
that is interpreted very simply in terms of the Huygens' principle. The
method has proved to be a robust alternative to conventional ray tracing,
while being faster and having a better ability to penetrate the shadow
zones.
\end{abstract}

\section{Introduction}

Though traveltime computation is widely used in seismic modeling and
routine data processing, attaining sufficient accuracy without compromising
speed and robustness is problematic. Moreover, there is no easy way to
obtain the traveltimes corresponding to the multiple arrivals that appear
in complex velocity media.
\par
The tradeoff between speed and accuracy becomes apparent in the choice
between the two most commonly used methods, ray tracing and numerical
solutions to the eikonal equation. Other methods reported in the literature
(dynamic programming \cite[]{GEO56-01-00590067}, wavefront construction
\cite[]{GEO58-08-11571166}, etc.) are less common in practice
\cite[]{Audebert.sep.80.25}.
\par
Eikonal solvers provide a relatively fast and robust method of
traveltime computations \cite[]{GEO55-05-05210526,GEO56-06-08120821}.
They also avoid the problem of traveltime interpolation to a regular
grid which imaging applications require. However, the eikonal solvers
compute first-arrival traveltimes and lack the important ability to
track multiple arrivals. In complex velocity structures, the first
arrival does not necessarily correspond to the most energetic wave,
and other arrivals can be crucially important for accurate modeling
and imaging \cite[]{GEO58-04-05640575,GEO59-05-08100817}.
\par
On the other hand, one-point ray tracing can compute multiple arrivals
with great accuracy. Unfortunately, it lacks the robustness of eikonal
solvers. Increasing the accuracy of ray tracing in the regions of
complex velocity variations raises the cost of the method and makes
it prohibitively expensive for routine large-scale applications.
Mathematically, ray tracing amounts to a numerical solution of the
initial value problem for a system of ordinary differential equations
\cite[]{Cerveny.SeiTom.99.87}.  These ray equations describe
characteristic lines of the eikonal partial differential equation.
\par
Here, we propose a somewhat different approach to traveltime computation,
that is both fast and accurate, and has the ability to find multiple
arrival traveltimes. The theoretical construction is based on a system of
differential equations, equivalent to the eikonal equation, but formulated
in the ray coordinate system. Unlike eikonal solvers, our method produces
the output in ray coordinates. Unlike ray tracing, it is computed by a
numerical solution of partial differential equations. We show that the
first-order discretization scheme has a remarkably simple interpretation in
terms of the Huygens' principle and propose a \emph{Huygens wavefront
tracing} (from now on referred to as \emph{HWT}) scheme as a robust
alternative to conventional ray tracing. Numerical examples demonstrate the
following properties of the method: stability in media with strong and
sharp lateral velocity variations, better coverage of the shadow zones, and
greater speed than paraxial ray tracing (from now on referred to as
\emph{PRT}).

\section{Continuous theory}

The eikonal equation, governing the traveltimes from a fixed source in
an isotropic heterogeneous medium, has the form
\begin{equation}
  \label{eqn:eikonal}
  \left(\frac{\partial \tau}{\partial x}\right)^2 +
  \left(\frac{\partial \tau}{\partial y}\right)^2 +
  \left(\frac{\partial \tau}{\partial z}\right)^2 =
  \frac{1}{v^2(x,y,z)}\;.
\end{equation}
Here $x$, $y$, and $z$ are spatial coordinates, $\tau$ is the
traveltime (eikonal), and $v$ stands for the velocity field.
Constant-traveltime contours in the traveltime field $\tau (x,y,z)$,
constrained by equation (\ref{eqn:eikonal}) and appropriate boundary
conditions, correspond to wavefronts of the propagating wave.
Additionally, each point on a wavefront can be parameterized by an
arbitrarily chosen ray parameter $\gamma$. In three dimensions, $\gamma$
includes a pair of independent parameters. For brevity, from now on we
will restrict the analysis to two dimensions. One can easily generalize
it to the 3-D case by considering $\gamma$ and $x$ as vector quantities.
Thus, we will refer to the following two-dimensional form of equation
(\ref{eqn:eikonal}):
\begin{equation}
  \label{eqn:eiko}
  \left(\frac{\partial \tau}{\partial x}\right)^2 +
  \left(\frac{\partial \tau}{\partial z}\right)^2 =
  \frac{1}{v^2(x,z)}\;.
\end{equation}

For a point source, $\gamma$ can be chosen as the initial ray angle at
the source.  \cite{Zhang.sepphd.76} shows that $\gamma$ as a
function of spatial coordinates satisfies the simple partial
differential equation
\begin{equation}
  {\frac{\partial \tau}{\partial x}\,\frac{\partial \gamma}{\partial x}}  +
  {\frac{\partial \tau}{\partial z}\,\frac{\partial \gamma}{\partial z}} = 0\;.
  \label{eqn:gamma}
\end{equation}
Equation (\ref{eqn:gamma}) merely expresses the fact that in an
isotropic medium, rays are locally orthogonal to wavefronts. The field
$\gamma (x,z)$ has not only theoretical interest as it provides one of
the possible ways for evaluating propagation amplitudes. In particular,
the geometrical spreading factor $J (x,z)$ is connected to $\gamma$ by
the simple relationship \cite[]{Zhang.sepphd.76}
\begin{equation}
  \label{eqn:spread}
  \left(\frac{\partial \gamma}{\partial x}\right)^2 +
  \left(\frac{\partial \gamma}{\partial z}\right)^2 =
  \frac{1}{J^2(x,z)}\;.
\end{equation}
It is important to note that for complex velocity fields, both $\tau$
and $\gamma$ as functions of $x$ and $z$ become multi-valued. In this
case, the multi-valued character of the ray parameter $\gamma$
corresponds to the situation, where more than one ray from the source
passes through a particular point $\{x,z\}$ in the subsurface. This
situation presents a very difficult problem when equations
(\ref{eqn:eiko}) and (\ref{eqn:gamma}) are solved numerically.
Typically, only the first-arrival branch of the traveltime is picked
in the numerical calculation. The ray tracing method is free from that
limitation because it operates in the ray coordinate system. Ray
tracing computes the traveltime $\tau$ and the corresponding ray
positions $x$ and $z$ for a fixed ray parameter $\gamma$.
\par
Since $x (\tau,\gamma)$ and $z (\tau, \gamma)$ are uniquely defined for
arbitrarily complex velocity fields, we can now make an important
mathematical transformation. Considering equations (\ref{eqn:eiko}) and
(\ref{eqn:gamma}) as a system and applying the general rules of calculus,
we can transform this system by substituting the inverse functions 
$x (\tau,\gamma)$ and $z (\tau, \gamma)$ for the original fields
$\tau (x,z)$ and $\gamma (x,z)$. The resultant expressions take the form
\begin{equation}
  \label{eqn:one}
  \left(\frac{\partial x}{\partial \tau}\right)^2 +
  \left(\frac{\partial z}{\partial \tau}\right)^2  =  
  v^2 \left(x(\tau,\gamma),z(\tau,\gamma)\right)
\end{equation}
and
\begin{equation}
  {\frac{\partial x}{\partial \tau}\,\frac{\partial x}{\partial \gamma}}  +
  {\frac{\partial z}{\partial \tau}\,\frac{\partial z}{\partial \gamma}} = 0\;.
  \label{eqn:two}
\end{equation}
\par
Comparing equations (\ref{eqn:one}) and (\ref{eqn:two}) with the original
system (\ref{eqn:eiko}-\ref{eqn:gamma}) shows that equations (\ref{eqn:one})
and (\ref{eqn:two}) again represent the dependence of ray coordinates and
Cartesian coordinates in the form of partial differential equations. 
However, the solutions of system (\ref{eqn:one}-\ref{eqn:two}) are better
behaved and have a unique value for every $\tau$ and $\gamma$.  These values
can be computed with the conventional ray tracing. However, the ray-tracing
approach is based on a system of ordinary differential equations, which
represents a different mathematical model.
\par
We use equations (\ref{eqn:one}) and (\ref{eqn:two}) as the basis of our
wavefront tracing algorithm. The next section discusses the discretization of
the differential equations and the physical interpretation we have given to
the scheme.

\section{A discretization scheme and the Huygens' principle}
\inputdir{XFig}
A natural first-order discretization scheme for equation (\ref{eqn:one}) leads
to the difference equation
\begin{equation}
  \label{eqn:circle}
  \left(x_{j+1}^i-x_j^i\right)^2 + 
  \left(z_{j+1}^i-z_j^i\right)^2 = \left(r_j^i\right)^2\;,
\end{equation}
where the index $i$ corresponds to the ray parameter $\gamma$, $j$
corresponds to the traveltime $\tau$, $r_j^i = \triangle \tau\,v_j^i$,
$\triangle \tau$ is the increment in time, and $v_j^i$ is the velocity
at the $\{i,j\}$ grid point. It is easy to notice that equation
(\ref{eqn:circle}) simply describes a sphere (or a circle in two
dimensions) with the center at $\{ x_j^i, z_j^i \}$ and the radius
$r_j^i$. This sphere is, of course, the wavefront of a secondary Huygens
source. 
\par
This observation suggests that we apply the Huygens' principle
directly to find an appropriate discretization for equation
(\ref{eqn:two}). Let us consider a family of Huygens spheres, centered
at the points along the current wavefront. Mathematically, this family
is described by an equation analogous to (\ref{eqn:circle}), as follows:
\begin{equation}
  \label{eqn:family}
  \left(x-x (\gamma)\right)^2 + 
  \left(z-z (\gamma)\right)^2 = r^2 (\gamma)\;.
\end{equation}
Here the ray parameter $\gamma$ serves as the parameter that
distinguishes a particular Huygens source. According to the Huygens'
principle, the next wavefront corresponds to the envelope of the
wavefront family. To find the envelop condition, we can simply
differentiate both sides of equation (\ref{eqn:family}) with respect to
the family parameter $\gamma$. The result takes the form 
\begin{equation}
  \label{eqn:envelop}
  \left(x (\gamma) - x \right)\, x'(\gamma) + 
  \left(z (\gamma) - z \right)\, z'(\gamma) = r (\gamma) r'(\gamma)\;,
\end{equation}
which is clearly a semidiscrete analog of equation (\ref{eqn:two}).
To complete the discretization, we can represent the
$\gamma$-derivatives in (\ref{eqn:envelop}) by a centered
finite-difference approximation. This representation yields the scheme
\begin{equation}
  \label{eqn:line}
  \left(x_j^i - x_{j+1}^i \right)\, \left(x_j^{i+1} - x_j^{i-1}\right) + 
  \left(z_j^i - z_{j+1}^i \right)\, \left(z_j^{i+1} - z_j^{i-1}\right)  
  = r_j^i \left(r_j^{i+1} - r_j^{i-1}\right)\;,
\end{equation}
which supplements the previously found scheme (\ref{eqn:circle}) for a
unique determination of the point $\{x_{j+1}^i,z_{j+1}^i\}$ on the
$i$-th ray and the $(j+1)$-th wavefront. Formulas (\ref{eqn:circle})
and (\ref{eqn:line}) define an update scheme, depicted in Figure
\ref{fig:scheme}. To fill the $\{\tau,\gamma\}$ plane, the scheme
needs to be initialized with one complete wavefront (around the wave
source) and two boundary rays.
\par
The solution of system (\ref{eqn:circle}-\ref{eqn:line}) has the
explicit form
\begin{eqnarray}
  \label{eqn:xsol}
  x_{j+1}^i & = & x_j^i - r_j^i \left(
    \alpha\,\left(x_j^{i+1} - x_j^{i-1}\right) \pm
    \beta\,\left(z_j^{i+1} - z_j^{i-1}\right)\right)\;, \\
   \label{eqn:zsol}
  z_{j+1}^i & = & z_j^i - r_j^i \left(
    \alpha\,\left(z_j^{i+1} - z_j^{i-1}\right) \mp
    \beta\,\left(x_j^{i+1} - x_j^{i-1}\right)\right)\;, 
\end{eqnarray}
where 
\begin{eqnarray}
  \label{eqn:alfa}
  \alpha & = & \frac{r_j^{i+1} - r_j^{i-1}}
  {\left(x_j^{i+1} - x_j^{i-1}\right)^2 + \left(z_j^{i+1} - z_j^{i-1}\right)^2} \;, \\ 
\end{eqnarray}
and 
\begin{eqnarray}
  \label{eqn:beta}
  \beta & = & 
  \mbox{sign}\left( x_j^{i+1} - x_j^{i-1} \right)
  \frac{\sqrt{
      \left(x_j^{i+1} - x_j^{i-1}\right)^2 + 
      \left(z_j^{i+1} - z_j^{i-1}\right)^2 -
      \left(r_j^{i+1} - r_j^{i-1}\right)^2
  }}
  {
  \left(x_j^{i+1} - x_j^{i-1}\right)^2 + 
  \left(z_j^{i+1} - z_j^{i-1}\right)^2 
  }
  \;.
\end{eqnarray}

\sideplot{scheme}{width=3in}{An updating scheme for HWT. Three
points on the current wavefront ($A$, $B$, and $C$) are used to advance
in the $\tau$ direction.}
\par
\inputdir{Sage}
Figure \ref{fig:huygens} shows a geometric interpretation of formulas
(\ref{eqn:circle}) and (\ref{eqn:line}). Formula (\ref{eqn:line}) is
clearly a line equation. Thus, the new point $D$ in Figure
\ref{fig:huygens} is defined as one of the two intersections of this
line with the $B$ sphere, defined by formula (\ref{eqn:circle}). It is
easy to show geometrically that the newly created ray segment $BD$ is
orthogonal to the common tangent of spheres $A$ and $C$. Within the
finite-difference approximation, the common tangent reflects local
wavefront behavior.

\sideplot{huygens}{width=2.5in}{A geometrical updating scheme for
HWT in the physical domain. Three points on the current
wavefront ($A$, $B$, and $C$) are used to compute the position of the
$D$ point. The bold lines represent equations (\ref{eqn:circle}) and
(\ref{eqn:line}). The tangent to circle $B$ at point $D$ is parallel to
the common tangent of circles $A$ and $C$.}

\section{Implementation details}

There are a few problems that have to be addressed for the successful
implementation of the algorithm described in the preceding section. The
most important are the boundary values, the existence of a double
solution (\ref{eqn:circle}-\ref{eqn:line}), and the complications of
finding the solution in the vicinity of the cusp points.

\subsection{Boundary values}

As mentioned in the preceding section, the application of formulas
(\ref{eqn:xsol}-\ref{eqn:zsol}) requires the existence of known boundary
values for both the first value of $\tau$ (next to the wave source)
and the extreme values of the take-off angle $\gamma$. Therefore,
we have to initialize the complete first wavefront as well as two
boundary rays that represent all the extreme points of each consequent
wavefront (that is, for the first and last considered take-off angle).
\par
To initialize the points on the first wavefront, we consider that the
velocity is constant around the source, and therefore this wavefront
becomes a circle centered at the source. This is a reasonable assumption
because we use a finite difference scheme with very small time steps, and
the velocity models have limited local variation.
\par
The values of the boundary rays are externally supplied. This apparent
problem is very easy to solve by using a ray tracing program to compute
the trajectories of these two boundary rays. We can shoot
several ``trial'' rays and select the ones that are the smoothest and that
penetrate the most into the model.

\sideplot{solution}{width=2.5in}{The double solution of the system of
equations (\ref{eqn:circle}-\ref{eqn:line}). D and E are the intersection
points between the circle given by equation (\ref{eqn:circle}) and the
line given by equation (\ref{eqn:line}). Point O is the previous point on
the ray going through B. The distance (OE) is smaller than the distance
(OD) and, therefore, D is the next selected point. The middle ray is
defined locally by the succession of points (-O-B-D-).}

\subsection{The double solution}

The system (\ref{eqn:circle}-\ref{eqn:line}) has two theoretical
solutions (\ref{eqn:xsol}-\ref{eqn:zsol}), though there is only one
that makes physical sense given a velocity map. Again, we used a
geometrical argument to select the appropriate solution. We observed 
that even though a wavefront can make a sharp turn, the corresponding
rays cannot (see the examples in the next section). We define a
turn as ``sharp'' if it happens over a very small number of samples (say,
three). Consequently, we decided to impose the condition that the correct
solution is the one represented by the point farthest away from the
preceding one on the same ray (Figure \ref{fig:solution}).

\subsection{Cusp Points}
   
The final problem to be solved is represented by the cusp
points, the case in which the three-point scheme doesn't provide a
satisfactory solution because it tends to decrease in an
unnatural way the sharpness of the wavefronts. In this case, we reduce
the three-point scheme to a two-point one by assuming that one of the
exterior points (either A or C, Figure \ref{fig:cusp}) is merged with
the point in the middle (B).

\sideplot{cusp}{width=2.5in}{Cusp points. A, B and C are the three
points on the current wavefront. Point O is the previous point on the
ray going through B. The angle CBA is smaller than the angle OBA, and
therefore B is a cusp point. If the angle CBA is closer to 90 degrees
than the angle OBA, then C is merged with B; otherwise, A is merged with
B. The three-point scheme becomes a two-point scheme without any change
in the program.}

\section{Examples}

This section presents three examples in which we applied the method
described in the last section. The first two applications are on
simple Gaussian velocity anomalies in a medium of constant velocity. We used
these models to check the validity, accuracy, and stability of the HWT
method. The third example concerns the very complex Marmousi 2-D
model, which is one of the most difficult benchmarks for ray tracing
methods. Throughout the test, we have compared our results with those
obtained with a ray tracing program
%Rekdal's PRT program \cite[]{Rekdal.sep.80.67} 
for accuracy,
speed, and stability.

\subsection{Gaussian velocity anomalies}
\inputdir{gauss}   
Our first two examples are Gaussian velocity anomalies (one positive and
one negative) with a magnitude of \mbox{2.0 km/s} in a constant velocity medium
of \mbox{2.0 km/s} for the positive anomaly, shown in Figure \ref{fig:gp-velocity},
and of \mbox{3.0 km/s} for the negative anomaly in Figure \ref{fig:gn-velocity}.
The anomaly is centered at a depth of 1.0km and has a half-width of \mbox{300 m}. The
source is placed on the surface directly above the anomaly (at \mbox{x=6.0 km}).

\plot{gp-velocity}{width=4in}{A Gaussian positive velocity anomaly.
The background velocity is \mbox{2.0 km/s}, and the maximum anomaly at the
center is \mbox{+2.0 km/s}.}

\plot{gn-velocity}{width=4in}{A Gaussian negative velocity anomaly.
The background velocity is \mbox{3.0 km/s}, and the maximum anomaly at the center
is \mbox{-2.0 km/s}.}
\par
We have selected these velocity models to test the way our method applies
to different patterns of velocity variation. In the case of the
negative anomaly, the rays focus inward, while in the case of the
positive anomaly the rays spread outward.
\par
The distribution of rays as obtained with the PRT and HWT methods are
presented in Figure \ref{fig:gp-velrw} for the positive anomaly, and in
Figure \ref{fig:gn-velrw} for the negative.

\plot{gp-velrw}{width=6in}{The rays obtained in the case of the Gaussian
positive velocity anomaly. We present the rays obtained with the PRT
method (left) and with the HWT method (right). The source is located on
the surface at \mbox{x=6.0 km}.}

\plot{gn-velrw}{width=6in}{The rays obtained in the case of the Gaussian
negative velocity anomaly. We present the rays obtained with the PRT
method (left) and with the HWT method (right). The source is
located on the surface at \mbox{x=6.0 km}.}
\par
One way to compare the two methods is to compute the distance between the
points that correspond to the same ray, identified by the same take-off
angle, at the same traveltimes. This is obviously not a perfect
quantitative comparison, because once two rays, obtained with the two 
methods, become slightly divergent, they keep going in different
directions, and thus the distance between corresponding points keeps
growing (Figures \ref{fig:gp-diff} and \ref{fig:gn-diff}). However, this
effect is not necessarily a manifestation of decreasing precision. It can
be easily seen that if such an angular mismatch doesn't occur, the rays
maintain practically the same path (see, for example, the rays shot in
the (-20,-40) and (20,40) degree intervals, where the distance decreases
in many cases to almost zero).  Even in the case of divergent rays, the
distance is kept to a reasonable level (less than 1\%). Consequently, we
do not interpret these differences as error.

\sideplot{gp-diff}{width=3in}{The distance between the corresponding
points on the rays obtained with the PRT method and with the HWT
method. Distances are given in meters.}

\sideplot{gn-diff}{width=3in}{The distance between the corresponding
points on the rays obtained with the PRT method and with the HWT
method. Distances are given in meters.}

\subsection{The Marmousi model}
\inputdir{marmousi}
   
In the third example, we have applied the same method to trace rays in
the far more complex Marmousi 2-D Model. Figure \ref{fig:m-velocity}
contains the true velocity (left) and a smoothed version using twice a
tridiagonal $5\times5$ filter (right). In Figure \ref{fig:m-velrw-raw} we present
the rays obtained on the unsmoothed Marmousi Model with the PRT method
(left) and with the HWT method (right). In Figure
\ref{fig:m-velrw-smooth} we present the rays obtained on the smoothed
Marmousi Model with the PRT method (left) and with the HWT method
(right).

\plot{m-velocity}{width=6in}{The Marmousi model. The true velocity
appears on the left,the smoothed velocity on the right.} 

\plot{m-velrw-raw}{width=6in}{The rays obtained in the true velocity Marmousi
model using the PRT method (left) and the HWT method (right).}

\plot{m-velrw-smooth}{width=6in}{The rays obtained in the smoothed Marmousi
model using the PRT method (left) and the HWT method (right).}
\par
As expected, the rays traced using the PRT method (Figure
\ref{fig:m-velrw-raw}, left), which represents a more exact solution to the
eikonal equation for the given velocity field, have a very rough
distribution. Since this erratic result is of no use in practice,
regardless of its accuracy, the only way to get a proper result is to
apply the ray tracing to a smoothed velocity model (Figure
\ref{fig:m-velrw-smooth}, left).
\par
On the other hand, the result obtained with the HWT method looks a lot
better, though some imperfections are still visible. For the case of the
unsmoothed velocity medium, the rays have a much smoother pattern, which
is less dependent on how rough the velocity model is (Figure
\ref{fig:m-velrw-raw}, right). This feature is preserved in the case of
the smoothed model (Figure \ref{fig:m-velrw-smooth}, right) where the
distributions of rays displayed by the two methods are much more
similar, though some differences remain (see, for example the zone
around x=6.5km, z=2.0km).
\par
As with the Gaussian model, we present the distances between the points
that correspond to the same ray, identified by the same take-off angle,
at the same traveltimes (Figure \ref{fig:m-diff-smooth}). This is another way
to interpret what we saw in Figure \ref{fig:m-velrw-smooth}, where most
of the rays have a consistent behavior, displaying similar paths regardless
of the method used, and therefore small distances, and a few have a
different trajectory, resulting in big distances that increase with
traveltime.

\sideplot{m-diff-smooth}{width=3in}{The distance between the
corresponding points on the rays obtained with the PRT method and with
the HWT method. Distances are given in meters.}

\section{Conclusions}

The results obtained so far have led us to the following conclusions:

\begin{enumerate}
\item  Stability: The HWT method is a lot more stable in rough
velocity media than the PRT method. The increased stability results
from the fact that HWT derives the points on the new wavefronts from 
three points on the preceding wavefront, compared to only one in the
usual PRT, which also means that a certain degree of smoothing is already
embedded in the method. This feature allows us to use the HWT
method in media of very sharp velocity variation and still
obtain results that are reasonable from a geophysical point of view.

\item  Coverage: Being more stable and giving smoother rays than the PRT
method, enables the HWT method to provide a better coverage of
the shadow zones. The idea is that since the wavefront is traced from one
ray to the other, it is very easy to introduce in the code a condition
to decrease the shooting angle as soon as the wavefront length exceeds
a specified upper limit.

\item Speed: Both methods were tested on an SGI 200. The execution time
for shooting 90 rays of 130 samples for each ray was 1.31s for the PRT method
and \mbox{0.22 s} for the HWT method. Even though in the current implementation
of HWT we do not compute the amplitudes of the waves, our method has
still yielded a big improvement in speed for the 2-D case, which gives us
hope of doing even better in the 3-D case. 
\end{enumerate}
\par
In our future work, we will implement the 3-D Huygens wavefront tracing
method. We expect to preserve its stability, while making it run even
faster in comparison to other conventional 3-D ray tracing methods.

%\section{Acknowledgments}

\bibliographystyle{seg} 
\bibliography{SEG,SEP2,paper}

%\APPENDIX{A}

%\plot{name}{width=6in,height=}{caption}
%\sideplot{name}{height=1.5in,width=}{caption}
%
%\begin{equation}
%\label{eqn:}
%\end{equation}
%
%\ref{fig:}
%(\ref{eqn:})

In this section I use elastic finite-difference-modeled wavefields
to illustrate some of the more interesting possible properties of
wave propagation in three-dimensional anisotropic media.

\subsection{Perturbing TI}
\label{Exam3-PertTI}
In section~\ref{Sec-Sing} we saw that relatively small changes to the
elastic constants could cause fairly dramatic topological changes in
the singularities.
Intuitively we expect that small perturbations in the material properties
of a medium should only be able to cause small changes in the associated
wavefield.
Is it possible for topological changes in the singularities to cause
noticeable effects in the resulting wavefield,
even when the change in the elastic constants is small?

\Tactiveplot{Exam3-Green}{width=5.75in}{\figdir}
{A Greenhorn Shale snapshot in three dimensions.}
{
\small
Four slices through a three-dimensional Greenhorn Shale
\refer{Jones and Wang}
{Ultrasonic velocities in Cretaceous shales from the Williston basin}
{1981} snapshot, corresponding to the medium shown in the top left plot
in Figure~\protect\ref{Sing-Crack}.
The {\qP} wave has been nulled to allow the maximum
resolution on the shear waves. (The original model had $128^3$ gridpoints;
the slices are through one octant.)
The source is a $z$ point force at the lower left corner.
The top row shows slices through the $x$-$z$ plane,
the bottom row shows slices rotated $30^\circ$ about the $z$ axis
out of the $x$-$z$ plane. (This slice is wider because the snapshot
cube is being cut at an oblique angle.)
The left column shows the component of displacement in the SV direction
(i.e. in the plane of the slice and
perpendicular to a vector pointing away from the source).
The right column shows the component of displacement in the SH direction
(i.e. perpendicular to the plane of the slice).
Theoretical impulse-response curves for the {\qSV} (finer dashes)
and SH (coarser dashes) modes for this medium are overlayed on the plots.
Since the medium is transversely isotropic and the source is aligned
with the symmetry axis, the $0^\circ$ and $30^\circ$ slices are equivalent.
\index{impulse response | 3D example}
\index{transverse isotropy | 3D example}
}

\Tactiveplot{Exam3-CGreen}{width=5.75in}{\figdir}
{How Figure~\protect\ref{Exam3-Green} changes
if the elastic constants are perturbed.}
{
\small
Four slices through a three-dimensional Cracked Greenhorn Shale
snapshot, corresponding to the medium shown in the middle left plot
in Figure~\protect\ref{Sing-Crack}. (The elastic constants are listed
in Table~\ref{Const-3D}.)
The parameters of the four plots are the same as for the corresponding
plots in Figure~\protect\ref{Exam3-Green}; the only difference is
the underlying elastic constants have been perturbed to break the
axisymmetry.
The same Greenhorn Shale theoretical impulse-response curves
used in the previous figure are overlayed again here. This time
the elastic constants are somewhat different, so there's no guarantee
they'll fit as well as they did in the previous figure.
For the $0^\circ$ ($x$-$z$) slice the overlayed curves fit exactly anyway,
and for the $30^\circ$ SV-component plot they are only slightly off.
For the $30^\circ$ SH-component plot, though, something unexpected happens.
Instead of some sorts of ``{\qSV}'' or ``{\qSH}'' waves as we might expect,
mostly we see a short nearly planar wavefront.
(Its amplitude is about $15\%$ of that of the primary ``{\qSV}'' wave.)
\index{impulse response | 3D example}
\index{connections | impulse-response example}
}

Figure~\ref{Exam3-Green} shows two slices through a three-dimensional
wavefield snapshot. (The {\qP} mode has been nulled so I can squeeze
as much resolution out of the $128^3$ finite-difference model as possible.)
One slice is oriented lies in the $x$-$z$ plane, the other is
oriented at $30^\circ$ to the $x$-$z$ plane.
For each orientation two plots are shown;
one shows displacement in the SV direction,
the other in the SH direction.
(Note I am using ``SV'' and ``SH'' here as directions,
not as labels for modes.)
The source is a pure $z$ point force.

There are two things to note in Figure~\ref{Exam3-Green}.
First, the two sets of plots appear to be identical.
This is because the medium is
transversely isotropic, and so the two orientations
are in fact symmetrically equivalent.
Second, there appears to be no SH motion at all.
Since the medium is transversely isotropic there is a pure
SH mode; this mode is orthogonal to the $z$ source and so is not excited.

We now take Figure~\ref{Exam3-Green} and perturb the elastic constants
by adding cracks in the $x$-$z$ plane. (The cracking is not severe,
only enough to slow the {\qSV} wave propagating
perpendicular to the plane of the cracks by about $5\%$.)
Given the {\qSV} and SH modes of the
original transversely isotropic medium shown in Figure~\ref{Exam3-Green}
and the relatively small perturbation to the elastic constants,
we would expect to see something like {\qSV} and {\qSH} modes
in the cracked medium.
Figure~\ref{Exam3-CGreen} shows the actual results.
As expected, the SV plots look much the same as before.
The SH plot in the $x$-$z$ symmetry plane is again zero.
The SH plot at $30^\circ$ appears quite strange, though.
There is a ``{\qSH}'' wavefront dimly perceptible (maybe not,
depending on the reproduction quality of your copy!),
but far stronger is a perplexing nearly planar event
that connects the expected ``{\qSV}'' and ``{\qSH}'' wavefronts,
but corresponds to nothing in transverse isotropy.
What is going on?

\Tactiveplot{Exam3-SlowGreen}{width=5.75in}{\figdir}
{The three-dimensional slowness surface for ``Cracked Greenhorn Shale''.}
{
The three-dimensional shear slowness surfaces for ``Cracked Greenhorn Shale'',
which is just Greenhorn Shale perturbed by adding fractures
in the $x$-$z$ plane.
Compare this figure with the unperturbed version, the top plot
in Figure~\protect\ref{Separ3-Wavedef}
on page~\protect\pageref{Separ3-Wavedef}.
This time I have rotated the medium $60^\circ$ so that the edges of
the cut-out octant do not line up with the symmetry axes of the medium.
At first glance, the topology appears the same as before.
Figure~\protect\ref{Exam3-SlowGreenclose} shows
a high-resolution view of the region inside the rectangle,
which shows the true topology.
(The middle left plot in Figure~\protect\ref{Sing-Crack} shows yet
another view of this same three-dimensional surface.)
\index{shear singularities | example}
\index{wave modes | perturbed transversely isotropic}
\index{slowness surface | 3D example}
\index{connections}
}

\Tactiveplot{Exam3-SlowGreenclose}{width=5.75in}{\figdir}
{A close-up view of a perturbed TI three-dimensional slowness surface.}
{
A close-up view of Figure~\protect\ref{Exam3-SlowGreen} showing the
true topology. (Remember, I use the darkness of the surfaces
to show the associated particle-motion direction,
SV motion being darker and SH motion lighter.)
Like the orthorhombic medium in Figure~\protect\ref{Separ3-Mobius},
this orthorhombic medium displays inner {\qS2} and outer {\qS1}
surfaces that touch at a small number of discrete points (singularities).
One such point, a singularity of order $-1$, is visible as the black dot
a little to the right and below top center.
\index{shear singularities | example}
\index{shear singularities | mode-mode coupling}
\index{wave modes | coupling}
\index{wave modes | perturbed transversely isotropic}
\index{slowness surface | 3D example}
\index{connections}
}

The problem is that our intuition has been learned from
studying highly symmetric two-dimensional examples
(or, even worse, isotropy).
In Chapter~\ref{Chap-2D} we encountered several examples
where events would disappear just at certain exceptional values
of the elastic constants.
(For example, compare cases 1 and~10 in Figure~\ref{TI-disp} on
page~\pageref{TI-disp} with the corresponding finite-difference
results in Figures \ref{TI-Exam1} and~\ref{TI-Exam2}
on pages \pageref{TI-Exam1} and~\pageref{TI-Exam2}.)
One could not understand transverse isotropy by studying only such
special cases.
\index{transverse isotropy | pitfalls}
Unfortunately two-dimensional anisotropy itself,
with many elastic constants held equal,
is just such a special case.\footnote{
Thus we find that
{\em all\/} two-dimensional anisotropy is inherently dishonorable!
(Recall the epigraph on page~\pageref{Quote-Helbig}.)
\index{Helbig, Klaus | quote}
[And no, I don't know why {\TeX} likes chopping this footnote into two
pieces 5 pages apart, either. In the hardcopy version I could get around
it with a brute-force {\tt {$\backslash$}clearpage},
but no such Potemkin-village hack is possible in an e-document!]
}

\subsubsection{``Connections''}
\index{connections}
So where does the mysterious event come from?

Figure~\ref{Exam3-SlowGreen} shows the three-dimensional slowness
surfaces for the cracked medium. It appears very much like the unperturbed
TI medium shown in the top plot in Figure~\ref{Separ3-Wavedef},
but there is a crucial difference, as can be seen clearly
in Figure~\ref{Exam3-SlowGreenclose}. The formerly distinct
{\qSV} and SH modes have crossed over and recombined to form
\label{Exam3-Chimera}
new {\qS1} and {\qS2} modes.
For the most part, these new shear surfaces can be thought of as
virtually unchanged patched-together fragments
of the old {\qSV} and SH surfaces, and hence I like to call them
``chimeras''\footnote{I'm using a biological analogy here;
the operative definition of ``chimera'' from Webster's
\refer{Woolf}
{Webster's new collegiate dictionary}
{1975}
is
``an individual, organ, or part consisting of tissues of diverse genetic
constitution and occurring esp. in plants at a graft union''.
(Although one of the other listed definitions,
``an imaginary monster compounded of incongruous parts'',
might be more appropriate here!)
}.

This ``exchanging identity without quite touching'' phenomenon is
quite common in physics. Spheroidal normal modes of the Earth are
another good geophysical example. A plot of frequency versus angular
order for these modes seems to show distinct sets of crisscrossing parallel
lines. Upon close examination, however, the parallel lines prove to
be an optical illusion formed by a series of noncrossing curves each shaped
like a set of staircase steps. (See for example
Figure~17 in
Gilbert and Dziewonski~\referna{Gilbert and Dziewonski}
{An application of normal mode theory to the retrieval of structural
parameters and source mechanisms from seismic spectra}
{1975}.)

This sort of behavior can be modeled using extremely simple mathematics;
take for example the equation
$y^2 = x^2$.
We would probably write the solutions to this equation as
\begin{equation}
y = \cases{x & \cr -x & \cr}
.
\end{equation}
However, if we used instead the perturbed equation
$y^2 = x^2 + \epsilon$,
we would write the solutions as
\begin{equation}
y = \cases{\sqrt{x^2 + \epsilon} & \cr -\sqrt{x^2 + \epsilon} & \cr}
.
\end{equation}
By letting $\epsilon \rightarrow 0$ we find
an alternate solution set for the original equation,
\begin{equation}
y = \cases{\vert x \vert & \cr -\vert x \vert & \cr}
.
\end{equation}
For our cracked Greenhorn Shale example, ``$\epsilon$'' is
the perturbation of the Christoffel matrix $\C$ away from
transverse isotropy. This ``$\epsilon$'' effectively vanishes
in the orthorhombic symmetry planes.

These topological details of solution sets are usually a mere curiosity,
but for our perturbed transversely isotropic example
they turn out to be significant.
It is the novel parts of the {\qS1} and {\qS2} surfaces in
Figure~\ref{Exam3-SlowGreenclose}
where they approach but don't quite touch
that create the new event seen in Figure~\ref{Exam3-CGreen}.
(Kawasaki~\referna{Kawasaki}
{Mode-mode coupling of surface waves in an anisotropic medium}
{in preparation}
similarly finds that for an azimuthally anisotropic earth such ``chimeric''
effects need to be taken into account
when inverting earthquake surface-wave data.)
To see how this relatively small change in the slowness surfaces
can be significant, we will now examine the relevant
three-dimensional {\em impulse-response surfaces\/}.

We begin with Figure~\ref{Exam3-Greenboth}, which shows
the three-dimensional shear impulse-response surfaces for unperturbed
Greenhorn Shale.
The elliptical SH mode in three dimensions becomes a prolate ellipsoid,
while the cusped {\qSV} mode becomes a sort of flanged barrel.
Figures \ref{Exam3-CGreeninner} and~\ref{Exam3-CGreenouter} show the
corresponding {\qS1} and {\qS2} modes for Cracked Greenhorn Shale.
The thick lines correspond to the thick lines in Figure~\ref{Exam3-SlowGreen}.
Note that although they lie in a plane in the slowness domain,
they definitely do not in the group domain.
(I would have shown both surfaces together,
but found the resulting tangle of
intersecting surfaces too complex to visualize easily.
To give some idea of the relative positions of the {\qS1} and {\qS2}
surfaces, I have shown the thick lines for both surfaces
in both sets of plots.)

The impulse-response surfaces shown in Figures
\ref{Exam3-CGreeninner} and~\ref{Exam3-CGreenouter}
combine gross elements of the surfaces visible
in Figure~\ref{Exam3-Greenboth}, but some new features are also visible.
The surface shown in Figure~\ref{Exam3-CGreeninner} is adorned with
thin disks hovering just above the remainder of the top surface,
and attached only at a relatively small region at the center.
The disks fit into the holes visible in Figure~\ref{Exam3-CGreenouter}.
The outer rims of the disks (and the inner edges of the holes)
correspond to the point
singularities visible in Figure~\ref{Exam3-SlowGreen}.
(The outer surface of the disks are usually called ``lids'' in the literature
(Crampin~\referna{Crampin}
{A review of wave motion in anisotropic and cracked elastic media}
{1981}, but there seems to be some confusion about what the term ``lid''
precisely means. I will avoid using the term until after I have
given better examples of the topology of the singularities, notably in
Figures~\ref{Exam3-HelbigSlice}, \ref{Exam3-FrancisAxis},
and~\ref{Exam3-FrancisBlowup}.)

\Tactiveplot{Exam3-Greenboth}{width=5.80in}{\figdir}
{
The three-dimensional shear impulse-response surfaces for Greenhorn Shale.
}
{
The three-dimensional shear impulse-response surfaces for Greenhorn Shale
\refer{Jones and Wang}
{Ultrasonic velocities in Cretaceous shales from the Williston basin}
{1981}.
The figure shows a light-colored prolate-ellipsoid pure SH mode
intersecting a flanged barrel-shaped {\qSV} mode.
\index{impulse-response surface | 3D example}
\index{transverse isotropy | 3D example}
}
\Tactiveplot{Exam3-CGreenouter}{height=6.08in}{\figdir}
{
The three-dimensional {\qS1} impulse-response surface for Cracked Greenhorn Shale
(Left view).
}
{
The three-dimensional {\qS1} impulse-response surface
for Cracked Greenhorn Shale
corresponding to the slowness surfaces shown
in Figure~\protect\ref{Exam3-SlowGreen}.
This figure is the left element in a stereo pair;
Figure~\protect\ref{Exam3-CGreenouter2} is the corresponding right element.
The isolated thick lines show the position of the omitted {\qS2} surface.
\index{impulse-response surface | 3D example}
}
\Tactiveplot{Exam3-CGreenouter2}{height=6.08in}{\figdir}
{
The three-dimensional {\qS1} impulse-response surface for
Cracked Greenhorn Shale (Right view).
}
{
The three-dimensional {\qS1} impulse-response
surface for Cracked Greenhorn Shale
corresponding to the slowness surfaces shown
in Figure~\protect\ref{Exam3-SlowGreen}.
This figure is the right element in a stereo pair;
Figure~\protect\ref{Exam3-CGreenouter} is the corresponding left element.
The isolated thick lines show the position of the omitted {\qS2} surface.
\index{impulse-response surface | 3D example}
}
\Tactiveplot{Exam3-CGreeninner}{height=6.08in}{\figdir}
{
The three-dimensional {\qS2} impulse-response
surface for Cracked Greenhorn Shale
(Left view).
}
{
The three-dimensional {\qS2} impulse-response
surface for Cracked Greenhorn Shale
corresponding to the slowness surfaces shown
in Figure~\protect\ref{Exam3-SlowGreen}.
This figure is the left element in a stereo pair;
Figure~\protect\ref{Exam3-CGreeninner2} is the corresponding right element.
The isolated thick lines show the position of the omitted {\qS1} surface.
\index{impulse-response surface | 3D example}
}
\Tactiveplot{Exam3-CGreeninner2}{height=6.08in}{\figdir}
{
The three-dimensional {\qS2} impulse-response surface
for Cracked Greenhorn Shale
(Right view).
}
{
The three-dimensional {\qS2} impulse-response surface
for Cracked Greenhorn Shale
corresponding to the slowness surfaces shown
in Figure~\protect\ref{Exam3-SlowGreen}.
This figure is the right element in a stereo pair;
Figure~\protect\ref{Exam3-CGreeninner} is the corresponding left element.
The isolated thick lines show the position of the omitted {\qS1} surface.
\index{impulse-response surface | 3D example}
}

Another apparent difference is the larger cusps in
Figure~\ref{Exam3-CGreeninner}.
The topology of the two intersecting shear surfaces is complex;
Figure~\ref{Exam3-CGreenImp} shows where the extended cusp comes
from more clearly.
The concavity visible on the {\qS2} surface
in Figure~\ref{Exam3-SlowGreenclose} corresponds to a cusp on
the associated impulse response,
just like the one on the {\qSV} surface of Greenhorn Shale did in
Chapter~\ref{Chap-2D}.
This cusp comes from a small region of the slowness
surface, and so should not be too significant energetically.
However,
this cusp is exactly what we were seeing back in Figure~\ref{Exam3-CGreen}.
%
% I deleted this figure as it was too slow to appear and wasn't worth it.
% 
%(Figure~\ref{Exam3-Slices} shows this more clearly.)
How is this possible?

\Tactiveplot{Exam3-CGreenImp}{width=5.75in}{\figdir}
{An off-axis slice through the 3D impulse-response surfaces
of Cracked Greenhorn Shale.}
{
A slice through the 3D impulse response surface for
Cracked Greenhorn Shale, oriented to correspond to
the $30^\circ$ slices in Figure~\protect\ref{Exam3-CGreen}.
Topologically there are distinct {\qS1} and {\qS2} surfaces,
but the effect is very much like that of {\qSV} and {\qSH} surfaces
with a ``connection'' spanning the tangents to those surfaces.
This figure also illustrates a couple of other interesting counterexamples.
Note that the {\qS2} wave arrives {\em first} for a small range
of angles near the upper edge of the old {\qSV} cusp, even though
by definition the {\qS2} mode has {\em slower} phase velocity
for all phase directions.
The connection itself consists of two closely spaced linear events;
the two parts are portions of the {\qS1} and {\qS2} modes.
\index{connections}
Particle-motion directions for a given phase direction are guaranteed
to be perpendicular for distinct modes. This is usually almost true
in the group domain as well. For this example, however,
at the top of the connection, where it roots into the former {\qSV}
mode, both modes have nearly pure SV polarization. At the bottom
of the connection, where it roots into the former SH mode,
both modes have nearly pure SH polarization. (In between, the particle
motions perform the transition by rotating in opposite directions.)
\index{particle motion | orthogonality}
\index{impulse-response surface | 3D slice example}
}

%
% This figure just isn't worth the excessive length of time it takes
% to appear on the screen! So deleted for the interactive version.
% -Joe Oct 21 1991
%
%\Tactiveplot{Exam3-Slices}{width=5.75in}{\figdir}
%{A reprise of Figure~\protect\ref{Exam3-CGreen}, showing the correct
%impulse-response surfaces for Cracked Greenhorn Shale.}
%{
%A reprise of Figure~\protect\ref{Exam3-CGreen}, this time showing
%the correct impulse-response surfaces for Cracked Greenhorn Shale
%(on the right).
%A gpow of $.7$ on the snapshots (left) has been applied to make weaker
%events more visible.
%An attempt has been made to show the proportion
%of the theoretical modes' particle-motion direction
%aligned with the pure SV and SH directions by shading the curves
%appropriately (although the shading has somewhat run afoul
%of the halftone dithering pattern).
%\index{impulse response | 3D example}
%\index{impulse-response surface | 3D slice example}
%}

Figure~\ref{Exam3-CGreenImp} very much appears like the unperturbed
{\qSV} and {SH} surfaces of Greenhorn Shale, plus a new event I
will call a ``connection''.
\index{definition | connection}
\index{connections | definition}
The connection comes from the ring along which the original {\qSV} and SH
surfaces intersected, but where the new {\qS1} and {\qS2} surfaces
merely appulse in a ``ring pinch'' (except at the few point singularities)
\refer{Crampin and Yedlin}
{Shear-wave singularities of wave propagation in anisotropic media}
{1981}.
\index{ring pinch}
\index{definition | ring pinch}
Points in the phase-slowness domain correspond to plane-wave components
in the impulse-response domain, so the connection lies where there
was a common plane-wave component of the original {\qSV} and SH modes
(i.e. it is tangent to both).

A connection can have a significant effect by providing a channel
for coupling the ersatz ``{\qSV}'' and ``{\qSH}'' modes.
\index{wave modes | coupling}
\index{connections | wave mode coupling}
In the Cracked Greenhorn Shale example used here, the connection roots
into the {\qSV} mode in the middle of the high-amplitude cusp,
and acts to provide a channel along which some of this energy can
drain away into the ``{\qSH}'' mode.
Since at the same time the particle-motion direction also slowly rotates
to include more and more SH motion, it manages to show up strongly
on the SH snapshot slice.
This sort of anisotropic behavior provides an ideal vehicle for
coupling a vertical P source into a horizontal SH receiver, even
for media that otherwise are virtually transversely isotropic.
Note that connections should only be significant well off of the
symmetry planes. To me this suggests that deliberately trying
to record seismic data
only along suspected symmetry planes is a questionable practice.

Ray tracers should note that while
the SH component in the $0^\circ$ symmetry plane
in Figure~\ref{Exam3-CGreen} is exactly zero for all wavetypes,
as it must be by symmetry, the SV component of the connection
in the symmetry plane is nonzero (although quite weak).
This event is in fact a non-ray wave.
Unfortunately the connection on the $0^\circ$ SV plot
in Figure~\ref{Exam3-CGreen} is not visible on the hardcopy,
although it is dimly visible on a good screen.

\subsection{Anatomy of a singularity}
We have seen that singularities are important topologically,
and that in certain cases the topology of the wave modes
case create significant events such as ``connections''.
But can the singularities themselves cause
any significant propagation effects?
I begin by examining the topology of point singularities
in the impulse-response domain.

\subsubsection{A canonical orthorhombic example}
\label{Exam3-HelbigSec}
Figure~\ref{Exam3-Helbig} shows the {\qS} slowness surfaces
for a simple orthorhombic medium.
This medium has four widely spaced singularities of order $+1$,
all lying in the $k_y$-$k_z$ plane. For the example here I have rotated
the medium about the $k_x$ axis to line up one of the singularities
with the $+k_z$ axis.
Figure~\ref{Exam3-Helbig2} shows the corresponding impulse-response
surfaces.
The geometries of the slowness and impulse-response surfaces seen here
seem to be much the same:
two surfaces intersecting at a point forming a shallow double cone.
In one case the center of the cone is on the $k_z$ axis; in the other
the center is off to one side.

\Tactiveplot{Exam3-Helbig}{width=5.75in}{\figdir}
{Anatomy of a canonical singularity in the slowness domain.}
{
The {\qS} slowness surfaces for a particularly symmetric orthorhombic
medium. This medium has a global pure P mode and pure SH and SV modes
in each symmetry plane. In each symmetry plane one of
the shear modes is elliptical and the other is circular.
The medium has been rotated about the $x$ axis to position one
of the singularities on the $+k_z$ axis; thus, only the $k_y$-$k_z$
axial plane is still one of the three planes of symmetry.
In order to show the topology of the singularity more clearly,
only alternate latitude bands are shown, and those only
from latitudes $45^\circ$ to $90^\circ$.
One $90^\circ$ sector has been omitted to allow a view of the singularity.
\index{slowness surface | 3D example}
\index{shear singularities | example}
}
\Tactiveplot{Exam3-Helbig2}{width=5.75in}{\figdir}
{Anatomy of a canonical singularity in the group domain.}
{
The {\qS} impulse-response surfaces corresponding to the slowness surfaces
in Figure~\protect\ref{Exam3-Helbig}. The point singularity in the
previous figure corresponds to the annular gap containing the $+z$ axis.
(The line leaving the figure to the left is the $-y$ axis.)
Note that the dark axial lines marking the $k_x$-$k_z$ plane abruptly bend
as they approach the annulus marking the location of the singularity.
This is because the singularity in the slowness domain was a fraction
of a degree out of the $k_x$-$k_z$ plane. The axial lines
do not cross the ring of the singularity in the group domain
but detour around it instead.
\index{impulse-response surface | 3D example}
\index{shear singularities | example}
}

In either domain one could imagine cutting the inner and outer
surfaces apart at the intersection point and labeling
those as the two {\qS} pure modes.
Unfortunately the resulting classification would not be consistent
between the two domains, because the intersection point in
the group domain does not correspond to the singularity.
What on the impulse-response surfaces does correspond to the singularity?

As is shown in Appendix~\ref{Chap-Group},
the group direction is perpendicular to the tangent to the slowness surface
at each point. As can be seen in Figure~\ref{Exam3-Helbig},
at the singularity the tangent becomes discontinuous.
In fact the point singularity in the slowness domain maps
to a cone of directions in the impulse reponse.
\index{shear singularities | internal conical refraction}
(This phenomenon is well known in crystals,
where it can be directly observed
as the phenomenon of internal conical refraction
\refer{de Klerk and Musgrave}
{Internal conical refraction of transverse elastic waves in a cubic crystal}
{1955}.
For our example in Figure~\ref{Exam3-Helbig},
one of the possible tangents at the singularity
is for a constant-velocity surface
(circular in two-dimensional cross section).
Since for that case the group and phase directions are the same,
the $+z$ axis in the impulse-response domain in our example
must mark one point along the cone of singular directions.
The singularity in Figure~\ref{Exam3-Helbig2}
is the annular gap containing the $+z$ axis.

%
% I really want these next two to appear togehter on the same page,
% but TeX has a mind of its own and with the \psfigs stuff there
% is no provision for lying to TeX about figure sizes to fool it
% into doing what you want! -Joe
%
\Tactiveplotpage{Exam3-HelbigSlice}{width=5.75in}{\figdir}
{The plane lid of a canonical singularity.}
{
A cross section through the $y$-$z$ plane of the surface shown in
Figure~\protect\ref{Exam3-Helbig2}.
The {\qS1} wave mode is shown by a black line,
the {\qS2} wave mode is shown by a dark stippled line,
and the plane lid by a thin dotted line.
\index{shear singularities | plane lid example}
\index{shear singularities | example}
\index{impulse-response surface | 3D slice example}
}
\Tactiveplotpage{Exam3-HelbigSlice2}{width=5.75in}{\figdir}
{The finite-difference model corresponding to the previous figure.}
{
The finite-difference model result corresponding
to Figure~\protect\ref{Exam3-HelbigSlice}.
The modeled wavefield was generated by a $y$ point source,
but the $x$ component of particle motion is shown.
This emphasizes wave components with intermediate polarization
directions, such as occur around the singularity.
The slice is displaced slightly out of the plane
containing the source (in the symmetry plane the $x$ component
is zero).
\index{shear singularities | impulse response example}
\index{impulse response | 3D slice example}
}

In fact there is more to the singularity than that.
As shown by Burridge~\referna{Burridge}
{The singularity of the plane lids of the wave surface of elastic media with cubic symmetry}
{1967},
the ``dimple'' on the impulse-response surface
is made convex by the additional presence of a planar event
stretching across the concavity.
\index{shear singularities | plane lids}
This event, called a ``plane lid'' by Burridge, 
is the plane-wave component of the impulse response
corresponding to the singularity.
The cone of singular directions previously described marks the edges
of the plane lid, in this case shaped like a circular disk.
Figure~\ref{Exam3-HelbigSlice} shows the position of the plane lid
on a cross section through the $y$-$z$ symmetry plane
of the impulse-response surface.
(The plane lid is omitted in Figure~\ref{Exam3-Helbig2}.)
Figure~\ref{Exam3-HelbigSlice2} shows a corresponding
cross section through a finite-difference model snapshot.

\subsubsection{A triply connected example}

Burridge~\referna{Burridge}
{The singularity of the plane lids of the wave surface of elastic media with cubic symmetry}
{1967} found that the singularities of cubic nickel had ``plane lids'',
and derived their amplitudes.
Later Crampin~\referna{Crampin}
{A review of wave motion in anisotropic and cracked elastic media}
{1981} pointed out that the ``lids'' of orthopyroxene are shallow cones,
not planar.
Are they both right?

\Tactiveplot{Exam3-FrancisImp}{width=5.85in}{\figdir}
{Impulse-response surfaces for a ``triply connected'' orthorhombic medium.}
{
Impulse-response surfaces for a ``triply connected'' anomalously polarized
orthorhombic medium. All three surfaces have singularities,
even the first-arriving one.
Since the surfaces are anomalously polarized, the standard anisotropic
notation ``\{{\qP}, {\qS1}, {\qS2}\}'' does not apply.
A general notation proposed by Muir~\referna{Muir}
{pers. comm.}{1989}
extends the original
meaning of ``P'' and ``S'' to sort
the three wavetypes noncommittally by phase velocity as
``\{{\sl P, S, T}\}'', for {\it primero}, {\it secundo}, and {\it tertio}.
\index{wavetype separation | 3D inseparable example}
\index{wave modes | ordering}
\index{wave modes | general}
\index{shear singularities | example}
\index{impulse-response surface | 3D example}
}

Figure~\ref{Exam3-FrancisImp} shows all three impulse-response surfaces for
a complex orthorhombic medium.
This medium is interesting in several ways.
All three wave modes are anomalously polarized.
All three wave modes are connected by singularities;
there is no separable {\qP} mode.
\index{wavetype separation | 3D inseparable example}
In the next few figures I will examine the properties of
the singularity in the $y$-$z$ symmetry plane
on the {\em fastest\/} wavetype; it is particularly amenable
to study since it is part of the first-arriving wavefront.

\Tactiveplot{Exam3-FrancisAxis}{width=5.75in}{\figdir}
{A slice through the $y$-$z$ plane of the previous figure.}
{
An annotated slice through the $y$-$z$ plane of the surfaces
in Figure~\protect\ref{Exam3-FrancisImp}.
The three surfaces are labeled following Muir's \{{\sl P, S, T}\} convention.
The {\sl P} and {\sl T} surfaces are drawn thick and stippled,
while the intermediate {\sl S} surface is drawn thin and black.
The lid is part of the {\sl S} wavefront.
(The plane lid is omitted in Figure~\protect\ref{Exam3-FrancisImp},
and is shown here with a thin dotted line.)
Note the two points marked ``Singularity'' correspond to the same point
singularity in the slowness domain.
\index{shear singularities | plane lid example}
\index{shear singularities | lid example}
\index{shear singularities | example}
\index{impulse-response surface | 3D slice example}
}
\Tactiveplot{Exam3-FrancisBlowup}{width=5.75in}{\figdir}
{As Figure~\protect\ref{Exam3-FrancisAxis}, but rotated $5^\circ$ out
of the $y$-$z$ symmetry plane.}
{
As Figure~\protect\ref{Exam3-FrancisAxis}, but rotated $5^\circ$ out
of the $y$-$z$ symmetry plane to better show the true topology.
The intermediate-velocity {\sl S} surface appears as two disjoint
pieces in this slice, although in three dimensions the parts form
one continuous surface that intersects itself in the $y$-$z$ plane.
\index{shear singularities | plane lid example}
\index{shear singularities | lid example}
\index{shear singularities | example}
\index{impulse-response surface | 3D slice example}
}

Figure~\ref{Exam3-FrancisAxis} shows an annotated slice of the impulse-response
surface in Figure~\ref{Exam3-FrancisImp} along the $y$-$z$ symmetry plane.
Note this example shows two distinct kinds of ``lids'':
Burridge and Crampin were both right, but were talking about different
things. I will follow their lead by calling one kind a ``plane lid''
versus just plain ``lid'' for the other.

I define a {\em plane lid\/} as the portion of the impulse-response
surface corresponding to the point singularity in the slowness domain.
\index{definition | plane lid}
\index{shear singularities | plane lids}
It is a planar surface that stretches across what otherwise would be a
concavity in the impulse-response surface, and is bounded by the cone
of singular directions associated with the singularity.
Its presence keeps the surface convex.
I define a {\em lid\/} as the outermost part of the impulse-response
surface within the region bounded by the cone of singular directions
associated with the point singularity not counting the ``plane lid''
generated by the singularity itself.
\index{definition | lid}
\index{shear singularities | lids}
This part of the impulse-response surface will be concave,
and although faster in the group domain
will be part of the slower of the two wavetypes
in the phase-slowness domain.
Some highly symmetric media, such as Burridge's cubic nickel example
and the example in Figure~\ref{Exam3-Helbig} do not exhibit a lid
because it has contracted to a point in the slowness domain and merged
with the plane lid.

\Tactiveplot{Exam3-FrancisCartoon}{height=7.36in}{\figdir}
{Various non-symmetry-plane slices with corresponding
finite-difference results.}
{
Various non-symmetry-plane slices through the impulse-response surface
in Figure~\protect\ref{Exam3-FrancisImp} with corresponding
finite-difference results. The source for the finite-difference model
is a $y$ point force. The SV component of the particle-motion
direction is shown.
The plane lid also shows exists in
the longitude~$75^\circ$ slice, but is too close to the lid to be
visible in either representation.
\index{shear singularities | plane lid example}
\index{impulse-response surface | 3D slice example}
\index{impulse response | 3D slice example}
}
\Tactiveplot{Exam3-FrancisSlice}{height=7.36in}{\figdir}
{More finite-difference examples
corresponding to those in Figure~\protect\ref{Exam3-FrancisCartoon}.}
{
Finite-difference results corresponding to three non-symmetry-plane
slices through the impulse-response surface
in Figure~\protect\ref{Exam3-FrancisImp}.
The slices are the same ones
used in Figure~\protect\ref{Exam3-FrancisCartoon},
but for this plot the P and SH components of the particle-motion
direction are shown.
\index{impulse response | 3D slice example}
}

What about the three-dimensional topology of the singularity shown
in Figure~\ref{Exam3-FrancisAxis}?
In the example in the previous section (Figure~\ref{Exam3-Helbig2}),
we saw that there was a point where the impulse-response surfaces
crossed through themselves forming a shallow double cone.
This was an accident of the high
symmetry of that example; Figure~\ref{Exam3-FrancisBlowup} shows
the true topology of the surfaces in Figure~\ref{Exam3-FrancisImp}
revealed by cutting on a non-symmetry plane. The intermediate {\sl S}
surface in this example intersects itself in a line, not a point.

\subsubsubsection{Finite-difference model results}
The lid and plane lid are distinguishable in Figure~\ref{Exam3-FrancisBlowup},
but can they also be distinguished in a finite-difference model result?
Figure~\ref{Exam3-FrancisCartoon} shows several non-symmetry-plane slices
of the impulse-response surfaces in Figure~\ref{Exam3-FrancisImp}
along with the corresponding finite-difference snapshot slices.
The top right plot shows a dark ($-$) event preceding
the main lid (which appears white ($+$) here)
and filling in the concavity in it.
This should be the plane lid, since it occurs in the right place
and seems to exhibit the necessary properties, but the result needs
to be backed up by further analytical calculations to be sure.
Figure~\ref{Exam3-FrancisSlice} shows the other two particle-motion
components of the finite-difference model results
at the same clip levels as in Figure~\ref{Exam3-FrancisCartoon}.
The plane lid exists in these plots as well, but is too weak to show
up on these plots.

\subsubsection{Can singularities cause visible effects?}
I still have not answered the original question of this section:
can the singularities themselves cause significant propagation effects?

In Figures \ref{Exam3-FrancisCartoon} and~\ref{Exam3-FrancisSlice}
the direct effect of the singularity, the plane lid, was a subtle
feature of the wavefront.
On the other hand,
the singularity showed up very strongly in Figure~\ref{Exam3-HelbigSlice2}.
In that example the $y$ source and $x$ receiver are at right angles;
the swirl of particle-motion directions around the singularity
provides a good coupling mechanism and makes the singularity stand out.

\index{shear singularities}
These examples indicate that direct effects such as plane lids are
not very significant, but that the local disturbance to the particle-motion
direction field caused by a singularity can be.
This latter effect will show up again in the next section.

\subsection{What should 3D anisotropy look like?}
So far in this chapter I have shown only model snapshots.
In this section I will show some seismograms to go with two of the
models examined in the previous section. This is meant to give some
idea how the effects previously noted might appear on surface data.

\subsubsection{What should a connection look like?}
Figure~\ref{Exam3-CGreenHyp} shows model sections recorded over
a homogeneous half-space of ``Cracked Greenhorn Shale''
(described in section~\ref{Exam3-PertTI})
at two different azimuths, along the cracks (azimuth~$0^\circ$)
and at an acute angle to the cracks (azimuth~$30^\circ$).
The model and plot parameters of this plot are the same as those for
Figure~\ref{TI-Green3NMO} on page~\pageref{TI-Green3NMO},
although the example in Chapter~\ref{Chap-2D} is not directly
comparable. That model medium had a different value
of {\cac} and so does not triplicate as strongly as the example here.

\Tactiveplot{Exam3-CGreenHyp}{height=7.52in}{\figdir}
{Model sections for ``cracked Greenhorn Shale'' showing ``connections''.}
{
Model sections for ``cracked Greenhorn Shale'' recorded along two
different lines of azimuth.
The cracks run along azimuth~$0^\circ$ (the $x$-$z$ plane).
The source is a vertical point force buried at a depth of 500~meters.
The {\qP} wave has been removed (leaving a slight artifact near the top left
of each plot).
Only the ``SH,~$30^\circ$'' plot shows any distinctive orthorhombic effects.
\index{connections | section example}
\index{impulse response | 3D slice example}
}

The diagnostic orthorhombic effect, the ``connection'', shows up strongly
on the ``SH, $30^\circ$'' plot. The other three plots, however,
reveal very little evidence
that the model medium is not transversely isotropic Greenhorn Shale.
The connection event also nearly disappears for sections oriented
along azimuths $90^\circ$ and~$45^\circ$.
\index{connections | observing}
This suggests that to observe orthorhombic effects in the field
it is necessary to record both along suspected symmetry directions
as inferred from the regional crack directions
and at other arbitrary angles unlikely to correspond to any special symmetry.
It is also beneficial to record the ``null'' sections such as Z source into
SH receiver.
\index{shear singularities | mode-mode coupling}
These ``off diagonal'' sections are most likely to show energy
coupling from one mode into another via the particle-motion direction
disturbance associated with singularities.

\subsubsection{What should a singularity look like?}
Figures \ref{Exam3-HelbigSeis} and~\ref{Exam3-HelbigSeis2}
show model sections recorded over
a homogeneous half-space of the medium from
Figure~\ref{Exam3-Helbig2}
(described in section~\ref{Exam3-HelbigSec}).
This time instead of showing sections
recorded along lines passing over the source at differing azimuths,
a series of sections recorded along parallel lines are shown.
Instead of showing off-diagonal sections,
for all plots both the source and receiver are oriented along the
$y$ axis.

Figure~\ref{Exam3-HelbigSeis} shows lines parallel to the $x$ axis.
(Since the source and receiver are oriented along the $y$ axis,
these could be called ``SH'' lines.)
Four different lines are shown. They were selected to cut through
different parts of the cone of singular directions.
In terms of Figure~\ref{Exam3-Helbig2} on page~\pageref{Exam3-Helbig2},
the $y=.25$ plot completely misses the singularity to the right.
The $y=0.$ plot was recorded along the $x$ axis and passes tangentially
through one edge of the cone of singular directions at $x=0$, directly
over the source.
The $y=.125$ plot passes through the center of the cone of singular
directions, and the $y=-.25$ plot passes tangentially through the
left edge of the cone.
The singularity mostly shows up in the strange flatness of the top
of the hyperbola, especially on the $y=-.125$ plot.

Figure~\ref{Exam3-HelbigSeis2} shows lines parallel to the $y$ axis.
(Since the source and receiver are oriented along the $y$ axis,
these could be called ``SV'' lines.)
In terms of Figure~\ref{Exam3-Helbig2},
the $x=0.$ plot passes over the source and
through the center of the cone of singular directions.
The $x=.125$ plot passes tangentially through the near edge of the
cone of singular directions.
The $x=.313$ and $x=.656$ plots completely miss the singularity.
Here the singularity shows up in the strange breaks in the events
as the {\qS1} and {\qS2} modes reassert their separate identities after
we move past the singularity, especially on the $x=.313$ plot.

\Tactiveplot{Exam3-HelbigSeis}{width=5.80in}{\figdir}
{Modeled $x$ sections of the medium
from Figure~\protect\ref{Exam3-HelbigSlice}.}
{
Modeled $x$ sections of the medium
from Figure~\protect\ref{Exam3-HelbigSlice}.
The source is a Y point force buried at a depth of $1$~unit;
the $y$ component of particle-motion direction is shown.
\index{impulse response | 3D slice example}
\index{shear singularities | impulse response example}
}
\Tactiveplot{Exam3-HelbigSeis2}{width=5.80in}{\figdir}
{Modeled $y$ sections of the medium
from Figure~\protect\ref{Exam3-HelbigSlice}.}
{
Modeled $y$ sections of the medium
from Figure~\protect\ref{Exam3-HelbigSlice}.
The source is a Y point force buried at a depth of $1$~unit;
the $y$ component of particle-motion direction is shown.
\index{impulse response | 3D slice example}
\index{shear singularities | impulse response example}
}

\index{shear singularities | observing}
Unfortunately from these examples it appears the direct effects
of a singularity likely to show up on standard SV and SH sections
are not very clear cut. Stacking such mangled events along hyperbolas
is likely to produce something that looks like merely poor-quality or
noisy data, even as the corresponding {\qP} events
stack consistently and well.


\section{HORIZONTAL REFLECTOR BENEATH A HOMOGENEOUS VTI MEDIUM}
%%%%%%%%%%%%%%%%%%%%%%%%%%%%%%%%%%%%%%%%%%%%%%%%%%%%%%%

\inputdir{XFig}

To exemplify the use of weak anisotropy, let us consider the
simplest model of a homogeneous VTI medium above a horizontal
reflector. For an isotropic medium, the reflection traveltime curve is
an exact hyperbola, as follows directly from the Pythagorean theorem
(Figure \ref{fig:nmoone})
\begin{equation}
t^2(l) = {{4\,z^2 + l^2} \over V_z^2} = t_0^2 + {l^2 \over V_z^2}\;,
\label{eqn:pifagor}
\end{equation}
where $z$ denotes the depth of reflector, $l$ is the offset,
$t_0=t(0)$ is the zero-offset traveltime, and $V_z$ is the
isotropic velocity. For a homogeneous VTI medium, the
velocity $V_z$ in equation (\ref{eqn:pifagor}) is replaced by the
angle-dependent group velocity $V_g$. This replacement leads to the
exact traveltimes if no approximation for the group velocity is used,
since the ray trajectories in homogeneous VTI media remain straight,
and the reflection point does not move. 
We can also obtain an approximate traveltime using the
approximate velocity $V_g$ defined by equations (\ref{eqn:vg}) or
(\ref{eqn:vgeta}), where the ray angle $\psi$ is given by 
\begin{equation}
\sin^2{\psi} = {{l^2} \over {4\,z^2 + l^2}}\;.
\label{eqn:sinpsi}
\end{equation}
Substituting equation (\ref{eqn:sinpsi}) into (\ref{eqn:vgeta}) and
linearizing the expression
\begin{equation}
  t^2(l) = {{4\,z^2 + l^2} \over V_g^2(\psi)}
\label{eqn:TIpifagor}
\end{equation}
with respect to the anisotropic parameters $\delta$ and $\eta$, we arrive
at the three-parameter nonhyperbolic approximation \cite[]{tsvantom}
\begin{equation}
t^2(l) = t_0^2 + {l^2 \over V_n^2} - {{2\,\eta\,l^4} \over 
{V_n^2\,\left(V_n^2 t_0^2 + l^2\right)}}\;,
\label{eqn:TIapprox}
\end{equation}
where the normal-moveout velocity $V_n$ is defined by equation
(\ref{eqn:vn}).  At small offsets $(l \ll z)$, the influence of the
parameter $\eta$ is negligible, and the traveltime curve is nearly
hyperbolic. At large offsets $(l \gg z)$, the third term in equation
(\ref{eqn:TIapprox}) has a clear influence on the traveltime behavior.
The Taylor series expansion of equation (\ref{eqn:TIapprox}) in the vicinity 
of the vertical zero-offset ray has the form
\begin{equation}
t^2(l) = t_0^2 + {l^2 \over V_n^2} - {{2\,\eta\,l^4} \over 
{V_n^4\,t_0^2}} + {{2\,\eta\,l^6} \over 
{V_n^6\,t_0^4}} - \ldots \;.
\label{eqn:TItaylor}
\end{equation}
When the offset $l$ approaches infinity, the traveltime
approximately satisfies an intuitively reasonable relationship
\begin{equation}
\lim_{l \rightarrow \infty} t^2(l) = {l^2 \over V_x^2}\;,
\label{eqn:hlimit}
\end{equation}
where the horizontal velocity $V_x$ is defined by equation~(\ref{eqn:vx}).  
Approximation (\ref{eqn:TIapprox}) is analogous, within the weak-anisotropy assumption, to the ``skewed hyperbola'' equation
\cite[]{GEO54-12-15641574} which uses the three velocities $V_z$, $V_n$,
and $V_x$ as the parameters of the approximation:
\begin{equation}
t^2(l) = t_0^2 + {l^2 \over V_n^2} - {{l^4} \over 
{V_n^2 t_0^2 + l^2}}\,
\left({1 \over V_n^2} - {1 \over V_x^2}\right)\;.
\label{eqn:BHapprox}
\end{equation}
The accuracy of equation (\ref{eqn:TIapprox}), which usually lies within 1\% error up to offsets twice as large as reflector depth, can be further improved at any
finite offset by modifying the denominator of the third term
\cite[]{aktsvan,grektsvan}.

\plot{nmoone}{width=6in,height=3in}{Reflected rays
in a homogeneous VTI layer above a horizontal reflector (a scheme).}
%??? -- Move the vertical dashed line exactly to the reflection point.}

\par
\cite{Muir.sep.44.55} suggested a different nonhyperbolic moveout
approximation in the form
%\begin{equation}
%t^2(l) = {{t_0^4 + (1 + f)\,{l^2 \over V_n^2} + f^2 {{l^4} \over
%{V_n^4}}} \over {{t_0^2 + f\,{l^2 \over V_n^2}}}} =
%t_0^2 + {l^2 \over V_n^2} - {{f\,(1-f)\,l^4} \over
%{V_n^2\,\left(V_n^2 t_0^2 + f\,l^2\right)}}\;,
%\label{eqn:DMapprox}
%\end{equation}
\begin{equation}
t^2(l) = 
t_0^2 + {l^2 \over V_n^2} - {{f\,(1-f)\,l^4} \over 
{V_n^2\,\left(V_n^2 t_0^2 + f\,l^2\right)}}\;,
\label{eqn:DMapprox}
\end{equation}
where $f$ is the dimensionless parameter of anellipticity. At large
offsets, equation (\ref{eqn:DMapprox}) approaches
\begin{equation}
\lim_{l \rightarrow \infty} t^2(l) = f\,{l^2 \over V_n^2}\;.
\label{eqn:DMhlimit}
\end{equation}
Comparing equations (\ref{eqn:hlimit}) and (\ref{eqn:DMhlimit}), we
can establish the correspondence
\begin{equation}
f = {{V_n^2} \over {V_x^2}} = {{1 + 2\,\delta} \over {1 +
2\,\epsilon}} \approx 1 - 2\,\eta\;.
\label{eqn:f2eta}
\end{equation}
Taking this equality into account, we see that equation
(\ref{eqn:DMapprox}) is approximately equivalent to equation
(\ref{eqn:TIapprox}) in the sense that their difference has the order
of $\eta^2$.


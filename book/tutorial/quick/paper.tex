\title{Quick reference}
\author{Jeff Godwin}
\maketitle

\section{Environmental variables}

These are the environmental variables that are used in Madagascar programs:

\begin{tabular}{| l | l |}
    \hline
    RSFROOT & location of the main Madagascar installation \\
    RSFSRC & location of the source for Madagascar \\
    DATAPATH & location where to put the binary RSF files \\
    RSF\_MEMSIZE & maximum memory size for some programs to use \\
    PYTHONPATH & set to point to Python libraries \\
    LD\_LIBRARY\_PATH & set to point to RSF dynamic libraries \\
    \hline
\end{tabular}

\section{Command line usage}

Using programs on the command line:
\begin{verbatim}
sfprogram < input.rsf arg1=val1 arg2=val2 ... > output.rsf
\end{verbatim}
Example:
\begin{verbatim}
sftransp < file.rsf plane=12 > file2.rsf
\end{verbatim}
Example, with Pipes:
\begin{verbatim}
sftransp < file.rsf plane=12 | sftransp plane=23 | sftransp plane=34 > file2.rsf
\end{verbatim}

\section{Some useful programs}

\begin{tabular}{| l | l |}
   \hline 
   sfdoc -k . $|$ less & show program descriptions \\
   sfbrowser & show program browser \\
   sfgui & show tkMadagascar GUI \\
   sfspike & create RSF files \\
   sfmath & create and manipulate RSF files \\
   sfadd & add, subtract, multiply datasets together \\
   sftransp & change the order of axes in RSF files \\
   sfwindow & window out portions of RSF files \\
   sfricker1 & create a Ricker wavelet \\
   sffft1 & FFT on the first axis (real to complex) \\
   sffft3 & FFT on other axes (complex) \\
   sfnoise & add noise  \\
   sfdd & convert datasets \\
   sfgrey & make raster plots \\
   sfcat & concatenate datasets together \\
   sfput & modify header values \\
   sfsegyread & read SEGY/SU files \\
   sfsegywrite & write SEGY/SU files \\
   \hline
\end{tabular}

\section{SCons commands}

\begin{tabular}{| l | l |}
    \hline
    scons & run an SConstruct \\
    scons view & view the results from an SConstruct, run if necessary \\
    scons lock & lock the results from an SConstruct \\
    scons -c & clean the local directory, delete all files \\
    scons -n & dry-run of an SConstruct \\
    scons --debug=explain & explain why SCons is doing what it does \\
    pscons & parallel execution of an SConstruct  \\
    \hline
\end{tabular}

\title{Introduction}
Welcome to a brief introduction to Madagascar.  The purpose of this
document is to teach new users how to use the powerful tools in
Madagascar to: process data, produce documents and build your own
Madagascar programs.

To gradually introduce you to Madagascar, we have created a series of
tutorials that are targeted to distinct audiences and designed to make
you an experienced Madagascar user in a short-time period.  The
tutorials are divided by interest into three main categories:
\begin{description}
    \item[Users] learn about Madagascar, how to use the processing programs, and build scripts.
    \item[Authors] learn how to build reproducible documents using Madagascar.
    \item[Developers] build new Madagascar programs that add additional functionality to Madagascar.
\end{description}

Each tutorial is designed to be completed in a short period of time.
Additionally, each tutorial has hands-on examples that you should be
able to reproduce on your computer as you go along with the tutorials.
Most tutorials will use scripts that you can edit, modify, or play
with to further gain experience and understanding.  By the end of the
tutorial series, you should be able to use all of the tools inside of
Madagascar.  Please note that this tutorial series does not explicitly
show you how to process certain types of data, or how to perform common data
processing operations (e.g. CMP semblance picking,
time migration,etc.).  Additional tutorials on those specific subjects
will be added over time.  The purpose of this document is simply to
familiarize you with the Madagascar framework in a general sense.

Before you go on, here are some notes on notation:
\begin{itemize}
    \item important names, or program names are usually bold in the text.  For example: \textbf{sfwindow}
    \item code snippets are always in the following formatting: \begin{verbatim} sfwindow \end{verbatim}
\end{itemize}
